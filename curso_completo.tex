\documentclass[12pt,a4paper]{{article}}
\usepackage[utf8]{{inputenc}}
\usepackage{{fontspec}}
\usepackage{{polyglossia}}
\setmainlanguage{{spanish}}
\setotherlanguage{{hindi}}

% Font setup - IMPORTANT: You must have 'Noto Sans Devanagari' installed or change this.
\newfontfamily\hindifont{{Noto Sans Devanagari}}[Script=Devanagari, AutoFakeBold=1.5]
\setmainfont{{Arial}}

\usepackage{{geometry}}
\geometry{{margin=2.5cm}}
\usepackage{{hyperref}}
\usepackage{{xcolor}}
\usepackage{{tcolorbox}}
\usepackage{{tabularx}}
\usepackage{{float}}
\usepackage{{booktabs}}

\setlength{{\parskip}}{{1em}}
\setlength{{\parindent}}{{0pt}}

\begin{{document}}


\title{Curso Completo de Hindi}
\maketitle
\tableofcontents
\newpage


Compendio de todas las unidades de aprendizaje.



\hrulefill

\section{Unidad 1: Introducción al hindi y al alfabeto devanagari (I)}


Objetivo: Introducir el sistema de escritura devanagari y las fórmulas básicas de cortesía en hindi

\subsection{1. El Alfabeto Devanagari: Sistema de Escritura del Hindi}

\subsubsection{1.1 Introducción al Devanagari}


El \textbf{Devanagari} (देवनागरी) es el sistema de escritura utilizado para escribir hindi y varios otros idiomas indoarios. Es un sistema silábico, lo que significa que cada carácter representa una consonante con una vocal inherente. El nombre "Devanagari" proviene de las palabras sánscritas "deva" (देव), que significa "dios", y "nagar" (नगर), que significa "ciudad", sugiriendo que es la escritura de los dioses.


\begin{tcolorbox}[colback=blue!5!white,colframe=blue!75!black,title=Nota/Clarificación]

\paragraph{Importancia del Devanagari} 

El dominio del Devanagari es esencial para cualquier estudiante de hindi, ya que:


\begin{itemize}

  \item Es el único sistema de escritura oficial para el hindi en la India

  \item Permite una correcta pronunciación de las palabras

  \item Es la base para entender la gramática y estructura del idioma

  \item Facilita la lectura y escritura auténticas en hindi

\end{itemize}


\end{tcolorbox}

\subsubsection{1.2 Características Generales del Devanagari}


Antes de explorar los detalles del sistema de escritura, es importante entender sus características fundamentales:


\begin{tcolorbox}[colback=gray!5!white,colframe=gray!75!black,title=Explicación]

\paragraph{Características Principales} 

\begin{enumerate}

  \item \textbf{Escritura de izquierda a derecha}: Al igual que el alfabeto latino, el devanagari se escribe de izquierda a derecha.

  \item \textbf{Forma cursiva}: Muchos caracteres se conectan entre sí al escribir, especialmente en la forma cursiva.

  \item \textbf{Línea superior horizontal}: La mayoría de los caracteres devanagari comparten una línea superior horizontal denominada \textit{shirorekha}.

  \item \textbf{Sistema silábico}: Cada caracter principal representa una consonante con la vocal "a" inherente.

  \item \textbf{Matras para otras vocales}: Para representar otras vocales, se utilizan marcas diacríticas llamadas \textit{matras}.

\end{enumerate}


\end{tcolorbox}

\subsubsection{1.3 La Shirorekha (शिरोरेखा): La Línea Superior}


La \textbf{Shirorekha} (शिरोरेखा), que literalmente significa "línea principal", es una línea horizontal que corre sobre muchos de los caracteres devanagari. Es una característica distintiva de este sistema de escritura.


\begin{tcolorbox}[colback=yellow!10!white,colframe=orange!75!black,title=Regla de Oro]

\paragraph{Regla Importante de la Shirorekha} 

La Shirorekha \textbf{se dibuja después} de escribir los caracteres individuales. No se traza antes como si fuera una línea base. Se coloca en la parte superior de los caracteres después de que la palabra esté formada.


\end{tcolorbox}

\subsubsection{1.4 Vocales (स्वर - Swar)}


El devanagari tiene 11 vocales básicas, aunque se representan mediante 13 símbolos (algunas vocales tienen formas largas y cortas). Estas vocales son:


\begin{center}
\begin{tabularx}{\textwidth}{| X | X | X | X | X |}
\hline
\textbf{Número} & \textbf{Nombre} & \textbf{Vocal Devanagari} & \textbf{Transliteración} & \textbf{Ejemplo} \ \hline
1 & Akar (अकार) & अ & a & Vocal 'a' corta y neutra (schwa) \ \hline
2 & Aakar (आकार) & आ & ā & 'a' larga y abierta, como en "casa" \ \hline
3 & Ikār (इकार) & इ & i & 'i' corta, como en "sí" \ \hline
4 & Īkār (ईकार) & ई & ī & 'i' larga, como en "sí" en español \ \hline
5 & Ukār (उकार) & उ & u & 'u' corta, como en "tú" \ \hline
6 & Ūkār (ऊकार) & ऊ & ū & 'u' larga, como en "tú" en español \ \hline
7 & Rikār (ऋकार) & ऋ & ṛ & Similar a 'ri' en "cristal" \ \hline
8 & Ekkār (एकार) & ए & e & 'e' cerrada, como en "tengas", nunca abierta \ \hline
9 & Aikār (ऐकार) & ऐ & ai & Como en "aire" \ \hline
10 & Okkār (ओकार) & ओ & o & 'o' cerrada, como en "torre" \ \hline
11 & Aukkār (औकार) & औ & au & Como en "pausa" \ \hline
\end{tabularx}
\end{center}

\subsubsection{1.5 Consonantes (व्यंजन - Vyanjan)}


El devanagari tiene 33 consonantes principales divididas en 5 grupos articulatorios, cada uno con 5 consonantes:

\paragraph{Grupos de Consonantes} 

\begin{enumerate}

  \item \textbf{Velares (क-वर्ग - ka-varga)}: Producidas con la parte posterior de la lengua tocando el paladar blando

  \item \textbf{Panatrales (च-वर्ग - ca-varga)}: Producidas con la parte central de la lengua tocando el paladar duro

  \item \textbf{Retroflejas (ट-वर्ग - ṭa-varga)}: Producidas con la punta de la lengua doblada hacia atrás

  \item \textbf{Dentales (त-वर्ग - ta-varga)}: Producidas con la punta de la lengua tocando los dientes

  \item \textbf{Labiales (प-वर्ग - pa-varga)}: Producidas con los labios

\end{enumerate}

\paragraph{Primeros Grupos: Velares y Palatales} 

Según los contenidos de la programación, en esta unidad se introducen los primeros grupos de consonantes:


\begin{center}
\begin{tabularx}{\textwidth}{| X | X | X | X | X |}
\hline
\textbf{Grupo} & \textbf{Consonante Devanagari} & \textbf{Transliteración} & \textbf{Nombre} & \textbf{Sonido} \ \hline
Velares (क-वर्ग) & क & k & Ka (क) & Como 'k' en "carro" (sin aspiración) \ \hline
ख & kh & Kha (ख) & Como 'k' en "carro" (con aspiración) &  \ \hline
ग & g & Ga (ग) & Como 'g' en "gato" &  \ \hline
घ & gh & Gha (घ) & Como 'g' en "globo" (voz más fuerte) &  \ \hline
ङ & ṅ & Na (ङ) & Como 'ng' en "tango" (sonido nasal) &  \ \hline
Palatales (च-वर्ग) & च & c & Ca (च) & Como 'ch' en "chocolate" (sin aspiración) \ \hline
छ & ch & Cha (छ) & Como 'ch' en "chocolate" (con aspiración) &  \ \hline
ज & j & Ja (ज) & Como 'j' en "jardín" (en español) o 'g' en "girar" &  \ \hline
झ & jh & Jha (झ) & Versión aspirada de Ja &  \ \hline
ञ & ñ & Nya (ञ) & Como 'ñ' en "señor" &  \ \hline
\end{tabularx}
\end{center}

\subsubsection{1.6 El Concepto de Matra (मात्रा)}


Las \textbf{matras} (मात्रा) son marcas diacríticas que se adjuntan a las consonantes para modificar la vocal inherente 'a'. Cada matra representa una vocal específica y puede aparecer en diferentes posiciones alrededor de la consonante: arriba, debajo, antes o después.


\begin{tcolorbox}[colback=blue!5!white,colframe=blue!75!black,title=Nota/Clarificación]

\paragraph{Importancia de las Matras} 

Las matras son esenciales porque permiten:


\begin{itemize}

  \item Modificar la vocal inherente 'a' de las consonantes

  \item Representar todas las vocales en combinación con consonantes

  \item Formar sílabas completas con diferentes sonidos vocálicos

\end{itemize}


\end{tcolorbox}

\subsubsection{1.7 Tabla Completa de Matras}


A continuación se muestra la correspondencia entre las vocales completas y sus respectivas matras:


\begin{center}
\begin{tabularx}{\textwidth}{| X | X | X | X | X |}
\hline
\textbf{Vocal (Forma Completa)} & \textbf{Matra (Signo Diacrítico)} & \textbf{Ubicación} & \textbf{Ejemplo con क (ka)} & \textbf{Transliteración} \ \hline
Sin vocal (solo consonante) & Halant (◌्) & Debajo de la consonante & {\hindifont क्} & k \ \hline
अ (a) & (Ninguna, forma básica) & (Natural) & {\hindifont क} & ka \ \hline
आ (ā) & ा & Después de la consonante & {\hindifont का} & kā \ \hline
इ (i) & ि & Antes de la consonante & {\hindifont कि} & ki \ \hline
ई (ī) & ी & Después de la consonante & {\hindifont की} & kī \ \hline
उ (u) & ु & Debajo de la consonante & {\hindifont कु} & ku \ \hline
ऊ (ū) & ू & Debajo de la consonante & {\hindifont कू} & kū \ \hline
ऋ (ṛ) & ृ & Después de la consonante & {\hindifont कृ} & kṛ \ \hline
ए (e) & े & Después de la consonante & {\hindifont के} & ke \ \hline
ऐ (ai) & ै & Después de la consonante & {\hindifont कै} & kai \ \hline
ओ (o) & ो & Después de la consonante & {\hindifont को} & ko \ \hline
औ (au) & ौ & Después de la consonante & {\hindifont कौ} & kau \ \hline
\end{tabularx}
\end{center}

\subsubsection{1.8 La Vocal Inherente 'a' (अ)}


Este es el concepto más importante del devanagari. Cada consonante (p. ej., {\hindifont क}) se considera una sílaba completa que ya incluye el sonido de la vocal 'a' corta (schwa).


\begin{tcolorbox}[colback=blue!5!white,colframe=blue!75!black,title=Nota/Clarificación]

\paragraph{Comprensión de la Vocal Inherente} 

Es importante entender que:


\begin{itemize}

  \item {\hindifont क} no es el sonido 'k', se lee como 'ka'

  \item {\hindifont म} no es el sonido 'm', se lee como 'ma'

  \item {\hindifont न} no es el sonido 'n', se lee como 'na'

  \item Este sonido es muy corto y neutro, similar a la 'a' en la palabra inglesa "about" o "sofa"

\end{itemize}


\end{tcolorbox}

\subsubsection{1.9 El Halant (हलन्त)}


El \textbf{Halant} (हलन्त), también conocido como \textbf{Virama}, es un pequeño trazo diagonal ({\hindifont ्}) que se coloca debajo de una consonante. Su función es cancelar la vocal inherente 'a', dejando la consonante "pura".


\begin{tcolorbox}[colback=green!5!white,colframe=green!75!black,title=Ejemplo]

\paragraph{Ejemplos de Uso del Halant} 

\begin{itemize}

  \item {\hindifont क} = ka (consonante + vocal inherente)

  \item {\hindifont क्} = k (solo el sonido consonántico 'k')

  \item Se usa principalmente para formar grupos de consonantes (ligaduras), como veremos más adelante. Ej: {\hindifont नमस्ते} (namaste) se forma de {\hindifont न + म् + स् + ते}.

\end{itemize}


\end{tcolorbox}

\subsubsection{1.10 Fonética: Aspiración}


La fonética hindi distingue entre sonidos \textbf{sin aspirar} (sin soplo de aire) y \textbf{aspirados} (con un soplo de aire). Esta diferencia es crucial para el significado de las palabras.


\begin{center}
\begin{tabularx}{\textwidth}{| X | X | X | X |}
\hline
\textbf{Tipo} & \textbf{Consonante (Devanagari)} & \textbf{Transliteración} & \textbf{Ejemplo de Sonido} \ \hline
Sin aspirar & {\hindifont क}, {\hindifont ग}, {\hindifont च}, {\hindifont ज} & k, g, c, j & Como en español, sin soplo de aire \ \hline
Aspirado & {\hindifont ख}, {\hindifont घ}, {\hindifont छ}, {\hindifont झ} & kh, gh, ch, jh & Con un claro soplo de aire al pronunciar \ \hline
\end{tabularx}
\end{center}


\begin{tcolorbox}[colback=blue!5!white,colframe=blue!75!black,title=Nota/Clarificación]

\paragraph{Consejo Práctico: Sentir la Aspiración} 

Para distinguir entre sonidos aspirados y no aspirados, coloca la palma de tu mano frente a tu boca. Al decir una consonante aspirada como {\hindifont ख} (kh), deberías sentir un claro soplo de aire. Al decir una sin aspirar como {\hindifont क} (k), apenas deberías sentirlo.


\end{tcolorbox}

\subsubsection{1.11 Formas de Cortesía y Fórmulas Básicas}


En esta unidad también se introducen fórmulas básicas de cortesía, que son esenciales para comunicarse adecuadamente en hindi:


\begin{center}
\begin{tabularx}{\textwidth}{| X | X | X | X | X |}
\hline
\textbf{Expresión} & \textbf{Devanagari} & \textbf{Transliteración} & \textbf{Significado} & \textbf{Uso} \ \hline
Saludos & {\hindifont नमस्ते} & Namaste & Saludos (literalmente: "Me inclino ante ti") & Saludo formal e informal \ \hline
Gracias & {\hindifont धन्यवाद} & Dhanyavaad & Agradecimiento & Expresar gratitud \ \hline
Por favor & {\hindifont कृपया} & Krupaya & Polite request marker & Para hacer peticiones con cortesía \ \hline
Perdóneme/Disculpe & {\hindifont माफ कीजिए} & Maaf kijiye & Pedir disculpas & Cuando cometes un error o molestar \ \hline
Hasta luego & {\hindifont फिर मिलेंगे} & Phir milenge & Nos vemos después & Despedida formal \ \hline
\end{tabularx}
\end{center}

\subsubsection{1.12 La Partícula de Respeto: जी (Jī)}


{\hindifont जी} (Jī) es una partícula de respeto fundamental en hindi. No tiene una traducción directa, pero añade un sentido de cortesía y respeto. Se puede usar de varias formas:


\begin{itemize}

  \item Después de un nombre: {\hindifont राहुल जी} (Rahul-jī)

  \item Después de "sí" o "no": {\hindifont जी हाँ} (Jī hān - Sí, con respeto), {\hindifont जी नहीं} (Jī nahin - No, con respeto).

  \item Solo, como respuesta para decir "¿Sí?" o "Dígame" de forma educada.

\end{itemize}

\subsubsection{1.13 Números del 1 al 10}


Para completar esta primera unidad, introducimos los números del 1 al 10 en devanagari:


\begin{center}
\begin{tabularx}{\textwidth}{| X | X | X | X |}
\hline
\textbf{Número} & \textbf{Devanagari} & \textbf{Transliteración} & \textbf{Pronunciación} \ \hline
1 & {\hindifont एक} & Ek & AYK (como "ace" sin la 'c') \ \hline
2 & {\hindifont दो} & Do & DOH (como "dough") \ \hline
3 & {\hindifont तीन} & Tiin & TEEN (como "teen") \ \hline
4 & {\hindifont चार} & Char & CHAA-r (con 'aa' largo) \ \hline
5 & {\hindifont पाँच} & Paanch & PAANCH (con 'aa' largo y 'n' nasal) \ \hline
6 & {\hindifont छह} & Chhah & CHHAH (sonido de 'ch' geminado) \ \hline
7 & {\hindifont सात} & Saat & SAAT (con 'aa' largo) \ \hline
8 & {\hindifont आठ} & Aath & AA-th (con 'aa' largo) \ \hline
9 & {\hindifont नौ} & Nau & now (como "cow" pero con 'n') \ \hline
10 & {\hindifont दस} & Das & dahs (como "mass" con 'd') \ \hline
\end{tabularx}
\end{center}

\subsubsection{1.14 Actividades Prácticas}


Ahora que has aprendido los conceptos básicos del devanagari, inténtalo:


\begin{enumerate}

  \item Escribe tu nombre en devanagari (usa los sonidos equivalentes en inglés si no existe una correspondencia exacta)

  \item Identifica las matras en estas palabras: {\hindifont गाय} (gāy - vaca), {\hindifont पीना} (pīnā - beber), {\hindifont कहना} (kahnā - decir)

  \item Practica escribiendo los números del 1 al 10 en devanagari

  \item Usa las fórmulas de cortesía en situaciones imaginarias

\end{enumerate}


\hrulefill

\section{Unidad 2: Pronombres, Verbo 'ser/estar' e Identidad}


Objetivo: Dominar los pronombres personales, la conjugación del verbo होना y formular frases de identidad básica

\subsection{2.1 Pronombres Personales en Hindi}


Los pronombres personales en hindi son fundamentales para la comunicación y tienen diferentes niveles de formalidad, lo cual es muy importante en la cultura india. A diferencia de otros idiomas, en hindi distinguimos entre varios registros de tratamiento que van desde lo íntimo hasta lo formal.

\subsubsection{2.1.1 Tabla Completa de Pronombres Personales}


\begin{center}
\begin{tabularx}{\textwidth}{| X | X | X | X | X |}
\hline
\textbf{Pronombre Devanagari} & \textbf{Transliteración} & \textbf{Español} & \textbf{Nivel de Formalidad y Uso} & \textbf{Guía de Uso Contextual} \ \hline
{\hindifont मैं} & main & Yo & Neutro - Se usa para todos & Forma estándar para referirse a uno mismo \ \hline
{\hindifont तू} & tū & Tú (íntimo/muy informal) & Íntimo - Muy familiar o rudo si se usa incorrectamente & Usado con niños, amigos muy cercanos, en rezos o con personas con las que tienes una relación especial. EVITAR EN SITUACIONES FORMALES O CON DESCONOCIDOS. \ \hline
{\hindifont तुम} & tum & Tú (informal) & Informal - Estándar con amigos, familiares y conocidos & El pronombre "tú" por excelencia. Es el registro estándar entre amigos, familiares y compañeros. \ \hline
{\hindifont आप} & āp & Usted (formal) & Formal - Con extraños, mayores, autoridades o para mostrar respeto & Es el pronombre seguro para usar en cualquier situación. Siempre es respetuoso y adecuado. \ \hline
{\hindifont यह} & yah & Él/Ella/Esto & Proximal - Se refiere a alguien o algo cerca del hablante & Usado para referirse a cosas o personas próximas o que están siendo presentadas. \ \hline
{\hindifont वह} & vah & Él/Ella/Aquello & Distal - Se refiere a alguien o algo lejos del hablante & Usado para cosas o personas que no están cerca o ya se han mencionado anteriormente. \ \hline
{\hindifont हम} & ham & Nosotros/Nosotras & Neutro - Se usa para todos & Forma estándar para referirse al grupo que incluye al hablante \ \hline
{\hindifont ये} & ye & Ellos/Ellas/Estos & Proximal Plural - Forma plural de {\hindifont यह} & Plural de proximidad, para referirse a cosas o personas cerca del hablante \ \hline
{\hindifont वे} & ve & Ellos/Ellas/Aquellos & Distal Plural - Forma plural de {\hindifont वह} & Plural de distalidad, para referirse a cosas o personas lejos del hablante \ \hline
\end{tabularx}
\end{center}


\begin{tcolorbox}[colback=blue!5!white,colframe=blue!75!black,title=Nota/Clarificación]

\paragraph{Niveles de Formalidad: Regla de Oro} 

La elección del pronombre correcto es crucial en hindi porque refleja el nivel de respeto y la relación entre las personas que interactúan. La elección equivocada puede causar incomodidad o incluso ofensa.

Guía de Selección por Situación

\begin{itemize}

  \item \textbf{Usa {\hindifont तू} (tū) solo si:}

\begin{itemize}

  \item Hablas con un niño muy pequeño

  \item Estás en una relación muy cercana y la otra persona te invita explícitamente a usarlo

  \item Te diriges a una divinidad en una oración personal

  \item Es un contexto artístico/poético muy especializado (muy raro en conversación diaria)

\end{itemize}



  \item \textbf{Usa {\hindifont तुम} (tum) para:}

\begin{itemize}

  \item Amigos y compañeros

  \item Miembros de la familia del mismo rango de edad

  \item Personas a las que conoces bien y que te han invitado a usar un tratamiento informal

  \item Personas relativamente jóvenes con las que tienes confianza

\end{itemize}



  \item \textbf{Usa {\hindifont आप} (āp) si:}

\begin{itemize}

  \item Hay una diferencia de edad significativa (mayores)

  \item Estás en un entorno profesional o académico

  \item Hablas con extraños

  \item Quieres ser educado o mostrar respeto

  \item No estás seguro del nivel de formalidad apropiado

\end{itemize}


ESTE ES EL PRONOMBRE SEGUR@ PARA CUALQUIER SITUACIÓN. Cuando tengas dudas, usa {\hindifont आप}.



\end{itemize}


\end{tcolorbox}

\subsubsection{2.1.2 Diferenciación Proximal-Distal: {\hindifont यह/वह} y {\hindifont ये/वे}}


En hindi, hay una distinción fundamental entre proximidad y distancia que no existe en español:


\begin{itemize}

  \item {\hindifont यह} (yah) y {\hindifont ये} (ye): Se refieren a algo/some que está cerca del hablante

  \item {\hindifont वह} (vah) y {\hindifont वे} (ve): Se refieren a algo/some que está lejos del hablante

\end{itemize}



\paragraph{Ejemplos de Uso Proximal-Distal} 

\begin{itemize}

  \item Si apuntas a una persona que está cerca de ti: {\hindifont यह राम है।} (Yah Ram hai. - Éste es Ram.)

  \item Si apuntas a una persona que está lejos de ti: {\hindifont वह राम है।} (Vah Ram hai. - Aquél es Ram.)

  \item Si apuntas a varias personas que están cerca de ti: {\hindifont ये मेरे दोस्त हैं।} (Ye mere dost hain. - Estos son mis amigos.)

  \item Si apuntas a varias personas que están lejos de ti: {\hindifont वे मेरे दोस्त हैं।} (Ve mere dost hain. - Aquéllos son mis amigos.)

\end{itemize}



\subsection{2.2 El Verbo 'Ser/Estar' ({\hindifont होना} - honā)}


El verbo {\hindifont होना} (honā) es el verbo más importante en hindi. Se traduce como "ser" o "estar" y se usa para expresar identidad, estado o existencia. Aprender su conjugación es fundamental para comenzar a construir frases completas.

\subsubsection{2.2.1 Conjugación del Verbo {\hindifont होना} en Presente}


La conjugación del verbo {\hindifont होना} debe coincidir con el sujeto de la oración en número y persona. Veamos cómo se conjuga:


\begin{center}
\begin{tabularx}{\textwidth}{| X | X | X | X | X | X |}
\hline
\textbf{Sujeto} & \textbf{Pronombre} & \textbf{Forma del Verbo} & \textbf{Transliteración} & \textbf{Ejemplo Completo} & \textbf{Traducción} \ \hline
Primera Persona Singular & {\hindifont मैं} & {\hindifont हूँ} & hū̃ & {\hindifont मैं छात्र हूँ।} & Yo soy estudiante. \ \hline
Segunda Persona Familiar & {\hindifont तू} & {\hindifont है} & hai & {\hindifont तू अच्छा है।} & Tú eres bueno. \ \hline
Segunda Persona Informal & {\hindifont तुम} & {\hindifont हो} & ho & {\hindifont तुम कौन हो?} & ¿Quién eres tú? \ \hline
Segunda Persona Formal & {\hindifont आप} & {\hindifont हैं} & ha̱ & {\hindifont आप अध्यापक हैं।} & Usted es profesor. \ \hline
Tercera Persona Singular (Próximo) & {\hindifont यह} & {\hindifont है} & hai & {\hindifont यह मेरी किताब है।} & Ésta es mi libro. \ \hline
Tercera Persona Singular (Distante) & {\hindifont वह} & {\hindifont है} & hai & {\hindifont वह मेरा भाई है।} & Aquél es mi hermano. \ \hline
Primera Persona Plural & {\hindifont हम} & {\hindifont हैं} & ha̱ & {\hindifont हम दोस्त हैं।} & Nosotros somos amigos. \ \hline
Tercera Persona Plural (Próximo) & {\hindifont ये} & {\hindifont हैं} & ha̱ & {\hindifont ये मेरे माता-पिता हैं।} & Éstos son mis padres. \ \hline
Tercera Persona Plural (Distante) & {\hindifont वे} & {\hindifont हैं} & ha̱ & {\hindifont वे अध्यापक हैं।} & Aquéllos son profesores. \ \hline
\end{tabularx}
\end{center}


\begin{tcolorbox}[colback=blue!5!white,colframe=blue!75!black,title=Nota/Clarificación]

\paragraph{Observaciones Importantes} 

\begin{itemize}

  \item \textbf{Notación nasalizada:} La diferencia entre {\hindifont है} (hai) y {\hindifont हैं} (hain) se indica con el bindi (signo nasal) encima de la letra. Esta distinción es crucial para la comprensión correcta de la oración.

  \item \textbf{Formalidad y respeto:} Usar {\hindifont हैं} con el pronombre {\hindifont आप} (en lugar de {\hindifont है}) es una forma adicional de mostrar respeto.

  \item \textbf{Concordancia:} En hindi, no hay concordancia en género como en español. Solo hay concordancia en número y persona.

\end{itemize}


\end{tcolorbox}

\subsubsection{2.2.2 Estructura de la Oración Sujeto-Objeto-Verbo (SOV)}


Una de las características distintivas del hindi es su estructura de oración S-O-V (Sujeto-Objeto-Verbo), en contraste con la estructura S-V-O del español.


\begin{tcolorbox}[colback=yellow!10!white,colframe=orange!75!black,title=Regla de Oro]

\paragraph{Regla Fundamental de la Estructura del Hindi} 

En hindi, el verbo \textbf{siempre} va al final de la oración o cláusula. Esta es una regla invariable que debes recordar constantemente.

Comparación entre Español e Hindi

\begin{itemize}

  \item \textbf{Español:} Yo (Sujeto) soy (Verbo) estudiante (Objeto/Complemento). - Estructura S-V-O

  \item \textbf{Hindi:} {\hindifont मैं (Sujeto) छात्र (Objeto/Complemento) हूँ (Verbo).} - Estructura S-O-V

\end{itemize}


\end{tcolorbox}

\subsubsection{2.3 El Conector {\hindifont और} (aur)}


El conector {\hindifont और} (aur) significa "y" en español y se usa para unir palabras o frases similares:


\begin{itemize}

  \item {\hindifont राम और श्याम} (Ram aur Shayam) - Ram y Shayam

  \item {\hindifont मैं खाता और पीता हूँ।} (Main khaata aur peeta huun.) - Yo como y bebo.

\end{itemize}

\subsection{2.4 Vocabulario: Miembros de la Familia}


Conocer los términos para los miembros de la familia es esencial para expresar la identidad y las relaciones:


\begin{center}
\begin{tabularx}{\textwidth}{| X | X | X | X |}
\hline
\textbf{Español} & \textbf{Hindi} & \textbf{Transliteración} & \textbf{Pronunciación} \ \hline
Madre & {\hindifont माँ} & mā̃ & MAAN (con nasal) \ \hline
Padre & {\hindifont पिता} & pita & PIT-aa \ \hline
Hermano & {\hindifont भाई} & bhaaee & BHAI (con 'aa' largo) \ \hline
Hermana & {\hindifont बहन} & bahan & BAH-han \ \hline
Hijo & {\hindifont बेटा} & beta & BET-aa \ \hline
Hija & {\hindifont बेटी} & betii & BET-eee \ \hline
Abuela & {\hindifont दादी} & daadee & DAA-dee \ \hline
Abuelo & {\hindifont दादा} & daada & DAA-daa \ \hline
Nieto/Nieta & {\hindifont पोता/पोती} & pota/potii & POT-aa / POT-eee \ \hline
\end{tabularx}
\end{center}

\subsection{2.5 Comunicación Básica: Presentaciones y Expresiones de Identidad}


Con los conceptos aprendidos en esta unidad, podemos formular frases básicas de presentación e identidad:

\subsubsection{2.5.1 Preguntar y Decir el Nombre}


\begin{itemize}

  \item {\hindifont आपका नाम क्या है?} (Aapkaa naam kya hai?) - ¿Cómo se llama Ud.?

  \item {\hindifont तुम्हारा नाम क्या है?} (Tumhaaraa naam kya hai?) - ¿Cómo te llamas?

  \item {\hindifont मेरा नाम राम है।} (Meraa naam Ram hai.) - Me llamo Ram.

  \item {\hindifont मैं राम हूँ।} (Main Ram huun.) - Yo soy Ram.

\end{itemize}

\subsubsection{2.5.2 Presentar a la Familia}


\begin{itemize}

  \item {\hindifont यह मेरी माँ है।} (Yah merii maa hai.) - Ésta es mi madre.

  \item {\hindifont यह मेरे पिता हैं।} (Yah mere pita hain.) - Éste es mi padre.

  \item {\hindifont ये मेरे माता-पिता हैं।} (Ye mere mataa-pita hain.) - Éstos son mis padres.

\end{itemize}

\subsubsection{2.5.3 Expresar Estados Básicos}


\begin{itemize}

  \item {\hindifont मैं ठीक हूँ।} (Main theek huun.) - Estoy bien.

  \item {\hindifont मैं अच्छा हूँ।} (Main achchhaa huun.) - Estoy bien/Estoy bueno.

  \item {\hindifont मुझे समझ में नहीं आ रहा है।} (Mujhe samajh men nahin aa raha hai.) - No entiendo.

  \item {\hindifont मैं ठीक नहीं हूँ।} (Main theek nahin huun.) - No estoy bien.

\end{itemize}

\subsection{2.6 Componente Cultural: Niveles de Cortesía}


La elección del pronombre adecuado en hindi no es solo una cuestión gramatical, sino también cultural. Mostrar respeto a través del lenguaje es fundamental en la sociedad india.

\subsubsection{2.6.1 Registro de Formalidad: {\hindifont तू} vs {\hindifont तुम} vs {\hindifont आप}}


La elección entre estos tres pronombres depende del contexto:


\begin{itemize}

  \item {\hindifont तू} (tū): Uso extremadamente íntimo o informal. Puede ser ofensivo si se usa con desconocidos o personas mayores.

  \item {\hindifont तुम} (tum): Uso informal entre amigos, familiares cercanos y personas del mismo nivel social/edad.

  \item {\hindifont आप} (āp): Uso formal, respetuoso y apropiado para cualquier situación.

\end{itemize}



\paragraph{Ejemplo Cultural} 

Imagina que conoces a alguien por primera vez en un entorno profesional. El uso de {\hindifont आप} mostraría respeto y educación. Si esa persona te dice "{\hindifont तुम} कहो" (tum kaho - dime tú), está invitándote a usar un tratamiento más informal. En hindi, el paso de {\hindifont आप} a {\hindifont तुम} es un indicativo de acercamiento y confianza mutua.



\subsection{2.7 Nota Gramatical: La Nasalización de {\hindifont हूँ} y {\hindifont हैं}}


Es importante prestar atención a la nasalización de los verbos {\hindifont हूँ} (hū̃) y {\hindifont हैं} (ha̱). Esta característica es distintiva y ayuda a diferenciarlos de sus formas no nasales {\hindifont हू} (hū) y {\hindifont है} (hai).



\paragraph{Ejercicio de Práctica} 

\begin{enumerate}

  \item Completa las siguientes oraciones con la forma correcta del verbo {\hindifont होना}:
                
\begin{itemize}

  \item मैं अध्यापक \_\_\_। (Main adhyaapak \_\_\_.) - Yo soy profesor.

  \item तुम कौन \_\_\_? (Tum kaun \_\_\_?) - ¿Quién eres tú?

  \item यह मेरी पुस्तक \_\_\_। (Yah merii pustak \_\_\_.) - Ésta es mi libro.

  \item हम मित्र \_\_\_। (Ham mitra \_\_\_.) - Nosotros somos amigos.

\end{itemize}



  \item Identifica el nivel de formalidad apropiado para las siguientes situaciones:
                
\begin{itemize}

  \item Hablar con tu profesor

  \item Dirección a un amigo de la infancia

  \item Dirigirse a tu hermano menor

  \item Comunicación con un extraño en la calle

\end{itemize}



\end{enumerate}




\hrulefill

\section{Unidad 3: Interrogativos, Origen y Profesiones}


Objetivo: Formular preguntas sobre identidad y origen, y hablar sobre profesiones y nacionalidades

\subsection{3.1 Palabras Interrogativas Fundamentales}


Las palabras interrogativas son esenciales para obtener información en cualquier idioma. En hindi, las preguntas se forman colocando la palabra interrogativa en la posición donde iría la información desconocida.

\subsubsection{3.1.1 Palabras Interrogativas Básicas}


\begin{center}
\begin{tabularx}{\textwidth}{| X | X | X | X | X | X |}
\hline
\textbf{Interrogativo} & \textbf{Devanagari} & \textbf{Transliteración} & \textbf{Traducción} & \textbf{Función} & \textbf{Uso en Oraciones} \ \hline
Qué & {\hindifont क्या} & kya & Qué & Preguntar sobre cosas o para formular preguntas de sí/no & {\hindifont यह क्या है?} (Yah kya hai? - ¿Qué es esto?) o {\hindifont क्या आप ठीक हैं?} (Kya aap theek hain? - ¿Está Ud. bien?) \ \hline
Dónde & {\hindifont कहाँ} & kahaan & ¿Dónde? & Preguntar sobre lugar/localización & {\hindifont आप कहाँ रहते हैं?} (Aap kahaan rehte hain? - ¿Dónde vive Ud.?) \ \hline
Quién & {\hindifont कौन} & kaun & ¿Quién? & Preguntar sobre persona & {\hindifont यह कौन है?} (Yah kaun hai? - ¿Quién es éste?) \ \hline
Cuál & {\hindifont कौन} & kaun & ¿Cuál? & Preguntar por opción específica & {\hindifont कौन सा रास्ता?} (Kaun saa raasta? - ¿Cuál camino?) \ \hline
Cómo & {\hindifont कैसे} & kaise & ¿Cómo? & Preguntar sobre método/modo & {\hindifont आप कैसे हैं?} (Aap kaise hain? - ¿Cómo está Ud.?) \ \hline
Cuándo & {\hindifont कब} & kab & ¿Cuándo? & Preguntar sobre tiempo/momento & {\hindifont आप कब आए?} (Aap kab aaye? - ¿Cuándo llegó Ud.?) \ \hline
Por qué & {\hindifont क्यों} & kyon & ¿Por qué? & Preguntar por razón/causa & {\hindifont आप क्यों आये?} (Aap kyon aaye? - ¿Por qué vino Ud.?) \ \hline
Cuánto/a/os & {\hindifont कितना/कितनी/कितने} & kitnaa/kitni/kitne & ¿Cuánto/a/os? & Preguntar sobre cantidad & {\hindifont यह कितना है?} (Yah kitnaa hai? - ¿Cuánto cuesta esto?) \ \hline
\end{tabularx}
\end{center}


\begin{tcolorbox}[colback=yellow!10!white,colframe=orange!75!black,title=Regla de Oro]

\paragraph{Importante: Estructura de las Preguntas en Hindi} 

A diferencia del español, en hindi las palabras interrogativas \textbf{no cambian} según el género o número. La palabra {\hindifont क्या} (kya) sirve tanto para preguntas de "qué" como para preguntas de "si/no".


\begin{itemize}

  \item {\hindifont क्या आप ठीक हैं?} (Kya aap theek hain?) - ¿Está Ud. bien? (pregunta de sí/no)

  \item {\hindifont यह क्या है?} (Yah kya hai?) - ¿Qué es esto? (pregunta de qué)

\end{itemize}


\end{tcolorbox}

\subsubsection{3.1.2 Patrones Comunes de Preguntas}


Los patrones de preguntas en hindi varían según el tipo de información que se busca:

\paragraph{A. Preguntas de Identidad} 

\begin{itemize}

  \item {\hindifont आपका नाम क्या है?} (Aapkaa naam kya hai?) - ¿Cómo se llama Ud.?

  \item {\hindifont आप कौन हैं?} (Aap kaun hain?) - ¿Quién es Ud.?

  \item {\hindifont आप कौन से देश के हैं?} (Aap kaun se desh ke hain?) - ¿De qué país es Ud.?

\end{itemize}

\paragraph{B. Preguntas de Localización} 

\begin{itemize}

  \item {\hindifont शौचालय कहाँ है?} (Shauchaalay kahaan hai?) - ¿Dónde está el baño?

  \item {\hindifont यहाँ स्कूल कहाँ है?} (Yahaan skul kahaan hai?) - ¿Dónde está la escuela aquí?

  \item {\hindifont रेस्तरां कहाँ है?} (Restaraan kahaan hai?) - ¿Dónde está el restaurante?

\end{itemize}

\subsection{3.2 Estructura de la Negación Básica}


La negación en hindi se forma con la palabra {\hindifont नहीं} (nahin), que significa "no". La posición de {\hindifont नहीं} es \textbf{antes} del verbo principal en la oración.


\begin{tcolorbox}[colback=blue!5!white,colframe=blue!75!black,title=Nota/Clarificación]

\paragraph{Regla de Ubicación de नहीं} 

{\hindifont नहीं} se coloca \textbf{inmediatamente antes} del verbo principal que se niega:


\begin{itemize}

  \item {\hindifont मैं ठीक नहीं हूँ।} (Main theek nahin huun.) - Yo no estoy bien.

  \item {\hindifont वह घर नहीं है।} (Vah ghar nahin hai.) - Él/Ella no está en casa. (o: Ésa no es la casa.)

  \item {\hindifont हम भारतीय नहीं हैं।} (Ham bharatiya nahin hain.) - Nosotros no somos indios.

  \item {\hindifont क्या आप ठीक नहीं हैं?} (Kya aap theek nahin hain?) - ¿No está Ud. bien? (Negación en pregunta)

\end{itemize}


\end{tcolorbox}

\subsubsection{3.2.1 Negación con Verbos Compuestos}


Cuando una oración tiene un verbo compuesto o auxiliar, {\hindifont नहीं} se coloca antes del verbo auxiliar o el verbo principal:


\begin{itemize}

  \item {\hindifont मैं नहीं समझता।} (Main nahin samajhta.) - Yo no entiendo.

  \item {\hindifont मैं नहीं जा रहा हूँ।} (Main nahin ja raha huun.) - Yo no estoy yendo. (No voy ahora)

  \item {\hindifont मैं पढ़ रहा नहीं हूँ।} (Main padh raha nahin huun.) - Yo no estoy estudiando. (No estudio ahora)

\end{itemize}

\subsection{3.3 Vocabulario: Profesiones y Ocupaciones}


El vocabulario profesional es esencial para expresar identidad y ocupación. A continuación se presentan las profesiones más comunes:


\begin{center}
\begin{tabularx}{\textwidth}{| X | X | X | X | X |}
\hline
\textbf{Profesión} & \textbf{Hindi} & \textbf{Transliteración} & \textbf{Traducción} & \textbf{Comentarios} \ \hline
Profesor/a & {\hindifont शिक्षक}/{\hindifont शिक्षिका} & shikshak/shikshika & Profesor/Profesora & Forma masculina/femenina \ \hline
Médico/a & {\hindifont डॉक्टर} o {\hindifont चिकित्सक} & doctor / chikitsak & Doctor/Doctora & "Doctor" es más común; "chikitsak" es más formal \ \hline
Ingeniero/a & {\hindifont इंजीनियर} o {\hindifont अभियंता} & ingeeniyaar / abhiyantaa & Ingeniero/Ingeniera & "Ingeeniyaar" es más común; "abhiyantaa" es más formal \ \hline
Estudiante & {\hindifont छात्र}/{\hindifont छात्रा} & chatra/chatraa & Estudiante & Forma masculina/femenina \ \hline
Escritor/a & {\hindifont लेखक}/{\hindifont लेखिका} & lekhak/lekhikaa & Escritor/Escritora & Forma masculina/femenina \ \hline
Cocinero/a & {\hindifont रसोईया} o {\hindifont पक्का} & rasoiyaa / pakkaa & Cook/Cocinero & "Rasoiyaa" es la forma más común \ \hline
Empleado/a & {\hindifont कर्मचारी} & karmachaari & Employee/Empleado & Forma neutra para ambos géneros \ \hline
Empresario/a & {\hindifont व्यवसायी} & vyyavasaayii & Businessperson/Empresario & Forma neutra para ambos géneros \ \hline
Comerciante & {\hindifont व्यापारी} & vyyapaaree & Trader/Comerciante & Forma neutra para ambos géneros \ \hline
Artesano & {\hindifont कारीगर} & kaarigar & Artisan/Artesano & Forma neutra para ambos géneros \ \hline
\end{tabularx}
\end{center}

\subsubsection{3.3.1 Expresar Profesión en Oraciones}


Para expresar la profesión, se usa la estructura: [Sujeto] + [Profesión] + {\hindifont है}/{\hindifont हूँ}/{\hindifont हैं} + [extensión si es necesario]

\paragraph{Ejemplos de Declaraciones de Profesión} 

\begin{itemize}

  \item {\hindifont मैं शिक्षक हूँ।} (Main shikshak huun.) - Yo soy profesor(a).

  \item {\hindifont वह डॉक्टर है।} (Vah doctor hai.) - Él/Ella es médico(a).

  \item {\hindifont हम इंजीनियर हैं।} (Ham ingeeniyaar hain.) - Nosotros somos ingenieros.

  \item {\hindifont आप क्या करते हैं?} (Aap kya karte hain?) - ¿Qué hace Ud.? (¿Cuál es su profesión?)

  \item {\hindifont मैं एक छात्र हूँ।} (Main ek chatra huun.) - Soy un estudiante.

\end{itemize}

\subsection{3.4 Vocabulario: Números del 11 al 20}


Continuando con la numeración, aquí están los números del 11 al 20 en devanagari:


\begin{center}
\begin{tabularx}{\textwidth}{| X | X | X | X | X |}
\hline
\textbf{Número} & \textbf{Devanagari} & \textbf{Transliteración} & \textbf{Pronunciación} & \textbf{Notas Importantes} \ \hline
11 & {\hindifont ग्यारह} & gyaarah & GYAAR-ah & Forma irregular, no se forma de 10+1 \ \hline
12 & {\hindifont बारह} & baarah & BAR-ah & Forma irregular, no se forma de 10+2 \ \hline
13 & {\hindifont तेरह} & teraah & TER-ah & Forma irregular, no se forma de 10+3 \ \hline
14 & {\hindifont चौदह} & chaudaah & CHOU-daah & Forma irregular (no diez+catorce como en español) \ \hline
15 & {\hindifont पंद्रह} & pandraah & PAN-draah & Otra forma irregular \ \hline
16 & {\hindifont सोलह} & solah & SO-lah & Forma irregular \ \hline
17 & {\hindifont सत्रह} & satraah & SAT-raah & Forma irregular \ \hline
18 & {\hindifont अठारह} & athaarah & ATH-aarah & Forma irregular \ \hline
19 & {\hindifont उन्नीस} & unnais & UN-nais & Forma irregular \ \hline
20 & {\hindifont बीस} & bees & BEEs & Primera forma regular: 2+10 (du+dash en sánscrito) \ \hline
\end{tabularx}
\end{center}

\subsection{3.5 Vocabulario: Países, Nacionalidades y Lenguas}


Para expresar el origen y la nacionalidad, es importante conocer los nombres de países, las nacionalidades y los idiomas:

\subsubsection{3.5.1 Países Comunes}


\begin{center}
\begin{tabularx}{\textwidth}{| X | X | X | X |}
\hline
\textbf{País} & \textbf{Devanagari} & \textbf{Transliteración} & \textbf{Traducción} \ \hline
India & {\hindifont भारत} & bharat & India (nombre nativo) \ \hline
España & {\hindifont स्पेन} & span & Spain (forma prestada) \ \hline
China & {\hindifont चीन} & cheen & China \ \hline
Estados Unidos & {\hindifont अमेरिका} & amerikaa & América \ \hline
Francia & {\hindifont फ्रांस} & france & France \ \hline
Reino Unido & {\hindifont इंग्लैंड} & ingland & England \ \hline
Rusia & {\hindifont रूस} & rus & Russia \ \hline
Japón & {\hindifont जापान} & jaapaan & Japan \ \hline
Brasil & {\hindifont ब्राज़ील} & braazeel & Brazil \ \hline
México & {\hindifont मैक्सिको} & meksiiko & Mexico \ \hline
\end{tabularx}
\end{center}

\subsubsection{3.5.2 Nacionalidades}


\begin{center}
\begin{tabularx}{\textwidth}{| X | X | X | X | X |}
\hline
\textbf{Nacionalidad} & \textbf{Devanagari} & \textbf{Transliteración} & \textbf{Traducción} & \textbf{Forma Adjetival} \ \hline
Indio/a & {\hindifont भारतीय} & bharatiya & Indian & Sigue la concordancia de género (bharatiya/bharatiyaa) \ \hline
Español/a & {\hindifont स्पैनिश} & spanish & Spanish & Forma prestada, no cambia por género \ \hline
Chino/a & {\hindifont चीनी} & cheenii & Chinese & Sufijo -ii indica nacionalidad \ \hline
Estadounidense & {\hindifont अमेरिकी} & amerikii & American & Sufijo -ii indica nacionalidad \ \hline
Francés/a & {\hindifont फ्रांसीसी} & franseese & French & Forma adaptada al hindi \ \hline
Inglés/a & {\hindifont अंग्रेज़ी} & angrezii & English & Forma adaptada al hindi \ \hline
Ruso/a & {\hindifont रूसी} & rusee & Russian & Sufijo -ii indica nacionalidad \ \hline
Japonés/a & {\hindifont जापानी} & jaapaanii & Japanese & Sufijo -ii indica nacionalidad \ \hline
Brasileño/a & {\hindifont ब्राज़ीलियाई} & brazeeliyaaee & Brazilian & Forma compleja con sufijos \ \hline
\end{tabularx}
\end{center}

\subsubsection{3.5.3 Idiomas}


\begin{center}
\begin{tabularx}{\textwidth}{| X | X | X | X |}
\hline
\textbf{Idioma} & \textbf{Devanagari} & \textbf{Transliteración} & \textbf{Traducción} \ \hline
Hindi & {\hindifont हिंदी} & hindee & Hindi \ \hline
Inglés & {\hindifont अंग्रेज़ी} & angrejee & English \ \hline
Español & {\hindifont स्पेनिश} & spanish & Spanish \ \hline
Francés & {\hindifont फ्रेंच} & frenc & French \ \hline
Chino & {\hindifont चीनी} & cheenee & Chinese \ \hline
Árabe & {\hindifont अरबी} & arabee & Arabic \ \hline
Ruso & {\hindifont रूसी} & rusee & Russian \ \hline
\end{tabularx}
\end{center}

\subsection{3.6 Expresiones para Preguntar y Decir el Origen}


El tema del origen es fundamental para las presentaciones personales y las conversaciones sobre identidad:

\subsubsection{3.6.1 Frases para Preguntar el Origen}


\begin{itemize}

  \item {\hindifont आप कहाँ से हैं?} (Aap kahaan se hain?) - ¿De dónde es Ud.? (formal)

  \item {\hindifont तुम कहाँ से हो?} (Tum kahaan se ho?) - ¿De dónde eres? (informal)

  \item {\hindifont आप किस देश के हैं?} (Aap kis desh ke hain?) - ¿De qué país es Ud.? (formal)

  \item {\hindifont तुम किस शहर से हो?} (Tum kis shehar se ho?) - ¿De qué ciudad eres? (informal)

  \item {\hindifont यहाँ कहाँ के आए हैं?} (Yahan kahaan ke aaye hain?) - ¿De dónde han venido aquí? (formal/plural)

\end{itemize}

\subsubsection{3.6.2 Frases para Expresar el Origen}


\begin{itemize}

  \item {\hindifont मैं भारत से हूँ।} (Main bharat se huun.) - Soy de la India.

  \item {\hindifont मैं स्पेन से हूँ।} (Main span se huun.) - Soy de España.

  \item {\hindifont मैं भारतीय हूँ।} (Main bharatiya huun.) - Soy indio/a. (expresa nacionalidad)

  \item {\hindifont मैं मुंबई से हूँ।} (Main Mumbai se huun.) - Soy de Mumbai.

  \item {\hindifont हम चीन से हैं।} (Ham cheen se hain.) - Somos de China.

  \item {\hindifont वे अमेरिका के हैं।} (Ve amerikaa ke hain.) - Ellos son de Estados Unidos.

\end{itemize}


\begin{tcolorbox}[colback=blue!5!white,colframe=blue!75!black,title=Nota/Clarificación]

\paragraph{Distinción Importante: Origen vs. Nacionalidad} 

En hindi, hay una distinción entre el lugar de origen y la nacionalidad:


\begin{itemize}

  \item Usar "{\hindifont ... से हूँ}" ({\hindifont ... se huun}) para expresar de dónde \textbf{viene} una persona (origen geográfico).

  \item Usar "{\hindifont ... हूँ}" (con adjetivo de nacionalidad) para expresar la \textbf{nacionalidad} o ciudadanía.

\end{itemize}


Ejemplos:


\begin{itemize}

  \item Un indio que vive en España podría decir: {\hindifont मैं भारत से हूँ, लेकिन मैं भारतीय हूँ।} (Main bharat se huun, lekin main bharatiya huun.) - Soy de la India, pero soy indio.

\end{itemize}


\end{tcolorbox}

\subsubsection{3.6.3 Preguntas Comunes sobre Identidad y Origen}


En conversaciones reales, las preguntas suelen combinarse para conocer mejor a alguien:


\begin{itemize}

  \item {\hindifont आपका नाम क्या है और कहाँ से हैं?} (Aapkaa naam kya hai aur kahaan se hain?) - ¿Cómo se llama Ud. y de dónde es?

  \item {\hindifont आप कौन हैं और क्या करते हैं?} (Aap kaun hain aur kya karte hain?) - ¿Quién es Ud. y qué hace? (profesión)

  \item {\hindifont तुम यहाँ क्यों आए हो?} (Tum yahan kyon aaye ho?) - ¿Por qué has venido aquí?

  \item {\hindifont तुम्हें यहाँ कैसा लग रहा है?} (Tumhe yahan kaisaa lag raha hai?) - ¿Cómo te parece este lugar?

\end{itemize}

\subsection{3.7 Actividades Prácticas}


Para reforzar los contenidos de esta unidad, realiza las siguientes actividades:

\subsubsection{3.7.1 Ejercicios de Traducción}


\begin{enumerate}

  \item Traduce al hindi:
            
\begin{itemize}

  \item ¿De dónde eres?

  \item Yo soy profesor.

  \item ¿Cuál es tu nombre?

  \item No entiendo.

  \item Soy de España.

\end{itemize}



  \item Traduce del hindi:
            
\begin{itemize}

  \item {\hindifont तुम कौन हो?}

  \item {\hindifont मैं स्पेनिश हूँ।}

  \item {\hindifont यह क्या है?}

  \item {\hindifont कृपया मदद करो।}

  \item {\hindifont मैं ठीक नहीं हूँ।}

\end{itemize}



\end{enumerate}

\subsubsection{3.7.2 Conversaciones de Práctica}


Realiza diálogos con estas estructuras:


\begin{itemize}

  \item Presentación personal (nombre, origen, profesión)

  \item Pregunta y responde sobre nacionalidad

  \item Pregunta y responder sobre idiomas que se hablan

  \item Descripción de familia y relaciones cercanas

\end{itemize}


\hrulefill

\section{Unidad 4: Adjetivos, Concordancia y Presente Habitual}


Objetivo: Describir personas y objetos usando adjetivos, comprender la concordancia de género y número, y expresar hábitos y rutinas con el presente habitual

\subsection{4.1 Género y Número: Conceptos Fundamentales}


En hindi, todos los sustantivos tienen género (masculino o femenino), y no existe el género neutro. Esta característica es fundamental y afecta a los adjetivos y, como veremos más adelante, a los verbos.

\subsubsection{4.1.1 Determinantes del Género de los Sustantivos}


En general, el género de los sustantivos en hindi se determina por diferentes factores:


\begin{itemize}

  \item \textbf{Sexo biológico:} En el caso de seres animados, el género suele corresponder al sexo biológico.
            
\begin{itemize}

  \item Humano masculino: {\hindifont लड़का} (laRkaa - chico), {\hindifont पुरुष} (puruṣ - hombre)

  \item Humano femenino: {\hindifont लड़की} (laRkii - chica), {\hindifont महिला} (mahilaa - mujer)

\end{itemize}



  \item \textbf{Patrón de terminación:} Muchos sustantivos siguen patrones regulares de género:
            
\begin{itemize}

  \item \textbf{Masculino:} Sustantivos que terminan en {\hindifont -आ} (-aa) suelen ser masculinos: {\hindifont लड़का} (laRkaa - chico), {\hindifont कमरा} (kamraa - habitación)

  \item \textbf{Femenino:} Sustantivos que terminan en {\hindifont -ई} ({\hindifont -इ} en hindi, pero se pronuncia como 'ii') suelen ser femeninos: {\hindifont लड़की} (laRkii - chica), {\hindifont किताब} (kitaab - libro)

\end{itemize}



  \item \textbf{Conceptos abstractos:} Muchos conceptos abstractos tienen un género fijo, pero no hay regla universal.

\end{itemize}


\begin{tcolorbox}[colback=yellow!10!white,colframe=orange!75!black,title=Regla de Oro]

\paragraph{Regla de Oro: Concordancia de Género} 

En hindi, los adjetivos \textbf{siempre} deben concordar con el sustantivo al que modifican en \textbf{género} y \textbf{número}. Esta es una de las reglas gramaticales más importantes en hindi.


\end{tcolorbox}

\subsubsection{4.1.2 Concordancia de Adjetivos con Sustantivos}


Veamos cómo cambian los adjetivos según el género y número del sustantivo que describen:


\begin{center}
\begin{tabularx}{\textwidth}{| X | X | X | X | X |}
\hline
\textbf{Sustantivo} & \textbf{Género} & \textbf{Adjetivo "bueno/a"} & \textbf{Oración Completa} & \textbf{Traducción} \ \hline
{\hindifont लड़का} (chico) & Masculino Singular & {\hindifont अच्छा} (achchhaa) & {\hindifont अच्छा लड़का} & Un buen chico \ \hline
{\hindifont लड़की} (chica) & Femenino Singular & {\hindifont अच्छी} (achchhii) & {\hindifont अच्छी लड़की} & Una buena chica \ \hline
{\hindifont लड़के} (chicos) & Masculino Plural & {\hindifont अच्छे} (achchhe) & {\hindifont अच्छे लड़के} & Buenos chicos \ \hline
{\hindifont लड़कियाँ} (chicas) & Femenino Plural & {\hindifont अच्छी} (achchhii) & {\hindifont अच्छी लड़कियाँ} & Buenas chicas \ \hline
\end{tabularx}
\end{center}

\subsubsection{4.1.3 Patrones Regulares de Formación de Adjetivos}


Los adjetivos que terminan en {\hindifont -आ} (-aa) en masculino siguen un patrón de cambio regular:


\begin{center}
\begin{tabularx}{\textwidth}{| X | X | X | X |}
\hline
\textbf{Género y Número} & \textbf{Cambio de Final} & \textbf{Ejemplo con "bueno/a" ({\hindifont अच्छा})} & \textbf{Otro ejemplo con "grande" ({\hindifont बड़ा})} \ \hline
Masculino Singular & Permanece como -aa & {\hindifont अच्छा} (achchhaa) & {\hindifont बड़ा} (badaa) \ \hline
Femenino Singular & -aa cambia a -ii & {\hindifont अच्छी} (achchhii) & {\hindifont बड़ी} (badii) \ \hline
Masculino Plural & -aa cambia a -e & {\hindifont अच्छे} (achchhe) & {\hindifont बड़े} (bade) \ \hline
Femenino Plural & Igual que fem. sing. (-ii) & {\hindifont अच्छी} (achchhii) & {\hindifont बड़ी} (badii) \ \hline
\end{tabularx}
\end{center}


\begin{tcolorbox}[colback=blue!5!white,colframe=blue!75!black,title=Nota/Clarificación]

\paragraph{Advertencia Importante sobre los Adjetivos} 

A diferencia del español, en hindi el adjetivo \textbf{siempre} precede al sustantivo que modifica, y la concordancia es obligatoria. No se puede decir "{\hindifont लड़का अच्छा}" para significar "chico bueno". La estructura correcta es "{\hindifont अच्छा लड़का}".


\end{tcolorbox}

\subsection{4.2 El Presente Habitual (Verbo + ता/ती/ते + होना)}


El presente habitual (también llamado presente indefinido o presente simple) se utiliza para expresar acciones habituales o verdades generales. Se forma agregando la partícula {\hindifont ता/ती/ते} (taa/tii/te) a la raíz del verbo, seguida del verbo auxiliar {\hindifont होना} (honaa) en la forma adecuada.

\subsubsection{4.2.1 Estructura del Presente Habitual}


La estructura básica es:


\textbf{Sujeto + Raíz del verbo + Partícula (ता/ती/ते) + Verbo auxiliar होना conjugado}


\begin{tcolorbox}[colback=blue!5!white,colframe=blue!75!black,title=Nota/Clarificación]

\paragraph{¿Por qué esta forma tan particular?} 

El presente habitual en hindi no es simplemente una conjugación de verbo, sino que es una construcción gramatical que indica una acción habitual. La partícula {\hindifont ता/ती/ते} acompaña al sujeto y no al verbo, y concuerda con él en género y número, no con el verbo auxiliar.


\end{tcolorbox}

\subsubsection{4.2.2 Formación de la Raíz del Verbo}


La raíz del verbo se forma quitando la terminación {\hindifont -ना} (-naa) de los infinitivos:


\begin{itemize}

  \item {\hindifont करना} (karnaa - hacer) → Raíz: {\hindifont कर} (kar)

  \item {\hindifont जाना} (jaanaa - ir) → Raíz: {\hindifont जा} (jaa)

  \item {\hindifont पढ़ना} (paRhnnaa - leer/estudiar) → Raíz: {\hindifont पढ़} (paRh)

  \item {\hindifont खाना} (khaanaa - comer) → Raíz: {\hindifont खा} (khaa)

  \item {\hindifont पीना} (piinaa - beber) → Raíz: {\hindifont पी} (pii)

\end{itemize}

\subsubsection{4.2.3 Las Partículas ता/ती/ते}


La elección de la partícula depende del género y número del sujeto:


\begin{itemize}

  \item {\hindifont ता} (taa) - Usado con sujetos masculinos singulares

  \item {\hindifont ती} (tii) - Usado con sujetos femeninos singulares

  \item {\hindifont ते} (te) - Usado con sujetos masculinos plurales o con {\hindifont हम}, {\hindifont आप}, {\hindifont ये}, {\hindifont वे}

\end{itemize}

\subsubsection{4.2.4 Tabla Completa del Presente Habitual}


Mostrando el verbo {\hindifont पढ़ना} (paRhnnaa - leer/estudiar) como ejemplo:


\begin{center}
\begin{tabularx}{\textwidth}{| X | X | X | X | X | X | X |}
\hline
\textbf{Sujeto} & \textbf{Pronombre} & \textbf{Partícula} & \textbf{Forma del Verbo} & \textbf{Transliteración} & \textbf{Ejemplo Completo} & \textbf{Traducción} \ \hline
1ra Pers. Sg. & {\hindifont मैं} & ता/ती & {\hindifont पढ़ता/पढ़ती हूँ} & paRhta/paRhtii huun & {\hindifont मैं पढ़ता हूँ।} & Yo leo/estudio. (masc.) \ \hline
2da Pers. Familiar & {\hindifont तू} & ता/ती & {\hindifont पढ़ता/पढ़ती है} & paRhta/paRhtii hai & {\hindifont तू पढ़ता है।} & Tú lees/estudias. (masc.) \ \hline
2da Pers. Informal & {\hindifont तुम} & ता/ती & {\hindifont पढ़ता/पढ़ती हो} & paRhta/paRhtii ho & {\hindifont तुम पढ़ते हो।} & Tú lees/estudias. (masc.) \ \hline
2da Pers. Formal & {\hindifont आप} & ते & {\hindifont पढ़ते हैं} & paRhte hain & {\hindifont आप पढ़ते हैं।} & Ud. lee/estudia. \ \hline
3ra Pers. Sg. Cercano & {\hindifont यह} & ता/ती & {\hindifont पढ़ता/पढ़ती है} & paRhta/paRhtii hai & {\hindifont यह पढ़ता है।} & Él/Ella lee/estudia. (masc.) \ \hline
3ra Pers. Sg. Lejano & {\hindifont वह} & ता/ती & {\hindifont पढ़ता/पढ़ती है} & paRhta/paRhtii hai & {\hindifont वह पढ़ता है।} & Él/Ella lee/estudia. (masc.) \ \hline
1ra Pers. Pl. & {\hindifont हम} & ते & {\hindifont पढ़ते हैं} & paRhte hain & {\hindifont हम पढ़ते हैं।} & Nosotros leemos/estudiamos. \ \hline
3ra Pers. Pl. Cercano & {\hindifont ये} & ते & {\hindifont पढ़ते हैं} & paRhte hain & {\hindifont ये पढ़ते हैं।} & Ellos/Ellas leen/estudian. \ \hline
3ra Pers. Pl. Lejano & {\hindifont वे} & ते & {\hindifont पढ़ते हैं} & paRhte hain & {\hindifont वे पढ़ते हैं।} & Ellos/Ellas leen/estudian. \ \hline
\end{tabularx}
\end{center}

\subsubsection{4.2.5 Reglas Importantes del Presente Habitual}


\begin{itemize}

  \item \textbf{Concordancia del sujeto:} La partícula {\hindifont ता/ती/ते} concuerda con el sujeto de la oración, no con el objeto.

  \item \textbf{Regla de género:} El adjetivo o partícula toma la forma según el género del sujeto, no del objeto.

  \item \textbf{Orden S-O-V:} Sigue la estructura típica del hindi: sujeto-objeto-verbo.

  \item \textbf{Forma verbal auxiliar:} El verbo auxiliar {\hindifont होना} concuerda con el pronombre personal.

\end{itemize}

\subsection{4.3 Uso de la Conjunción या (ya)}


La palabra {\hindifont या} (ya) significa "o" en español y se usa para presentar alternativas:

\subsubsection{4.3.1 Ejemplos de Uso de या}


\begin{itemize}

  \item {\hindifont राम या श्याम} (Ram ya Shayam) - Ram o Shayam

  \item {\hindifont चाय या कॉफी} (Chaay ya kaapi) - Té o café

  \item {\hindifont तुम कहाँ जा रहे हो? घर या कार्यालय?} (Tum kahaan ja rahe ho? Ghar ya karyaalya?) - ¿A dónde vas? ¿A casa o a la oficina?

  \item {\hindifont क्या तुम खाना या नाश्ता चाहते हो?} (Kya tum khaanaa ya naashTaa chaahate ho?) - ¿Quieres comida o refrigerio?

\end{itemize}

\subsubsection{4.3.2 Diferencia con और (aur)}


Aunque ambas son conjunciones, {\hindifont या} (ya) y {\hindifont और} (aur) tienen usos diferentes:


\begin{itemize}

  \item {\hindifont या} (ya) - Indica alternativas excluyentes (A o B, pero no ambos)

  \item {\hindifont और} (aur) - Indica adición o inclusión (A y B, ambos)

\end{itemize}


Ejemplos:
        
\begin{itemize}

  \item {\hindifont चाय या कॉफी पी सकते हो।} (Chaay ya kaafee pee sakte ho.) - Puedes beber té o café (una opción).

  \item {\hindifont चाय और कॉफी दोनों मुझे पसंद हैं।} (Chaay aur kaafee dono mujhe pasand hain.) - Me gustan tanto el té como el café.

\end{itemize}



\subsection{4.4 Vocabulario: Colores Adjetivos y Adjetivos Comunes}


Los adjetivos son fundamentales para describir personas, objetos y estados. Veamos algunos adjetivos comunes y cómo cambian según el género y número:


\begin{center}
\begin{tabularx}{\textwidth}{| X | X | X | X | X | X |}
\hline
\textbf{Español} & \textbf{Hindi (Masc. Sg.)} & \textbf{Hindi (Fem. Sg.)} & \textbf{Hindi (Masc. Pl.)} & \textbf{Transliteración} & \textbf{Uso} \ \hline
Bueno/a & {\hindifont अच्छा} & {\hindifont अच्छी} & {\hindifont अच्छे} & achchhaa/achchhii/achchhe & Descripción general positiva \ \hline
Malo/a & {\hindifont बुरा} & {\hindifont बुरी} & {\hindifont बुरे} & buraa/burii/bure & Descripción general negativa \ \hline
Grande & {\hindifont बड़ा} & {\hindifont बड़ी} & {\hindifont बड़े} & badaa/badii/bade & Tamaño físico o importancia \ \hline
Pequeño/a & {\hindifont छोटा} & {\hindifont छोटी} & {\hindifont छोटे} & chhoTaa/chhoTii/chhoTe & Tamaño físico reducido \ \hline
Nuevo/a & {\hindifont नया} & {\hindifont नई} & {\hindifont नए} & nayaa/naii/naye & Lo reciente o fabricado \ \hline
Viejo/a & {\hindifont पुराना} & {\hindifont पुरानी} & {\hindifont पुराने} & puranaa/purani/purane & Lo anterior o antiguo \ \hline
Alto/a & {\hindifont लंबा} & {\hindifont लंबी} & {\hindifont लंबे} & lambaa/lambii/lambe & Longitud o altura \ \hline
Corto/a & {\hindifont छोटा} & {\hindifont छोटी} & {\hindifont छोटे} & chhoTaa/chhoTii/chhoTe & Longitud o duración reducida \ \hline
Gordo/a & {\hindifont मोटा} & {\hindifont मोटी} & {\hindifont मोटे} & motaa/motii/mote & Con más peso o volumen \ \hline
Delgado/a & {\hindifont पतला} & {\hindifont पतली} & {\hindifont पतले} & patlaa/patlii/patle & Con menos peso o volumen \ \hline
\end{tabularx}
\end{center}

\subsubsection{4.4.1 Colores como Adjetivos}


Los colores en hindi funcionan como adjetivos y también deben concordar con el sustantivo en género y número:


\begin{center}
\begin{tabularx}{\textwidth}{| X | X | X | X | X | X |}
\hline
\textbf{Color} & \textbf{Masculino Sing.} & \textbf{Femenino Sing.} & \textbf{Masculino Plural} & \textbf{Transliteración} & \textbf{Ejemplo con libro ({\hindifont पुस्तक} - fem.)} \ \hline
Rojo & {\hindifont लाल} & {\hindifont लाल} & {\hindifont लाल} & laal & {\hindifont लाल पुस्तक} - Libro rojo \ \hline
Azul & {\hindifont नीला} & {\hindifont नीली} & {\hindifont नीले} & niilaa/niilii/niile & {\hindifont नीली पुस्तक} - Libro azul \ \hline
Verde & {\hindifont हरा} & {\hindifont हरी} & {\hindifont हरे} & haraa/harii/hare & {\hindifont हरी पुस्तक} - Libro verde \ \hline
Amarillo & {\hindifont पीला} & {\hindifont पीली} & {\hindifont पीले} & piilaa/piilii/piile & {\hindifont पीली पुस्तक} - Libro amarillo \ \hline
Blanco & {\hindifont सफेद} & {\hindifont सफेद} & {\hindifont सफेद} & safed & {\hindifont सफेद पुस्तक} - Libro blanco \ \hline
Negro & {\hindifont काला} & {\hindifont काली} & {\hindifont काले} & kaalaa/kaalii/kaale & {\hindifont काली पुस्तक} - Libro negro \ \hline
Marrón & {\hindifont भूरा} & {\hindifont भूरी} & {\hindifont भूरे} & bhooraa/bhoorii/bhoore & {\hindifont भूरी पुस्तक} - Libro marrón \ \hline
Rosa & {\hindifont गुलाबी} & {\hindifont गुलाबी} & {\hindifont गुलाबी} & gulaabee & {\hindifont गुलाबी पुस्तक} - Libro rosa \ \hline
\end{tabularx}
\end{center}


\begin{tcolorbox}[colback=blue!5!white,colframe=blue!75!black,title=Nota/Clarificación]

\paragraph{Excepciones en los Colores} 

Algunos colores como {\hindifont लाल} (rojo), {\hindifont सफेद} (blanco), y {\hindifont गुलाबी} (rosa) son invariables y no cambian según el género y número del sustantivo al que modifican.


\end{tcolorbox}

\subsubsection{4.4.2 Vocabulario Ampliado: Adjetivos y Opuestos}


Pares de adjetivos útiles para descripciones más precisas:


\begin{center}
\begin{tabularx}{\textwidth}{| X | X | X | X |}
\hline
\textbf{Adjetivo} & \textbf{Hindi} & \textbf{Opuesto} & \textbf{Hindi (Opuesto)} \ \hline
Fácil & {\hindifont आसान} (aasaan) & Difícil & {\hindifont मुश्किल} (mushkil) \ \hline
Barato & {\hindifont सस्ता} (sastaa) & Caro & {\hindifont महंगा} (mahangaa) \ \hline
Limpio & {\hindifont साफ़} (saaf) & Sucio & {\hindifont गंदा} (gandaa) \ \hline
Abierto & {\hindifont खुला} (khulaa) & Cerrado & {\hindifont बंद} (band) \ \hline
Nuevo & {\hindifont नया} (nayaa) & Viejo (cosas) & {\hindifont पुराना} (puraanaa) \ \hline
Joven & {\hindifont जवान} (javaan) & Viejo (personas) & {\hindifont बूढ़ा} (boodhaa) \ \hline
Feliz & {\hindifont खुश} (khush) & Triste & {\hindifont दुखी} (dukhi) \ \hline
Vivo & {\hindifont ज़िंदा} (zindaa) & Muerto & {\hindifont मरा हुआ} (maraa hua) \ \hline
\end{tabularx}
\end{center}

\subsection{4.5 Vocabulario Adicional: Días de la Semana y Números 21-50}

\subsubsection{4.5.1 Días de la Semana}


\begin{center}
\begin{tabularx}{\textwidth}{| X | X | X | X | X |}
\hline
\textbf{Día} & \textbf{Devanagari} & \textbf{Transliteración} & \textbf{Literalmente} & \textbf{Traducción} \ \hline
Lunes & {\hindifont सोमवार} & somvaar & Día de la luna & Lunes \ \hline
Martes & {\hindifont मंगलवार} & mangalvaar & Día de Marte & Martes \ \hline
Miércoles & {\hindifont बुधवार} & budhvaar & Día de Mercurio & Miércoles \ \hline
Jueves & {\hindifont गुरुवार} & guruvaar & Día del maestro & Jueves \ \hline
Viernes & {\hindifont शुक्रवार} & shukravaar & Día de Venus & Viernes \ \hline
Sábado & {\hindifont शनिवार} & shanivaar & Día de Saturno & Sábado \ \hline
Domingo & {\hindifont रविवार} & ravivaar & Día del sol & Domingo \ \hline
\end{tabularx}
\end{center}

\subsubsection{4.5.2 Números del 21 al 50}


\begin{center}
\begin{tabularx}{\textwidth}{| X | X | X | X | X |}
\hline
\textbf{Número} & \textbf{Devanagari} & \textbf{Transliteración} & \textbf{Pronunciación} & \textbf{Patrón} \ \hline
21 & {\hindifont इक्कीस} & ikkis & EK-ees (irregular) & Forma irregular \ \hline
22 & {\hindifont बाईस} & baais & BAI-eese & Forma irregular \ \hline
23 & {\hindifont तेईस} & teis & TEE-eese & Forma irregular \ \hline
24 & {\hindifont चौबीस} & chaubis & CHOW-beese & Forma irregular \ \hline
25 & {\hindifont पच्चीस} & paccis & PAC-cheese & Forma irregular \ \hline
26 & {\hindifont छब्बीस} & chhabbis & CHHAB-beese & Forma irregular \ \hline
27 & {\hindifont सत्ताईस} & sattais & SAT-tai-se & Forma irregular \ \hline
28 & {\hindifont अठाईस} & aththais & ATH-tai-se & Forma irregular \ \hline
29 & {\hindifont उनतीस} & unatis & OO-na-ti-se & Literalmente "treinta menos uno" \ \hline
30 & {\hindifont तीस} & tis & TISE & Base para los treintas \ \hline
31 & {\hindifont इकतीस} & ikatis & EE-ka-ti-se & Literalmente "uno de los treinta" \ \hline
32 & {\hindifont बत्तीस} & batthis & BAT-ti-se & Literalmente "dos de los treinta" \ \hline
35 & {\hindifont पैंतीस} & paintis & PAIN-ti-se & Literalmente "cinco de los treinta" \ \hline
40 & {\hindifont चालीस} & chaalis & CHA-lee-se & Base para los cuarenta \ \hline
45 & {\hindifont पैंतालीस} & paintaalees & PAIN-taa-lee-se & Literalmente "cinco de los cuarenta" \ \hline
50 & {\hindifont पचास} & pachaas & PA-chaa-se & Base para los cincuenta \ \hline
\end{tabularx}
\end{center}

\subsection{4.6 Comunicación: Descripción de Personas, Objetos y Rutinas}


Con los conceptos aprendidos, podemos formar frases para describir a personas y objetos:

\subsubsection{4.6.1 Frases Ejemplares con Adjetivos}


\begin{itemize}

  \item {\hindifont यह अच्छा लड़का है।} (Yah achchhaa laRkaa hai.) - Éste es un buen chico.

  \item {\hindifont वह छोटी किताब है।} (Vah chhoTii kitaab hai.) - Ésa es una pequeña libro.

  \item {\hindifont हम बड़े घर में रहते हैं।} (Ham bade ghar men rahte hain.) - Vivimos en una casa grande.

  \item {\hindifont ये लाल कमीज हैं।} (Ye laal kamiz hain.) - Éstas son camisas rojas.

\end{itemize}

\subsubsection{4.6.2 Expresión de Hábitos y Rutinas Diarias}


Usando el presente habitual, podemos expresar hábitos y rutinas:


\begin{itemize}

  \item {\hindifont मैं सुबह छह बजे उठता हूँ।} (Main subah chheh baje uthata huun.) - Yo me levanto a las seis de la mañana. (masc.)

  \item {\hindifont वह रोज़ पढ़ाई करता है।} (Vah roz padhaaee karata hai.) - Él estudia todos los días. (masc.)

  \item {\hindifont हम शाम को टहलने जाते हैं।} (Ham shaam ko tahalne jaate hain.) - Salimos a caminar por la tarde.

  \item {\hindifont तुम कलेजा कैसे जाते हो?} (Tum kalejaa kaise jaate ho?) - ¿Cómo vas al colegio/universidad?

\end{itemize}


\begin{tcolorbox}[colback=blue!5!white,colframe=blue!75!black,title=Nota/Clarificación]

\paragraph{Notas sobre las Rutinas Diarias} 

Para expresar rutinas diarias, se pueden usar expresiones temporales como:


\begin{itemize}

  \item {\hindifont रोज़} (roz) - Todos los días

  \item {\hindifont हर रोज़} (har roz) - Cada día

  \item {\hindifont प्रतिदिन} (pratidin) - Diariamente

  \item {\hindifont कभी-कभी} (kabhi-kabhi) - A veces

  \item {\hindifont अक्सर} (aksar) - A menudo

  \item {\hindifont कभी नहीं} (kabhi nahin) - Nunca

\end{itemize}


\end{tcolorbox}

\subsubsection{4.6.3 Ejemplos de Conversación}


Diálogo sobre hábitos diarios:


\begin{itemize}

  \item राम: तुम सुबह क्या करते हो? (Ram: Tum subah kya karte ho?) - Ram: ¿Qué haces por la mañana?

  \item {\hindifont श्याम: मैं सुबह जल्दी उठता हूँ और व्यायाम करता हूँ। (Shyam: Main subah jaldi uthata huun aur vyaayaam karata huun.) - Shyam: Yo me levanto temprano por la mañana y hago ejercicio.}

  \item {\hindifont राम: तुम क्या पढ़ते हो? (Ram: Tum kya padhte ho?) - Ram: ¿Qué lees?}

  \item {\hindifont श्याम: मैं अख़बार और कहानियाँ पढ़ता हूँ। (Shyam: Main akhbaar aur kahaaniyan padhata huun.) - Shyam: Leo periódicos y cuentos.}

\end{itemize}

\subsection{4.7 Matices Verbales: Verbos de Habla}


En hindi hay varios verbos para "hablar" o "decir" que no son intercambiables. Es crucial entender sus diferencias:


\begin{center}
\begin{tabularx}{\textwidth}{| X | X | X | X |}
\hline
\textbf{Verbo} & \textbf{Significado Principal} & \textbf{Uso y Gramática} & \textbf{Ejemplo} \ \hline
{\hindifont बोलना} (bolnaa) & Hablar, emitir sonidos, saber un idioma & Intransitivo (generalmente). Se usa para la capacidad de hablar o el acto físico. & {\hindifont वह हिंदी बोलता है।}\newline (Él habla hindi.) \ \hline
{\hindifont बात करना} (baat karnaa) & Conversar, charlar & Implica interacción con otra persona. Rige la postposición {\hindifont से} (con). & {\hindifont मैं राम से बात कर रहा हूँ।}\newline (Estoy conversando con Ram.) \ \hline
{\hindifont कहना} (kahnaa) & Decir & Transitivo. Se usa para transmitir un mensaje específico. Rige {\hindifont से} (a alguien). & {\hindifont उसने मुझसे कहा।}\newline (Él me dijo.) \ \hline
{\hindifont बताना} (bataanaa) & Contar, informar, narrar & Transitivo. Implica dar información o explicar. Rige {\hindifont को} (a alguien). & {\hindifont मुझे अपना नाम बताओ।}\newline (Dime/Cuéntame tu nombre.) \ \hline
\end{tabularx}
\end{center}


\hrulefill

\section{Unidad 5: Posesivos, Demostrativos y Objetos del Aula}


Objetivo: Expresar posesión con pronombres posesivos, usar demostrativos y vocabulario de objetos escolares

\subsection{5.1 Pronombres Posesivos en Hindi}


Los pronombres posesivos en hindi funcionan como adjetivos y concuerdan en género y número con el \textbf{objeto poseído}, no con el poseedor como en español. Esta es una característica fundamental que diferencia el hindi del español.

\subsubsection{5.1.1 Sistema de Posesivos en Hindi}


Los posesivos en hindi tienen diferentes formas según el género y número del objeto poseído:


\begin{center}
\begin{tabularx}{\textwidth}{| X | X | X | X | X | X |}
\hline
\textbf{Forma Básica} & \textbf{Forma Masculina Sg.} & \textbf{Forma Femenina Sg.} & \textbf{Forma Oblicua/Masculina Pl.} & \textbf{Traducción} & \textbf{Comentario} \ \hline
Base: -रा (-ra) & {\hindifont -रा} (-raa) & {\hindifont -री} (-rii) & {\hindifont -रे} (-re) & Forma base que se adapta al objeto poseído & Este patrón se combina con pronombres \ \hline
Base: -का (-ka) & {\hindifont -का} (-kaa) & {\hindifont -की} (-kii) & {\hindifont -के} (-ke) & Forma base que se adapta al objeto poseído & Forma más común que -ra/-ri/-re \ \hline
\end{tabularx}
\end{center}

\subsubsection{5.1.2 Tabla de Pronombres Posesivos}


\begin{center}
\begin{tabularx}{\textwidth}{| X | X | X | X | X | X |}
\hline
\textbf{Pronombre Personal} & \textbf{Posesivo Masc. Sing.} & \textbf{Posesivo Fem. Sing.} & \textbf{Posesivo Masc. Plural} & \textbf{Ejemplo con Objeto} & \textbf{Traducción} \ \hline
{\hindifont मैं} (maiṁ) - Yo & {\hindifont मेरा} (meraa) & {\hindifont मेरी} (merii) & {\hindifont मेरे} (mere) & {\hindifont मेरा कमरा} (meraa kamraa) / {\hindifont मेरी किताब} (merii kitaab) & Mi cuarto / Mi libro \ \hline
{\hindifont तू} (tū) - Tú (íntimo) & {\hindifont तेरा} (teraa) & {\hindifont तेरी} (terii) & {\hindifont तेरे} (tere) & {\hindifont तेरा घर} (teraa ghar) / {\hindifont तेरी कार} (terii kaar) & Tu casa (tú íntimo) / Tu coche (tú íntimo) \ \hline
{\hindifont तुम} (tum) - Tú (informal) & {\hindifont तुम्हारा} (tumhaaraa) & {\hindifont तुम्हारी} (tumhaarii) & {\hindifont तुम्हारे} (tumhaare) & {\hindifont तुम्हारा भाई} (tumhaaraa bhaaii) / {\hindifont तुम्हारी बहन} (tumhaarii bahan) & Tu hermano / Tu hermana \ \hline
{\hindifont आप} (aap) - Usted (formal) & {\hindifont आपका} (aapkaa) & {\hindifont आपकी} (aapki) & {\hindifont आपके} (aapke) & {\hindifont आपका नाम} (aapkaa naam) / {\hindifont आपकी किताब} (aapki kitaab) & Su nombre / Su libro \ \hline
{\hindifont हम} (ham) - Nosotros & {\hindifont हमारा} (hamaaraa) & {\hindifont हमारी} (hamaarii) & {\hindifont हमारे} (hamaare) & {\hindifont हमारा देश} (hamaaraa desh) / {\hindifont हमारी माता} (hamaarii maataa) & Nuestro país / Nuestra madre \ \hline
{\hindifont वह} (vah) - Él/Ella (lejano) & {\hindifont उसका} (uskaa) & {\hindifont उसकी} (uskii) & {\hindifont उसके} (uske) & {\hindifont उसका कुत्ता} (uskaa kutttaa) / {\hindifont उसकी कार} (uskii kaar) & Su perro / Su coche \ \hline
{\hindifont यह} (yah) - Él/Ella (cercano) & {\hindifont इसका} (iskaa) & {\hindifont इसकी} (iskii) & {\hindifont इसके} (iske) & {\hindifont इसका नाम} (iskaa naam) / {\hindifont इसकी किताब} (iskii kitaab) & Su nombre (de éste) / Su libro (de éste) \ \hline
{\hindifont वे} (ve) - Ellos/Ellas (lejanos) & {\hindifont उनका} (unkaa) & {\hindifont उनकी} (unkii) & {\hindifont उनके} (unke) & {\hindifont उनका घर} (unkaa ghar) / {\hindifont उनकी किताबें} (unkii kitaaben) & Su casa (de ellos) / Sus libros (de ellos) \ \hline
{\hindifont ये} (ye) - Ellos/Ellas (cercanos) & {\hindifont इनका} (inkaa) & {\hindifont इनकी} (inkii) & {\hindifont इनके} (inke) & {\hindifont इनका काम} (inkaa kaam) / {\hindifont इनकी गाड़ी} (inki gaaRii) & Su trabajo (de éstos) / Su coche (de éstos) \ \hline
\end{tabularx}
\end{center}


\begin{tcolorbox}[colback=blue!5!white,colframe=blue!75!black,title=Nota/Clarificación]

\paragraph{Concepto Fundamental: Concordancia con el Objeto Poseído} 

A diferencia del español donde los posesivos concuerdan con el poseedor (mi \textit{casa} = mi casa femenina, mi \textit{libro} = mi libro masculino), en hindi los posesivos concuerdan con el \textbf{objeto poseído}.


Por ejemplo:


\begin{itemize}

  \item {\hindifont तुम्हारा नाम} (tumhaaraa naam) - Tu nombre (nombre es masculino)

  \item {\hindifont तुम्हारी किताब} (tumhaarii kitaab) - Tu libro (libro es femenino)

  \item {\hindifont तुम्हारे दोस्त} (tumhaare dost) - Tus amigos (amigos es masculino plural)

\end{itemize}


\end{tcolorbox}

\subsection{5.2 Demostrativos en Hindi}


Los demostrativos indican la proximidad o lejanía del objeto al que se refieren. Hindi distingue entre objetos próximos y distantes al hablante.

\subsubsection{5.2.1 Tipos de Demostrativos}


\begin{center}
\begin{tabularx}{\textwidth}{| X | X | X | X | X | X | X | X |}
\hline
\textbf{Forma} & \textbf{Devanagari} & \textbf{Transliteración} & \textbf{Singular Masc.} & \textbf{Singular Fem.} & \textbf{Plural Masc.} & \textbf{Plural Fem.} & \textbf{Significado} \ \hline
Próximo & {\hindifont यह} & yah & {\hindifont यह} & {\hindifont यही} & {\hindifont ये} & {\hindifont ये} & Este/Esta/Estos/Estas (cercano al hablante) \ \hline
Distante & {\hindifont वह} & vah & {\hindifont वह} & {\hindifont वही} & {\hindifont वे} & {\hindifont वे} & Ese/Esa/Esos/Esas (lejano del hablante) \ \hline
\end{tabularx}
\end{center}


Notas importantes sobre los demostrativos:


\begin{itemize}

  \item El género femenino singular para {\hindifont यह} es {\hindifont यही} y para {\hindifont वह} es {\hindifont वही}

  \item En plural, {\hindifont ये} y {\hindifont वे} no cambian según el género

  \item Los demostrativos pueden funcionar como pronombres o adjetivos

\end{itemize}

\subsubsection{5.2.2 Uso de los Demostrativos}


\begin{itemize}

  \item Como \textbf{adjetivos}: {\hindifont यह किताब अच्छी है।} (Yah kitaab achchhii hai.) - Este libro es bueno/a.

  \item Como \textbf{pronombres}: {\hindifont यह मेरी है।} (Yah merii hai.) - Esta es mía.

  \item Para \textbf{presentar personas u objetos}: {\hindifont यह राम है।} (Yah Ram hai.) - Éste es Ram.

\end{itemize}

\subsection{5.3 Interrogativo: किसका (kiska)}


La palabra {\hindifont किसका} (kiska) significa "de quién" o "cuyo/cuya" y se utiliza para preguntar a quién pertenece algo:

\subsubsection{5.3.1 Formación y Uso}


\begin{center}
\begin{tabularx}{\textwidth}{| X | X | X | X | X |}
\hline
\textbf{Forma} & \textbf{Devanagari} & \textbf{Transliteración} & \textbf{Significado} & \textbf{Ejemplo} \ \hline
Singular Masc. & {\hindifont किसका} & kiska & ¿De quién? (masc. sing.) & {\hindifont यह किसका घर है?} (Yah kiska ghar hai?) - ¿De quién es esta casa? \ \hline
Singular Fem. & {\hindifont किसकी} & kiskii & ¿De quién? (fem. sing.) & {\hindifont यह किसकी किताब है?} (Yah kiskii kitaab hai?) - ¿De quién es este libro? \ \hline
Plural & {\hindifont किसके} & kaske & ¿De quién? (masc. plural) & {\hindifont ये किसके हैं?} (Ye kaske hain?) - ¿De quién son éstos? \ \hline
\end{tabularx}
\end{center}

\subsection{5.4 Vocabulario: Objetos del Aula}


El vocabulario de objetos escolares es fundamental para expresar posesión y descripción:


\begin{center}
\begin{tabularx}{\textwidth}{| X | X | X | X | X | X |}
\hline
\textbf{Español} & \textbf{Hindi} & \textbf{Transliteración} & \textbf{Género} & \textbf{Pronunciación} & \textbf{Comentario} \ \hline
Libro & {\hindifont किताब} & kitaab & Femenino & KI-taab & Forma árabe de préstamo \ \hline
Pluma/Lápiz & {\hindifont कलम} & qalam & Femenino & QA-lam & Forma árabe de préstamo \ \hline
Mesa & {\hindifont मेज़} & mez & Femenino & mez & Forma persa de préstamo \ \hline
Silla & {\hindifont कुर्सी} & kursii & Femenino & KUR-see & Forma persa de préstamo \ \hline
Pizarra & {\hindifont श्यामपट्ट} & shyaampaTT & Masculino & SHAAM-pat & Literalmente "tablero oscuro" \ \hline
Escuela & {\hindifont स्कूल} & skul & Masculino & SKOOL & Forma inglesa de préstamo \ \hline
Estudiante & {\hindifont छात्र}/{\hindifont छात्रा} & chatra/chatrii & Masc./Fem. & CHAA-tra/CHAA-tree & Masc. sing./Fem. sing. \ \hline
Profesor & {\hindifont अध्यापक}/{\hindifont अध्यापिका} & adhyaapak/adhyaapikaa & Masc./Fem. & ad-HYA-pak/ad-HYA-pee-kaa & Masc. sing./Fem. sing. \ \hline
Director & {\hindifont प्रिंसिपल} & prinsipal & Masculino & prin-SIP-al & Forma inglesa de préstamo \ \hline
Salón de clase & {\hindifont कक्षा} & kakshaa & Femenino & KAK-shaa & Literalmente "sala" \ \hline
Cuaderno & {\hindifont नोटबुक} & notbuk & Femenino & NOT-buk & Forma inglesa de préstamo \ \hline
Bolígrafo & {\hindifont पेन} & pen & Masculino & pen & Forma inglesa de préstamo \ \hline
Borrador & {\hindifont पेनसिल} & penkil & Masculino & pen-KIL & Forma inglesa de préstamo (lápiz) \ \hline
Regla & {\hindifont पैमाना} & paimaanaa & Masculino & pie-MAN-aa & Literalmente "medida" \ \hline
Pegamento & {\hindifont गोंद} & gond & Masculino & GOND & Palabra nativa \ \hline
\end{tabularx}
\end{center}

\subsubsection{5.4.1 Expresiones Comunes con Objetos del Aula}


\begin{itemize}

  \item {\hindifont मेरी किताब} - Mi libro

  \item {\hindifont तुम्हारा पेन} - Tu bolígrafo

  \item {\hindifont यह मेरा नोटबुक है।} - Este es mi cuaderno.

  \item {\hindifont वह तुम्हारी कलम है।} - Esa es tu pluma.

  \item {\hindifont हमारी कक्षा} - Nuestra clase

  \item {\hindifont किसकी किताब?} - ¿De quién es el libro?

\end{itemize}

\subsection{5.5 Vocabulario Adicional: Objetos de la Casa}


Complementamos con vocabulario de objetos domésticos:


\begin{center}
\begin{tabularx}{\textwidth}{| X | X | X | X | X |}
\hline
\textbf{Español} & \textbf{Hindi} & \textbf{Transliteración} & \textbf{Género} & \textbf{Uso} \ \hline
Casa & {\hindifont घर} & ghar & Masculino & Residencia \ \hline
Habitación & {\hindifont कमरा} & kamraa & Masculino & División de la casa \ \hline
Puerta & {\hindifont दरवाज़ा} & darvaazaa & Masculino & Entrada/salida \ \hline
Ventana & {\hindifont खिड़की} & khidkii & Femenino & Apertura en pared \ \hline
Cama & {\hindifont बिस्तर} & bistar & Masculino & Mueble para dormir \ \hline
Cocina & {\hindifont रसोई} & rasoi & Femenino & Área para cocinar \ \hline
Baño & {\hindifont शौचालय} & shauchaalay & Masculino & WC/toilet \ \hline
Comedor & {\hindifont भोजन कक्ष} & bhojan kaksh & Masculino & Área para comer \ \hline
Armario & {\hindifont अलमारी} & almarii & Femenino & Para almacenar cosas \ \hline
Espejo & {\hindifont दर्पण} & darpan & Masculino & Para verse reflejado \ \hline
\end{tabularx}
\end{center}

\subsection{5.6 Comunicación: Expresar Posesión y Preguntar por Posesión}


Ahora que conocemos los posesivos y los interrogativos, veamos cómo usarlos en contextos comunicativos:

\subsubsection{5.6.1 Frases para Expresar Posesión}


\begin{itemize}

  \item {\hindifont यह मेरी किताब है।} - Este libro es mío.

  \item {\hindifont वह तुम्हारी कलम है।} - Esa pluma es tuya.

  \item {\hindifont हमारा घर बड़ा है।} - Nuestra casa es grande.

  \item {\hindifont आपके नाम के साथ क्या है?} - ¿Cómo está Ud. junto con su nombre?

  \item {\hindifont ये हमारे दोस्त हैं।} - Éstos son nuestros amigos.

  \item {\hindifont इनकी कार बहुत अच्छी है।} - El coche de ellos es muy bueno.

\end{itemize}

\subsubsection{5.6.2 Frases para Preguntar por Posesión}


\begin{itemize}

  \item {\hindifont यह किसका घर है?} - ¿De quién es esta casa?

  \item {\hindifont किताब किसकी है?} - ¿De quién es el libro?

  \item {\hindifont ये किसके पेन हैं?} - ¿De quién son estos bolígrafos?

  \item {\hindifont तुम्हारा नाम क्या है?} - ¿Cuál es tu nombre?

  \item {\hindifont आपका फ़ोन नंबर क्या है?} - ¿Cuál es su número de teléfono?

\end{itemize}

\subsubsection{5.6.3 Diálogos Ejemplares}


Diálogo 1 - En la escuela:


\begin{itemize}

  \item A: {\hindifont यह किताब किसकी है?} (Yah kitaab kiskii hai?) - ¿De quién es este libro?

  \item B: {\hindifont यह मेरी है।} (Yah merii hai.) - Éste es mío.

  \item A: {\hindifont तुम्हारा नाम क्या है?} (Tumhaaraa naam kya hai?) - ¿Cómo te llamas?

  \item B: {\hindifont मेरा नाम राम है।} (Meraa naam Ram hai.) - Me llamo Ram.

  \item A: {\hindifont यह तुम्हारा पेन है?} (Yah tumhaaraa pen hai?) - ¿Este es tu bolígrafo?

  \item B: {\hindifont हाँ, यह मेरा पेन है।} (Haan, yah meraa pen hai.) - Sí, este es mi bolígrafo.

\end{itemize}


Diálogo 2 - Presentaciones:


\begin{itemize}

  \item A: {\hindifont नमस्ते, मैं सीता हूँ।} (Namaste, main Seetaa huun.) - Hola, soy Sita.

  \item B: {\hindifont नमस्कार सीता जी। मैं राम हूँ।} (Namaskaar Seetaa jii. Main Ram huun.) - Saludos, Sita ji. Soy Ram.

  \item A: {\hindifont आप कहाँ से हैं?} (Aap kahaan se hain?) - ¿De dónde es Ud.?

  \item B: {\hindifont मैं भारत से हूँ। आप?} (Main Bharat se huun. Aap?) - Soy de la India. ¿Y Ud.?

  \item A: {\hindifont मैं स्पेन से हूँ। आपका घर कहाँ है?} (Main Spain se huun. Aapkaa ghar kahaan hai?) - Soy de España. ¿Dónde está su casa?

  \item B: {\hindifont मेरा घर दिल्ली में है।} (Meraa ghar Dillee men hai.) - Mi casa está en Delhi.

\end{itemize}


\begin{tcolorbox}[colback=yellow!10!white,colframe=orange!75!black,title=Regla de Oro]

\paragraph{Regla de Oro: Posesivos en Hindi} 

Recuerda que en hindi, la concordancia del posesivo es con el \textbf{objeto poseído} no con el poseedor como en español. Por ejemplo:


\begin{itemize}

  \item En español: "Mi \textbf{libro} es bueno" - Concordancia con "libro" (masc.)

  \item En hindi: {\hindifont मेरी किताब अच्छी है।} - Concordancia con "kitaab" que es femenino, por eso usamos {\hindifont मेरी} (femenino) no {\hindifont मेरा} (masculino)

\end{itemize}


\end{tcolorbox}


\hrulefill

\section{Unidad 6: Expresiones de Posesión y el Cuerpo}


Objetivo: Dominar las diferentes formas de expresar posesión y aprender el vocabulario del cuerpo humano

\subsection{6.1 Expresión de Posesión con la Partícula के पास (ke paas)}


La construcción {\hindifont के पास} (ke paas) se utiliza para expresar posesión de objetos movibles o elementos tangibles que una persona tiene en su posesión.

\subsubsection{6.1.1 Estructura de la Construcción के पास}


La estructura es: \textbf{Sujeto + के पास + Objeto + है/हैं}


\begin{center}
\begin{tabularx}{\textwidth}{| X | X | X | X |}
\hline
\textbf{Elemento} & \textbf{Explicación} & \textbf{Ejemplo} & \textbf{Traducción} \ \hline
Sujeto & El poseedor, en forma oblicua & {\hindifont मेरे पास} (mere paas) & En mí (tengo) \ \hline
Ke paas & Literalmente "cerca de" o "junto a" & {\hindifont के पास} (ke paas) & "Cerca de" / "Con" \ \hline
Objeto & El elemento poseído & {\hindifont एक किताब} (ek kitaab) & Un libro \ \hline
Verbo & Forma conjugada de होना & {\hindifont है / हैं} (hai / hain) & es / son \ \hline
\end{tabularx}
\end{center}

\subsubsection{6.1.2 Ejemplos de Uso}


\begin{itemize}

  \item {\hindifont मेरे पास एक किताब है।} (Mere paas ek kitaab hai.) - Tengo un libro. (Literalmente: En mí hay un libro.)

  \item {\hindifont तुम्हारे पास पैसा है?} (Tumhaare paas paise hai?) - ¿Tienes dinero?

  \item {\hindifont उसके पास एक कार है।} (Uske paas ek kaar hai.) - Él/Ella tiene un coche.

  \item {\hindifont हमारे पास समय नहीं है।} (Hamaare paas samay nahin hai.) - No tenemos tiempo.

  \item {\hindifont आपके पास कितने भाई हैं?} (Aapke paas kitne bhaai hain?) - ¿Cuántos hermanos tiene Ud.?

\end{itemize}


\begin{tcolorbox}[colback=yellow!10!white,colframe=orange!75!black,title=Regla de Oro]

\paragraph{Regla de Oro: के पास vs. का/की/के} 

Existe una diferencia importante entre el uso de {\hindifont के पास} y las formas posesivas regulares {\hindifont का/की/के}:


\begin{itemize}

  \item \textbf{के पास}: Se usa para expresar posesión temporal o posesión de objetos movibles (cosas que se pueden tener, llevar o poseer en un momento dado).

  \item \textbf{का/की/के}: Se usa para relaciones permanentes, partes del cuerpo o posesión más estable.

\end{itemize}


Ejemplos comparativos:


\begin{itemize}

  \item {\hindifont मेरे पास पेन है।} - Tengo un bolígrafo. (Objeto movible)

  \item {\hindifont मेरा पेन लाल है।} - Mi bolígrafo es rojo. (Posesión más específica)

  \item {\hindifont मेरे पास एक कुत्ता है।} - Tengo un perro. (Objeto poseído)

  \item {\hindifont मेरा कुत्ता नाम रामू है।} - El nombre de mi perro es Ramu. (Relación más permanente)

\end{itemize}


\end{tcolorbox}

\subsection{6.2 Posesión con la Construcción Genitiva का/के/की}


La construcción genitiva con {\hindifont का/के/की} (kaa/ke/ki) expresa posesión o relación entre dos sustantivos, similar al "'s" en inglés o "del/de la" en español.

\subsubsection{6.2.1 Formas del Genitivo}


\begin{center}
\begin{tabularx}{\textwidth}{| X | X | X | X |}
\hline
\textbf{Objeto Poseído} & \textbf{Terminación Genitiva} & \textbf{Ejemplo} & \textbf{Traducción} \ \hline
Masculino Singular & {\hindifont का} (kaa) & {\hindifont राम का घर} (Ram kaa ghar) & La casa de Ram / La casa de él \ \hline
Femenino (sing. o plural) & {\hindifont की} (ki) & {\hindifont सीता की किताब} (Seetaa kii kitaab) & El libro de Sita / El libro de ella \ \hline
Masculino Plural o en Oblicuo & {\hindifont के} (ke) & {\hindifont लड़कों के खिलौने} (laRkon ke khilone) & Los juguetes de los niños \ \hline
Partes del cuerpo & {\hindifont का/की/के} (según objeto) & {\hindifont मेरा सिर} (meraa sir) & Mi cabeza \ \hline
Relaciones familiares & {\hindifont का/की/के} (según objeto) & {\hindifont तुम्हारी माँ} (tumhaarii maa) & Tu madre \ \hline
\end{tabularx}
\end{center}

\subsubsection{6.2.2 Uso del Genitivo}


El genitivo se utiliza en varios contextos:


\begin{itemize}

  \item \textbf{Relaciones:} {\hindifont राम की माँ} (Ram kii maa) - La madre de Ram

  \item \textbf{Partes del cuerpo:} {\hindifont मेरा हाथ दर्द कर रहा है।} (Meraa haath dard kar raha hai.) - Me duele la mano.

  \item \textbf{Propiedades:} {\hindifont इमारत का दरवाज़ा} (imaarat kaa darvaazaa) - La puerta del edificio

  \item \textbf{Descripción:} {\hindifont लाल रंग की कमीज़} (laal rang kii kamiz) - La camisa de color rojo

\end{itemize}

\subsection{6.3 Doblaje de la Posesión: के पास vs का/की/के}


Es importante entender cuándo utilizar cada forma y cómo se diferencian:


\begin{tcolorbox}[colback=blue!5!white,colframe=blue!75!black,title=Nota/Clarificación]

\paragraph{Guía de Elección para Expresar Posesión} 

Usa \textbf{के पास} cuando:


\begin{itemize}

  \item El objeto poseído es movible/tangible (dinero, libros, coches, animales)

  \item La posesión es temporal o posiblemente cambiante

  \item Quieres enfatizar que "tengo" algo de forma más general

\end{itemize}


Usa \textbf{का/की/के} cuando:


\begin{itemize}

  \item La relación es más permanente (familia, posesiones fijas)

  \item Es una parte del cuerpo

  \item Estás describiendo características específicas de algo

\end{itemize}


\end{tcolorbox}

\subsubsection{6.3.1 Ejemplos de Contraste}


\begin{center}
\begin{tabularx}{\textwidth}{| X | X | X | X |}
\hline
\textbf{Construcción} & \textbf{Ejemplo en Hindi} & \textbf{Traducción} & \textbf{Comentario} \ \hline
के पास & {\hindifont राम के पास पैसा है।} & Ram tiene dinero. & Enfocado en la posesión general \ \hline
का/की/के & {\hindifont राम का पैसा अधिक है।} & El dinero de Ram es mucho. & Enfocado en el objeto mismo \ \hline
के पास & {\hindifont मेरे पास समय नहीं है।} & No tengo tiempo. & Tiempo como posesión disponible \ \hline
का/की/के & {\hindifont मेरा समय कम है।} & Mi tiempo es poco. & Característica del tiempo poseído \ \hline
के पास & {\hindifont उसके पास कुत्ता है।} & Él tiene un perro. & En general posee un perro \ \hline
का/की/के & {\hindifont उसका कुत्ता बोलता है।} & Su perro habla. & Atributo específico del perro \ \hline
\end{tabularx}
\end{center}

\subsection{6.4 Pronombre Posesivo Reflexivo: अपना/अपनी/अपने (apnaa/apnii/apne)}


El pronombre posesivo reflexivo {\hindifont अपना} es una forma especial que se utiliza cuando el poseedor y el sujeto de la oración son la misma persona.


\begin{tcolorbox}[colback=yellow!10!white,colframe=orange!75!black,title=Regla de Oro]

\paragraph{Regla Fundamental de अपना (apnaa)} 

Cuando el sujeto de la oración es también el poseedor de algo mencionado en la misma oración, se utiliza \textbf{अपना/अपनी/अपने} en lugar del posesivo normal ({\hindifont मेरा/तुम्हारा/etc.}).


\end{tcolorbox}

\subsubsection{6.4.1 Tabla de Apnaa}


\begin{center}
\begin{tabularx}{\textwidth}{| X | X | X | X |}
\hline
\textbf{Objeto Poseído} & \textbf{Forma de Apnaa} & \textbf{Ejemplo con "libro"} & \textbf{Traducción} \ \hline
Masculino Singular & {\hindifont अपना} (apnaa) & {\hindifont मैं अपनी किताब पढ़ता हूँ।} & Leo \textbf{mi} (propia) libro. \ \hline
Femenino Singular & {\hindifont अपनी} (apnii) & {\hindifont वह अपनी किताब पढ़ता है।} & Él lee \textbf{su} (propia) libro. \ \hline
Masculino Plural & {\hindifont अपने} (apne) & {\hindifont हम अपने काम करते हैं।} & Hacemos \textbf{nuestros} (propios) trabajos. \ \hline
\end{tabularx}
\end{center}

\subsubsection{6.4.2 Ejemplos Contrastivos}


\begin{itemize}

  \item \textbf{Con अपना:} {\hindifont मैं अपनी किताब पढ़ता हूँ।} (Main apnii kitaab paRhtaa huun.) - Yo leo \textbf{mi propia} libro.

  \item \textbf{Sin अपना:} {\hindifont मैं मेरी किताब पढ़ता हूँ।} (Main merii kitaab paRhtaa huun.) - Yo leo la libro que es mía.

  \item \textbf{Con अपना:} {\hindifont वह अपने घर में है।} (Vah apne ghar men hai.) - Él está en \textbf{su propio} casa.

  \item \textbf{Sin अपना:} {\hindifont वह उसके घर में है।} (Vah uske ghar men hai.) - Él está en la casa de él/ella.

\end{itemize}


\begin{tcolorbox}[colback=blue!5!white,colframe=blue!75!black,title=Nota/Clarificación]

\paragraph{Atención: Errores Comunes con अपना} 

El uso incorrecto de {\hindifont अपना} es un error común. Solo se utiliza cuando el sujeto y el poseedor son la misma persona:


\begin{itemize}

  \item ❌ {\hindifont राम अपनी माँ को देखता है।} (Incorrecto si nos referimos a la madre de alguien más)

  \item ✅ {\hindifont राम अपनी माँ को देखता है।} (Correcto si hablamos de la madre de Ram)

  \item ✅ {\hindifont राम उसकी माँ को देखता है।} (Correcto si hablamos de la madre de otra persona)

\end{itemize}


\end{tcolorbox}

\subsection{6.5 Vocabulario: Partes del Cuerpo Humano}


A continuación se presentan las partes principales del cuerpo humano con su género gramatical en hindi:


\begin{center}
\begin{tabularx}{\textwidth}{| X | X | X | X | X | X |}
\hline
\textbf{Español} & \textbf{Hindi} & \textbf{Transliteración} & \textbf{Género} & \textbf{Pronunciación} & \textbf{Comentario} \ \hline
Cabeza & {\hindifont सिर} & sir & Masculino & SIR & Parte superior del cuerpo \ \hline
Ojo & {\hindifont आँख} & aankh & Femenino & AAANKH & Órgano de la vista \ \hline
Nariz & {\hindifont नाक} & naak & Femenino & NA-AK & Órgano del olfato \ \hline
Boca & {\hindifont मुँह} & muh & Masculino & MUH & Abertura facial \ \hline
Lengua & {\hindifont जीभ} & jiibh & Femenino & JII-BH & Órgano en la boca \ \hline
Dientes & {\hindifont दाँत} & daant & Masculino & DA-AANT & En la boca \ \hline
Oreja & {\hindifont कान} & kaan & Masculino & KAA-N & Órgano del oído \ \hline
Mano & {\hindifont हाथ} & haath & Masculino & HAA-TH & Extremidad superior \ \hline
Dedo & {\hindifont उँगली} & ungalii & Femenino & UNG-ga-lee & En la mano/pie \ \hline
Brazo & {\hindifont बाँह} & baah & Femenino & BAAH & Extremidad superior \ \hline
Pecho & {\hindifont छाती} & chhaatii & Femenino & CHHAA-tee & Parte frontal superior \ \hline
Corazón & {\hindifont दिल} & dil & Masculino & DIL & Órgano vital \ \hline
Vientre & {\hindifont पेट} & pet & Masculino & PET & Parte media del torso \ \hline
Pierna & {\hindifont टाँग} & taang & Femenino & TAA-NG & Extremidad inferior \ \hline
Pie & {\hindifont पाँव} & paav & Masculino & PA-AV & Extremidad inferior \ \hline
Dedo del pie & {\hindifont पैर की उंगली} & pair kiiungalii & Femenino & PAR kee OONG-ga-lee & En el pie \ \hline
Garganta & {\hindifont गला} & galaa & Masculino & GA-laa & Entre boca y pecho \ \hline
Cuello & {\hindifont गर्दन} & gardan & Femenino & GAR-dan & Entre cabeza y torso \ \hline
\end{tabularx}
\end{center}

\subsubsection{6.5.1 Expresiones con Partes del Cuerpo}


Las partes del cuerpo se usan comúnmente con el genitivo posesivo y el verbo होना:


\begin{itemize}

  \item {\hindifont मेरा सिर दर्द कर रहा है।} (Meraa sir dard kar raha hai.) - Me duele la cabeza.

  \item {\hindifont उसकी आँखें बहुत सुंदर हैं।} (Uskii aankhen bahut sundar hain.) - Sus ojos son muy bonitos.

  \item {\hindifont मेरे दाँत दर्द कर रहे हैं।} (Mere daant dard kar rahe hain.) - Me duelen los dientes.

  \item {\hindifont हमारे दिल खुशी से भर गए।} (Hamaare dil khushi se bhar gae.) - Nuestros corazones se llenaron de alegría.

  \item {\hindifont तुम्हारा चेहरा खुशी से चमक रहा है।} (Tumhaaraa chehra khushi se chamak raha hai.) - Tu cara brilla de felicidad.

\end{itemize}

\subsection{6.6 Comunicación: Expresar Posesión con Diferentes Construcciones}


Veamos cómo usar las diferentes formas de posesión en contextos comunicativos:

\subsubsection{6.6.1 Frases Comunes}


\begin{itemize}

  \item {\hindifont मुझे एक किताब चाहिए।} (Mujhe ek kitaab chaahie.) - Necesito un libro. (Literalmente: A mí me es necesaria una libro)

  \item {\hindifont तुम्हारे पास क्या है?} (Tumhaare paas kya hai?) - ¿Qué tienes?

  \item {\hindifont यह राम का घर है।} (Yah Ram kaa ghar hai.) - Ésta es la casa de Ram.

  \item {\hindifont मैं अपना काम कर रहा हूँ।} (Main apnaa kaam kar raha huun.) - Estoy haciendo \textbf{mi propio} trabajo.

  \item {\hindifont मुझे सिर दर्द है।} (Mujhe sir dard hai.) - Tengo dolor de cabeza. (Literalmente: A mí me duele la cabeza)

  \item {\hindifont तुम्हारी मुस्कान सुंदर है।} (Tumhaarii muskaan sundar hai.) - Tu sonrisa es bonita.

\end{itemize}

\subsubsection{6.6.2 Diálogos Ejemplares}


Diálogo 1 - En la clase:


\begin{itemize}

  \item A: {\hindifont यह किसकी किताब है?} (Yah kiskii kitaab hai?) - ¿De quién es este libro?

  \item B: {\hindifont यह मेरी है।} (Yah merii hai.) - Éste es mío.

  \item A: {\hindifont क्या तुम्हारे पास एक पेन है?} (Kya tumhaare paas ek pen hai?) - ¿Tienes un bolígrafo?

  \item B: {\hindifont हाँ, मेरे पास है। यह लो।} (Haan, mere paas hai. Yah lo.) - Sí, lo tengo. Toma.

  \item A: {\hindifont धन्यवाद। यह आपका है।} (Dhanyavaad. Yah aapkaa hai.) - Gracias. Éste es suyo.

\end{itemize}


Diálogo 2 - En la tienda:


\begin{itemize}

  \item Vendedor: {\hindifont आपके पास पैसा है?} (Aapke paas paise hai?) - ¿Tiene Ud. dinero?

  \item Cliente: {\hindifont हाँ, मेरे पास पैसा है। यह मेरी कीमत सूची है।} (Haan, mere paas paise hai. Yah merii keemat suchii hai.) - Sí, tengo dinero. Ésta es mi lista de precios.

  \item Vendedor: {\hindifont आपका चुनाव क्या है?} (Aapkaa chunaav kya hai?) - ¿Cuál es su elección?

  \item Cliente: {\hindifont मुझे यह कमीज़ चाहिए।} (Mujhe yah kamiz chaahie.) - Quiero esta camisa.

  \item Vendedor: {\hindifont यह आपके लिए बहुत अच्छी है।} (Yah aapke liye bahut achchhii hai.) - Ésta es muy buena para Ud.

\end{itemize}

\subsubsection{6.6.3 Frases con Partes del Cuerpo}


Diálogo 3 - En la clínica:


\begin{itemize}

  \item Paciente: {\hindifont डॉक्टर साहब, मुझे सिर दर्द है।} (Doctor sahab, mujhe sir dard hai.) - Doctor sahib, tengo dolor de cabeza.

  \item Dr.: {\hindifont ठीक है। मुझे आपके सिर की जाँच करनी है।} (Theek hai. Mujhe aapke sir kii jaanch karni hai.) - Bien. Tengo que examinar su cabeza.

  \item Paciente: {\hindifont मुझे मेरी आँख दिखाई नहीं दे रही है।} (Mujhe merii aankh dikhaaii nahin de rahi hai.) - No me veo bien el ojo.

  \item Dr.: {\hindifont हाँ, मैं देख रहा हूँ। मेरी जुबान पर थोड़ा ध्यान दें।} (Haan, main dekh raha huun. Merii jubaan par thoda dhyaan den.) - Sí, lo veo. Preste atención a mi lengua.

\end{itemize}

\subsection{6.7 Vocabulario Adicional: Reino Animal}


Lista ampliada de animales comunes en el contexto indio:


\begin{center}
\begin{tabularx}{\textwidth}{| X | X | X | X |}
\hline
\textbf{Español} & \textbf{Hindi} & \textbf{Transliteración} & \textbf{Género} \ \hline
Elefante & {\hindifont हाथी} & haathii & Masc. \ \hline
Perro & {\hindifont कुत्ता} & kuttaa & Masc. \ \hline
Gato/a & {\hindifont बिल्ली} & billii & Fem. \ \hline
Vaca & {\hindifont गाय} & gaay & Fem. \ \hline
Caballo & {\hindifont घोड़ा} & ghodaa & Masc. \ \hline
Mono & {\hindifont बन्दर} & bandar & Masc. \ \hline
Ratón/Rata & {\hindifont चूहा} & chuhaa & Masc. \ \hline
Pájaro & {\hindifont पक्षी} / {\hindifont चिड़िया} & pakshii / chidiyaa & Masc. / Fem. \ \hline
Pavo real & {\hindifont मोर} & mor & Masc. \ \hline
Loro & {\hindifont तोता} & totaa & Masc. \ \hline
Camello & {\hindifont ऊँट} & uunt & Masc. \ \hline
Tigre & {\hindifont बाघ} & baagh & Masc. \ \hline
León & {\hindifont शेर} & sher & Masc. \ \hline
Oso & {\hindifont भालू} & bhaaloo & Masc. \ \hline
Serpiente & {\hindifont साँप} & saanp & Masc. \ \hline
\end{tabularx}
\end{center}


\hrulefill

\section{Unidad 7: Caso Oblicuo (Singular), Postposiciones (I) y Presente Continuo}


Objetivo: Dominar el caso oblicuo singular, aprender las primeras postposiciones y formar el presente continuo

\subsection{7.1 Introducción al Caso Oblicuo Singular}


El caso oblicuo es una forma gramatical que adoptan los sustantivos, pronombres y adjetivos en hindi cuando van seguidos de una postposición. Es una de las características más importantes del sistema gramatical del hindi.


\begin{tcolorbox}[colback=yellow!10!white,colframe=orange!75!black,title=Regla de Oro]

\paragraph{Regla de Oro: El Caso Oblicuo} 

Cuando un sustantivo o pronombre va seguido de una \textbf{postposición} (como {\hindifont में, पर, को}), la palabra debe cambiar a su forma \textbf{"oblicua"}. Piensa que las postposiciones son como "mochilas" que una palabra carga. Para poder cargar una mochila, la palabra necesita adoptar una forma más estable y fuerte: la forma oblicua.


\end{tcolorbox}

\subsubsection{7.1.1 ¿Qué es el Caso Oblicuo?}


El caso oblicuo es la forma que adopta un sustantivo o pronombre para poder aceptar una postposición. Es equivalente al caso dativo o preposicional en otros idiomas. La forma oblicua se forma de acuerdo con reglas específicas según el género y terminación del sustantivo.

\subsection{7.2 Formas Oblicuas de Sustantivos (Singular)}


Las formas oblicuas varían según el género y terminación del sustantivo. Presentamos aquí las formas singulares:


\begin{center}
\begin{tabularx}{\textwidth}{| X | X | X | X | X |}
\hline
\textbf{Categoría} & \textbf{Forma Nominativa (Directa)} & \textbf{Forma Oblicua} & \textbf{Ejemplo} & \textbf{Ejemplo con Postposición} \ \hline
Masculino terminado en -ा (-aa) & {\hindifont लड़का} (chico) & {\hindifont लड़के} (laRke) & {\hindifont लड़के को} (laRke ko - al chico) & {\hindifont लड़के को पुस्तक मिली।} (El chico recibió el libro) \ \hline
Masculino no terminado en -ा & {\hindifont घर} (casa, masc.) & {\hindifont घर} (ghar) & {\hindifont घर में} (ghar men - en la casa) & {\hindifont लड़का घर में है।} (El chico está en la casa) \ \hline
Femenino terminado en -ी (-ii) & {\hindifont लड़की} (chica) & {\hindifont लड़की} (laRkii) & {\hindifont लड़की को} (laRkii ko - a la chica) & {\hindifont लड़की को पुस्तक मिली।} (La chica recibió el libro) \ \hline
Femenino no terminado en -ी & {\hindifont किताब} (libro, fem.) & {\hindifont किताब} (kitaab) & {\hindifont किताब पर} (kitaab par - sobre el libro) & {\hindifont कलम किताब पर है।} (La pluma está sobre el libro) \ \hline
\end{tabularx}
\end{center}

\subsubsection{7.2.1 Reglas Detalladas de Formación del Caso Oblicuo (Singular)}


\begin{enumerate}

  \item \textbf{Sustantivos masculinos que terminan en -ा (-aa)}: Cambian la terminación a -े (-e)
            
\begin{itemize}

  \item {\hindifont कमरा} (cuarto) → {\hindifont कमरे} (kamre) - {\hindifont कमरे में} (en el cuarto)

  \item {\hindifont लड़का} (chico) → {\hindifont लड़के} (laRke) - {\hindifont लड़के को} (al chico)

  \item {\hindifont स्टूडेंट} (estudiante, masc.) → {\hindifont स्टूडेंट} (student) - {\hindifont स्टूडेंट को} (al estudiante)

\end{itemize}



  \item \textbf{Todos los demás sustantivos (femeninos y masculinos que no terminan en -ा)}: \textbf{No cambian} en singular oblicuo
            
\begin{itemize}

  \item {\hindifont लड़की} (chica) → {\hindifont लड़की} (laRkii) - {\hindifont लड़की को} (a la chica)

  \item {\hindifont किताब} (libro) → {\hindifont किताब} (kitaab) - {\hindifont किताब पर} (sobre el libro)

  \item {\hindifont घर} (casa) → {\hindifont घर} (ghar) - {\hindifont घर में} (en la casa)

  \item {\hindifont पुस्तक} (libro formal) → {\hindifont पुस्तक} (pustak) - {\hindifont पुस्तक के बारे में} (sobre el libro)

\end{itemize}



\end{enumerate}


\begin{tcolorbox}[colback=blue!5!white,colframe=blue!75!black,title=Nota/Clarificación]

\paragraph{Regla Importante: Concordancia con el Objeto} 

Los sustantivos en caso oblicuo se usan siempre antes de una postposición. La postposición define la relación gramatical entre el sustantivo y otras partes de la oración:


\begin{itemize}

  \item {\hindifont घर में} - En casa (relación de lugar)

  \item {\hindifont किताब पर} - Sobre el libro (relación de posición)

  \item {\hindifont लड़के को} - A/para el chico (relación de destinatario)

\end{itemize}


\end{tcolorbox}

\subsection{7.3 Tabla de Pronombres en Caso Oblicuo}


Los pronombres también tienen formas oblicuas que se usan con postposiciones:


\begin{center}
\begin{tabularx}{\textwidth}{| X | X | X | X | X |}
\hline
\textbf{Pronombre Nominativo} & \textbf{Forma Oblicua} & \textbf{Transliteración} & \textbf{Traducción} & \textbf{Uso con Postposición} \ \hline
{\hindifont मैं} (main) & {\hindifont मुझ} (mujh) & mujh & mí & {\hindifont मुझ पर} (sobre mí), {\hindifont मुझ से} (de mí) \ \hline
{\hindifont तू} (tū) & {\hindifont तुझ} (tujh) & tujh & ti & {\hindifont तुझ पर} (sobre ti), {\hindifont तुझ से} (de ti) \ \hline
{\hindifont तुम} (tum) & {\hindifont तुम्हारा} & tumhaaraa & a tí & {\hindifont तुम्हारे साथ} (contigo), {\hindifont तुम्हारे लिए} (para ti) \ \hline
{\hindifont आप} (āp) & {\hindifont आप} & āp & usted & {\hindifont आपके साथ} (con usted), {\hindifont आपके लिए} (para usted) \ \hline
{\hindifont यह} (yah) & {\hindifont इस} (is) & is & este/esta (cercano) & {\hindifont इसमें} (en esto), {\hindifont इस पर} (en esto) \ \hline
{\hindifont वह} (vah) & {\hindifont उस} (us) & us & ese/esa (distante) & {\hindifont उसमें} (en aquello), {\hindifont उस पर} (en aquello) \ \hline
{\hindifont हम} (ham) & {\hindifont हम} & ham & nosotros & {\hindifont हमारे साथ} (con nosotros) \ \hline
{\hindifont ये} (ye) & {\hindifont इन} (in) & in & éstos/éstas & {\hindifont इनमें} (en éstos), {\hindifont इन पर} (sobre éstos) \ \hline
{\hindifont वे} (ve) & {\hindifont उन} (un) & un & aquéllos/aquéllas & {\hindifont उनमें} (en aquéllos), {\hindifont उन पर} (sobre aquéllos) \ \hline
\end{tabularx}
\end{center}

\subsubsection{7.3.1 Formas Comunes de los Pronombres con Postposiciones}


Algunas combinaciones comunes de pronombres con postposiciones son irregulares o tienen formas especiales:


\begin{itemize}

  \item {\hindifont मैं} (main) → {\hindifont मुझे} (mujhe)/{\hindifont मुझको} (mujhko) con {\hindifont को} (a/para)

  \item {\hindifont तू} (tū) → {\hindifont तुझे} (tujhe)/{\hindifont तुझको} (tujhko) con {\hindifont को}

  \item {\hindifont तुम} (tum) → {\hindifont तुम्हें} (tumhe)/{\hindifont तुमको} (tumko) con {\hindifont को}

  \item {\hindifont हम} (ham) → {\hindifont हमें} (hame)/{\hindifont हमको} (hamko) con {\hindifont को}

  \item {\hindifont आप} (āp) → {\hindifont आपको} (aapko) con {\hindifont को}

\end{itemize}

\subsection{7.4 Postposiciones Fundamentales (Grupo 1)}


Las postposiciones en hindi equivalen a las preposiciones en español, pero se colocan \textbf{después} del sustantivo o pronombre:

\subsubsection{7.4.1 Postposiciones Básicas}


\begin{center}
\begin{tabularx}{\textwidth}{| X | X | X | X | X | X |}
\hline
\textbf{Postposición} & \textbf{Devanagari} & \textbf{Transliteración} & \textbf{Significado} & \textbf{Uso} & \textbf{Ejemplo} \ \hline
Dentro/en & {\hindifont में} & men & en, dentro de, en medio de & Lugar o tiempo & {\hindifont घर में} (ghar men - en casa), {\hindifont रात में} (rat men - por la noche) \ \hline
En/sobre & {\hindifont पर} & par & sobre, encima de, en & Superficie o contacto & {\hindifont मेज़ पर} (mez par - sobre la mesa), {\hindifont पत्र पर हस्ताक्षर} (patr par hastakshar - firmar la carta) \ \hline
A/para (destinatario) & {\hindifont को} & ko & a (indirecto), para & Objeto indirecto & {\hindifont राम को} (Ram ko - a Ram), {\hindifont उसको दे दो} (usko de do - dárselo) \ \hline
\end{tabularx}
\end{center}

\subsubsection{7.4.2 Ejemplos de Uso de Postposiciones}


\begin{itemize}

  \item {\hindifont किताब मेज़ पर है।} (Kitaab mez par hai.) - El libro está sobre la mesa.

  \item {\hindifont लड़का घर में है।} (LaRkaa ghar men hai.) - El chico está en casa.

  \item {\hindifont मैं राम को किताब दूंगा।} (Main Ram ko kitaab doonga.) - Le daré el libro a Ram.

  \item {\hindifont हवा बहुत तेज़ है।} (Havaa bahut tej hai.) - El viento es muy fuerte.

  \item {\hindifont कलम पुस्तक के पास है।} (Kalam pustak ke paas hai.) - La pluma está cerca del libro.

\end{itemize}

\subsection{7.5 Formación del Presente Continuo}


El presente continuo en hindi se forma con la raíz del verbo + {\hindifont रहा}/{\hindifont रही}/{\hindifont रहे} (rahaa/rahi/rahe) + la forma conjugada del verbo {\hindifont होना} (honaa - ser/estar).

\subsubsection{7.5.1 Estructura del Presente Continuo}


La estructura es: \textbf{Sujeto + Raíz del Verbo + रहा/रही/रहे + Verbo 'Ser/Estar' conjugado}


La partícula {\hindifont रहा/रही/रहे} concuerda en género y número con el \textbf{sujeto} de la oración, no con el objeto.

\subsubsection{7.5.2 Tabla del Presente Continuo}


\begin{center}
\begin{tabularx}{\textwidth}{| X | X | X | X | X | X |}
\hline
\textbf{Sujeto} & \textbf{Pronombre} & \textbf{Forma del Verbo} & \textbf{Transliteración} & \textbf{Ejemplo con {\hindifont पढ़ना} (padhnaa - leer)} & \textbf{Traducción} \ \hline
1ra Pers. Sg. & {\hindifont मैं} & रहा/रही हूँ & rahaa/rahi huun & {\hindifont मैं पढ़ रहा हूँ।} & Yo estoy leyendo. (masc.) \ \hline
2da Pers. Cercano & {\hindifont तू} & रहा/रही है & rahaa/rahi hai & {\hindifont तू पढ़ रहा है।} & Tú estás leyendo. (masc.) \ \hline
2da Pers. Inf. & {\hindifont तुम} & रहा/रही हो & rahaa/rahi ho & {\hindifont तुम पढ़ रहे हो।} & Tú estás leyendo. (masc.pl.) \ \hline
2da Pers. Form. & {\hindifont आप} & रहे/रही हैं & rahe/rahi hain & {\hindifont आप पढ़ रहे हैं।} & Ud. está leyendo. (masc.pl.) \ \hline
3ra Pers. Sg. Cercano & {\hindifont यह} & रहा/रही है & rahaa/rahi hai & {\hindifont यह पढ़ रहा है।} & Él/Ella está leyendo. (masc.) \ \hline
3ra Pers. Sg. Distante & {\hindifont वह} & रहा/रही है & rahaa/rahi hai & {\hindifont वह पढ़ रहा है।} & Él/Ella está leyendo. (masc.) \ \hline
1ra Pers. Pl. & {\hindifont हम} & रहे/रही हैं & rahe/rahi hain & {\hindifont हम पढ़ रहे हैं।} & Nosotros estamos leyendo. (masc.) \ \hline
3ra Pers. Pl. Cercano & {\hindifont ये} & रहे/रही हैं & rahe/rahi hain & {\hindifont ये पढ़ रहे हैं।} & Ellos están leyendo. (masc.) \ \hline
3ra Pers. Pl. Distante & {\hindifont वे} & रहे/रही हैं & rahe/rahi hain & {\hindifont वे पढ़ रहे हैं।} & Ellos están leyendo. (masc.) \ \hline
\end{tabularx}
\end{center}

\subsubsection{7.5.3 Ejemplos de Presente Continuo}


\begin{itemize}

  \item {\hindifont मैं खाना बना रहा हूँ।} (Main khaanaa banaa raha huun.) - Yo estoy cocinando comida. (masc.)

  \item {\hindifont तुम पानी पी रहे हो।} (Tum paanii pee rahe ho.) - Tú estás bebiendo agua. (masc.)

  \item {\hindifont वह घर जा रही है।} (Vah ghar ja rahi hai.) - Ella está yendo a casa. (fem.)

  \item {\hindifont हम पढ़ाई कर रहे हैं।} (Ham padhaaee kar rahe hain.) - Estamos estudiando. (masc.)

  \item {\hindifont ये खेल रहे हैं।} (Ye khel rahe hain.) - Éstos están jugando. (masc.)

\end{itemize}


\begin{tcolorbox}[colback=blue!5!white,colframe=blue!75!black,title=Nota/Clarificación]

\paragraph{Diferencia entre Presente Simple y Presente Continuo} 

Es importante distinguir entre los dos tiempos verbales:


\begin{itemize}

  \item \textbf{Presente Simple/Habitual:} {\hindifont मैं हिंदी बोलता हूँ।} (Main hindi boltta huun.) - Yo hablo hindi. (Hábito/general)

  \item \textbf{Presente Continuo:} {\hindifont मैं हिंदी बोल रहा हूँ।} (Main hindi bol raha huun.) - Yo estoy hablando hindi. (Ahora mismo)

\end{itemize}


\end{tcolorbox}

\subsection{7.6 Verbos देना (denaa) y लेना (lenaa)}


Estos dos verbos son fundamentales y se usan con el caso oblicuo:

\subsubsection{7.6.1 Verbo देना (denaa) - Dar}


Cuando se da algo a alguien, el receptor va en caso oblicuo con la postposición {\hindifont को}:


\begin{itemize}

  \item {\hindifont मैं राम को किताब देता हूँ।} (Main Ram ko kitaab detaa huun.) - Yo doy el libro a Ram.

  \item {\hindifont तुम मुझे समय दो।} (Tum mujhe samay do.) - Dame tiempo. (Literally: Tú dame tiempo.)

  \item {\hindifont वह उसको पैसा देगा।} (Vah usko paise degaa.) - Él le dará dinero a él/ella.

\end{itemize}

\subsubsection{7.6.2 Verbo लेना (lenaa) - Tomar/Recibir}


Cuando se toma algo de alguien, la fuente va en caso oblicuo con la postposición {\hindifont से}:


\begin{itemize}

  \item {\hindifont मैं राम से किताब लेता हूँ।} (Main Ram se kitaab letaa huun.) - Yo tomo/el libro de Ram.

  \item {\hindifont तुम मुझ से धोखा मत लो।} (Tum mujh se dhokhaa mat lo.) - No me engañes. (No tomes engaño de mí.)

  \item {\hindifont वह उससे सलाह लेता है।} (Vah usse salaah letaa hai.) - Él toma consejo de él/ella.

\end{itemize}


\begin{tcolorbox}[colback=blue!5!white,colframe=blue!75!black,title=Nota/Clarificación]

\paragraph{Patrón Importante de देना y लेना} 

Los patrones de estos verbos son opuestos:


\begin{itemize}

  \item \textbf{देना} (dar): Sujeto + Objeto + {\hindifont को} + Destinatario + {\hindifont देता/देती/देते है/हैं}

  \item \textbf{लेना} (tomar): Sujeto + Objeto + {\hindifont लेता/लेती/लेते है/हैं} + {\hindifont से} + Fuente

\end{itemize}


\end{tcolorbox}

\subsection{7.7 Vocabulario: Lugares de la Ciudad, Expresiones Temporales, Clima y Ropa}

\subsubsection{7.7.1 Lugares de la Ciudad}


\begin{center}
\begin{tabularx}{\textwidth}{| X | X | X | X | X |}
\hline
\textbf{Español} & \textbf{Hindi} & \textbf{Transliteración} & \textbf{Género} & \textbf{Comentarios} \ \hline
Mercado & {\hindifont बाज़ार} & baazaar & Masculino & Área comercial \ \hline
Tienda & {\hindifont दुकान} & dukaan & Femenino & Comercial pequeña \ \hline
Escuela & {\hindifont स्कूल} & skul & Masculino & Forma prestada del inglés \ \hline
Hospital & {\hindifont अस्पताल} & aspataal & Masculino & De origen sánscrito \ \hline
Templo & {\hindifont मंदिर} & mandir & Masculino & Lugar sagrado \ \hline
Restaurante & {\hindifont रेस्तरां} & restaraan & Masculino & Forma prestada del francés \ \hline
Banco & {\hindifont बैंक} & bank & Masculino & Forma prestada del inglés \ \hline
Parque & {\hindifont पार्क} & park & Masculino & Área recreativa \ \hline
Biblioteca & {\hindifont पुस्तकालय} & pustakaalay & Masculino & De "pustak" (libro) \ \hline
Cine & {\hindifont सिनेमा} & sinemaa & Masculino & Forma prestada \ \hline
\end{tabularx}
\end{center}

\subsubsection{7.7.2 Expresiones Temporales}


\begin{itemize}

  \item {\hindifont अभी} (abhi) - ahora, ahora mismo

  \item {\hindifont इस समय} (is samay) - en este momento

  \item {\hindifont अभी तुरंत} (abhi turant) - inmediatamente

  \item {\hindifont थोड़ा समय पहले} (thoda samay pahle) - hace un rato

  \item {\hindifont बाद में} (baad men) - después

  \item {\hindifont जल्दी} (jaldee) - rápidamente, pronto

  \item {\hindifont धीरे} (dheere) - lentamente, despacio

  \item {\hindifont समय पर} (samay par) - a tiempo

  \item {\hindifont देर से} (der se) - tarde

  \item {\hindifont जल्दी से} (jaldee se) - temprano

\end{itemize}

\subsubsection{7.7.3 Clima}


\begin{center}
\begin{tabularx}{\textwidth}{| X | X | X | X | X |}
\hline
\textbf{Español} & \textbf{Hindi} & \textbf{Transliteración} & \textbf{Género} & \textbf{Uso} \ \hline
Calor & {\hindifont गर्मी} & garmii & Femenino & Estación o sensación \ \hline
Frio & {\hindifont ठंढ} & thand & Femenino & Sensación física \ \hline
Agua (lluvia) & {\hindifont बारिश} & baarish & Femenino & Precipitación \ \hline
Viento & {\hindifont हवा} & havaa & Femenino & Aire en movimiento \ \hline
Sol & {\hindifont धूप} & dhoop & Femenino & Luz solar \ \hline
Nieve & {\hindifont बर्फ़} & barf & Femenino & Hielo en polvo \ \hline
Nube & {\hindifont बादल} & baadal & Masculino & Formación atmosférica \ \hline
Trueno & {\hindifont गरज} & garaj & Femenino & Sonido del cielo \ \hline
Relámpago & {\hindifont बिजली} & bijalii & Femenino & Fenómeno eléctrico \ \hline
Arco iris & {\hindifont इंद्रधनुष} & indradhanush & Masculino & Fenómeno óptico \ \hline
\end{tabularx}
\end{center}

\subsubsection{7.7.4 Ropa}


\begin{itemize}

  \item \textbf{Camisa:} {\hindifont कमीज़} (kamiz) - Femenino

  \item \textbf{Pantalón:} {\hindifont पतलून} (patloon) - Masculino

  \item \textbf{Sari:} {\hindifont साड़ी} (saarii) - Femenino

  \item \textbf{Kurta:} {\hindifont कुर्ता} (kurtaa) - Masculino

  \item \textbf{Zapatos:} {\hindifont जूते} (jute) - Masculino Plural

  \item \textbf{Sandalias:} {\hindifont चप्पल} (chappal) - Femenino

  \item \textbf{Camiseta:} {\hindifont टी-शर्ट} (tee-shirt) - Masculino

  \item \textbf{Jeans:} {\hindifont जींस} (jiins) - Masculino

  \item \textbf{Medias:} {\hindifont मोज़े} (moje) - Masculino Plural

  \item \textbf{Bufanda:} {\hindifont स्कार्फ़} (skaarf) - Masculino

  \item \textbf{Guantes:} {\hindifont दस्ताने} (dastaane) - Masculino Plural

  \item \textbf{Bufanda:} {\hindifont गला ठंढा करने वाला कपड़ा} (gala thandaa karne vaalaa kapraa) - Literalmente "ropa para enfriar el cuello"

\end{itemize}

\subsection{7.8 Comunicación: Describir Ubicación, Dar y Recibir Objetos, Acciones en Progreso}


Con los nuevos elementos gramaticales, se pueden formular expresiones más complejas:

\subsubsection{7.8.1 Frases para Describir Ubicación}


\begin{itemize}

  \item {\hindifont किताब मेज़ पर है।} - El libro está sobre la mesa.

  \item {\hindifont कुर्सी घर में है।} - La silla está en la casa.

  \item {\hindifont कलम डॉक्टर के पास है।} - La pluma está con el doctor.

  \item {\hindifont राम स्कूल में है।} - Ram está en la escuela.

  \item {\hindifont मैं घर पर हूँ।} - Estoy en casa.

\end{itemize}

\subsubsection{7.8.2 Expresiones para Dar y Recibir}


\begin{itemize}

  \item {\hindifont कृपया यह राम को दे दीजिए।} - Por favor, déle esto a Ram.

  \item {\hindifont मैं तुम्हें एक किताब दूंगा।} - Te daré un libro.

  \item {\hindifont उससे पानी माँगो।} - Pide agua a él/ella.

  \item {\hindifont लो, यह तुम्हारा पैसा है।} - Toma, éste es tu dinero.

  \item {\hindifont मैंने तुम से झूठ नहीं सुना।} - No escuché mentiras de ti.

\end{itemize}

\subsubsection{7.8.3 Describir Acciones en Progreso}


\begin{itemize}

  \item {\hindifont मैं पढ़ रहा हूँ।} - Estoy leyendo.

  \item {\hindifont तुम क्या बना रहे हो?} - ¿Qué estás cocinando?

  \item {\hindifont वह गाना गा रहा है।} - Él está cantando una canción.

  \item {\hindifont हम भारत जा रहे हैं।} - Estamos yendo a la India.

  \item {\hindifont ये लोग हँस रहे हैं।} - Estas personas están riéndose.

\end{itemize}

\subsubsection{7.8.4 Diálogos Ejemplares}


\textbf{Diálogo 1 - En la biblioteca:}


\begin{itemize}

  \item Bibliotecario: {\hindifont आप क्या ढूंढ रहे हैं?} (Aap kya dhoondh rahe hain?) - ¿Qué está buscando Ud.?

  \item Usuario: {\hindifont मैं हिंदी की किताब ढूंढ रहा हूँ।} (Main hindi kii kitaab dhoondh raha huun.) - Estoy buscando un libro de hindi.

  \item Bibliotecario: {\hindifont यह किताब आपके लिए है।} (Yah kitaab aapke liye hai.) - Este libro es para Ud.

  \item Usuario: {\hindifont धन्यवाद, यह मुझे पसंद है।} (Dhanyavaad, yah mujhe pasand hai.) - Gracias, me gusta este.

\end{itemize}


\textbf{Diálogo 2 - En la calle:}


\begin{itemize}

  \item Turista: {\hindifont क्षमा कीजिए, स्कूल कहाँ है?} (Kshamaa kijiye, skul kahaan hai?) - Disculpe, ¿dónde está la escuela?

  \item Local: {\hindifont यहाँ से सीधे जाओ, आपको दाहिने हाथ पर दिखाई देगा।} (Yahan se seedhe jao, aapko daahine haath par dikhaai degaa.) - Desde aquí vaya recto, lo verá a su derecha.

  \item Turista: {\hindifont कितनी दूर है?} (Kitni door hai?) - ¿A qué distancia está?

  \item Local: {\hindifont तीन मिनट की पैदल दूरी है।} (Teen minute kii paidal doorii hai.) - Es una distancia de tres minutos a pie.

\end{itemize}


\hrulefill

\section{Unidad 8: Caso Oblicuo (Pronombre) y Postposiciones (II)}


Objetivo: Consolidar el uso del caso oblicuo con pronombres e introducir postposiciones de propósito y compañía

\subsection{8.1 Caso Oblicuo de los Pronombres}


En la unidad anterior vimos el caso oblicuo para sustantivos, ahora profundizaremos en el caso oblicuo de los pronombres, que tienen formas más irregulares y específicas.

\subsubsection{8.1.1 Tabla Completa de Pronombres en Caso Directo y Oblicuo}


La siguiente tabla muestra todos los pronombres personales, demostrativos e interrogativos en sus formas directa (nominativo) y oblicua:


\begin{center}
\begin{tabularx}{\textwidth}{| X | X | X | X | X | X |}
\hline
\textbf{Tipo de Pronombre} & \textbf{Caso Directo\newline } & \textbf{Caso Oblicuo} & \textbf{Forma con Postposición को} & \textbf{Transliteración} & \textbf{Ejemplo con Postposición} \ \hline
1ra Persona Sg. & {\hindifont मैं} (main) & Masculino/Femenino & {\hindifont मुझको} & mujhko & {\hindifont मुझको समझ नहीं आ रहा है।} (mujhko samajh nahin aa raha hai. - No entiendo.) \ \hline
{\hindifont मुझ} & {\hindifont मुझे} & mujhe & {\hindifont मुझे पानी चाहिए।} (mujhe paanii chaahie. - Necesito agua.) &  &  \ \hline
2da Persona Íntima & {\hindifont तू} (tū) & Masculino/Femenino & {\hindifont तुझको} & tujhko & {\hindifont तुझको देखकर अच्छा लगा।} (tujhko dekhkar achchhaa laga. - Me alegré de verte.) \ \hline
{\hindifont तुझ} & {\hindifont तुझे} & tujhe & {\hindifont तुझे मिस किया।} (tujhe mis kiya. - Te extrañé.) &  &  \ \hline
2da Persona Inf. & {\hindifont तुम} (tum) & Masculino & {\hindifont तुमको} & tumko & {\hindifont मैं तुमको याद करता हूँ।} (main tumko yaad karata huun. - Te recuerdo.) \ \hline
Femenino & {\hindifont तुमको} & tumko & {\hindifont मैं तुमको याद करती हूँ।} (main tumko yaad karatii huun. - Te recuerdo.) &  &  \ \hline
Oblicuo Común & {\hindifont तुम्हें} & tumhe̱ & {\hindifont तुम्हें आराम करना चाहिए।} (tumhe̱ aaraam karnaa chaahie. - Deberías descansar.) &  &  \ \hline
2da Persona Formal & {\hindifont आप} (āp) & Masculino & {\hindifont आपको} & aapko & {\hindifont आपको समझना चाहिए।} (aapko samajhnaa chaahie. - Ud. debería entender.) \ \hline
Femenino & {\hindifont आपको} & aapko & {\hindifont मैं आपको बहुत सम्मान करता हूँ।} (main aapko bahut samaan karata huun. - Le respeto mucho.) &  &  \ \hline
Oblicuo Común & {\hindifont आपको} & aapko & {\hindifont आपको यह जानकारी उपयोगी होगी।} (aapko yah jaanakaari upayogii hogii. - Esta información será útil para Ud.) &  &  \ \hline
3ra Persona Cercana & {\hindifont यह} (yah) & Singular Masc. & {\hindifont इसको} & isko & {\hindifont इसको देखो।} (isko dekho. - Mire esto.) \ \hline
Singular Fem. & {\hindifont इसको} & isko & {\hindifont इसको उठाओ।} (isko uthaao. - Levante esta.) &  &  \ \hline
3ra Persona Distante & {\hindifont वह} (vah) & Singular Masc. & {\hindifont उसको} & usko & {\hindifont उसको बुलाओ।} (usko bulao. - Llámele.) \ \hline
Singular Fem. & {\hindifont उसको} & usko & {\hindifont उसको देखा।} (usko dekhaa. - La vi.) &  &  \ \hline
1ra Persona Pl. & {\hindifont हम} (ham) & Común & {\hindifont हमको}/{\hindifont हमें} & hamko/hame̱ & {\hindifont हमें जाना है।} (hame̱ jaanaa hai. - Debemos ir.) \ \hline
3ra Persona Pl. Cercana & {\hindifont ये} (ye) & Plural Masc. & {\hindifont इनको} & inko & {\hindifont इनको खाना दो।} (inko khaanaa do. - Déles comida.) \ \hline
3ra Persona Pl. Distante & {\hindifont वे} (ve) & Plural Masc. & {\hindifont उनको} & unko & {\hindifont उनको बुलाया गया।} (unko bulayaa gayaa. - Les llamaron.) \ \hline
\end{tabularx}
\end{center}


\begin{tcolorbox}[colback=yellow!10!white,colframe=orange!75!black,title=Regla de Oro]

\paragraph{Regla Importante: Variantes de Formas Oblicuas} 

Existen dos formas principales para expresar el dativo con la postposición {\hindifont को}:


\begin{itemize}

  \item \textbf{Forma extendida:} {\hindifont मुझको, तुझको, उसको, इनको, उनको} - Más enfática, común en el habla coloquial

  \item \textbf{Forma contrahida:} {\hindifont मुझे, तुझे, उसे, इन्हें, उन्हें} - Forma más estándar y formal

\end{itemize}


Ambas formas son correctas, pero la forma contrahida es más común en el habla formal y escrita.


\end{tcolorbox}

\subsubsection{8.1.2 Notas Especiales sobre Pronombres Oblicuos}


Algunos pronombres cambian su forma en el caso oblicuo pero no con todas las postposiciones:


\begin{itemize}

  \item {\hindifont हम} (ham - nosotros) y {\hindifont आप} (aap - ustedes) no cambian en oblicuo cuando no están con {\hindifont को}: {\hindifont हमें} vs {\hindifont हम पर}

  \item Los pronombres demostrativos {\hindifont ये} y {\hindifont वे} se convierten en {\hindifont इन} y {\hindifont उन} en oblicuo: {\hindifont इनमें, उनमें}

  \item Algunas postposiciones requieren la forma completamente oblicua: {\hindifont से} (de, desde, con): {\hindifont मुझसे, तुमसे, उससे}

\end{itemize}

\subsection{8.2 Postposiciones Grupo 2: से (de, desde, con) y को (a, marcador de objeto/dativo)}


Estas postposiciones son fundamentales para expresar relaciones espaciales, temporales y de dirección:

\subsubsection{8.2.1 Postposición: से (se) - de, desde, con}


La postposición {\hindifont से} tiene múltiples usos:


\begin{itemize}

  \item \textbf{Origen:} {\hindifont भारत से} (India se - de la India)

  \item \textbf{Medio:} {\hindifont पैदल से} (a pie se - a pie)

  \item \textbf{Agente (en voz pasiva):} {\hindifont राम से कहा गया} (dicen por Ram se)

  \item \textbf{Instrumento:} {\hindifont कलम से लिखो} (escribe con pluma se)

  \item \textbf{Comparación:} {\hindifont तुमसे बड़ा} (mayor que tú se)

\end{itemize}


\begin{center}
\begin{tabularx}{\textwidth}{| X | X | X | X |}
\hline
\textbf{Uso} & \textbf{Forma Oblícua} & \textbf{Ejemplo} & \textbf{Traducción} \ \hline
Origen (cercano) & {\hindifont मुझसे} & {\hindifont तुम मुझ से लेते हो।} & Tú tomas de mí / Tú tomas de mí. \ \hline
Origen (distante) & {\hindifont उससे} & {\hindifont लेकिन वह उस से नहीं सीखता।} & Pero él no aprende de él/ella. \ \hline
Medio/Instrumento & {\hindifont तुमसे} & {\hindifont मैं तुमसे सीखता हूँ।} & Aprendo de tí / Aprendo contigo. \ \hline
Comparación & {\hindifont हमसे} & {\hindifont वे हम से तेज़ हैं।} & Ellos son más rápidos que nosotros. \ \hline
\end{tabularx}
\end{center}

\subsubsection{8.2.2 Postposición: को (ko) - a (marcador de objeto indirecto)}


La postposición {\hindifont को} tiene dos usos principales:


\begin{enumerate}

  \item \textbf{Objeto indirecto:} {\hindifont राम को किताब दो।} (Dale el libro a Ram.)

  \item \textbf{Objeto directo específico:} {\hindifont मैं राम को देखता हूँ।} (Veo a Ram específicamente.)

\end{enumerate}

\subsubsection{8.2.3 Profundización: La Postposición KO (को)}


La partícula {\hindifont को} (ko) es fundamental en hindi. Aunque generalmente marca el objeto indirecto, su uso con el objeto directo depende de reglas específicas de "Definitud" y "Animacidad".

\paragraph{A. La Jerarquía de Animacidad} 

El uso de {\hindifont को} con el objeto directo no es aleatorio. Sigue una jerarquía estricta:


\begin{center}
\begin{tabularx}{\textwidth}{| X | X | X | X |}
\hline
\textbf{Categoría} & \textbf{Regla} & \textbf{Ejemplo} & \textbf{Traducción} \ \hline
\textbf{Humanos y Nombres Propios} & Casi siempre llevan {\hindifont को} (Máxima prioridad) & {\hindifont मैंने राम को देखा।} & Vi a Ram. \ \hline
\textbf{Animales (Mascotas/Personificados)} & Suelen llevar {\hindifont को} (Se tratan como "humanos") & {\hindifont कुत्ते को बुलाओ।} & Llama al perro. \ \hline
\textbf{Animales Genéricos} & Opcional / Sigue la regla de definitud & {\hindifont मैंने शेर देखा / शेर को देखा।} & Vi un león / Vi al león. \ \hline
\textbf{Objetos Inanimados} & Solo llevan {\hindifont को} si son específicos/definidos & {\hindifont किताब मेज़ पर रखो।} & Pon el libro en la mesa. \ \hline
\end{tabularx}
\end{center}

\paragraph{B. La Regla de Definitud (Para Inanimados)} 

Para cosas u objetos, el uso de {\hindifont को} funciona como el artículo determinado ("el/la" vs "un/una"):


\begin{tcolorbox}[colback=yellow!10!white,colframe=orange!75!black,title=Regla de Oro]

\paragraph{Definido vs. Indefinido} 

\begin{itemize}

  \item \textbf{Sin KO (Indefinido/Genérico):} Indica una clase de objeto o "un" objeto cualquiera.
                \newline Ej: {\hindifont मैं किताब पढ़ता हूँ।} (Leo libros / Leo un libro).

  \item \textbf{Con KO (Definido/Específico):} Indica "ese" objeto en particular.
                \newline Ej: {\hindifont मैं इस किताब को पढ़ता हूँ।} (Leo \textbf{este} libro específico).

\end{itemize}


\end{tcolorbox}

\paragraph{C. Restricción de Doble 'KO'} 

Si una oración tiene tanto Objeto Directo como Objeto Indirecto:


\begin{itemize}

  \item El Objeto Indirecto (persona) mantiene el {\hindifont को}.

  \item El Objeto Directo (cosa), aunque sea específico, \textbf{pierde} el {\hindifont को} para evitar la repetición cacofónica.

\end{itemize}


\textbf{Ejemplo:} {\hindifont मैं राम को किताब देता हूँ।} (Doy el libro a Ram.)\newline 
\textit{Incorrecto:} Main Ram ko kitaab ko deta huun.

\subsection{8.3 Postposiciones Grupo 3: के लिए (para) y के साथ (con)}


Estas postposiciones son compuestas y permiten expresar relaciones más complejas:

\subsubsection{8.3.1 के लिए (ke liye) - para}


Esta constricción expresa propósito o destinatario:


\begin{itemize}

  \item {\hindifont यह आपके लिए है।} - Esto es para Ud.

  \item {\hindifont हम भारत के लिए काम करते हैं।} - Trabajamos para la India.

  \item {\hindifont इस काम के लिए धन्यवाद।} - Gracias por este trabajo.

  \item {\hindifont मैं अपने बच्चों के लिए काम करता हूँ।} - Trabajo para mis hijos.

\end{itemize}

\subsubsection{8.3.2 के साथ (ke saath) - con}


Expresa compañía o asociación:


\begin{itemize}

  \item {\hindifont मैं दोस्तों के साथ जाता हूँ।} - Voy con amigos.

  \item {\hindifont वह मेरे साथ था।} - Él estaba conmigo.

  \item {\hindifont हम उनके साथ बाजार गए।} - Fuimos al mercado con ellos.

  \item {\hindifont काम करने के लिए उपकरण के साथ आया।} - Vino con herramientas para trabajar.

\end{itemize}

\subsubsection{8.3.3 Otros Ejemplos de Combinaciones Comunes}


\begin{itemize}

  \item {\hindifont मुझके साथ} (mujhke saath) - Conmigo

  \item {\hindifont तुम्हारे लिए} (tumhaare liye) - Para ti

  \item {\hindifont उनके साथ} (unka saath) - Con ellos

  \item {\hindifont इसके लिए} (iskaa liye) - Por esto

  \item {\hindifont यह तुम्हारे लिए सबसे अच्छा है।} - Ésta es la mejor para ti.

\end{itemize}


\begin{tcolorbox}[colback=blue!5!white,colframe=blue!75!black,title=Nota/Clarificación]

\paragraph{Recordatorio sobre Concordancia con Género y Número} 

Cuando usamos estas estructuras con adjetivos o en oraciones más complejas, recordemos las reglas de concordancia:


\begin{itemize}

  \item Los adjetivos deben concordar con el sustantivo que modifican en género y número

  \item El verbo debe concordar con el sujeto principal de la oración

  \item En el presente habitual, la partícula {\hindifont ता/ती/ते} concuerda con el sujeto

\end{itemize}


\end{tcolorbox}

\subsection{8.4 Comunicación: Expresar Propósito, Compañía, Origen y Destino}


Con las nuevas postposiciones podemos formular frases más complejas:

\subsubsection{8.4.1 Expresiones de Propósito}


\begin{itemize}

  \item {\hindifont मैं दवा लेने के लिए दुकान गया।} - Fui a la tienda para comprar medicina.

  \item {\hindifont वह भाषा सीखने के लिए पढ़ता है।} - Él estudia para aprender el idioma.

  \item {\hindifont हम आपके लिए कुछ अच्छा करेंगे।} - Haremos algo bueno para ti.

  \item {\hindifont यह आपके स्वास्थ्य के लिए अच्छा है।} - Esto es bueno para tu salud.

\end{itemize}

\subsubsection{8.4.2 Expresiones de Compañía}


\begin{itemize}

  \item {\hindifont मैं अपने दोस्त के साथ घूमता हूँ।} - Paseo con mi amigo.

  \item {\hindifont वह मेरी पत्नी के साथ था।} - Él estaba con mi esposa.

  \item {\hindifont हम सब एक साथ काम करते हैं।} - Todos trabajamos juntos.

  \item {\hindifont लोग एक दूसरे के साथ मिलते हैं।} - La gente se encuentra con otros.

\end{itemize}

\subsubsection{8.4.3 Expresiones de Origen y Destino}


\begin{itemize}

  \item {\hindifont मैं भारत से आया हूँ।} - Soy de la India. (Vengo de la India)

  \item {\hindifont वह स्पेन से है।} - Él/ella es de España.

  \item {\hindifont हम तुम्हारे घर से आए।} - Vinimos de tu casa.

  \item {\hindifont वह मेरे पास आया था।} - Él vino a mí.

  \item {\hindifont क्या तुम राम के पास जाओगे?} - ¿Irás a donde Ram?

  \item {\hindifont मैं तुम्हारे पास आऊँगा।} - Vendré a ti.

\end{itemize}

\subsubsection{8.4.4 Diálogos Ejemplares}


\textbf{Diálogo 1 - Planeando una reunión:}


\begin{itemize}

  \item A: {\hindifont क्या तुम आज शाम के लिए कुछ समय निकाल सकते हो?} - ¿Puedes sacar algo de tiempo para esta tarde?

  \item B: {\hindifont हाँ, मैं अपने दोस्त के साथ आ सकता हूँ।} - Sí, puedo venir con mi amigo.

  \item A: {\hindifont बहुत अच्छा! वह मेरे लिए भी जाना चाहता है?} - ¡Muy bien! ¿Él también quiere venir para mí? (¿Él también me quiere conocer?)

  \item B: {\hindifont हाँ, वह आपसे मिलना चाहता है।} - Sí, él quiere conocerle.

  \item A: {\hindifont तो 6 बजे मेरे घर पर मिलते हैं।} - Entonces nos encontramos a las 6 en mi casa.

  \item B: {\hindifont ठीक है। वह तुम्हारे घर से आएगा।} - Bien. Él vendrá de tu casa.

\end{itemize}


\textbf{Diálogo 2 - En la entrevista de trabajo:}


\begin{itemize}

  \item Entrevistador: {\hindifont आप कहाँ से हैं?} - ¿De dónde es Ud.?

  \item Candidato: {\hindifont मैं उत्तर प्रदेश से हूँ।} - Soy de Uttar Pradesh.

  \item Entrevistador: {\hindifont आपके लिए यह काम कैसा होगा?} - ¿Cómo será este trabajo para Ud.?

  \item Candidato: {\hindifont मुझे लगता है कि यह मेरे लिए एक अच्छा मौका होगा।} - Creo que será una buena oportunidad para mí.

  \item Entrevistador: {\hindifont आप किसके साथ काम करना पसंद करेंगे?} - ¿Con quién le gustaría trabajar?

  \item Candidato: {\hindifont मैं उन लोगों के साथ काम करना पसंद करता हूँ जो मेरी मदद कर सकते हैं।} - Me gusta trabajar con gente que pueda ayudarme.

\end{itemize}

\subsubsection{8.4.5 Frases Útiles con Postposiciones}


\begin{itemize}

  \item {\hindifont मैं आपके लिए इंतज़ार कर रहा हूँ।} - Te estoy esperando. (Estoy esperando para ti.)

  \item {\hindifont तुम मेरे लिए बहुत महत्वपूर्ण हो।} - Eres muy importante para mí.

  \item {\hindifont वह इस काम के लिए उत्तरदायी है।} - Él es responsable por este trabajo.

  \item {\hindifont हम तुम्हारे साथ बाजार जा रहे हैं।} - Estamos yendo al mercado contigo.

  \item {\hindifont यह तुम्हारे लिए एक अच्छा अवसर है।} - Ésta es una buena oportunidad para ti.

  \item {\hindifont क्या तुम उसके साथ काम कर सकते हो?} - ¿Puedes trabajar con él?

\end{itemize}


\hrulefill

\section{Unidad 9: Conectores y Preguntas Complejas}


Objetivo: Formar frases compuestas y hacer preguntas sobre causa, modo y tiempo

\subsection{9.1 Conectores y Partículas de Conexión}


Los conectores permiten unir ideas y construir frases más complejas. Son herramientas esenciales para expresar relaciones lógicas entre cláusulas.

\subsubsection{9.1.1 Conectores Fundamentales}


\begin{center}
\begin{tabularx}{\textwidth}{| X | X | X | X | X | X |}
\hline
\textbf{Conector} & \textbf{Devanagari} & \textbf{Transliteración} & \textbf{Traducción} & \textbf{Función} & \textbf{Ejemplo Completo} \ \hline
Pero & {\hindifont लेकिन} & lekin & pero & Contraste/Opuesto & {\hindifont मौसम अच्छा है, लेकिन मैं घर जाना चाहता हूँ।} - El tiempo es bueno, pero quiero ir a casa. \ \hline
Porque & {\hindifont क्योंकि} & kyonki & porque & Causalidad & {\hindifont मैं जल्दी जा रहा हूँ क्योंकि मुझे काम पर जल्दी पहुंचना है।} - Estoy yendo rápido porque necesito llegar temprano al trabajo. \ \hline
Por eso/por lo tanto & {\hindifont इसलिए} & isliye & por eso/por lo tanto & Conclusión/Resultado & {\hindifont बारिश हो रही है, इसलिए मैं घर पर ही रहूँगा।} - Está lloviendo, por eso me quedaré en casa. \ \hline
\end{tabularx}
\end{center}

\subsubsection{9.1.2 Otros Conectores Comunes}


\begin{itemize}

  \item {\hindifont और} (aur) - y: {\hindifont राम और श्याम} (Ram aur Shayam) - Ram y Shyam

  \item {\hindifont या} (ya) - o: {\hindifont चाय या कॉफी} (chaa ya koffee) - Té o café

  \item {\hindifont तो} (to) - entonces, pues: {\hindifont तुम यहाँ रहोगे, तो मैं आऊँगा।} (Tú permanecerás aquí, entonces yo vendré.)

  \item {\hindifont लेकिन} (lekin) - pero: {\hindifont पैसा है, लेकिन खरीदना नहीं चाहता।} (Tengo dinero, pero no quiero comprar.)

  \item {\hindifont हालांकि} (haalaanki) - aunque: {\hindifont हालांकि बीमार था, वह आया।} (Aunque estaba enfermo, él vino.)

  \item {\hindifont चूंकि} (chuunki) - ya que: {\hindifont चूंकि तुम समझदार हो, मैं बता रहा हूँ।} (Ya que eres inteligente, te lo estoy diciendo.)

\end{itemize}

\subsubsection{9.1.3 Expresiones Útiles con "Aur" (और)}


La palabra {\hindifont और} (aur) significa "y", pero también se usa para expresar "más" o "otro" en preguntas y afirmaciones:


\begin{itemize}

  \item {\hindifont और क्या?} (Aur kya?) - ¿Qué más?

  \item {\hindifont और कौन?} (Aur kaun?) - ¿Quién más?

  \item {\hindifont और कुछ?} (Aur kuch?) - ¿Algo más?

  \item {\hindifont और कहीं} (Aur kahin) - En algún otro lugar / A otro sitio

  \item {\hindifont और कोई} (Aur koi) - Alguien más / Algún otro

  \item {\hindifont एक और} (Ek aur) - Uno más

\end{itemize}


\begin{tcolorbox}[colback=yellow!10!white,colframe=orange!75!black,title=Regla de Oro]

\paragraph{Regla Importante: Uso de Conectores} 

Los conectores se colocan:


\begin{itemize}

  \item Entre palabras o frases cuando expresan una relación directa: {\hindifont राम और श्याम}

  \item Entre cláusulas cuando expresan una relación causal, contrastiva o consecuente: {\hindifont क्योंकि..., लेकिन..., इसलिए...}

  \item Al comienzo de una cláusula secundaria para conectar con la cláusula principal

\end{itemize}


\end{tcolorbox}

\subsection{9.2 Interrogativos Avanzados}


Además de los interrogativos básicos ({\hindifont क्या} - qué, {\hindifont कौन} - quién, {\hindifont कहाँ} - dónde), hay otros interrogativos que permiten hacer preguntas más complejas:

\subsubsection{9.2.1 Interrogativos de Tiempo, Causa, Modo y Cantidad}


\begin{center}
\begin{tabularx}{\textwidth}{| X | X | X | X | X | X |}
\hline
\textbf{Interrogativo} & \textbf{Devanagari} & \textbf{Transliteración} & \textbf{Traducción} & \textbf{Función} & \textbf{Ejemplo de Pregunta} \ \hline
Cuándo & {\hindifont कब} & kab & cuándo & Pregunta temporal & {\hindifont आप कब आएंगे?} (¿Cuándo vendrá Ud.?) \ \hline
Por qué & {\hindifont क्यों} & kyon & por qué & Pregunta causal & {\hindifont तुम क्यों देर से आए?} (¿Por qué viniste tarde?) \ \hline
Cómo & {\hindifont कैसे} & kaise & cómo & Pregunta modal & {\hindifont आप यहाँ कैसे पहुँचे?} (¿Cómo llegó Ud. aquí?) \ \hline
Cuánto/a/os & {\hindifont कितना/कितनी/कितने} & kitnaa/kitni/kitne & cuánto/a/os & Pregunta cuantitativa & {\hindifont तुम्हारे पास कितना पैसा है?} (¿Cuánto dinero tienes?) \ \hline
\end{tabularx}
\end{center}

\subsubsection{9.2.2 Uso de los Interrogativos Avanzados}


Estos interrogativos pueden usarse de diferentes maneras:


\begin{itemize}

  \item \textbf{Al comienzo de la oración:} {\hindifont क्यों तुम इस काम को कर रहे हो?} (¿Por qué estás haciendo este trabajo?)

  \item \textbf{En medio de la oración:} {\hindifont तुम इस काम को कैसे कर रहे हो?} (¿Cómo estás haciendo este trabajo?)

  \item \textbf{Combinados con otras palabras:}

\begin{itemize}

  \item {\hindifont कितने समय तक} - ¿Durante cuánto tiempo?

  \item {\hindifont कितना बार} - ¿Cuántas veces?

  \item {\hindifont किस तरह से} - ¿De qué manera?

  \item {\hindifont किस कारण से} - ¿Por qué causa?

\end{itemize}



\end{itemize}

\subsection{9.3 Postposiciones Grupo 4: के बारे में, के ऊपर, के नीचे}


Continuamos con nuevas postposiciones que permiten expresar relaciones espaciales y de tema más complejas:

\subsubsection{9.3.1 Postposiciones Compuestas}


\begin{center}
\begin{tabularx}{\textwidth}{| X | X | X | X | X | X |}
\hline
\textbf{Postposición} & \textbf{Devanagari} & \textbf{Transliteración} & \textbf{Traducción} & \textbf{Uso} & \textbf{Ejemplo} \ \hline
Sobre/Acerca de & {\hindifont के बारे में} & ke baare men & sobre/acercade & Temática & {\hindifont तुम यह किताब के बारे में क्या सोचते हो?} (¿Qué piensas sobre este libro?) \ \hline
Encima de & {\hindifont के ऊपर} & ke upar & encima de & Posición espacial & {\hindifont पुस्तक मेज़ के ऊपर है।} (El libro está encima de la mesa.) \ \hline
Debajo de & {\hindifont के नीचे} & ke niche & debajo de & Posición espacial & {\hindifont कुत्ता बिस्तर के नीचे सो रहा है।} (El perro está durmiendo debajo de la cama.) \ \hline
Ante/Enfrente de & {\hindifont के सामने} & ke saamne & ante/enfrente de & Posición espacial & {\hindifont वह घर के सामने खड़ा है।} (Él está de pie delante de la casa.) \ \hline
Detrás de & {\hindifont के पीछे} & ke piiche & atrás de & Posición espacial & {\hindifont बच्चा माँ के पीछे छिपा है।} (El niño está escondido detrás de su madre.) \ \hline
Cerca de & {\hindifont के पास} & ke paas & cerca de & Proximidad & {\hindifont मेरे घर के पास एक पार्क है।} (Hay un parque cerca de mi casa.) \ \hline
\end{tabularx}
\end{center}

\subsection{9.4 Vocabulario: Adverbios de Tiempo y Meses del Año}


Complementamos el vocabulario de expresiones temporales:

\subsubsection{9.4.1 Adverbios de Tiempo}


\begin{center}
\begin{tabularx}{\textwidth}{| X | X | X | X | X |}
\hline
\textbf{Español} & \textbf{Hindi} & \textbf{Transliteración} & \textbf{Pronunciación} & \textbf{Uso} \ \hline
Hoy & {\hindifont आज} & aaj & AHJ & Este día \ \hline
Mañana & {\hindifont कल} & kal & KAL & El próximo día o ayer (en contexto) \ \hline
Anteayer & {\hindifont परसों} & parson & PAR-sohn & Dos días antes de hoy \ \hline
Ayer & {\hindifont कल} & kal & KAL & El día anterior (contextual) \ \hline
Mañana (día siguiente) & {\hindifont सुबह} कल & subah kal & SUB-ah KAL & Mañana por la mañana \ \hline
Siempre & {\hindifont हमेशा} & hamesha & HA-me-shaa & Constantemente \ \hline
A veces & {\hindifont कभी-कभी} & kabhii-kabhii & KAB-ee KAB-ee & Ocasionalmente \ \hline
Temprano & {\hindifont जल्दी} & jaldi & JAL-dee & Antes del horario previsto \ \hline
Tarde & {\hindifont देर} & der & DER & Después del horario previsto \ \hline
Ahora & {\hindifont अभी} & abhi & AB-hee & En este momento \ \hline
Pronto & {\hindifont जल्द} & jald & JALD & En poco tiempo \ \hline
Siempre & {\hindifont हर वक्त} & har vakta & HAR VAK-ta & Todo el tiempo \ \hline
\end{tabularx}
\end{center}

\subsubsection{9.4.2 Meses del Año en Hindi}


\begin{center}
\begin{tabularx}{\textwidth}{| X | X | X | X | X |}
\hline
\textbf{Número} & \textbf{Mes en Hindi} & \textbf{Transliteración} & \textbf{Pronunciación} & \textbf{Origen} \ \hline
1 & {\hindifont जनवरी} & janvarii & JAN-var-ee & Del inglés "January" \ \hline
2 & {\hindifont फ़रवरी} & farvarii & FAR-var-ee & Del inglés "February" \ \hline
3 & {\hindifont मार्च} & maarch & MAARCH & Del inglés "March" \ \hline
4 & {\hindifont अप्रैल} & aprail & A-pra-ee-L & Del inglés "April" \ \hline
5 & {\hindifont मई} & maee & MA-ee & Del inglés "May" \ \hline
6 & {\hindifont जून} & june & JUNE & Del inglés "June" \ \hline
7 & {\hindifont जुलाई} & julaee & JU-la-ee & Del inglés "July" \ \hline
8 & {\hindifont अगस्त} & agast & A-gast & Del inglés "August" \ \hline
9 & {\hindifont सितंबर} & sitanbar & SIT-an-bar & Del inglés "September" \ \hline
10 & {\hindifont अक्टूबर} & aktubar & AK-too-bar & Del inglés "October" \ \hline
11 & {\hindifont नवंबर} & navanbar & NAA-van-bar & Del inglés "November" \ \hline
12 & {\hindifont दिसंबर} & disanbar & DI-san-bar & Del inglés "December" \ \hline
\end{tabularx}
\end{center}


\begin{tcolorbox}[colback=blue!5!white,colframe=blue!75!black,title=Nota/Clarificación]

\paragraph{Observación Importante sobre los Meses} 

Los meses del año en hindi provienen principalmente del inglés, a diferencia de otros idiomas como el francés o español que tienen sus propias formas. Esto se debe a la influencia del calendario gregoriano y la administración británica colonial.


\end{tcolorbox}

\subsection{9.5 Comunicación: Dar y Pedir Razones, Describir Cómo se Hace Algo, Hablar sobre Temas, Preguntar por Cantidades}


Con los nuevos elementos gramaticales, podemos formular expresiones más complejas:

\subsubsection{9.5.1 Expresiones con Conectores}


\begin{itemize}

  \item {\hindifont मैं पढ़ाई करता हूं, क्योंकि मुझे सफल बनना है।} - Estudio porque quiero tener éxito.

  \item {\hindifont मौसम अच्छा है, लेकिन मैं बाहर नहीं जा रहा हूं।} - El tiempo es bueno, pero no estoy saliendo.

  \item {\hindifont यह बहुत महंगा है, इसलिए मैं खरीदने वाला नहीं हूं।} - Esto es muy caro, por eso no lo compraré.

  \item {\hindifont तुम घर जाओ और आराम करो।} - Ve a casa y descansa.

  \item {\hindifont हालांकि बीमार था, वह काम पर गया।} - Aunque estaba enfermo, fue al trabajo.

\end{itemize}

\subsubsection{9.5.2 Preguntas con Interrogativos Avanzados}


\begin{itemize}

  \item {\hindifont आप कब भारत आए थे?} (¿Cuándo vino Ud. a la India?)

  \item {\hindifont तुम इस काम को कैसे करोगे?} (¿Cómo harás este trabajo?)

  \item {\hindifont यह कितने पैसे का है?} (¿De cuánto dinero es esto? / ¿Cuánto cuesta?)

  \item {\hindifont तुम्हें यह किताब के बारे में क्या लगता है?} (¿Qué opinas sobre este libro?)

  \item {\hindifont तुमने यह काम क्यों नहीं किया?} (¿Por qué no hiciste este trabajo?)

\end{itemize}

\subsubsection{9.5.3 Expresiones con Nuevas Postposiciones}


\begin{itemize}

  \item {\hindifont किताब मेज़ के ऊपर है।} - El libro está encima de la mesa.

  \item {\hindifont कुत्ता बिस्तर के नीचे है।} - El perro está debajo de la cama.

  \item {\hindifont वह घर के सामने खड़ा है।} - Él está de pie frente a la casa.

  \item {\hindifont बच्चा माँ के पीछे छिपा है।} - El niño está escondido detrás de su madre.

  \item {\hindifont यह तुम्हारे बारे में एक अच्छी किताब है।} - Éste es un buen libro sobre ti.

\end{itemize}

\subsubsection{9.5.4 Diálogos Ejemplares}


\textbf{Diálogo 1 - En la oficina:}


\begin{itemize}

  \item Gerente: {\hindifont आपने यह काम क्यों नहीं किया?} - ¿Por qué no hizo Ud. este trabajo?

  \item Empleado: {\hindifont मैं बीमार था, इसलिए नहीं कर सका।} - Estaba enfermo, por eso no pude hacerlo.

  \item Gerente: {\hindifont ठीक है, लेकिन कल समय पर आना।} - Bien, pero mañana llegue a tiempo.

  \item Empleado: {\hindifont हाँ साहब, मैं समय पर आऊंगा।} - Sí señor, llegaré a tiempo.

  \item Gerente: {\hindifont और यह काम आप कैसे करेंगे?} - Y ¿cómo hará Ud. este trabajo?

  \item Empleado: {\hindifont मैं इसे कल दोपहर तक पूरा कर लूंगा।} - Lo completaré para mañana al mediodía.

\end{itemize}


\textbf{Diálogo 2 - En la tienda:}


\begin{itemize}

  \item Cliente: {\hindifont यह कितने का है?} - ¿De cuánto es esto?

  \item Vendedor: {\hindifont यह पचास रुपये का है।} - Éste es de cincuenta rupias.

  \item Cliente: {\hindifont क्या आप कम कीमत पर दे सकते हैं?} - ¿Puede Ud. dármelo a un precio más bajo?

  \item Vendedor: {\hindifont हाँ, मैं चालीस में दे सकता हूं।} - Sí, puedo dárselo en cuarenta.

  \item Cliente: {\hindifont मेरे पास बस तीस है। क्या आप इतने में दे सकते हैं?} - Sólo tengo treinta. ¿Puede Ud. dármelo en eso?

  \item Vendedor: {\hindifont तीस में नहीं, लेकिन बत्तीस में दे सकता हूं।} - No en treinta, pero puedo dárselo en treinta y dos.

\end{itemize}


\hrulefill

\section{Unidad 10: Construcciones con Sujeto Dativo}


Objetivo: Expresar gustos, necesidades y conocimiento usando la estructura de sujeto dativo con को

\subsection{10.1 Estructura de Sujeto Dativo con को (ko)}


En hindi, existen construcciones verbales que en español usan un sujeto directo, pero en hindi utilizan un sujeto en \textbf{caso dativo} (con la postposición {\hindifont को}). Esta es una característica fundamental del sistema gramatical de hindi.


\begin{tcolorbox}[colback=yellow!10!white,colframe=orange!75!black,title=Regla de Oro]

\paragraph{Regla Fundamental: Sujeto Dativo} 

En las construcciones con sujeto dativo:


\begin{itemize}

  \item La persona que \textbf{experimenta} el sentimiento/deseo/necesidad/estado se coloca en el \textbf{caso dativo} con {\hindifont को}.

  \item El objeto sobre el que recae ese sentimiento/estado va en posición de sujeto.

  \item El verbo concuerda con el \textbf{objeto} (que está en posición de sujeto), no con la persona que experimenta el sentimiento.

\end{itemize}


\end{tcolorbox}

\subsubsection{10.1.1 Estructura Básica}


La estructura de las construcciones con sujeto dativo es:


\textbf{[Persona que experimenta] + को + [Objeto sobre el que recae la emoción/estado] + लगता है / याद है / पसंद है / चाहिए ...}

\subsection{10.2 Estructuras Fundamentales con Sujeto Dativo}

\subsubsection{10.2.1 पसंद होना (pasand honaa) - Gustar}


\begin{center}
\begin{tabularx}{\textwidth}{| X | X | X | X | X | X |}
\hline
\textbf{Sujeto (Persona que gusta)} & \textbf{Forma con को} & \textbf{Objeto (Lo que gusta)} & \textbf{Verbo} & \textbf{Ejemplo Completo} & \textbf{Traducción} \ \hline
{\hindifont मैं} (main) - Yo & {\hindifont मुझे} (mujhe) & {\hindifont चाय} (chaa) - té & {\hindifont पसंद है} (pasand hai) & {\hindifont मुझे चाय पसंद है।} & A mí me gusta el té. \ \hline
{\hindifont तू} (tū) - Tú (íntimo) & {\hindifont तुझे} (tujhe) & {\hindifont गाना} (gaanaa) - canción & {\hindifont पसंद है} (pasand hai) & {\hindifont तुझे गाना पसंद है।} & A ti te gusta la canción. \ \hline
{\hindifont तुम} (tum) - Tú (informal) & {\hindifont तुम्हें} (tumhe) & {\hindifont संगीत} (sangeet) - música & {\hindifont पसंद है} (pasand hai) & {\hindifont तुम्हें संगीत पसंद है।} & A ti te gusta la música. \ \hline
{\hindifont आप} (āp) - Usted & {\hindifont आपको} (aapko) & {\hindifont फल} (fal) - fruta & {\hindifont पसंद है} (pasand hai) & {\hindifont आपको फल पसंद है।} & A Ud. le gusta la fruta. \ \hline
{\hindifont यह/वह} (yah/vah) - Él/Ella & {\hindifont उसे} (use) & {\hindifont खेल} (khel) - juego/deporte & {\hindifont पसंद है} (pasand hai) & {\hindifont उसे खेल पसंद है।} & A él/ella le gusta el juego/deporte. \ \hline
{\hindifont हम} (ham) - Nosotros & {\hindifont हमें} (hame) & {\hindifont फिल्म} (film) - película & {\hindifont पसंद है} (pasand hai) & {\hindifont हमें फिल्म पसंद है।} & A nosotros nos gusta la película. \ \hline
{\hindifont ये/वे} (ye/ve) - Ellos/Ellas & {\hindifont उन्हें} (unhe) & {\hindifont पुस्तकें} (pustaken) - libros & {\hindifont पसंद हैं} (pasand hain) & {\hindifont उन्हें पुस्तकें पसंद हैं।} & A ellos les gustan los libros. \ \hline
\end{tabularx}
\end{center}

\subsubsection{10.2.2 चाहिए (chaahie) - Deber/Consejo/Necesidad}


Esta estructura se usa para expresar necesidad, consejo o deber. El sujeto (persona que necesita) va en dativo, y el objeto necesario va en posición de sujeto.


\begin{center}
\begin{tabularx}{\textwidth}{| X | X | X | X | X | X |}
\hline
\textbf{Persona que necesita} & \textbf{Forma con को} & \textbf{Objeto necesario} & \textbf{Verbo} & \textbf{Ejemplo Completo} & \textbf{Traducción} \ \hline
{\hindifont मैं} (main) - Yo & {\hindifont मुझे} (mujhe) & {\hindifont आराम} (aaraam) - descanso & {\hindifont चाहिए} (chaahie) & {\hindifont मुझे आराम करना चाहिए।} & Yo debería descansar. \ \hline
{\hindifont तू} (tū) - Tú (íntimo) & {\hindifont तुझे} (tujhe) & {\hindifont दूध} (doodh) - leche & {\hindifont चाहिए} (chaahie) & {\hindifont तुझे दूध पीना चाहिए।} & Tú deberías beber leche. \ \hline
{\hindifont तुम} (tum) - Tú (informal) & {\hindifont तुम्हें} (tumhe) & {\hindifont पानी} (paanii) - agua & {\hindifont चाहिए} (chaahie) & {\hindifont तुम्हें पानी पीना चाहिए।} & Tú deberías beber agua. \ \hline
{\hindifont आप} (āp) - Usted & {\hindifont आपको} (aapko) & {\hindifont चिकित्सा} (chikitsa) - tratamiento médico & {\hindifont चाहिए} (chaahie) & {\hindifont आपको चिकित्सा की आवश्यकता है।} & Ud. necesita tratamiento médico. \ \hline
{\hindifont यह/वह} (yah/vah) & {\hindifont उसे} (use) & {\hindifont समय} (samay) - tiempo & {\hindifont चाहिए} (chaahie) & {\hindifont उसे समय की आवश्यकता है।} & Él/Ella necesita tiempo. \ \hline
\end{tabularx}
\end{center}

\subsubsection{10.2.3 मालूम होना (maalum honaa) - Saber/Enterarse}


Se usa para expresar conocimiento o percepción de información. La persona que sabe va en dativo.


\begin{center}
\begin{tabularx}{\textwidth}{| X | X | X | X | X | X |}
\hline
\textbf{Persona que sabe} & \textbf{Forma con को} & \textbf{Información conocida} & \textbf{Verbo} & \textbf{Ejemplo Completo} & \textbf{Traducción} \ \hline
{\hindifont मैं} (main) & {\hindifont मुझे} (mujhe) & {\hindifont हिंदी भाषा} (hindi bhaashaa) - idioma hindi & {\hindifont मालूम है} (maalum hai) & {\hindifont मुझे हिंदी भाषा मालूम है।} & Yo sé el idioma hindi. \ \hline
{\hindifont तुम} (tum) & {\hindifont तुम्हें} (tumhe) & {\hindifont यह बात} (yah baat) - este asunto & {\hindifont मालूम है} (maalum hai) & {\hindifont तुम्हें यह बात मालूम है।} & Tú sabes este asunto. \ \hline
{\hindifont आप} (āp) & {\hindifont आपको} (aapko) & {\hindifont सच्चाई} (sacchaii) - la verdad & {\hindifont मालूम है} (maalum hai) & {\hindifont आपको सच्चाई मालूम है।} & Ud. sabe la verdad. \ \hline
\end{tabularx}
\end{center}

\subsubsection{10.2.4 याद होना (yaad honaa) - Recordar}


Expresa la capacidad de recordar algo. La persona que recuerda va en dativo.


\begin{center}
\begin{tabularx}{\textwidth}{| X | X | X | X | X | X |}
\hline
\textbf{Persona que recuerda} & \textbf{Forma con को} & \textbf{Lo recordado} & \textbf{Verbo} & \textbf{Ejemplo Completo} & \textbf{Traducción} \ \hline
{\hindifont मैं} (main) & {\hindifont मुझे} (mujhe) & {\hindifont यह गाना} (yah gaanaa) - esta canción & {\hindifont याद है} (yaad hai) & {\hindifont मुझे यह गाना याद है।} & Yo recuerdo esta canción. \ \hline
{\hindifont तुम} (tum) & {\hindifont तुम्हें} (tumhe) & {\hindifont मेरा नाम} (meraa naam) - mi nombre & {\hindifont याद है} (yaad hai) & {\hindifont तुम्हें मेरा नाम याद है?} & ¿Recuerdas mi nombre? \ \hline
{\hindifont आप} (āp) & {\hindifont आपको} (aapko) & {\hindifont यह तारीख़} (yah taariikh) - esta fecha & {\hindifont याद है} (yaad hai) & {\hindifont आपको यह तारीख़ याद है?} & ¿Recuerda Ud. esta fecha? \ \hline
\end{tabularx}
\end{center}


\begin{tcolorbox}[colback=blue!5!white,colframe=blue!75!black,title=Nota/Clarificación]

\paragraph{Comparación con el Español} 

Veamos un ejemplo de cómo cambia la estructura:


\begin{itemize}

  \item \textbf{Español:} "A mí me gusta el té" - El sujeto es "el té", pero la persona que experimenta el gusto es "mí"

  \item \textbf{Hindi:} {\hindifont मुझे चाय पसंद है।} - Literalmente: "A mí (mujhe) té (chaa) gustar (pasand hai)". Aquí, {\hindifont चाय} es el sujeto gramatical y {\hindifont मुझे} es el sujeto experiencial en dativo.

\end{itemize}


\end{tcolorbox}

\subsection{10.3 Vocabulario: Comida, Frutas, Verduras y Bebidas}


Este vocabulario es ideal para practicar las estructuras de sujeto dativo:

\subsubsection{10.3.1 Comida}


\begin{center}
\begin{tabularx}{\textwidth}{| X | X | X | X | X |}
\hline
\textbf{Español} & \textbf{Hindi} & \textbf{Transliteración} & \textbf{Género} & \textbf{Uso en Frases Dativas} \ \hline
Comida & {\hindifont खाना} & khaanaa & Masculino & {\hindifont मुझे खाना पसंद है।} - Me gusta la comida. \ \hline
Arroz & {\hindifont चावल} & chaaval & Masculino & {\hindifont उसे चावल खाना चाहिए।} - Él debería comer arroz. \ \hline
Pan (roti) & {\hindifont रोटी} & rotii & Femenino & {\hindifont तुम्हें रोटी पसंद है?} - ¿Te gusta el pan? \ \hline
Lentejas & {\hindifont दाल} & daal & Femenino & {\hindifont हमें दाल खानी चाहिए।} - Debemos comer lentejas. \ \hline
Verdura & {\hindifont सब्ज़ी} & sabzii & Femenino & {\hindifont ये सब्ज़ियाँ मुझे पसंद हैं।} - Estas verduras me gustan. \ \hline
Carne & {\hindifont माँस} & maas & Masculino & {\hindifont उन्हें माँस नहीं खाना चाहिए।} - Ellos no deberían comer carne. \ \hline
\end{tabularx}
\end{center}

\subsubsection{10.3.2 Frutas}


\begin{center}
\begin{tabularx}{\textwidth}{| X | X | X | X | X |}
\hline
\textbf{Español} & \textbf{Hindi} & \textbf{Transliteración} & \textbf{Género} & \textbf{Uso en Frases Dativas} \ \hline
Manzana & {\hindifont सेब} & seb & Masculino & {\hindifont मुझे सेब पसंद है।} - Me gusta la manzana. \ \hline
Plátano & {\hindifont केला} & kelaa & Masculino & {\hindifont मुझे केला खाना चाहिए।} - Debería comer plátano. \ \hline
Naranja & {\hindifont संतरा} & santraa & Masculino & {\hindifont तुम्हें संतरा पसंद है?} - ¿Te gusta la naranja? \ \hline
Uva & {\hindifont अंगूर} & angoor & Masculino & {\hindifont उसे अंगूर पसंद हैं।} - Él/Ella le gustan las uvas. \ \hline
Mango & {\hindifont आम} & aam & Masculino & {\hindifont तुम्हें आम याद है?} - ¿Recuerdas el mango? \ \hline
\end{tabularx}
\end{center}

\subsubsection{10.3.3 Verduras}


\begin{center}
\begin{tabularx}{\textwidth}{| X | X | X | X | X |}
\hline
\textbf{Español} & \textbf{Hindi} & \textbf{Transliteración} & \textbf{Género} & \textbf{Uso en Frases Dativas} \ \hline
Zanahoria & {\hindifont गाजर} & gaajar & Femenino & {\hindifont तुम्हें गाजर खानी चाहिए।} - Deberías comer zanahorias. \ \hline
Pimiento & {\hindifont मिर्च} & mirch & Femenino & {\hindifont उन्हें मिर्च पसंद नहीं है।} - Ellos no tienen gusto por el pimiento. \ \hline
Tomate & {\hindifont टमाटर} & tamaatar & Masculino & {\hindifont हमें टमाटर खरीदना है।} - Debemos comprar tomates. (No dativo, pero ejemplo real) \ \hline
Cebolla & {\hindifont प्याज़} & pyaaz & Femenino & {\hindifont मुझे प्याज़ काटनी है।} - Necesito cortar la cebolla. \ \hline
Espinacas & {\hindifont पालक} & paalak & Femenino & {\hindifont आपको पालक खानी चाहिए।} - Ud. debería comer espinacas. \ \hline
\end{tabularx}
\end{center}

\subsubsection{10.3.4 Bebidas}


\begin{center}
\begin{tabularx}{\textwidth}{| X | X | X | X | X |}
\hline
\textbf{Español} & \textbf{Hindi} & \textbf{Transliteración} & \textbf{Género} & \textbf{Uso en Frases Dativas} \ \hline
Agua & {\hindifont पानी} & paanii & Masculino & {\hindifont मुझे पानी पीना चाहिए।} - Debería beber agua. \ \hline
Té & {\hindifont चाय} & chaa & Femenino & {\hindifont तुम्हें चाय पसंद है।} - Te gusta el té. \ \hline
Leche & {\hindifont दूध} & doodh & Masculino & {\hindifont मुझे दूध पीना चाहिए।} - Debería beber leche. \ \hline
Jugo & {\hindifont जूस} & jus & Masculino & {\hindifont मुझे जूस पसंद है।} - Me gusta el jugo. \ \hline
Café & {\hindifont कॉफ़ी} & kaafee & Femenino & {\hindifont उसे कॉफ़ी पीनी चाहिए।} - Él/Ella debería beber café. \ \hline
\end{tabularx}
\end{center}

\subsection{10.4 Comunicación: Expresar Gustos, Necesidades y Dar Consejos}


Ahora que conocemos las construcciones con sujeto dativo, veamos cómo usarlas en contextos comunicativos:

\subsubsection{10.4.1 Expresar Gustos y Disgustos}


\begin{itemize}

  \item {\hindifont मुझे चाय पसंद है।} (Mujhe chaa pasand hai.) - Me gusta el té.

  \item {\hindifont मुझे कॉफ़ी पसंद नहीं है।} (Mujhe kaafee pasand nahin hai.) - No me gusta el café.

  \item {\hindifont तुम्हें फल खाने चाहिए।} (Tumhe fal khaane chaahie.) - Deberías comer fruta.

  \item {\hindifont उन्हें यह किताबें पसंद हैं।} (Unhe ye kitaaben pasand hain.) - Les gustan estos libros.

  \item {\hindifont हमें नई फिल्म देखनी चाहिए।} (Hame nayi film dekhnaa chaahie.) - Debemos ver una nueva película.

\end{itemize}

\subsubsection{10.4.2 Pedir y Ofrecer en un Restaurante}


\begin{itemize}

  \item Cliente: {\hindifont मुझे चावल और दाल चाहिए।} (Mujhe chaaval aur daal chaahie.) - Quiero arroz y lentejas.

  \item Cliente: {\hindifont क्या मुझे थोड़ा पानी मिल सकता है?} (Kya mujhe thoda paanii mil saktaa hai?) - ¿Me podría traer un poco de agua?

  \item Garzón: {\hindifont जी हाँ, मैं अभी लाता हूँ।} (Jii haan, main abhii laata huun.) - Sí, se lo traigo ahora.

  \item Garzón: {\hindifont आपको कुछ और चाहिए?} (Aapko kuch aur chaahie?) - ¿Necesita Ud. algo más?

  \item Cliente: {\hindifont नहीं, धन्यवाद। मुझे बिल चाहिए।} (Nahin, dhanyavaad. Mujhe bil chaahie.) - No, gracias. Quiero la cuenta.

\end{itemize}

\subsubsection{10.4.3 Dar Consejos y Expresar Necesidades}


\begin{itemize}

  \item {\hindifont तुम्हें आराम करना चाहिए।} (Tumhe aaraam karnaa chaahie.) - Deberías descansar.

  \item {\hindifont हमें समय पर काम पर जाना चाहिए।} (Hame samay par kaam par jaanaa chaahie.) - Debemos ir al trabajo a tiempo.

  \item {\hindifont उसे डॉक्टर के पास जाना चाहिए।} (Use doctor ke paas jaanaa chaahie.) - Él/Ella debería ir al doctor.

  \item {\hindifont आपको यहाँ ध्यान देना चाहिए।} (Aapko yahan dhyaan denaa chaahie.) - Ud. debería prestar atención aquí.

  \item {\hindifont हमें अपने अध्ययन पर ध्यान केंद्रित करना चाहिए।} (Hame apne adhyayan par dhyaan kendrit karnaa chaahie.) - Debemos concentrarnos en nuestros estudios.

\end{itemize}

\subsubsection{10.4.4 Preguntas Comunes con Sujeto Dativo}


\begin{itemize}

  \item {\hindifont तुम्हें क्या पसंद है?} (Tumhe kya pasand hai?) - ¿Qué te gusta?

  \item {\hindifont आपको कौन सी किताब पसंद है?} (Aapko kaun si kitaab pasand hai?) - ¿Cuál libro le gusta a Ud.?

  \item {\hindifont तुम्हें क्या खाना चाहिए?} (Tumhe kya khaana chaahie?) - ¿Qué deberías comer?

  \item {\hindifont उसे क्या करना चाहिए?} (Use kya karnaa chaahie?) - ¿Qué debería hacer él/ella?

  \item {\hindifont तुम्हें यह गाना याद है?} (Tumhe yah gaanaa yaad hai?) - ¿Recuerdas esta canción?

\end{itemize}

\subsubsection{10.4.5 Diálogos Ejemplares}


\textbf{Diálogo 1 - En la cafetería:}


\begin{itemize}

  \item Empleado: {\hindifont आपको क्या चाहिए?} (Aapko kya chaahie? - ¿Qué necesita Ud.?)

  \item Cliente: {\hindifont मुझे एक कप चाय चाहिए।} (Mujhe ek kap chaa chaahie. - Necesito una taza de té.)

  \item Empleado: {\hindifont मिलक या नहीं?} (Milk ya nahin? - ¿Con leche o sin?)

  \item Cliente: {\hindifont हाँ, मुझे दूध याद है।} (Haan, mujhe doodh yaad hai. - Sí, recuerdo la leche.)

  \item Empleado: {\hindifont अच्छा, और कुछ?} (Acchhaa, aur kuch? - Bien, ¿algo más?)

  \item Cliente: {\hindifont तुम्हें यह मिठाई पसंद है?} (Tumhe yah mitaaii pasand hai? - ¿Te gusta este dulce?)

  \item Empleado: {\hindifont हाँ, हमें सबको मिठाई पसंद है।} (Haan, hame sabko mitaaii pasand hai. - Sí, a todos nosotros nos gustan los dulces.)

\end{itemize}


\textbf{Diálogo 2 - Conversación sobre hábitos alimenticios:}


\begin{itemize}

  \item A: {\hindifont तुम्हें कौन सा फल पसंद है?} (Tumhe kaun sa fal pasand hai? - ¿Qué fruta te gusta?)

  \item B: {\hindifont मुझे सेब और केला पसंद है।} (Mujhe seb aur kelaa pasand hai. - Me gustan las manzanas y los plátanos.)

  \item A: {\hindifont और सब्ज़ियाँ?} (Aur sabziyan? - ¿Y las verduras?)

  \item B: {\hindifont तुम्हें सब्ज़ियाँ खानी चाहिए।} (Tumhe sabziyan khaanii chaahie. - Deberías comer verduras.)

  \item A: {\hindifont हाँ, लेकिन मुझे पालक पसंद नहीं है।} (Haan, lekin mujhe paalak pasand nahin hai. - Sí, pero no me gusta la espinaca.)

  \item B: {\hindifont उसे कैसे पकाना मालूम है?} (Use kaise pakaa naa maalum hai? - ¿Sabes cómo cocinarla?)

  \item A: {\hindifont मुझे नहीं मालूम है।} (Mujhe nahin maalum hai. - No lo sé.)

\end{itemize}


\hrulefill

\section{Unidad 11: El Imperativo (Todos los Registros), Futuro Simple y Verbo Modal 'saknā'}


Objetivo: Dominar las diferentes formas del imperativo según el nivel de formalidad, expresar acciones futuras y capacidad/habilidad con el verbo modal {\hindifont सकना}

\subsection{11.1 El Imperativo en Hindi}


El imperativo en hindi varía según el nivel de formalidad, respeto o cercanía con la persona a la que se dirige. Hay tres niveles principales:

\subsubsection{11.1.1 Imperativo Formal para आप (Aap) - इए (ie) / कीजिए (kijiye)}


Este es el nivel más respetuoso. Para formar el imperativo formal para {\hindifont आप}, hay dos formas comunes:


\begin{enumerate}

  \item Usando el sufijo {\hindifont -इए} ({\hindifont -ie}) añadido a la raíz del verbo o a la forma intransitiva del verbo (terminada en {\hindifont -ना}):


\begin{center}
\begin{tabularx}{\textwidth}{| X | X | X | X | X | X |}
\hline
\textbf{Verbo Infinitivo} & \textbf{Raíz} & \textbf{Forma Imperativa Formal} & \textbf{Transliteración} & \textbf{Ejemplo Completo} & \textbf{Traducción} \ \hline
{\hindifont जाना} (jaanaa) - ir & जा (jaa) & {\hindifont जाइए} & jaaiye & {\hindifont कृपया आप जाइए।} & Por favor, Ud. vaya. \ \hline
{\hindifont बैठना} (baithnaa) - sentarse & बैठ (baith) & {\hindifont बैठिए} & baithiye & {\hindifont कृपया बैठिए।} & Por favor, siéntese Ud. \ \hline
{\hindifont लिखना} (likhnaa) - escribir & लिख (likh) & {\hindifont लिखिए} & likhiye & {\hindifont यह नाम लिखिए।} & Escriba este nombre. \ \hline
{\hindifont देना} (denaa) - dar & दे (de) & {\hindifont दीजिए} & diijiye & {\hindifont कृपया किताब दीजिए।} & Por favor, déme el libro. \ \hline
{\hindifont लेना} (lenaa) - tomar, recibir & लो (lo) & {\hindifont लीजिए} & liijiye & {\hindifont यह पेन लीजिए।} & Coja este bolígrafo. \ \hline
\end{tabularx}
\end{center}

\end{enumerate}


\begin{enumerate}

  \item Usando la forma irregular pero común {\hindifont -कीजिए} ({\hindifont -kijiye}) con ciertos verbos:

\end{enumerate}


\begin{center}
\begin{tabularx}{\textwidth}{| X | X | X | X |}
\hline
\textbf{Verbo} & \textbf{Forma Regular} & \textbf{Forma Irregular con कीजिए} & \textbf{Traducción} \ \hline
{\hindifont करना} (karnaa) - hacer & {\hindifont करिए} (kariye) & {\hindifont कीजिए} (kiijiye) & Hágalo Ud./Hágala Ud. \ \hline
{\hindifont देना} (denaa) - dar & {\hindifont दीजिए} (diijiye) & {\hindifont दीजिए} (diijiye) & Démelo Ud. \ \hline
{\hindifont लेना} (lenaa) - tomar & {\hindifont लीजिए} (liijiye) & {\hindifont लीजिए} (liijiye) & Tómelo Ud. \ \hline
{\hindifont पीना} (piinaa) - beber & {\hindifont पीजिए} (piijiye) & {\hindifont पीजिए} (piijiye) & Bébalo Ud. \ \hline
\end{tabularx}
\end{center}


\begin{tcolorbox}[colback=yellow!10!white,colframe=orange!75!black,title=Regla de Oro]

\paragraph{Regla de Oro: Imperativo Formal} 

El imperativo formal {\hindifont कीजिए} es especialmente común con verbos cortos como {\hindifont कर}, {\hindifont दे}, {\hindifont ले}, {\hindifont पी}, {\hindifont खा} (comer), etc. Es una forma más refinada y segura de usar.


\end{tcolorbox}

\subsubsection{11.1.2 Imperativo Informal para तुम (Tum) - प्रत्यय -ओ (praatyay -o)}


Este es el nivel estándar para amigos y conocidos. Se forma añadiendo {\hindifont -ओ} a la raíz del verbo:


\begin{itemize}

  \item {\hindifont जाना} (jaanaa) → {\hindifont जा} (jaa) + {\hindifont -ओ} = {\hindifont जाओ} (jaao) - Ve (a ti)

  \item {\hindifont बैठना} (baithnaa) → {\hindifont बैठ} (baith) + {\hindifont -ओ} = {\hindifont बैठो} (baitho) - Siéntate (a ti)

  \item {\hindifont लिखना} (likhnaa) → {\hindifont लिख} (likh) + {\hindifont -ओ} = {\hindifont लिखो} (likho) - Escribe (a ti)

  \item {\hindifont देना} (denaa) → {\hindifont दे} (de) + {\hindifont -ओ} = {\hindifont दो} (do) - Da (a ti)

  \item {\hindifont लेना} (lenaa) → {\hindifont ले} (le) + {\hindifont -ओ} = {\hindifont लो} (lo) - Toma (a ti)

\end{itemize}

\subsubsection{11.1.3 Imperativo Íntimo para तू (Tū) - Raíz del Verbo}


Este es el nivel más íntimo y familiar, usado con niños o personas muy cercanas. Simplemente se usa la raíz del verbo:


\begin{itemize}

  \item {\hindifont जाना} (jaanaa) → {\hindifont जा} (jaa) - Ve

  \item {\hindifont बैठना} (baithnaa) → {\hindifont बैठ} (baith) - Siéntate

  \item {\hindifont लिखना} (likhnaa) → {\hindifont लिख} (likh) - Escribe

  \item {\hindifont देना} (denaa) → {\hindifont दे} (de) - Da

  \item {\hindifont लेना} (lenaa) → {\hindifont ले} (le) - Toma

\end{itemize}

\subsubsection{11.1.4 Negación del Imperativo}


La negación del imperativo también varía según el nivel de formalidad:


\begin{itemize}

  \item \textbf{Para तू y तुम:} Con {\hindifont मत} (mat) antes del verbo: {\hindifont मत जा} (mat jaa - no vayas), {\hindifont मत खा} (mat khaa - no comas)

  \item \textbf{Para आप:} Con {\hindifont न} (na) antes del verbo: {\hindifont न जाइए} (na jaaiye - no vaya Ud.)

\end{itemize}

\subsection{11.2 Futuro Simple en Hindi}


El futuro simple se forma con la raíz del verbo + partículas de futuro + verbo {\hindifont होना} (honaa) conjugado. Es fundamental para expresar planes e intenciones:

\subsubsection{11.2.1 Formación del Futuro Simple}


La estructura básica es: Raíz del verbo + partícula de futuro + verbo {\hindifont होना} conjugado

\subsubsection{11.2.2 Partículas de Futuro}


\begin{center}
\begin{tabularx}{\textwidth}{| X | X | X | X | X | X | X |}
\hline
\textbf{Sujeto} & \textbf{Partícula Futura Masc. Sing.} & \textbf{Partícula Futura Fem. Sing.} & \textbf{Partícula Futura Plural} & \textbf{Verbo होना Conjugado} & \textbf{Ejemplo Completo} & \textbf{Traducción} \ \hline
{\hindifont मैं} (main) - Yo (masc.) & {\hindifont -ऊँगा} (-uunga) & {\hindifont -ऊँगी} (-uungi) & {\hindifont -एँगे} (-enge) & {\hindifont हूँ} (huun) & {\hindifont मैं कल जाऊँगा।} & Yo iré mañana. (masc.) \ \hline
{\hindifont मैं} (main) - Yo (fem.) & {\hindifont -ऊँगा} (-uunga) & {\hindifont -ऊँगी} (-uungi) & {\hindifont -एँगी} (-engi) & {\hindifont हूँ} (huun) & {\hindifont मैं कल जाऊँगी।} & Yo iré mañana. (fem.) \ \hline
{\hindifont तू} (tū) - Tú (íntimo) & {\hindifont -एगा} (-ega) & {\hindifont -एगी} (-egi) & {\hindifont -एगे} (-ege) & {\hindifont है} (hai) & {\hindifont तू कल आएगा।} & Tú vendrás mañana. (masc.) \ \hline
{\hindifont तुम} (tum) - Tú (informal) & {\hindifont -ओगे} (-oge) & {\hindifont -ओगी} (-ogi) & {\hindifont -ओगे} (-oge) & {\hindifont हो} (ho) & {\hindifont तुम कल आओगे।} & Tú vendrás mañana. \ \hline
{\hindifont आप} (āp) - Usted & {\hindifont -एँगे} (-enge) & {\hindifont -एँगी} (-engi) & {\hindifont -एँगे} (-enge) & {\hindifont हैं} (hain) & {\hindifont आप कल आएँगे।} & Ud. vendrá mañana. \ \hline
{\hindifont यह/वह} (yah/vah) - Él/Ella & {\hindifont -एगा} (-ega) & {\hindifont -एगी} (-egi) & {\hindifont -एगे} (-ege) & {\hindifont है} (hai) & {\hindifont वह कल जाएगा।} & Él irá mañana. \ \hline
{\hindifont हम} (ham) - Nosotros/as & {\hindifont -एँगे} (-enge) & {\hindifont -एँगी} (-engi) & {\hindifont -एँगे} (-enge) & {\hindifont हैं} (hain) & {\hindifont हम कल जाएँगे।} & Nosotros iremos mañana. \ \hline
{\hindifont ये/वे} (ye/ve) - Ellos/as & {\hindifont -एँगे} (-enge) & {\hindifont -एँगी} (-engi) & {\hindifont -एँगे} (-enge) & {\hindifont हैं} (hain) & {\hindifont वे कल आएँगे।} & Ellos vendrán mañana. \ \hline
\end{tabularx}
\end{center}

\subsubsection{11.2.3 Ejemplos de Frases con Futuro Simple}


\begin{itemize}

  \item {\hindifont मैं कल भारत जाऊँगा।} (Main kal bharat jaunngaa.) - Yo iré a la India mañana. (masc.)

  \item {\hindifont तुम अगले हफ्ते यहाँ आओगे।} (Tum agle hafte yahan aao.ge.) - Tú vendrás aquí la próxima semana.

  \item {\hindifont आप कब घर जाएँगे?} (Aap kab ghar jaenge.) - ¿Cuándo irá Ud. a casa?

  \item {\hindifont वह कल ऑफिस नहीं जाएगा।} (Vah kal office nahin jayega.) - Él no irá a la oficina mañana.

  \item {\hindifont हम जल्दी यह काम करेंगे।} (Ham jaldi yah kaam karenge.) - Nosotros haremos este trabajo pronto.

\end{itemize}


\begin{tcolorbox}[colback=blue!5!white,colframe=blue!75!black,title=Nota/Clarificación]

\paragraph{Observación Importante sobre el Futuro} 

En hindi, el futuro se expresa con mucha más certeza que en español. No hay diferencia entre "iré" y "voy a ir". El verbo en futuro simple indica claramente que la acción ocurrirá en el futuro con certeza.


\end{tcolorbox}

\subsection{11.3 Verbo Modal सकना (saknaa) - Poder/Capacidad}


El verbo modal {\hindifont सकना} (saknaa) expresa habilidad, capacidad o posibilidad de realizar una acción. Se conjuga como un verbo terminado en {\hindifont -आ} ({\hindifont a}).

\subsubsection{11.3.1 Conjugación de सकना (saknaa)}


\begin{center}
\begin{tabularx}{\textwidth}{| X | X | X | X | X |}
\hline
\textbf{Sujeto} & \textbf{Forma del Verbo} & \textbf{Transliteración} & \textbf{Ejemplo con Infinitivo} & \textbf{Traducción} \ \hline
{\hindifont मैं} (main) - Yo & {\hindifont सकता/सकती हूँ} & saktaa/saktii huun & {\hindifont मैं हिंदी बोल सकता हूँ।} & Puedo hablar hindi. (masc.) \ \hline
{\hindifont तू} (tū) - Tú (íntimo) & {\hindifont सकता/सकती है} & saktaa/saktii hai & {\hindifont तू यह काम कर सकता है।} & Tú puedes hacer este trabajo. (masc.) \ \hline
{\hindifont तुम} (tum) - Tú (informal) & {\hindifont सकता/सकती हो} & saktaa/saktii ho & {\hindifont तुम यह पढ़ सकते हो।} & Tú puedes leer esto. (masc.pl.) \ \hline
{\hindifont आप} (aap) - Usted & {\hindifont सकते/सकती हैं} & sakte/sakti hain & {\hindifont आप यह समझ सकते हैं।} & Ud. puede entender esto. \ \hline
{\hindifont यह/वह} (yah/vah) - Él/Ella & {\hindifont सकता/सकती है} & saktaa/saktii hai & {\hindifont वह तैर सकता है।} & Él puede nadar. \ \hline
{\hindifont हम} (ham) - Nosotros/as & {\hindifont सकते/सकती हैं} & sakte/sakti hain & {\hindifont हम यह कर सकते हैं।} & Nosotros podemos hacer esto. \ \hline
{\hindifont ये/वे} (ye/ve) - Ellos/as & {\hindifont सकते/सकती हैं} & sakte/sakti hain & {\hindifont वे तेज़ दौड़ सकते हैं।} & Ellos pueden correr rápido. \ \hline
\end{tabularx}
\end{center}

\subsubsection{11.3.2 Uso de सकना (saknaa) con Infinitivo}


Para formar la estructura de habilidad o capacidad, se coloca la raíz del verbo principal + {\hindifont सकना} en la forma conjugada:


\begin{itemize}

  \item {\hindifont बोलना} (bolnaa - hablar) → {\hindifont बोल सकना} (bol saknaa - poder hablar)

  \item {\hindifont करना} (karnaa - hacer) → {\hindifont कर सकना} (kar saknaa - poder hacer)

  \item {\hindifont पढ़ना} (padhnaa - leer/estudiar) → {\hindifont पढ़ सकना} (padh saknaa - poder leer)

  \item {\hindifont चलना} (chalnaa - caminar) → {\hindifont चल सकना} (chal saknaa - poder caminar)

  \item {\hindifont लिखना} (likhnaa - escribir) → {\hindifont लिख सकना} (likh saknaa - poder escribir)

\end{itemize}

\subsection{11.4 Postposiciones Grupo 5}


Complementamos el vocabulario de postposiciones con el grupo 5, que permite expresar ubicaciones específicas y alcance:

\subsubsection{11.4.1 Nuevas Postposiciones}


\begin{center}
\begin{tabularx}{\textwidth}{| X | X | X | X | X | X | X |}
\hline
\textbf{Postposición} & \textbf{Devanagari} & \textbf{Transliteración} & \textbf{Traducción} & \textbf{Uso} & \textbf{Ejemplo} & \textbf{Traducción} \ \hline
Al lado de & {\hindifont के पास} & ke paas & al lado de, cerca de & Proximidad & {\hindifont घर के पास बाजार है।} & Al lado de la casa hay un mercado. \ \hline
Enfrente de & {\hindifont के सामने} & ke saamne & enfrente de & Posición opuesta & {\hindifont दुकान के सामने बस स्टॉप है।} & Enfrente de la tienda hay una parada de autobús. \ \hline
Detrás de & {\hindifont के पीछे} & ke piiche & detrás de & Posición trasera & {\hindifont पेड़ के पीछे लड़का छिपा है।} & Detrás del árbol hay un niño escondido. \ \hline
Hasta & {\hindifont ... तक} & ...tak & hasta & Límite espacial/temporal & {\hindifont मैं छह बजे तक ऑफिस में रहूँगा।} & Me quedaré en la oficina hasta las seis. \ \hline
Alrededor de & {\hindifont के आसपास} & ke aaspaas & alrededor de & Entorno & {\hindifont मेज़ के आसपास कुर्सियाँ हैं।} & Alrededor de la mesa hay sillas. \ \hline
Entre & {\hindifont के बीच} & ke beech & entre & Ubicación media & {\hindifont लड़कों के बीच एक लड़की है।} & Entre los chicos hay una chica. \ \hline
\end{tabularx}
\end{center}

\subsection{11.5 Vocabulario: Planificación y Tiempo Futuro}


Complementamos con vocabulario relacionado con la planificación y el tiempo futuro:

\subsubsection{11.5.1 Expresiones Temporales Futuras}


\begin{center}
\begin{tabularx}{\textwidth}{| X | X | X | X | X |}
\hline
\textbf{Español} & \textbf{Hindi} & \textbf{Transliteración} & \textbf{Uso} & \textbf{Ejemplo} \ \hline
Mañana & {\hindifont कल} & kal & Tiempo futuro & {\hindifont मैं कल आऊंगा।} - Yo vendré mañana. \ \hline
La próxima semana & {\hindifont अगले हफ्ते} & agle hafte & Tiempo futuro & {\hindifont अगले हफ्ते हम बाहर जाएँगे।} - La próxima semana saldremos. \ \hline
El próximo mes & {\hindifont अगले महीने} & agle maheene & Tiempo futuro & {\hindifont अगले महीने परीक्षा है।} - El próximo mes hay examen. \ \hline
El próximo año & {\hindifont अगले साल} & agle saal & Tiempo futuro & {\hindifont अगले साल यात्रा करेंगे।} - El próximo año viajaremos. \ \hline
Pronto & {\hindifont जल्दी} & jaldi & Temporalidad & {\hindifont वह जल्दी आएगा।} - Él vendrá pronto. \ \hline
Después & {\hindifont बाद में} & baad men & Secuencia temporal & {\hindifont काम के बाद में आराम करूँगा।} - Después del trabajo descansaré. \ \hline
Antes & {\hindifont पहले} & pahle & Secuencia temporal & {\hindifont खाने से पहले हाथ धोएँ।} - Lávese las manos antes de comer. \ \hline
Plan & {\hindifont योजना} & yojnaa & Planeación & {\hindifont हमारी योजना अच्छी है।} - Nuestro plan es bueno. \ \hline
Decidir & {\hindifont तय करना} & taay karnnaa & Acción & {\hindifont हम तय करेंगे।} - Decidiremos. \ \hline
Empezar & {\hindifont शुरू करना} & shuroo karnaa & Acción & {\hindifont हम नई चीज़ शुरू करेंगे।} - Empezaremos algo nuevo. \ \hline
\end{tabularx}
\end{center}

\subsubsection{11.5.2 Verbos de Acción e Instrucciones}


\begin{itemize}

  \item {\hindifont देना} (denaa) - dar

  \item {\hindifont लेना} (lenaa) - tomar/recibir

  \item {\hindifont जाना} (jaanaa) - ir

  \item {\hindifont आना} (aanaa) - venir

  \item {\hindifont करना} (karnaa) - hacer

  \item {\hindifont कहना} (kahnnaa) - decir

  \item {\hindifont सुनना} (sunnaa) - escuchar

  \item {\hindifont लाना} (laanaa) - traer

  \item {\hindifont भेजना} (bhejnaa) - enviar

  \item {\hindifont लौटना} (lautnaa) - regresar

\end{itemize}

\subsection{11.6 Comunicación: Dar Direcciones, Pedir Favores y Expresar Planes}


Con los nuevos elementos gramaticales, podemos formular expresiones más complejas:

\subsubsection{11.6.1 Frases para Dar y Pedir Direcciones}


\begin{itemize}

  \item {\hindifont सीधे जाओ।} (Siidhe jaao.) - Siga recto.

  \item {\hindifont यहाँ से सीधा जाओ।} (Yahan se seedhaa jaao.) - Desde aquí siga recto.

  \item {\hindifont दायाँ मुड़ो।} (Daayaañ muRoo.) - Gire a la derecha.

  \item {\hindifont बायाँ मुड़ो।} (Baayaañ muRoo.) - Gire a la izquierda.

  \item {\hindifont कृपया मेरी सहायता करें।} (Krupaya merii sahaayata karen.) - Por favor, ayúdeme.

  \item {\hindifont क्या आप मेरी कुछ सहायता कर सकते हैं?} (Kya aap merii kuchh sahaayata kar sakte hain?) - ¿Puede Ud. ayudarme en algo?

  \item {\hindifont कृपया यह पुस्तक उसे दे दें।} (Krupaya yah pustak use de den.) - Por favor, entréguele este libro.

  \item {\hindifont बस स्टॉप कहाँ है?} (Bus stop kahaan hai?) - ¿Dónde está la parada de autobús?

\end{itemize}

\subsubsection{11.6.2 Expresar Planes e Intenciones Futuras}


\begin{itemize}

  \item {\hindifont मैं कल भारत जाऊँगा।} (Main kal bharat jaunngaa.) - Yo iré a la India mañana.

  \item {\hindifont हम अगले महीने यात्रा करेंगे।} (Ham agle maheene yaatra karenge.) - Viajaremos el próximo mes.

  \item {\hindifont मैं इस साल शादी करूँगा।} (Main is saal shaadi karunngaa.) - Me casaré este año.

  \item {\hindifont वह अपनी पढ़ाई पूरी करेगा।} (Vah apni padhaai poori karega.) - Él completará sus estudios.

  \item {\hindifont हम इस योजना को लागू करेंगे।} (Ham is yojnaa ko laagu karenge.) - Pondremos en marcha este plan.

  \item {\hindifont मैं नई नौकरी शुरू करूँगा।} (Main nyi naukari shuroo karunngaa.) - Empezaré un nuevo trabajo.

\end{itemize}

\subsubsection{11.6.3 Pedir y Dar Órdenes}


\begin{itemize}

  \item {\hindifont कृपया बैठिए।} (Krupaya baithiye.) - Por favor, siéntese Ud.

  \item {\hindifont तुम बैठो।} (Tum baitho.) - Tú, siéntate.

  \item {\hindifont मत जा।} (Mat jaa.) - No vayas.

  \item {\hindifont काम करो।} (Kaam karo.) - Haced el trabajo.

  \item {\hindifont न जाइए।} (Na jaaiye.) - No vaya Ud.

  \item {\hindifont तुम यहाँ रहो।} (Tum yahan raho.) - Tú quédate aquí.

  \item {\hindifont हम लोग घर जाते हैं।} (Ham log ghar jaate hain.) - Nosotros nos vamos a casa.

\end{itemize}

\subsubsection{11.6.4 Expresar Habilidad con सकना}


\begin{itemize}

  \item {\hindifont मैं हिंदी बोल सकता हूँ।} (Main hindii bol saktaa huun.) - Puedo hablar hindi. (masc.)

  \item {\hindifont वह गाना गा सकता है।} (Vah gaan gaa saktaa hai.) - Él puede cantar canciones.

  \item {\hindifont क्या तुम इस किताब को पढ़ सकते हो?} (Kya tum is kitaab ko padh sakte ho?) - ¿Puedes leer este libro?

  \item {\hindifont हम यह काम कर सकते हैं।} (Ham yah kaam kar sakte hain.) - Podemos hacer este trabajo.

  \item {\hindifont वे तेज़ दौड़ सकते हैं।} (Ve tej daur sakte hain.) - Ellos pueden correr rápido.

  \item {\hindifont मैं तैरना नहीं सीख सका।} (Main tairnaa nahin seekh sakaa.) - No pude aprender a nadar.

  \item {\hindifont क्या वह इस काम को अकेले कर सकता है?} (Kya vah is kaam ko akelaa kar saktaa hai?) - ¿Puede él hacer este trabajo solo?

\end{itemize}

\subsubsection{11.6.5 Diálogos Ejemplares}


\textbf{Diálogo 1 - Pidiendo y dando direcciones:}


\begin{itemize}

  \item Turista: {\hindifont माफ़ कीजिए, बस स्टॉप कहाँ है?} (Maaf kijiye, bus stop kahaan hai?) - Disculpe, ¿dónde está la parada de autobús?

  \item Local: {\hindifont सीधे जाइए, फिर दायाँ मुड़िए। आपको बस स्टॉप दिख जाएगा।} (Seedhe jaaiye, phir daayaañ muRiye. Aapko bus stop dikhe jayega.) - Vaya recto, luego gire a la derecha. Verá la parada de autobús.

  \item Turista: {\hindifont यहाँ से कितना दूर है?} (Yahan se kitnaa door hai?) - ¿A qué distancia está desde aquí?

  \item Local: {\hindifont लगभग पांच मिनट की पैदल दूरी है।} (Lagbhag paanch minute kii paidal doorii hai.) - Es una distancia de aproximadamente cinco minutos a pie.

  \item Turista: {\hindifont धन्यवाद। क्या आप मेरी कुछ सहायता कर सकते हैं?} (Dhanyavaad. Kya aap merii kuchh sahaayata kar sakte hain?) - Gracias. ¿Puede Ud. ayudarme en algo?

  \item Local: {\hindifont हाँ, बेशक। क्या सहायता चाहिए?} (Haan, beshaq. Kya sahaayata chaahie?) - Sí, por supuesto. ¿Qué ayuda necesita?

\end{itemize}


\textbf{Diálogo 2 - Planeando planes futuros:}


\begin{itemize}

  \item A: {\hindifont अगले महीने क्या करना चाहते हो?} (Agal maheene kya karnaa chaahate ho?) - ¿Qué queréis hacer el próximo mes?

  \item B: {\hindifont हम यात्रा करना चाहते हैं। कहीं नई जगह घूमना चाहते हैं।} (Ham yaatra karnaa chaahate hain. Kahin nyi jagah ghoomnaa chaahate hain.) - Queremos viajar. Queremos visitar algún lugar nuevo.

  \item A: {\hindifont यह अच्छा विचार है। कहाँ जाओगे?} (Yah achchhaa vichaar hai. Kahaan jaoge?) - Esa es una buena idea. ¿Adónde iréis?

  \item B: {\hindifont हम उत्तराखंड जाना चाहते हैं। हिमालय के पास।} (Ham Uttarakhand jaana chaahate hain. Himalay ke paas.) - Queremos ir a Uttarakhand. Cerca del Himalaya.

  \item A: {\hindifont वाह! आप वहाँ क्या देख सकते हैं?} (Vaah! Aap vahan kya dekh sakte hain?) - ¡Guau! ¿Qué pueden ver Ud. allí?

  \item B: {\hindifont हम प्रकृति को देख सकते हैं, पहाड़ों में घूम सकते हैं।} (Ham prakruti ko dekh sakte hain, pahaaron men ghoom sakte hain.) - Podemos ver la naturaleza, podemos caminar por las montañas.

  \item A: {\hindifont क्या आप वहाँ तस्वीरें ले सकते हैं?} (Kya aap vahan tasviiren le sakte hain?) - ¿Pueden Ud. tomar fotos allí?

  \item B: {\hindifont हाँ, निश्चित रूप से। हम अपने दोस्तों को तस्वीरें दिखाएँगे।} (Haan, nishchit roop se. Ham apne doston ko tasviiren dikhayenge.) - Sí, ciertamente. Mostraremos las fotos a nuestros amigos.

\end{itemize}


\hrulefill

\section{Unidad 12: Pasado Simple (Intransitivo) y Consolidación A1}


Objetivo: Narrar eventos pasados simples con verbos intransitivos y consolidar los contenidos del nivel A1

\subsection{12.1 Introducción al Pasado Simple en Hindi}


El pasado simple en hindi se divide en dos grandes categorías:


\begin{enumerate}

  \item \textbf{Verbos intransitivos} (acciones que no requieren un objeto directo)

  \item \textbf{Verbos transitivos} (acciones que sí requieren un objeto directo)

\end{enumerate}


En esta unidad nos enfocaremos en el pasado simple de \textbf{verbos intransitivos}, que es más sencillo de aprender ya que el verbo concuerda con el \textbf{sujeto} en género y número.

\subsubsection{12.1.1 Verbos Intransitivos Comunes}


Los verbos intransitivos son aquellos que no necesitan un objeto directo:


\begin{itemize}

  \item {\hindifont जाना} (jaanaa) - ir (sin objeto directo)

  \item {\hindifont आना} (aanaa) - venir (sin objeto directo)

  \item {\hindifont सोना} (sonaa) - dormir

  \item {\hindifont उठना} (uthnaa) - levantarse

  \item {\hindifont रहना} (rahnnaa) - vivir/quedarse

  \item {\hindifont होना} (honaa) - ser/estar

  \item {\hindifont रहना} (rahnnaa) - permanecer

\end{itemize}

\subsection{12.2 Formación del Pasado Simple para Verbos Intransitivos}


La formación del pasado simple para verbos intransitivos depende del género y número del \textbf{sujeto}, no del verbo como en español. El verbo concuerda con el sujeto.

\subsubsection{12.2.1 Conjugación del Pasado Simple con Verbos Intransitivos}


\begin{center}
\begin{tabularx}{\textwidth}{| X | X | X | X | X | X |}
\hline
\textbf{Persona/Sujeto} & \textbf{Masculino Singular} & \textbf{Femenino Singular} & \textbf{Masculino Plural} & \textbf{Ejemplo con {\hindifont जाना} (jaanaa)} & \textbf{Traducción} \ \hline
{\hindifont मैं} (main) - Yo & {\hindifont गया} (gyaa) & {\hindifont गई} (gee) & {\hindifont गए} (ge) & {\hindifont मैं गया।} / {\hindifont मैं गई।} & Fui. (masc./fem.) \ \hline
{\hindifont तू} (tū) - Tú (íntimo) & {\hindifont गया} (gyaa) & {\hindifont गई} (gee) & {\hindifont गए} (ge) & {\hindifont तू गया।} / {\hindifont तू गई।} & Tú fuiste. (masc./fem.) \ \hline
{\hindifont तुम} (tum) - Tú (informal) & {\hindifont गए} (ge) & {\hindifont गईं} (gein) & {\hindifont गए} (ge) & {\hindifont तुम गए।} / {\hindifont तुम गईं।} & Tú fuiste. (masc./fem.) \ \hline
{\hindifont आप} (āp) - Usted & {\hindifont गए} (ge) & {\hindifont गईं} (gein) & {\hindifont गए} (ge) & {\hindifont आप गए।} & Ud. fue. \ \hline
{\hindifont यह}/{\hindifont वह} (yah/vah) - Él/Ella & {\hindifont गया} (gaayaa) & {\hindifont गई} (gaee) & {\hindifont गए} (gae) & {\hindifont यह गया।} / {\hindifont वह गई।} & Él/Ella fue. (masc./fem.) \ \hline
{\hindifont हम} (ham) - Nosotros/as & {\hindifont गए} (ge) & {\hindifont गईं} (gein) & {\hindifont गए} (ge) & {\hindifont हम गए।} / {\hindifont हम गईं।} & Nosotros/as fuimos. \ \hline
{\hindifont ये}/{\hindifont वे} (ye/ve) - Ellos/as & {\hindifont गए} (ge) & {\hindifont गईं} (gein) & {\hindifont गए} (ge) & {\hindifont वे गए।} & Ellos/as fueron. \ \hline
\end{tabularx}
\end{center}


\begin{tcolorbox}[colback=yellow!10!white,colframe=orange!75!black,title=Regla de Oro]

\paragraph{Regla de Oro: Concordancia en el Pasado Intransitivo} 

En el pasado simple con verbos intransitivos, el verbo \textbf{concuerda con el sujeto} en género (masculino/femenino) y número (singular/plural). Esto es diferente del español, donde el verbo permanece igual para todos los sujetos.


\begin{itemize}

  \item Yo fui (masc.) - {\hindifont मैं गया।}

  \item Yo fui (fem.) - {\hindifont मैं गई।}

\end{itemize}


\end{tcolorbox}

\subsubsection{12.2.2 Otros Verbos Intransitivos en Pasado}


\begin{center}
\begin{tabularx}{\textwidth}{| X | X | X | X | X | X | X |}
\hline
\textbf{Verbo} & \textbf{Infinitivo} & \textbf{Masculino Sing.} & \textbf{Femenino Sing.} & \textbf{Plural} & \textbf{Ejemplo} & \textbf{Traducción} \ \hline
Ir & {\hindifont जाना} & {\hindifont गया} & {\hindifont गई} & {\hindifont गए} & {\hindifont वह दिल्ली गया।} & Él fue a Delhi. \ \hline
Venir & {\hindifont आना} & {\hindifont आया} & {\hindifont आई} & {\hindifont आए} & {\hindifont वह दिल्ली आया।} & Él vino a Delhi. \ \hline
Dormir & {\hindifont सोना} & {\hindifont सोया} & {\hindifont सोई} & {\hindifont सोए} & {\hindifont मैं आठ घंटे सोया।} & Dormí ocho horas. (masc.) \ \hline
Levantarse & {\hindifont उठना} & {\hindifont उठा} & {\hindifont उठी} & {\hindifont उठे} & {\hindifont मैं छह बजे उठा।} & Me levanté a las seis. (masc.) \ \hline
Vivir/Quedarse & {\hindifont रहना} & {\hindifont रहा} & {\hindifont रही} & {\hindifont रहे} & {\hindifont वह मुंबई में रहा।} & Él se quedó en Mumbai. \ \hline
Ser/Estar & {\hindifont होना} & {\hindifont हुआ} & {\hindifont हुई} & {\hindifont हुए} & {\hindifont बारिश हुई।} & Llovió./Hubo lluvia. \ \hline
Ver & {\hindifont देखना} & {\hindifont देखा} & {\hindifont देखी} & {\hindifont देखे} & {\hindifont मैं चित्र देखा।} & Vi el cuadro. (masc.) \ \hline
Escuchar & {\hindifont सुनना} & {\hindifont सुना} & {\hindifont सुनी} & {\hindifont सुने} & {\hindifont तुम गाना सुने।} & Tú escuchaste la canción. \ \hline
\end{tabularx}
\end{center}

\subsubsection{12.2.3 Ejemplos de Frases en Pasado Simple}


\begin{itemize}

  \item {\hindifont मैं कल घर गया था।} (Main kal ghar gayaa thaa.) - Yo fui a casa ayer. (masc.)

  \item {\hindifont तुम बाजार गए थे।} (Tum baazaar ge the.) - Tú fuiste al mercado.

  \item {\hindifont उसने किताब पढ़ी।} (Usne kitaab padhii.) - Él/Ella leyó el libro. (femenino)

  \item {\hindifont हम दोस्तों के साथ घूमे।} (Ham doston ke saath ghume.) - Nosotros salimos con amigos.

  \item {\hindifont वे कल भारत आए।} (Ve kal bharat aaye.) - Ellos vinieron a la India ayer.

  \item {\hindifont मैंने नाश्ता किया।} (Mainne naashTaa kiya.) - Yo desayuné.

  \item {\hindifont रात को मैं जल्दी सो गया।} (Raat ko main jaldi so gayaa.) - Por la noche me dormí temprano. (masc.)

\end{itemize}


\begin{tcolorbox}[colback=blue!5!white,colframe=blue!75!black,title=Nota/Clarificación]

\paragraph{Importante: Verbo होना en Pasado} 

El verbo {\hindifont होना} (honaa - ser/estar) forma su pasado de manera especial:


\begin{itemize}

  \item Masculino singular: {\hindifont था} (thaa)

  \item Femenino singular: {\hindifont थी} (thii)

  \item Plural: {\hindifont थे} (the)

\end{itemize}


Ejemplos:


\begin{itemize}

  \item {\hindifont मैं ठीक था।} (Main theek thaa.) - Yo estaba bien. (masc.)

  \item {\hindifont वह अध्यापिका थी।} (Vah adhyaapikaa thii.) - Ella era profesora.

  \item {\hindifont हम घर पर थे।} (Ham ghar par the.) - Nosotros estábamos en casa.

\end{itemize}


\end{tcolorbox}

\subsection{12.3 Vocabulario: Expresiones Temporales Pasadas}


Usamos expresiones temporales para indicar cuándo ocurrieron los eventos:


\begin{center}
\begin{tabularx}{\textwidth}{| X | X | X | X | X |}
\hline
\textbf{Español} & \textbf{Hindi} & \textbf{Transliteración} & \textbf{Uso} & \textbf{Ejemplo} \ \hline
Ayer & {\hindifont कल} & kal & Para eventos de ayer & {\hindifont मैं कल दिल्ली गया।} - Fui a Delhi ayer. \ \hline
La semana pasada & {\hindifont पिछले हफ्ते} & pichhle hafte & Para eventos hace una semana & {\hindifont हम पिछले हफ्ते घूमने गए।} - Fuimos de excursión la semana pasada. \ \hline
El mes pasado & {\hindifont पिछले महीने} & pichhle maheene & Para eventos hace un mes & {\hindifont मैं पिछले महीने नौकरी शुरू की।} - Comencé trabajo el mes pasado. \ \hline
El año pasado & {\hindifont पिछले साल} & pichhle saal & Para eventos hace un año & {\hindifont वे पिछले साल भारत आए।} - Ellos vinieron a la India el año pasado. \ \hline
Cuando & {\hindifont जब} & jab & Para indicar tiempo en cláusulas subordinadas & {\hindifont जब मैं छोटा था... - Cuando yo era joven...} \ \hline
Antes & {\hindifont पहले} & pahle & Para indicar secuencia temporal & {\hindifont मैंने पहले काम किया।} - Yo hice el trabajo antes. \ \hline
Después & {\hindifont बाद में} & baad men & Para indicar secuencia temporal & {\hindifont काम के बाद में आराम किया।} - Descansé después del trabajo. \ \hline
\end{tabularx}
\end{center}

\subsection{12.4 Repaso General del Nivel A1}


Esta unidad es crucial para revisar y consolidar todos los contenidos estudiados en el nivel A1. Vamos a repasar los conceptos fundamentales:

\subsubsection{12.4.1 Escritura: Devanagari}


\begin{itemize}

  \item Vocales ({\hindifont अ, आ, इ, ई, उ, ऊ, ऋ, ए, ऐ, ओ, औ}) y su representación como \textit{matras}

  \item Consonantes (grupo velar {\hindifont क-varga}, palatal {\hindifont च-varga}, retrofleja {\hindifont ट-varga}, dental {\hindifont त-varga}, labial {\hindifont प-varga})

  \item \textit{Shirorekha} ({\hindifont शिरोरेखा}) - La línea superior que se escribe después de completar la palabra

  \item \textit{Halant} ({\hindifont हलन्त}) - El signo {\hindifont ्} que cancela la vocal inherente

  \item Aspiración (diferencia entre {\hindifont क} y {\hindifont ख}, {\hindifont ग} y {\hindifont घ}, etc.)

\end{itemize}

\subsubsection{12.4.2 Estructura Sujeto-Objeto-Verbo (SOV)}


\begin{tcolorbox}[colback=yellow!10!white,colframe=orange!75!black,title=Regla de Oro]


El orden fundamental en hindi es \textbf{Sujeto - Objeto - Verbo (SOV)}, y el verbo \textbf{siempre} va al final de la cláusula:


\begin{itemize}

  \item \textbf{Español:} Yo (S) como (V) arroz (O). - Yo como arroz.

  \item \textbf{Hindi:} {\hindifont मैं (S) चावल (O) खाता हूँ (V)।} - Main chaaval khaataa huun.

\end{itemize}


\end{tcolorbox}

\subsubsection{12.4.3 Verbos y Tiempos}


\begin{itemize}

  \item \textbf{Presente Habitual:} Raíz + {\hindifont ता/ती/ते} + {\hindifont होना} (concordancia con sujeto)

  \item \textbf{Presente Continuo:} Raíz + {\hindifont रहा/रही/रहे} + {\hindifont होना} (concordancia con sujeto)

  \item \textbf{Futuro Simple:} Raíz + {\hindifont एगा/एंगी/एँगे} (concordancia con sujeto)

  \item \textbf{Pasado Simple (Intransitivo):} Raíz + {\hindifont आ/ई/ए} (concordancia con sujeto)

  \item \textbf{Verbo "ser/estar":} {\hindifont हूँ, हो, है, हैं} (concordancia con el sujeto)

\end{itemize}

\subsubsection{12.4.4 Concordancia de Género y Número}


\begin{itemize}

  \item Los adjetivos concuerdan con el sustantivo en género y número: {\hindifont अच्छा लड़का} (chico bueno), {\hindifont अच्छी लड़की} (chica buena)

  \item Los posesivos concuerdan con el objeto poseído: {\hindifont मेरा घर} (mi casa masc.), {\hindifont मेरी किताब} (mi libro fem.)

  \item En el pasado intransitivo, el verbo concuerda con el sujeto: {\hindifont लड़का रोया।} (El chico lloró.), {\hindifont लड़की रोई।} (La chica lloró.)

\end{itemize}

\subsubsection{12.4.5 Postposiciones y Caso Oblicuo}


\begin{itemize}

  \item Cuando un sustantivo va seguido de una postposición, cambia a su forma oblicua

  \item Sustantivos masculinos terminados en {\hindifont -आ} ({\hindifont -aa}) cambian a {\hindifont -ए} ({\hindifont -e}): {\hindifont लड़का} → {\hindifont लड़के को}

  \item Otros sustantivos no cambian en singular oblicuo: {\hindifont किताब} → {\hindifont किताब पर}

  \item Pronombres cambian completamente en oblicuo: {\hindifont मैं} → {\hindifont मुझ}, {\hindifont तुम} → {\hindifont तुम्हारा}, etc.

  \item Postposiciones comunes: {\hindifont में} (en), {\hindifont पर} (sobre), {\hindifont को} (a/para), {\hindifont से} (de/desde), {\hindifont के लिए} (para)

\end{itemize}

\subsubsection{12.4.6 Estructuras de Sujeto Dativo}


\begin{itemize}

  \item {\hindifont मुझे पानी चाहिए।} - Literalmente "A mí me es agua necesaria". (Yo necesito agua.)

  \item {\hindifont उसे गाना पसंद है।} - Literalmente "A él/ella le gusta canción". (A él/ella le gusta la canción.)

  \item {\hindifont हमें काम करना है।} - Debemos trabajar.

\end{itemize}

\subsubsection{12.4.7 Conectores y Conjugaciones Verbales}


\begin{itemize}

  \item Conectores: {\hindifont और} (y), {\hindifont लेकिन} (pero), {\hindifont क्योंकि} (porque), {\hindifont इसलिए} (por eso)

  \item Interrogativos: {\hindifont क्या} (qué/sí-no), {\hindifont कहाँ} (dónde), {\hindifont कौन} (quién), {\hindifont कब} (cuándo), {\hindifont कैसे} (cómo), {\hindifont क्यों} (por qué)

  \item Imperativo: Formal ({\hindifont -इए}), Informal ({\hindifont -ओ}), Íntimo (raíz)

  \item Modal {\hindifont सकना}: Expresa habilidad/potencialidad: {\hindifont मैं हिंदी बोल सकता हूँ।} - Puedo hablar hindi.

\end{itemize}

\subsection{12.5 Práctica Integradora}


Vamos a aplicar todos los conceptos aprendidos en frases y diálogos completos:

\subsubsection{12.5.1 Frases que Integran Varias Estructuras}


\begin{itemize}

  \item {\hindifont मैं कल बाज़ार गया और फल खरीदे।} (Main kal baazaar gayaa aur phal kharide.) - Fui al mercado ayer y compré fruta. (Pasado intransitivo + Conector + Objeto)

  \item {\hindifont तुम्हारा घर कहाँ है? मुझे पता नहीं है।} (Tumhaaraa ghar kahaan hai? Mujhe pataa nahin hai.) - ¿Dónde está tu casa? No lo sé. (Pronombres + Interrogativo + Sujeto dativo)

  \item {\hindifont वह लड़की स्कूल में पढ़ रही है।} (Vah laRkii skul men padh rahi hai.) - Esa chica está estudiando en la escuela. (Demostrativo + Postposición + Presente continuo)

  \item {\hindifont मुझे यह किताब पसंद है। क्या तुम इसे पढ़ सकते हो?} (Mujhe yah kitaab pasand hai. Kya tum isse padh sakte ho?) - Me gusta este libro. ¿Puedes leerlo? (Sujeto dativo + Modal)

  \item {\hindifont हम अगले हफ्ते भारत जाने का इरादा रखते हैं।} - Planeamos ir a la India la próxima semana. (Futuro + Intención)

  \item {\hindifont आपको यह काम जल्दी से करना चाहिए।} (Aapko yah kaam jaldi se karnaa chaahie.) - Ud. debería hacer este trabajo rápidamente. (Sujeto dativo + Consejo)

  \item {\hindifont जब मैं छोटा था, मैं बहुत खेलता था।} - Cuando era niño, jugaba mucho. (Tiempo pasado + Subordinada)

\end{itemize}

\subsubsection{12.5.2 Diálogo de Ejemplo - Visita Turística}


\begin{itemize}

  \item Turista: {\hindifont नमस्ते। मैं दिल्ली घूमना चाहता हूँ। क्या आप मेरी मदद कर सकते हैं?} (Namaste. Main Dillee ghoomnaa chaahata huun. Kya aap merii madad kar sakte hain?) - Hola. Quiero visitar Delhi. ¿Puede Ud. ayudarme?

  \item Guía: {\hindifont हाँ, बेशक। आप जहाँ भी जाएँगे, मैं आपके साथ हूँ।} - Sí, por supuesto. A dondequiera que Ud. vaya, estaré con Ud.

  \item Turista: {\hindifont मुझे लाल किला देखना है। कल सुबह चल सकते हैं?} - Quiero ver el Fuerte Rojo. ¿Podemos salir mañana por la mañana?

  \item Guía: {\hindifont बेशक! कल सुबह नौ बजे मैं आपके होटल पर होऊँगा।} - ¡Por supuesto! Mañana por la mañana a las nueve estaré en su hotel.

  \item Turista: {\hindifont यह अच्छा है। मैं आपके साथ लाल किला तथा इंडिया गेट देखना चाहता हूँ।} - Esto es bueno. Quiero ver el Fuerte Rojo y el India Gate con Ud.

  \item Guía: {\hindifont मैं आपको समझा सकता हूँ कि यहाँ का इतिहास कैसा है।} - Puedo explicarle cómo es la historia de aquí.

  \item Turista: {\hindifont वाह! यह बहुत दिलचस्प है। क्या आपके पास मानचित्र है?} - ¡Guau! Esto es muy interesante. ¿Tiene Ud. un mapa?

  \item Guía: {\hindifont हाँ, मेरे पास मानचित्र है। यह लीजिए।} - Sí, tengo un mapa. Tome Ud.

\end{itemize}

\subsubsection{12.5.3 Mini-diálogo - Rutina Diaria Pasada}


\begin{itemize}

  \item A: {\hindifont कल आप जल्दी उठे?} (Kal aap jaldi uthhe?) - ¿Ud. se levantó temprano ayer?

  \item B: {\hindifont नहीं, मैं नौ बजे उठा। फिर मैंने नाश्ता किया और पढ़ाई शुरू की।} - No, me levanté a las nueve. Luego desayuné y comencé a estudiar.

  \item A: {\hindifont आपने किस समय तक पढ़ाई की?} - ¿Hasta qué hora estudió Ud.?

  \item B: {\hindifont साढ़े दस बजे तक मैं पढ़ रहा था। फिर दोपहर के भोजन के लिए बैठा।} - Hasta las diez y media estuve estudiando. Luego me senté para el almuerzo.

  \item A: {\hindifont क्या कल आपने कुछ समय आराम किया?} - ¿Pasó Ud. algo de tiempo descansando ayer?

  \item B: {\hindifont हाँ, शाम को मैं सोया।} - Sí, por la tarde dormí.

\end{itemize}

\subsection{12.6 Evaluación Final A1: Autoevaluación}


Para confirmar que has dominado el nivel A1, debes ser capaz de:

\subsubsection{12.6.1 Habilidades Lingüísticas}


\begin{itemize}

  \item Leer y escribir en devanagari sin dificultad.

  \item Formar oraciones básicas usando la estructura S-O-V.

  \item Expresar identidad, origen, profesión y posesión.

  \item Hablar de rutinas diarias con el presente habitual.

  \item Describir acciones en progreso con el presente continuo.

  \item Expresar planes futuros con el futuro simple.

  \item Narrar eventos simples del pasado con verbos intransitivos.

  \item Usar correctamente los pronombres, posesivos y estructuras dativas.

  \item Formular preguntas básicas y comprensión de las respuestas.

  \item Utilizar adecuadamente postposiciones y el caso oblicuo.

  \item Expresar gustos, necesidades y consejos con estructuras dativas.

  \item Combinar frases con conectores como {\hindifont और}, {\hindifont लेकिन}, {\hindifont क्योंकि}.

  \item Dar y recibir órdenes en diferentes registros de formalidad.

  \item Expresar habilidades y posibilidades con el verbo modal {\hindifont सकना}.

\end{itemize}

\subsubsection{12.6.2 Vocabulario}


\begin{itemize}

  \item Números del 1 al 100.

  \item Días de la semana y meses del año.

  \item Miembros de la familia.

  \item Partes del cuerpo.

  \item Comida, frutas, verduras y bebidas.

  \item Objetos del aula y de la casa.

  \item Lugares de la ciudad.

  \item Profesiones y nacionalidades.

  \item Colores y adjetivos básicos.

  \item Adverbios de tiempo, lugar y modo.

  \item Expresiones temporales pasadas, presentes y futuras.

\end{itemize}

\subsubsection{12.6.3 Comunicación Funcional}


\begin{itemize}

  \item Presentarse y presentar a otros.

  \item Preguntar y decir el nombre.

  \item Expresar origen y nacionalidad.

  \item Describir personas y objetos.

  \item Hablar sobre gustos y preferencias.

  \item Expresar necesidades y dar consejos.

  \item Pedir y dar direcciones.

  \item Describir lugares y dar información sobre ellos.

  \item Narrar eventos simples del pasado.

  \item Expresar planes e intenciones futuras.

  \item Expresar habilidades y capacidades.

  \item Formular preguntas sobre tiempo, modo, causa y cantidad.

  \item Usar diferentes niveles de formalidad según el contexto.

\end{itemize}

\subsubsection{12.6.4 Actividad Final: Redacción de un Texto Corto}


Escribe un párrafo breve (5-8 oraciones) sobre uno de estos temas usando todos los elementos gramaticales aprendidos:


\begin{itemize}

  \item Mi rutina diaria

  \item Mi visita a un nuevo lugar

  \item Mis planes para las próximas vacaciones

  \item Una experiencia pasada interesante

\end{itemize}


Considera incluir:


\begin{itemize}

  \item Pronombres personales y posesivos

  \item Concordancia de género y número

  \item Al menos un tiempo verbal en pasado, presente y futuro

  \item Algunas postposiciones

  \item Una estructura con sujeto dativo

  \item Conectores para unir ideas

\end{itemize}


Ejemplo de respuesta posible:


{\hindifont मैं राम हूँ। मैं भारतीय हूँ। मैं हिंदी बोल सकता हूँ। हर रोज़ मैं सुबह छह बजे उठता हूँ। मैं नाश्ता करता हूँ और फिर पढ़ाई करता हूँ। कल मैं नयी किताब खरीदा। अगले हफ्ते मैं दिल्ली जाऊँगा। मुझे यह यात्रा पसंद होगी।}


Ram soy yo. Soy indio. Puedo hablar hindi. Todos los días me levanto a las seis de la mañana. Desayuno y luego estudio. Ayer compré un libro nuevo. La próxima semana iré a Delhi. Me gustará este viaje.


\begin{tcolorbox}[colback=blue!5!white,colframe=blue!75!black,title=Nota/Clarificación]

\paragraph{Consejos para el Aprendizaje Continuo} 

\begin{itemize}

  \item Practica diariamente la escritura en devanagari para consolidar la memoria muscular.

  \item Lee textos simples en hindi para mejorar la comprensión lectora.

  \item Conversa con hablantes nativos o practica en voz alta para mejorar la pronunciación.

  \item Usa aplicaciones o recursos digitales para reforzar vocabulario y gramática.

  \item Consulta regularmente las tablas gramaticales para consolidar las estructuras.

  \item Escucha música o videos en hindi con subtítulos para mejorar la comprensión auditiva.

\end{itemize}


\end{tcolorbox}


\hrulefill

\section{Unidad 13: Repaso A1, Adverbios y el sufijo 'vālā'}


\textbf{Objetivo:} Revisar conceptos de A1, ampliar el uso de adverbios y aprender el sufijo 'vālā'.

\subsection{1. Repaso de Conceptos Fundamentales del Nivel A1}


Antes de introducir nuevos contenidos, consolidamos los elementos gramaticales fundamentales del nivel A1:

\subsubsection{1.1 Estructura Sujeto - Objeto - Verbo (SOV)}


Este es el orden fundamental de la oración en hindi. A diferencia del español (SVO), el verbo siempre va al final de la cláusula.


\begin{center}
\begin{tabularx}{\textwidth}{| X | X | X | X | X |}
\hline
\textbf{Estructura} & \textbf{Ejemplo en Español} & \textbf{Estructura en Hindi} & \textbf{Ejemplo en Hindi} & \textbf{Traducción} \ \hline
Sujeto - Verbo - Objeto & Yo como arroz & Sujeto - Objeto - Verbo & {\hindifont मैं चावल खाता हूँ।} & Yo como arroz. \ \hline
Compramos una mesa & Nosotros compramos una mesa & Sujeto - Objeto - Verbo & {\hindifont हम एक मेज़ खरीदते हैं।} & Nosotros compramos una mesa. \ \hline
\end{tabularx}
\end{center}

\subsubsection{1.2 Tiempos Verbales Básicos del Nivel A1}


\begin{center}
\begin{tabularx}{\textwidth}{| X | X | X | X | X |}
\hline
\textbf{Tiempo} & \textbf{Estructura} & \textbf{Ejemplo} & \textbf{Traducción} & \textbf{Uso} \ \hline
Presente Habitual & Raíz + ta/ti/te + honaa & {\hindifont मैं पढ़ता हूँ।} & Yo leo/estudio. (masc.) & Acciones habituales o verdades generales \ \hline
Presente Continuo & Raíz + raha/rhi/rhe + honaa & {\hindifont मैं पढ़ रहा हूँ।} & Yo estoy leyendo. (masc.) & Acciones en progreso en el momento de hablar \ \hline
Futuro Simple & Raíz + ega/egi/ege & {\hindifont मैं पढ़ूँगा।} & Yo leeré/estudiaré. (masc.) & Acciones futuras \ \hline
Pasado Simple (Intransitivo) & Verbo conjugado (concordancia con sujeto) & {\hindifont मैं गया।} / {\hindifont मैं गई।} & Yo fui. (masc./fem.) & Acciones pasadas completadas (verbos intransitivos) \ \hline
\end{tabularx}
\end{center}

\subsubsection{1.3 Concordancia de Género y Número}


Los adjetivos y formas verbales concuerdan con el género y número del sustantivo o sujeto al que modifican:


\begin{itemize}

  \item \textbf{Adjetivos:} {\hindifont अच्छा लड़का} (buen chico masc.), {\hindifont अच्छी लड़की} (buena chica fem.), {\hindifont अच्छे लड़के} (buenos chicos masc.pl.)

  \item \textbf{Verbos en presente habitual:} {\hindifont लड़का पढ़ता है।} (el chico estudia), {\hindifont लड़की पढ़ती है।} (la chica estudia), {\hindifont लड़के पढ़ते हैं।} (los chicos estudian)

\end{itemize}

\subsubsection{1.4 Estructuras de Sujeto Dativo}


Algunas construcciones usan un sujeto en caso dativo para expresar gustos, necesidades, conocimiento:


\begin{itemize}

  \item {\hindifont मुझे चाय पसंद है।} - A mí me gusta el té. (Literalmente: A mí té gusta.)

  \item {\hindifont तुम्हें आराम करना चाहिए।} - Tú deberías descansar.

  \item {\hindifont उसे हिंदी आती है।} - Él/Ella sabe hindi. (Literally: A él/ella hindi viene.)

\end{itemize}

\subsection{2. Adverbios en Hindi}


Los adverbios en hindi modifican verbos, adjetivos u otros adverbios. Vienen en varias categorías:

\subsubsection{2.1 Adverbios de Frecuencia}


\begin{center}
\begin{tabularx}{\textwidth}{| X | X | X | X | X |}
\hline
\textbf{Español} & \textbf{Hindi} & \textbf{Transliteración} & \textbf{Uso} & \textbf{Ejemplo} \ \hline
Siempre & {\hindifont हमेशा} & hameshaa & Indica acción constante & {\hindifont वह हमेशा स्कूल जल्दी जाता है।} (Él siempre va temprano a la escuela.) \ \hline
A veces & {\hindifont कभी-कभी} & kabhi-kabhi & Indica frecuencia ocasional & {\hindifont हम कभी-कभी पार्क में घूमते हैं।} (Nosotros a veces paseamos por el parque.) \ \hline
Todos los días & {\hindifont हर रोज़} & har roz & Indica acción diaria & {\hindifont मैं हर रोज़ खाना बनाता हूँ।} (Yo cocino todos los días.) \ \hline
A menudo & {\hindifont अक्सर} & aksar & Indica frecuencia alta & {\hindifont तुम अक्सर किताबें पढ़ते हो।} (Tú a menudo lees libros.) \ \hline
Raramente & {\hindifont शायद} & shaayad & Indica baja frecuencia & {\hindifont वह शायद बाहर जाता है।} (Él raramente sale.) \ \hline
Nunca & {\hindifont कभी नहीं} & kabhi nahin & Indica ausencia total & {\hindifont मैं कभी झूठ नहीं बोलता।} (Yo nunca miento.) \ \hline
\end{tabularx}
\end{center}

\subsubsection{2.2 Adverbios de Lugar}


\begin{center}
\begin{tabularx}{\textwidth}{| X | X | X | X | X |}
\hline
\textbf{Español} & \textbf{Hindi} & \textbf{Transliteración} & \textbf{Uso} & \textbf{Ejemplo} \ \hline
Aquí & {\hindifont यहाँ} & yahaan & Indica proximidad al hablante & {\hindifont किताबें यहाँ हैं।} (Los libros están aquí.) \ \hline
Allí & {\hindifont वहाँ} & vahaan & Indica lejanía del hablante & {\hindifont मेरा घर वहाँ है।} (Mi casa está allí.) \ \hline
Cerca & {\hindifont पास} & paas & Indica proximidad & {\hindifont दुकान मेरे घर के पास है।} (La tienda está cerca de mi casa.) \ \hline
Lejos & {\hindifont दूर} & door & Indica distancia & {\hindifont मंदिर बहुत दूर है।} (El templo está muy lejos.) \ \hline
Delante & {\hindifont आगे} & aage & Indica dirección hacia adelante & {\hindifont बस स्टॉप आगे है।} (La parada de autobús está adelante.) \ \hline
Detrás & {\hindifont पीछे} & piche & Indica dirección hacia atrás & {\hindifont मेरा दोस्त पीछे खड़ा है।} (Mi amigo está de pie detrás.) \ \hline
\end{tabularx}
\end{center}

\subsubsection{2.3 Adverbios de Modo}


\begin{center}
\begin{tabularx}{\textwidth}{| X | X | X | X | X |}
\hline
\textbf{Español} & \textbf{Hindi} & \textbf{Transliteración} & \textbf{Uso} & \textbf{Ejemplo} \ \hline
Bien & {\hindifont अच्छा} & acchhaa & Indica calidad positiva & {\hindifont वह हिंदी अच्छा बोलता है।} (Él habla hindi bien.) \ \hline
Mal & {\hindifont ख़राब} & kharaab & Indica calidad negativa & {\hindifont मैं गाना ख़राब गाता हूँ।} (Yo canto canciones mal.) \ \hline
rápido & {\hindifont तेज़} & tez & Indica velocidad & {\hindifont तुम तेज़ चलते हो।} (Tú caminas rápido.) \ \hline
lento & {\hindifont धीमा} & dheemaa & Indica lentitud & {\hindifont वह धीमा चलता है।} (Él camina lento.) \ \hline
claramente & {\hindifont स्पष्ट} & spashtt & Indica claridad & {\hindifont कृपया स्पष्ट बोलो।} (Por favor, habla claramente.) \ \hline
\end{tabularx}
\end{center}

\subsection{3. El Sufijo 'वाला' (vālā)}


El sufijo {\hindifont वाला} (vālā) es una herramienta gramatical versátil en hindi que se puede usar tanto como adjetivo como sustantivo para expresar pertenencia, característica o profesión.

\subsubsection{3.1 Uso del Sufijo 'वाला' como Adjetivo}


Este sufijo se agrega a adjetivos, sustantivos o incluso frases enteras para crear adjetivos que denotan posesión o cualidad:


\begin{center}
\begin{tabularx}{\textwidth}{| X | X | X | X | X |}
\hline
\textbf{Elemento Base} & \textbf{Con 'वाला'} & \textbf{Genero} & \textbf{Ejemplo en Oración} & \textbf{Traducción} \ \hline
{\hindifont लाल} (rojo) & {\hindifont लाल वाला} (rojo-vala) & Masculino & {\hindifont लाल वाली कमीज़} & La camisa roja (camisa que es roja) \ \hline
{\hindifont सफेद} (blanco) & {\hindifont सफेद वाला} (blanco-vala) & Masculino & {\hindifont सफेद वाली कार} & El coche blanco (coche que es blanco) \ \hline
{\hindifont चश्मा} (anteojos) & {\hindifont चश्मा वाला} (anteojos-vala) & Masculino & {\hindifont चश्मा वाला आदमी} & El hombre con anteojos (hombre que tiene anteojos) \ \hline
{\hindifont किताब} (libro) & {\hindifont किताब वाला} (libro-vala) & Masculino & {\hindifont किताब वाला दुकानदार} & El librero (comerciante que vende libros) \ \hline
\end{tabularx}
\end{center}

\subsubsection{3.2 Declinación del Sufijo 'वाला'}


Al igual que otros adjetivos en hindi, el sufijo {\hindifont वाला} concuerda en género y número con el sustantivo que modifica:


\begin{center}
\begin{tabularx}{\textwidth}{| X | X | X | X | X | X |}
\hline
\textbf{Singular} & \textbf{Masculino} & \textbf{Femenino} & \textbf{Plural} & \textbf{Masculino} & \textbf{Femenino} \ \hline
Con 'rojo' & {\hindifont लाल वाला} & {\hindifont लाल वाली} & Con 'rojo' & {\hindifont लाल वाले} & {\hindifont लाल वाली} \ \hline
Con 'libro' & {\hindifont किताब वाला} & {\hindifont किताब वाली} & Con 'libros' & {\hindifont किताब वाले} & {\hindifont किताब वाली} \ \hline
Ejemplo & {\hindifont लड़का चश्मा वाला है।} & {\hindifont लड़की चश्मा वाली है।} & Ejemplo & {\hindifont लड़के चश्मा वाले हैं।} & {\hindifont लड़कियाँ चश्मा वाली हैं।} \ \hline
\end{tabularx}
\end{center}

\subsubsection{3.3 Uso del Sufijo 'वाला' como Sustantivo}


El sufijo también puede formar sustantivos que denotan profesiones o roles:


\begin{itemize}

  \item {\hindifont चाय वाला} (chaa vaalaa) - el vendedor de té

  \item {\hindifont किताब वाला} (kitaab vaalaa) - el librero

  \item {\hindifont सब्ज़ी वाला} (sabzii vaalaa) - el verdulero

  \item {\hindifont फल वाला} (phal vaalaa) - el frutero

  \item {\hindifont मिठाई वाला} (mitaaii vaalaa) - el pastelero

  \item {\hindifont दूध वाला} (doodh vaalaa) - el lechero

\end{itemize}

\subsubsection{3.4 Uso del Sufijo como Sustantivo con Pronombres}


Cuando se usa con pronombres posesivos, puede indicar posesión:


\begin{itemize}

  \item {\hindifont मेरा घर वाला} (meraa ghar vaalaa) - la persona de mi casa / mi criado

  \item {\hindifont तुम्हारा काम वाला} (tumhaaraa kaam vaalaa) - la persona relacionada con tu trabajo

  \item {\hindifont उसका पैसा वाला} (uskaa paise vaalaa) - La persona con su dinero

\end{itemize}


\begin{tcolorbox}[colback=yellow!10!white,colframe=orange!75!black,title=Regla de Oro]

\paragraph{Regla de Oro: Uso del Sufijo 'वाला'} 

El sufijo {\hindifont वाला} indica que algo o alguien \textbf{tiene}, \textbf{posee}, \textbf{está relacionado con} o \textbf{es característico de} lo que precede al sufijo.


\begin{itemize}

  \item {\hindifont गाड़ी वाला} = Algo/persona \textbf{que tiene} un coche

  \item {\hindifont लाल वाला} = Algo que \textbf{es} rojo

  \item {\hindifont चाय वाला} = Persona \textbf{que vende} té

\end{itemize}


\end{tcolorbox}

\subsubsection{3.5 Ejemplos de Frases con el Sufijo 'वाला'}


\begin{itemize}

  \item {\hindifont लाल वाली कमीज़ मुझे पसंद है।} (La camisa roja me gusta.)

  \item {\hindifont वह चश्मा वाला लड़का मेरा दोस्त है।} (Ese chico con anteojos es mi amigo.)

  \item {\hindifont मेरे पास धन वाला आदमी है।} (Tengo un hombre con dinero.)

  \item {\hindifont किताब वाला दुकानदार मेरा पुराना जानना है।} (El librero es una vieja conocida mía.)

  \item {\hindifont रोटी वाला आदमी अच्छा इंसान है।} (El panadero es una buena persona.)

\end{itemize}

\subsection{4. Comunicación: Descripción de Rutinas y Expresiones de Características}

\subsubsection{4.1 Frases Comunes para Describir Rutinas Diarias con Adverbios}


\begin{itemize}

  \item {\hindifont मैं हर रोज़ सुबह छह बजे उठता हूँ।} - Yo me levanto todos los días a las seis de la mañana.

  \item {\hindifont वह कभी-कभी स्कूल जल्दी जाता है।} - Él a veces va temprano a la escuela.

  \item {\hindifont हम अक्सर शाम को पार्क में घूमते हैं।} - Nosotros a menudo paseamos por el parque por la tarde.

  \item {\hindifont तुम तेज़ काम करते हो।} - Tú trabajas rápidamente.

  \item {\hindifont वह धीमा बोलता है।} - Él habla lentamente.

\end{itemize}

\subsubsection{4.2 Expresiones con el Sufijo 'वाला'}


\begin{itemize}

  \item {\hindifont मुझे चावल वाला खाना पसंद है।} - Me gusta comer arroz.

  \item {\hindifont यह चाय वाली दुकान काफी लोकप्रिय है।} - Esta tienda de té es bastante popular.

  \item {\hindifont सफेद वाला कुर्ता मेरा पसंदीदा है।} - La kurta blanca es mi favorita.

  \item {\hindifont इस गाँव में फल वाला एक आदमी है।} - En este pueblo hay una persona que vende fruta.

  \item {\hindifont कल किताब वाले आदमी से मिलूँगा।} - Mañana me reuniré con el librero.

\end{itemize}

\subsubsection{4.3 Diálogos de Práctica}


\textbf{Diálogo 1: Describiendo rutinas}


\begin{itemize}

  \item A: {\hindifont तुम हर रोज़ कैसे रहते हो?} - ¿Cómo pasas tus días normalmente?

  \item B: {\hindifont हर रोज़ मैं सुबह जल्दी उठता हूँ। फिर मैं खाना अच्छा खाता हूँ।} - Todos los días me levanto temprano. Luego como buena comida.

  \item A: {\hindifont अच्छा, क्या तुम हर रोज़ व्यायाम करते हो?} - Bien, ¿haces ejercicio todos los días?

  \item B: {\hindifont कभी-कभी मैं व्यायाम तेज़ करता हूँ।} - A veces hago ejercicio rápidamente.

  \item A: {\hindifont मैं तेज़ चलना पसंद करता हूँ।} - A mí me gusta caminar rápido.

  \item B: {\hindifont हाँ, तेज़ चलना स्वास्थ्य के लिए अच्छा है।} - Sí, caminar rápido es bueno para la salud.

\end{itemize}


\textbf{Diálogo 2: Hablando sobre objetos y personas}


\begin{itemize}

  \item A: {\hindifont यह किताब वाला आदमी कौन है?} - ¿Quién es ese hombre de los libros?

  \item B: {\hindifont यह राम है। वह एक पुस्तकालय वाला है।} - Éste es Ram. Él es el bibliotecario.

  \item A: {\hindifont राम के पास सफेद वाला कुर्ता क्यों है?} - ¿Por qué Ram tiene una kurta blanca?

  \item B: {\hindifont वह अक्सर सफेद वाला कुछ पहनता है। वह उसे पसंद करता है।} - Él a menudo viste algo blanco. Le gusta eso.

  \item A: {\hindifont मैं चश्मा वाली औरत को जानता हूँ।} - Yo conozco a la mujer con gafas.

  \item B: {\hindifont क्या वह भी किताब वाली है?} - ¿Ella también es una librera?

\end{itemize}


\hrulefill

\section{Unidad 14: Caso Oblicuo (Plural)}


\textbf{Objetivo:} Dominar la formación y el uso del caso oblicuo en plural para todos los sustantivos.

\subsection{1. Introducción al Caso Oblicuo Plural}


Después de haber aprendido en el nivel A1 cómo los sustantivos cambian a su forma oblicua en singular cuando van seguidos de una postposición, ahora extenderemos ese conocimiento al plural. El caso oblicuo plural es necesario cuando un sustantivo en plural va acompañado de una postposición como {\hindifont में} (en), {\hindifont पर} (sobre), {\hindifont से} (de), etc.


\begin{tcolorbox}[colback=yellow!10!white,colframe=orange!75!black,title=Regla de Oro]

\paragraph{Regla Fundamental: El Caso Oblicuo Plural} 

Cuando un sustantivo en plural va seguido de una \textbf{postposición}, casi siempre debe cambiar a su forma \textbf{"Oblicua Plural"}.


\textbf{Memorizar:} Las postposiciones son como "mochilas". Para poder llevar una mochila, las palabras plurales necesitan cambiar a una forma más fuerte y estable: la forma oblicua plural.


\end{tcolorbox}

\subsection{2. Formas Oblicuas de Sustantivos en Plural}


La formación del caso oblicuo en plural varía según el género y la terminación del sustantivo en singular:

\subsubsection{2.1 Sustantivos Femeninos en Plural}


Los sustantivos femeninos forman su plural oblicuo de forma similar a su plural nominativo:


\begin{itemize}

  \item Sustantivos femeninos terminados en {\hindifont -ई} ({\hindifont -ii}): \textbf{No cambian} en plural oblicuo
            
\begin{itemize}

  \item {\hindifont लड़की} (chica) → {\hindifont लड़कियाँ} (chicas) → {\hindifont लड़कियों} (con postposición - oblicuo plural)

  \item {\hindifont किताब} (libro fem.) → {\hindifont किताबें} (libros) → {\hindifont किताबों} (con postposición - oblicuo plural)

  \item {\hindifont महिला} (mujer) → {\hindifont महिलाएँ} (mujeres) → {\hindifont महिलाओं} (con postposición - oblicuo plural)

\end{itemize}



  \item Sustantivos femeninos que no terminan en {\hindifont -ई}: \textbf{No cambian} en plural oblicuo
            
\begin{itemize}

  \item {\hindifont बहन} (hermana) → {\hindifont बहनें} (hermanas) → {\hindifont बहनों} (con postposición - oblicuo plural)

  \item {\hindifont कुर्सी} (silla) → {\hindifont कुर्सियाँ} (sillas) → {\hindifont कुर्सियों} (con postposición - oblicuo plural)

\end{itemize}



\end{itemize}

\subsubsection{2.2 Sustantivos Masculinos en Plural}


Los sustantivos masculinos tienen una regla especial para la formación del oblicuo plural:


\begin{itemize}

  \item Sustantivos masculinos terminados en {\hindifont -आ} ({\hindifont -aa}): Cambian la terminación de {\hindifont -ए} ({\hindifont -e}) en singular oblicuo a {\hindifont -ों} ({\hindifont -on}) en plural oblicuo
            
\begin{itemize}

  \item {\hindifont लड़का} (chico) → {\hindifont लड़के} (chicos) → {\hindifont लड़कों} (con postposición - oblicuo plural)

  \item {\hindifont घर} (casa masc.) → {\hindifont घर} (casas) → {\hindifont घरों} (con postposición - oblicuo plural)

  \item {\hindifont कमरा} (habitación) → {\hindifont कमरे} (habitaciones) → {\hindifont कमरों} (con postposición - oblicuo plural)

  \item {\hindifont दिन} (día) → {\hindifont दिन} (días) → {\hindifont दिनों} (con postposición - oblicuo plural)

\end{itemize}



  \item Todos los demás sustantivos masculinos (que no terminan en {\hindifont -आ}): \textbf{No cambian} en plural oblicuo
            
\begin{itemize}

  \item {\hindifont किताब} (libro) → {\hindifont किताबें} (libros) → {\hindifont किताबों} (con postposición - oblicuo plural)

  \item {\hindifont नाम} (nombre) → {\hindifont नाम} (nombres) → {\hindifont नामों} (con postposición - oblicuo plural)

\end{itemize}



\end{itemize}

\subsubsection{2.3 Tabla Completa de Sustantivos en Oblicuo Plural}


\begin{center}
\begin{tabularx}{\textwidth}{| X | X | X | X | X | X |}
\hline
\textbf{Singular Nominativo} & \textbf{Plural Nominativo} & \textbf{Plural Oblicuo} & \textbf{Transliteración} & \textbf{Ejemplo con Postposición} & \textbf{Traducción} \ \hline
{\hindifont लड़का} (chico) & {\hindifont लड़के} (chicos) & {\hindifont लड़कों} & laRkon & {\hindifont लड़कों में} (laRkon men) & Entre los chicos \ \hline
{\hindifont लड़की} (chica) & {\hindifont लड़कियाँ} (chicas) & {\hindifont लड़कियों} & laRkiyon & {\hindifont लड़कियों के लिए} (laRkiyon ke liye) & Para las chicas \ \hline
{\hindifont कमरा} (habitación) & {\hindifont कमरे} (habitaciones) & {\hindifont कमरों} & kamron & {\hindifont कमरों में} (kamron men) & En las habitaciones \ \hline
{\hindifont कुर्सी} (silla) & {\hindifont कुर्सियाँ} (sillas) & {\hindifont कुर्सियों} & kursiyon & {\hindifont कुर्सियों पर} (kursiyon par) & En las sillas \ \hline
{\hindifont घर} (casa) & {\hindifont घर} (casas) & {\hindifont घरों} & gharon & {\hindifont घरों के बाहर} (gharon ke baahar) & Fuera de las casas \ \hline
{\hindifont नाम} (nombre) & {\hindifont नाम} (nombres) & {\hindifont नामों} & naamon & {\hindifont नामों के बारे में} (naamon ke baare men) & Sobre los nombres \ \hline
{\hindifont भाषा} (idioma) & {\hindifont भाषाएँ} (idiomas) & {\hindifont भाषाओं} & bhaashaon & {\hindifont भाषाओं के लिए} (bhaashaon ke liye) & Para los idiomas \ \hline
\end{tabularx}
\end{center}

\subsubsection{2.4 Sustantivos Irregulares en Plural}


Algunos sustantivos tienen formas irregulares tanto en plural como en oblicuo plural:


\begin{center}
\begin{tabularx}{\textwidth}{| X | X | X | X | X | X |}
\hline
\textbf{Singular Nominativo} & \textbf{Plural Nominativo} & \textbf{Plural Oblicuo} & \textbf{Transliteración} & \textbf{Ejemplo con Postposición} & \textbf{Traducción} \ \hline
{\hindifont आदमी} (hombre) & {\hindifont आदमियों} (hombres) & {\hindifont आदमियों} & aadamiyon & {\hindifont आदमियों से} (aadamiyon se) & De los hombres \ \hline
{\hindifont बच्चा} (niño) & {\hindifont बच्चे} (niños) & {\hindifont बच्चों} & bachhon & {\hindifont बच्चों के साथ} (bachhon ke saath) & Con los niños \ \hline
{\hindifont बच्ची} (niña) & {\hindifont बच्चियाँ} (niñas) & {\hindifont बच्चियों} & bachchhiyon & {\hindifont बच्चियों के लिए} (bachchhiyon ke liye) & Para las niñas \ \hline
\end{tabularx}
\end{center}

\subsection{3. Caso Oblicuo Plural de los Pronombres}


Los pronombres también tienen formas oblicuas en plural. A continuación se muestra la tabla completa:


\begin{center}
\begin{tabularx}{\textwidth}{| X | X | X | X | X |}
\hline
\textbf{Pronombre Nominativo} & \textbf{Forma Oblicua} & \textbf{Transliteración} & \textbf{Ejemplo con Postposición} & \textbf{Traducción} \ \hline
{\hindifont हम} (ham - nosotros) & {\hindifont हम} (ham) & ham & {\hindifont हम पर} (ham par) & Sobre nosotros \ \hline
{\hindifont तुम} (tum - vosotros) & {\hindifont तुम} (tum) & tum & {\hindifont तुम से} (tum se) & De vosotros \ \hline
{\hindifont आप} (āp - ustedes) & {\hindifont आप} (āp) & āp & {\hindifont आप के लिए} (āp ke liye) & Para ustedes \ \hline
{\hindifont ये} (ye - éstos) & {\hindifont इन} (in) & in & {\hindifont इन में} (in men) & En éstos \ \hline
{\hindifont वे} (ve - aquéllos) & {\hindifont उन} (un) & un & {\hindifont उन पर} (un par) & Sobre aquéllos \ \hline
\end{tabularx}
\end{center}


\begin{tcolorbox}[colback=blue!5!white,colframe=blue!75!black,title=Nota/Clarificación]

\paragraph{Nota Importante sobre Pronombres} 

Los pronombres {\hindifont हम}, {\hindifont तुम}, {\hindifont आप} ya tienen formas robustas y no cambian en oblicuo. Los pronombres demostrativos {\hindifont ये} y {\hindifont वे} cambian a {\hindifont इन} e {\hindifont उन} respectivamente en oblicuo.


\end{tcolorbox}

\subsection{4. Uso del Caso Oblicuo Plural con Postposiciones}


Todas las postposiciones aprendidas en el nivel A1 se pueden usar con el caso oblicuo plural:

\subsubsection{4.1 Ejemplos con las Postposiciones del Nivel A1}


\begin{itemize}

  \item {\hindifont लड़कों में} (laRkon men) - Entre los chicos

  \item {\hindifont लड़कियों पर} (laRkiyon par) - Sobre las chicas

  \item {\hindifont कमरों में} (kamron men) - En las habitaciones

  \item {\hindifont कुर्सियों पर} (kursiyon par) - Sobre las sillas

  \item {\hindifont लड़कों को} (laRkon ko) - A los chicos

  \item {\hindifont लड़कियों को} (laRkiyon ko) - A las chicas

  \item {\hindifont किताबों से} (kitaabon se) - De los libros

  \item {\hindifont घरों से} (gharon se) - De las casas

\end{itemize}

\subsubsection{4.2 Ejemplos Completos de Uso}


\begin{center}
\begin{tabularx}{\textwidth}{| X | X | X |}
\hline
\textbf{Oración} & \textbf{Traducción} & \textbf{Explicación} \ \hline
{\hindifont लड़कों में से कई खेलने गए।} & De entre los chicos, muchos fueron a jugar. & Uso de {\hindifont से} con plural oblicuo {\hindifont लड़कों} \ \hline
{\hindifont घरों के बाहर पेड़ लगाए गए।} & Árboles fueron plantados fuera de las casas. & Uso de {\hindifont के} con plural oblicuo {\hindifont घरों} \ \hline
{\hindifont किताबों के बारे में बात की।} & Habló sobre los libros. & Uso de {\hindifont के बारे में} con plural oblicuo {\hindifont किताबों} \ \hline
{\hindifont लड़कियों के साथ बाजार गया।} & Fui al mercado con las chicas. & Uso de {\hindifont के साथ} con plural oblicuo {\hindifont लड़कियों} \ \hline
{\hindifont नामों के लिए जगह थी।} & Había espacio para los nombres. & Uso de {\hindifont के लिए} con plural oblicuo {\hindifont नामों} \ \hline
{\hindifont इनको दिल्ली के पास रहता है।} & Éstos viven cerca de Delhi. & Uso de {\hindifont के पास} con pronombre oblicuo plural {\hindifont इन} \ \hline
\end{tabularx}
\end{center}

\subsubsection{4.3 Frases Comunes con Caso Oblicuo Plural}


\begin{itemize}

  \item {\hindifont लोगों से बातचीत करना आसान होता है।} - Hablar con la gente es fácil.

  \item {\hindifont लड़कों में से कई तैरना जानते हैं।} - Muchos de los chicos saben nadar.

  \item {\hindifont हम सभी एक दूसरे के लिए जिम्मेदार हैं।} - Todos nosotros somos responsables los unos de los otros.

  \item {\hindifont इमारतों के पास बहुत शोर होता है।} - Cerca de los edificios hay mucho ruido.

  \item {\hindifont इन किताबों में से कौन सी तुम्हारी पसंद की है?} - ¿Cuál de estos libros es de tu agrado?

  \item {\hindifont महिलाओं के समर्थन पर निर्भर करता है समाज का विकास।} - El desarrollo de la sociedad depende del apoyo de las mujeres.

\end{itemize}

\subsection{5. Comunicación: Hablar de Acciones que Involucran a Múltiples Personas u Objetos}


Con el caso oblicuo plural, podemos hablar de acciones que involucran a grupos de personas u objetos en diferentes roles gramaticales:

\subsubsection{5.1 Expresiones con Multiples Actores}


\begin{itemize}

  \item {\hindifont मैंने दोस्तों को पत्र लिखे।} - Les escribí cartas a mis amigos.

  \item {\hindifont उन्होंने बच्चों के लिए खिलौने खरीदे।} - Compraron juguetes para los niños.

  \item {\hindifont हम सभी एक दूसरे के साथ मिलकर काम करते हैं।} - Todos nosotros trabajamos juntos.

  \item {\hindifont लड़कों को लड़कियों के साथ खेलना चाहिए।} - Los chicos deberían jugar con las chicas.

  \item {\hindifont स्कूल के लिए छात्रों का उत्साह महत्वपूर्ण है।} - La entusiasmo de los estudiantes es importante para la escuela.

\end{itemize}

\subsubsection{5.2 Diálogos Ejemplares con Caso Oblicuo Plural}


\textbf{Diálogo 1: Comentando sobre un evento con multitudes}


\begin{itemize}

  \item A: {\hindifont कल त्योहार में बहुत लोग शामिल हुए।} (Kya kal tyohaar men bahut log shaamil hue?) - ¿Muchas personas participaron en la fiesta de ayer?

  \item B: {\hindifont हाँ, लोगों ने खूब मस्ती की।} (Haan, logon ne khoob mastii ki.) - Sí, la gente se divirtió mucho.

  \item A: {\hindifont क्या तुमने लोगों को नृत्य करते देखा?} (Kya tumne logon ko nrity karthe dekhaa?) - ¿Viste a la gente bailando?

  \item B: {\hindifont हाँ, लोगों ने बहुत अच्छा नृत्य किया। लड़कों और लड़कियों ने सबने मिलकर नृत्य किया।} (Haan, logon ne bahut achchhaa nrity kiya. LaRkon aur laRkiyon ne sabne milkar nrity kiya.) - Sí, la gente bailó muy bien. Todos los chicos y chicas bailaron juntos.

  \item A: {\hindifont क्या तुमने उन लोगों के साथ गाना गाया?} (Kya tumne un logon ke saath gaanaa gaayaa?) - ¿Cantaste canciones con aquellas personas?

  \item B: {\hindifont हाँ, हमने उनके साथ गाना गाया। हमारे गाने सभी को पसंद आए।} (Haan, hamanne unke saath gaanaa gaayaa. Hamaare gaane sabhi ko pasand aaye.) - Sí, cantamos con ellos. A todos les gustaron nuestras canciones.

\end{itemize}


\textbf{Diálogo 2: Discutiendo sobre una escuela o lugar de trabajo}


\begin{itemize}

  \item A: {\hindifont स्कूल के छात्रों से कैसे संबंध हैं?} (Skul ke chhaatron se kaise sambandh hain?) - ¿Cómo son las relaciones con los estudiantes de la escuela?

  \item B: {\hindifont छात्रों के साथ हमेशा अच्छा वातावरण रहता है।} (Chhaatron ke saath hamesha achchhaa vaatavarann rahataa hai.) - Siempre hay un buen ambiente con los estudiantes.

  \item A: {\hindifont क्या छात्रों को पुरस्कार दिए जाते हैं?} (Kya chhaatron ko puraskaar diye jaate hain?) - ¿Se dan premios a los estudiantes?

  \item B: {\hindifont हाँ, अच्छे काम करने वाले छात्रों के लिए पुरस्कार होते हैं।} (Haan, achchhe kaam karne vaale chhaatron ke liye puraskaar hote hain.) - Sí, hay premios para los estudiantes que hacen buen trabajo.

  \item A: {\hindifont क्या तुम उन छात्रों की सहायता करते हो?} (Kya tum un chhaatron ki sahaayataa karte ho?) - ¿Ayudas a aquellos estudiantes?

  \item B: {\hindifont हाँ, मैं उन छात्रों की सहायता करता हूँ जिन्हें समस्या है।} (Haan, main un chhaatron ki sahaayataa karta huun jinhen samasya hai.) - Sí, ayudo a los estudiantes que tienen problemas.

\end{itemize}


\hrulefill

\section{Unidad 15: Pasado Simple (Transitivo y Ergativo {\hindifont ने})}


\textbf{Objetivo:} Usar la construcción ergativa {\hindifont ने} con verbos transitivos en pasado.

\subsection{1. Introducción al Sistema Ergativo}


El sistema ergativo es una de las características más importantes y distintivas del hindi. A diferencia de las lenguas nomino-acusativas como el español, en las lenguas ergativas (como el hindi en pasado con verbos transitivos), el sujeto de una oración transitiva se marca de forma diferente al sujeto de una oración intransitiva.


\begin{tcolorbox}[colback=yellow!10!white,colframe=orange!75!black,title=Regla de Oro]

\paragraph{Regla de Oro: El Caso Ergativo (Postposición {\hindifont ने})} 

El sujeto de una oración usa la postposición \textbf{{\hindifont ने} (ne)} si y solo si se cumplen \textbf{DOS} condiciones a la vez:


\begin{enumerate}

  \item El verbo es \textbf{TRANSITIVO} (requiere un objeto directo, como {\hindifont खाना} comer, {\hindifont लिखना} escribir, {\hindifont देखना} ver).

  \item El tiempo verbal es \textbf{PERFECTO} o \textbf{PASADO SIMPLE} (acciones ya completadas).

\end{enumerate}


\textbf{Verbos Intransitivos} (como {\hindifont जाना} ir, {\hindifont सोना} dormir, {\hindifont उठना} levantarse) \textbf{NUNCA} usan {\hindifont ने}.


\end{tcolorbox}

\subsection{2. La Construcción Ergativa en el Pasado Simple}


En el pasado simple con verbos transitivos, la estructura es:


\textbf{Sujeto + ने + Objeto + Verbo (concordando con el Objeto)}


\begin{tcolorbox}[colback=blue!5!white,colframe=blue!75!black,title=Nota/Clarificación]

\paragraph{Concepto Fundamental} 

En la construcción ergativa, el verbo concuerda con el \textbf{objeto directo}, no con el sujeto. Esto es opuesto al español y al presente del hindi.


\begin{itemize}

  \item \textbf{Español:} "Él comió un plátano" - El verbo "comió" concuerda con el sujeto "él".

  \item \textbf{Hindi:} {\hindifont उसने केला खाया।} (Usne kela khayaa.) - El verbo "khayaa" concuerda con el objeto "kela" (plátano masculino).

\end{itemize}


\end{tcolorbox}

\subsubsection{2.1 Formación del Pasado Simple con Verbos Transitivos}


La formación del participio pasado para verbos transitivos es:


\begin{itemize}

  \item \textbf{Masculino Singular:} Forma base femenina + {\hindifont -आ} (-aa)
            
\begin{itemize}

  \item Ej: {\hindifont खाना} (khaanaa - comer) → {\hindifont खाया} (khayaa)

  \item Ej: {\hindifont लिखना} (likhnaa - escribir) → {\hindifont लिखा} (likhaa)

  \item Ej: {\hindifont लेना} (lena - tomar) → {\hindifont लिया} (liyaa)

\end{itemize}



  \item \textbf{Femenino Singular:} Forma base femenina + {\hindifont -ई} (-ii)
            
\begin{itemize}

  \item Ej: {\hindifont खाना} (khaanaa) → {\hindifont खाई} (khayii)

  \item Ej: {\hindifont लिखना} (likhnaa) → {\hindifont लिखी} (likhii)

  \item Ej: {\hindifont लेना} (lena) → {\hindifont ली} (li)

\end{itemize}



  \item \textbf{Masculino/Neutro Plural:} Forma base femenina + {\hindifont -ए} (-e)
            
\begin{itemize}

  \item Ej: {\hindifont खाना} (khaanaa) → {\hindifont खाए} (khaye)

  \item Ej: {\hindifont लिखना} (likhnaa) → {\hindifont लिखे} (likhe)

  \item Ej: {\hindifont लेना} (lena) → {\hindifont लिए} (liye)

\end{itemize}



  \item \textbf{Femenino Plural:} Igual que femenino singular (por defecto)
            
\begin{itemize}

  \item Se usa la forma femenina singular para objetos femeninos en plural.

\end{itemize}



\end{itemize}

\subsubsection{2.2 Tabla Comparativa: Verbos Transitivos vs. Intransitivos en Pasado Simple}


\begin{center}
\begin{tabularx}{\textwidth}{| X | X | X | X | X | X | X |}
\hline
\textbf{Tipo de Verbo} & \textbf{Sujeto} & \textbf{Objeto} & \textbf{Verbo} & \textbf{Hindi} & \textbf{Traducción} & \textbf{Concordancia} \ \hline
Intransitivo & {\hindifont वह} (vah) - Él & (No requiere objeto) & Verbo concuerda con el \textbf{sujeto} & {\hindifont वह गया।} (Vah gayaa.) & Él fue. & Verbo (gayaa) concuerda con sujeto (vah masc.sing.) \ \hline
Transitivo & {\hindifont वहने} (vahne) - Él (con ergativo) & {\hindifont सेब} (seb) - manzana (masc.) & Verbo concuerda con el \textbf{objeto} & {\hindifont उसने सेब खाया।} (Usne seb khayaa.) & Él comió la manzana. & Verbo (khayaa) concuerda con objeto (seb masc.sing.) \ \hline
Transitivo & {\hindifont वहने} (vahne) - Él (con ergativo) & {\hindifont रोटी} (rotii) - pan (fem.) & Verbo concuerda con el \textbf{objeto} & {\hindifont उसने रोटी खाई।} (Usne rotii khayii.) & Él comió el pan. & Verbo (khayii) concuerda con objeto (rotii fem.sing.) \ \hline
Transitivo & {\hindifont वहने} (vahne) - Él (con ergativo) & {\hindifont पुस्तकें} (pustaken) - libros (fem.pl.) & Verbo concuerda con el \textbf{objeto} & {\hindifont उसने पुस्तकें पढ़ीं।} (Usne pustaken padhii.) & Él leyó los libros. & Verbo (padhii) concuerda con objeto (pustaken fem.pl.) \ \hline
\end{tabularx}
\end{center}


\begin{tcolorbox}[colback=blue!5!white,colframe=blue!75!black,title=Nota/Clarificación]

\paragraph{Nota del Profesor: La Concordancia Invertida} 

Imagina que en hindi, cuando una acción se completa en el pasado y afecta directamente a algo/some (verbo transitivo), el \textbf{objeto directo} toma control y el verbo obedece al objeto directo en vez de obedecer al sujeto. Es como si el objeto directo se convirtiera temporalmente en el "sujeto gramatical" del verbo.


El sujeto original "cede" su poder de control al objeto directo, simbolizado por el uso de {\hindifont ने}.


\end{tcolorbox}

\subsubsection{2.3 Regla Especial: El Objeto con को}


\begin{tcolorbox}[colback=yellow!10!white,colframe=orange!75!black,title=Regla de Oro]


Cuando el \textbf{objeto directo} lleva la postposición {\hindifont को} (indicando que es específico o animado), el verbo \textbf{NO concuerda} con él. En su lugar, el verbo se queda en su forma \textbf{Masculina Singular (MS)} por defecto.


\end{tcolorbox}


\begin{center}
\begin{tabularx}{\textwidth}{| X | X | X | X | X | X | X |}
\hline
\textbf{Tipo de Objeto} & \textbf{Sujeto} & \textbf{Objeto} & \textbf{Verbo} & \textbf{Ejemplo en Hindi} & \textbf{Traducción} & \textbf{Explicación} \ \hline
Simple/No animado & {\hindifont वहने} (vahne) & {\hindifont सेब} (seb masc.) & Concordancia con objeto: {\hindifont खाया} (khayaa - masc.) & {\hindifont उसने सेब खाया।} & Él comió la manzana. & Verbo concuerda con el objeto \ \hline
Simple/No animado & {\hindifont वहने} (vahne) & {\hindifont रोटी} (rotii fem.) & Concordancia con objeto: {\hindifont खाई} (khayii - fem.) & {\hindifont उसने रोटी खाई।} & Él comió el pan. & Verbo concuerda con el objeto \ \hline
Con को (específico/animado) & {\hindifont वहने} (vahne) & {\hindifont सीता को} (Seetaa ko) & Forma MS por defecto: {\hindifont देखा} (dekhaa) & {\hindifont उसने सीता को देखा।} & Él vio a Sita. & Verbo no concuerda con objeto + को, se queda en forma MS \ \hline
Con को (específico/animado) & {\hindifont मैंने} (mainne - yo masc.) & {\hindifont उस लड़की को} (us laRkii ko) & Forma MS por defecto: {\hindifont पूछा} (poochhaa) & {\hindifont मैंने उस लड़की को पूछा।} & Yo pregunté a esa chica. & Verbo no concuerda con objeto fem. + को, se queda en forma MS \ \hline
\end{tabularx}
\end{center}

\subsection{3. Verbos Transitivos Comunes en la Lista T}


Esta es una lista de verbos transitivos comunes que se usan con la construcción ergativa en pasado simple:


\begin{center}
\begin{tabularx}{\textwidth}{| X | X | X | X | X | X | X |}
\hline
\textbf{Español} & \textbf{Hindi} & \textbf{Transliteración} & \textbf{Participio Pasado Masc.Sing.} & \textbf{Participio Pasado Fem.Sing.} & \textbf{Participio Pasado Masc.Pl.} & \textbf{Uso en Frase} \ \hline
Hacer & {\hindifont करना} & karnaa & {\hindifont किया} (kiyaa) & {\hindifont की} (kii) & {\hindifont किए} (kiye) & {\hindifont मैंने काम किया।} (Hice el trabajo.) \ \hline
Comer & {\hindifont खाना} & khaanaa & {\hindifont खाया} (khayaa) & {\hindifont खाई} (khayii) & {\hindifont खाए} (khaye) & {\hindifont रामने रोटी खाई।} (Ram comió pan.) \ \hline
Beber/Tomar & {\hindifont पीना} & piinaa & {\hindifont पिया} (piyaa) & {\hindifont पी} (pii) & {\hindifont पिए} (piye) & {\hindifont वहने दूध पिया।} (Él bebió leche.) \ \hline
Leer & {\hindifont पढ़ना} & padhnaa & {\hindifont पढ़ा} (padhaa) & {\hindifont पढ़ी} (padhii) & {\hindifont पढ़े} (padhe) & {\hindifont लड़कियों ने कविता पढ़ी।} (Las chicas leyeron poesía.) \ \hline
Ver & {\hindifont देखना} & dekhnaa & {\hindifont देखा} (dekhaa) & {\hindifont देखी} (dekhhii) & {\hindifont देखे} (dekhe) & {\hindifont हमने चिड़िया देखी।} (Vimos un pájaro.) \ \hline
Dar & {\hindifont देना} & denaa & {\hindifont दिया} (diyaa) & {\hindifont दी} (dii) & {\hindifont दिए} (diye) & {\hindifont मैंने तुम्हें किताब दी।} (Te di el libro.) \ \hline
Tomar/Recibir & {\hindifont लेना} & lena & {\hindifont लिया} (liyaa) & {\hindifont ली} (lii) & {\hindifont लिए} (liye) & {\hindifont उसने पैसा लिया।} (Él tomó/recebí dinero.) \ \hline
Escribir & {\hindifont लिखना} & likhnaa & {\hindifont लिखा} (likhaa) & {\hindifont लिखी} (likhii) & {\hindifont लिखे} (likhe) & {\hindifont रामने पत्र लिखा।} (Ram escribió una carta.) \ \hline
Caminar & {\hindifont चलना} & chalnaa & {\hindifont चला} (chalaa) & {\hindifont चली} (chalii) & {\hindifont चले} (chale) & {\hindifont लड़की ने बाज़ार चली।} (La chica caminó al mercado.) \ \hline
Correr & {\hindifont दौड़ना} & dauranaa & {\hindifont दौड़ा} (dauraa) & {\hindifont दौड़ी} (daurii) & {\hindifont दौड़े} (daure) & {\hindifont लड़कों ने पार्क में दौड़े।} (Los chicos corrieron en el parque.) \ \hline
\end{tabularx}
\end{center}

\subsection{4. Uso del Caso Ergativo con Pronombres}


El ergativo {\hindifont ने} también se usa con pronombres. Aquí están las formas:


\begin{center}
\begin{tabularx}{\textwidth}{| X | X | X | X | X |}
\hline
\textbf{Pronombre Nominativo} & \textbf{Forma Ergativa (con ने)} & \textbf{Transliteración} & \textbf{Ejemplo con Verbo} & \textbf{Traducción} \ \hline
{\hindifont मैं} (main) - Yo & {\hindifont मैंने} (mainne) & mainne & {\hindifont मैंने नाश्ता किया।} & Yo desayuné. \ \hline
{\hindifont तू} (tū) - Tú (íntimo) & {\hindifont तूने} (tūne) & tūne & {\hindifont तूने कल घर क्यों नहीं आया?} & ¿Por qué no viniste a casa ayer? \ \hline
{\hindifont तुम} (tum) - Tú (informal) & {\hindifont तुमने} (tumne) & tumne & {\hindifont तुमने फिल्म देखी?} & ¿Viste la película? \ \hline
{\hindifont आप} (āp) - Usted & {\hindifont आपने} (āpne) & āpne & {\hindifont आपने क्या कहा?} & ¿Qué dijo Ud.? \ \hline
{\hindifont यह} (yah) - Él/Ella (cercano) & {\hindifont इसने} (isne) & isne & {\hindifont इसने किताब पढ़ी।} & Él/Ella leyó el libro. \ \hline
{\hindifont वह} (vah) - Él/Ella (distante) & {\hindifont उसने} (usne) & usne & {\hindifont उसने खाना बनाया।} & Él/Ella cocinó comida. \ \hline
{\hindifont हम} (ham) - Nosotros/as & {\hindifont हमने} (hamne) & hamne & {\hindifont हमने अच्छा समय बिताया।} & Disfrutamos buen tiempo. \ \hline
{\hindifont ये} (ye) - Ellos/as (cercanos) & {\hindifont इन्होंने} (inhon-ne) & inhōnne & {\hindifont इन्होंने गाना गाया।} & Éstos cantaron una canción. \ \hline
{\hindifont वे} (ve) - Ellos/as (distantes) & {\hindifont उन्होंने} (unhon-ne) & unhōnne & {\hindifont उन्होंने यात्रा की।} & Ellos hicieron un viaje. \ \hline
\end{tabularx}
\end{center}

\subsection{5. Narración de Anécdotas Pasadas Completas}


Con la construcción ergativa, ahora podemos narrar eventos pasados completos con acciones que afectaron directamente a personas u objetos:

\subsubsection{5.1 Frases de Ejemplo}


\begin{itemize}

  \item {\hindifont मैंने कल स्कूल जाकर पुस्तकालय में किताबें देखीं।} - Ayer fui a la escuela y miré libros en la biblioteca. (mainne - masc. sing., kitaaben - fem. pl., dekhiin - fem. pl.)

  \item {\hindifont राम ने अपनी माँ को पत्र लिखा।} - Ram escribió una carta a su madre. (ram - masc., maa ko - fem. obj. with ko, likhaa - masc. sing. default)

  \item {\hindifont सीता ने बाज़ार में सब्ज़ियाँ खरीदीं।} - Sita compró verduras en el mercado. (seetaa - fem., sabziyan - fem. pl., kharidiin - fem. pl.)

  \item {\hindifont अध्यापक ने छात्रों को प्रश्न पूछे।} - El profesor hizo preguntas a los estudiantes. (adhyaapak - masc., chatron ko - obj. with ko, pooche - masc. sing. default)

  \item {\hindifont मेरे दोस्त ने मेरी मदद की।} - Mi amigo me ayudó. (dost - masc., madad - fem., ki - fem. sing.)

  \item {\hindifont लोगों ने अच्छी तरह से नाचा।} - La gente bailó bien. (logon - masc. pl., nachaa - masc. sing. default)

  \item {\hindifont डॉक्टर ने रोगी को दवा दी।} - El doctor le dio medicina al paciente. (doctor - masc., rogii ko - obj. with ko, dii - masc. sing. default)

\end{itemize}

\subsubsection{5.2 Diálogos con Construcción Ergativa}


\textbf{Diálogo 1: Conversación sobre una fiesta pasada}


\begin{itemize}

  \item A: {\hindifont क्या आपने कल की पार्टी में शामिल होने का निर्णय लिया?} - ¿Tomó Ud. la decisión de asistir a la fiesta de ayer?

  \item B: {\hindifont हाँ, मैंने शामिल होने का निर्णय लिया और मज़े की।} - Sí, tomé la decisión de asistir y me divertí.

  \item A: {\hindifont क्या आपने मेरे बारे में पूछा था?} - ¿Preguntó Ud. por mí?

  \item B: {\hindifont हाँ, मैंने आपके बारे में पूछा था, लेकिन किसी ने नहीं बताया।} - Sí, pregunté por Ud., pero nadie me dijo.

  \item A: {\hindifont तो मैंने आपको मैसेज क्यों नहीं पाया?} - ¿Por qué no le encontré a Ud. mensaje?

  \item B: {\hindifont क्योंकि मैंने अपना फोन घर पर छोड़ दिया था।} - Porque dejé mi teléfono en casa.

  \item A: {\hindifont लेकिन आपने किताब पढ़ने का वादा किया था। क्या आपने किताब पढ़ी?} - Pero Ud. prometió leer el libro. ¿Leyó Ud. el libro?

  \item B: {\hindifont नहीं, उस दिन मैंने किताब नहीं पढ़ी। मैंने अन्य काम किए।} - No, ese día no leí el libro. Hice otros trabajos.

\end{itemize}


\textbf{Diálogo 2: Narración de un viaje}


\begin{itemize}

  \item A: {\hindifont जब आप छुट्टियों पर गए थे, तो कहाँ गए थे?} - ¿A dónde fueron cuando estuvieron de vacaciones?

  \item B: {\hindifont हमने उत्तराखंड जाने का निर्णय लिया था।} - Tomamos la decisión de ir a Uttarakhand.

  \item A: {\hindifont क्या आपने पहाड़ देखे?} - ¿Vieron Ud. las montañas?

  \item B: {\hindifont हाँ, हमने सुंदर पहाड़ देखे और वहाँ की प्रकृति का आनंद लिया।} - Sí, vimos hermosas montañas y disfrutamos la naturaleza allí.

  \item A: {\hindifont मैं भी उस स्थान को देखना चाहता हूँ। क्या आपने फोटो खींचे?} - Yo también quiero ver ese lugar. ¿Sacaron Ud. fotos?

  \item B: {\hindifont हाँ, मैंने कई फोटो खींचे। आपको दिखाएँगे।} - Sí, tomé muchas fotos. Se las mostraré.

\end{itemize}

\subsection{6. Comunicación: Narrar Eventos Pasados Concretos}


Ahora puedes comunicar experiencias específicas del pasado con mayor precisión:


\begin{itemize}

  \item {\hindifont मैंने अपना काम पूरा किया और फिर आराम किया।} - Completé mi trabajo y luego descansé.

  \item {\hindifont उसने मुझे सलाह दी कि मैं अधिक पढ़ाई करूँ।} - Él me dio el consejo de que estudiara más.

  \item {\hindifont हमने उस दिन बहुत कुछ सीखा।} - Aprendimos muchas cosas ese día.

  \item {\hindifont मेरे मित्र ने बहुत मेहनत से परीक्षा लिखी।} - Mi amigo escribió el examen con mucho esfuerzo.

  \item {\hindifont वह लड़कियों को गाना सुनाया।} - Él les cantó una canción a las chicas.

  \item {\hindifont कल उसने मुझे बहुत अच्छा उपहार दिया।} - Ayer me dio un muy buen regalo.

  \item {\hindifont अध्यापक ने छात्रों को गृह कार्य दिया।} - El profesor les dio tarea a los estudiantes.

  \item {\hindifont मैंने अपने घर की सफाई की।} - Limpie mi casa.

\end{itemize}


\hrulefill

\section{Unidad 16: Pasado Imperfecto}


\textbf{Objetivo:} Describir hábitos y situaciones pasadas con el pasado imperfecto.

\subsection{1. Introducción al Pasado Imperfecto}


El pasado imperfecto en hindi se forma con la raíz del verbo + {\hindifont ता/ती/ते} (taa/tii/te) + una forma conjugada del verbo {\hindifont होना} en pasado ({\hindifont था/थी/थे}).


\begin{tcolorbox}[colback=yellow!10!white,colframe=orange!75!black,title=Regla de Oro]

\paragraph{Construcción del Pasado Imperfecto} 

\textbf{Sujeto + Verbo (raíz) + ता/ती/ते + था/थी/थे + (resto de la oración)}


La partícula {\hindifont ता/ती/ते} concuerda con el \textbf{sujeto} en género y número, no con el objeto como en el pretérito perfecto con ergativo.


\end{tcolorbox}

\subsubsection{1.1 Diferencia entre Pasado Simple e Imperfecto}


Es fundamental entender la diferencia entre los dos tiempos pasados:


\begin{center}
\begin{tabularx}{\textwidth}{| X | X | X | X |}
\hline
\textbf{Construcción} & \textbf{Uso} & \textbf{Ejemplo en Hindi} & \textbf{Traducción al Español} \ \hline
\textbf{Pasado Simple (Intransitivo)}\newline & Acción \textbf{puntual} o \textbf{completada} en el pasado & {\hindifont मैं गया।}\newline & Fui. / Yo fui. \ \hline
\textbf{Pasado Simple (Transitivo con Ergativo)}\newline & Acción \textbf{completada} realizada sobre un objeto & {\hindifont मैंने किताब पढ़ी।}\newline & Leí el libro. / Yo leí el libro. \ \hline
\textbf{Pasado Imperfecto}\newline & Acción \textbf{habitual}, \textbf{repetitiva} o \textbf{descriptiva} en el pasado & {\hindifont मैं किताब पढ़ता था।}\newline & Yo leía libros. / Yo solía leer libros. \ \hline
\end{tabularx}
\end{center}


\begin{tcolorbox}[colback=blue!5!white,colframe=blue!75!black,title=Nota/Clarificación]

\paragraph{Explicación del Pasado Imperfecto} 

El pasado imperfecto se usa para:


\begin{itemize}

  \item Describir \textbf{rutinas habituales} del pasado

  \item Mostrar \textbf{lo que solía ocurrir} en el pasado

  \item Establecer el \textbf{contexto} o \textbf{marco descriptivo} de una historia en pasado

  \item Describir \textbf{condiciones o estados} que existían en el pasado

\end{itemize}


\end{tcolorbox}

\subsection{2. Formación del Pasado Imperfecto}


La formación sigue exactamente el mismo patrón que el presente habitual, pero con el verbo {\hindifont होना} en forma PASADA:

\subsubsection{2.1 Tabla de Conjugación del Pasado Imperfecto}


\begin{center}
\begin{tabularx}{\textwidth}{| X | X | X | X | X | X |}
\hline
\textbf{Sujeto} & \textbf{Forma del Verbo} & \textbf{Transliteración} & \textbf{Ejemplo con {\hindifont पढ़ना} (leer)} & \textbf{Traducción} & \textbf{Comentario} \ \hline
{\hindifont मैं} (main) - Yo masc. & Raíz + {\hindifont ता} + {\hindifont था} & Root + taa + thaa & {\hindifont मैं पढ़ता था।} & Yo leía. (masc.) & Concordancia con sujeto masc. sing. \ \hline
{\hindifont मैं} (main) - Yo fem. & Raíz + {\hindifont ती} + {\hindifont थी} & Root + tii + thii & {\hindifont मैं पढ़ती थी।} & Yo leía. (fem.) & Concordancia con sujeto fem. sing. \ \hline
{\hindifont तू} (tū) - Tú masc. & Raíz + {\hindifont ता} + {\hindifont था} & Root + taa + thaa & {\hindifont तू पढ़ता था।} & Tú leías. (masc.) & Concordancia con sujeto masc. sing. \ \hline
{\hindifont तू} (tū) - Tú fem. & Raíz + {\hindifont ती} + {\hindifont थी} & Root + tii + thii & {\hindifont तू पढ़ती थी।} & Tú leías. (fem.) & Concordancia con sujeto fem. sing. \ \hline
{\hindifont तुम} (tum) - Tú masc./fem. & Raíz + {\hindifont ते} + {\hindifont थे} & Root + te + the & {\hindifont तुम पढ़ते थे।} & Tú leías. / Vosotros leíais. & Forma plural usada informalmente con "tum". \ \hline
{\hindifont आप} (āp) - Usted/Uds. & Raíz + {\hindifont ते} + {\hindifont थे} & Root + te + the & {\hindifont आप पढ़ते थे।} & Ud. leía. / Uds. leían. & Forma formal, concordancia con plural. \ \hline
{\hindifont यह/वह} masc. - Él & Raíz + {\hindifont ता} + {\hindifont था} & Root + taa + thaa & {\hindifont वह पढ़ता था।} & Él leía. / Ella leía. & Sujeto masc. sing. concuerda con verbo masc. sing. \ \hline
{\hindifont यह/वह} fem. - Ella & Raíz + {\hindifont ती} + {\hindifont थी} & Root + tii + thii & {\hindifont वह पढ़ती थी।} & Ella leía. / Él leía. & Sujeto fem. sing. concuerda con verbo fem. sing. \ \hline
{\hindifont हम} (ham) - Nosotros masc. & Raíz + {\hindifont ते} + {\hindifont थे} & Root + te + the & {\hindifont हम पढ़ते थे।} & Nosotros leíamos. (masc.) & Concordancia con sujeto masc. pl. \ \hline
{\hindifont हम} (ham) - Nosotras fem. & Raíz + {\hindifont ती} + {\hindifont थीं} & Root + tii + thiin & {\hindifont हम पढ़ती थीं।} & Nosotras leíamos. (fem.) & Concordancia con sujeto fem. pl. \ \hline
{\hindifont ये/वे} masc. - Ellos & Raíz + {\hindifont ते} + {\hindifont थे} & Root + te + the & {\hindifont वे पढ़ते थे।} & Ellos leían. (masc.) & Concordancia con sujeto masc. pl. \ \hline
{\hindifont ये/वे} fem. - Ellas & Raíz + {\hindifont ती} + {\hindifont थीं} & Root + tii + thiin & {\hindifont वे पढ़ती थीं।} & Ellas leían. (fem.) & Concordancia con sujeto fem. pl. \ \hline
\end{tabularx}
\end{center}

\subsubsection{2.2 Ejemplos con Diferentes Verbos}


\begin{center}
\begin{tabularx}{\textwidth}{| X | X | X | X | X |}
\hline
\textbf{Verbo Infinitivo} & \textbf{Raíz} & \textbf{Sujeto (masc.)} & \textbf{Ejemplo Pasado Imperfecto} & \textbf{Traducción} \ \hline
{\hindifont खाना} (comer) & खा (khaa) & {\hindifont मैं} (main) & {\hindifont मैं रोटी खाता था।} (Main roti khataa thaa.) & Yo comía pan. (masc.) \ \hline
{\hindifont पीना} (beber) & पी (pii) & {\hindifont वह} (vah) fem. & {\hindifont वह दूध पीती थी।} (Vah doodh peetii thii.) & Ella bebía leche. (fem.) \ \hline
{\hindifont लिखना} (escribir) & लिख (likh) & {\hindifont हम} (ham) & {\hindifont हम पत्र लिखते थे।} (Ham patr likhte the.) & Nosotros escribíamos cartas. (masc.) \ \hline
{\hindifont देखना} (ver) & देख (dekh) & {\hindifont ये} (ye) masc. & {\hindifont ये फिल्म देखते थे।} (Ye film dekhte the.) & Éstos veían películas. (masc.) \ \hline
{\hindifont चलना} (andar/caminar) & चल (chal) & {\hindifont आप} (aap) & {\hindifont आप पार्क में चलते थे।} (Aap park men chalte the.) & Ud. caminaba en el parque. \ \hline
{\hindifont जाना} (ir) & जा (jaa) & {\hindifont वह} (vah) masc. & {\hindifont वह घर जाता था।} (Vah ghar jaataa thaa.) & Él iba a casa. (masc.) \ \hline
\end{tabularx}
\end{center}

\subsection{3. Uso del Pasado Imperfecto}


El pasado imperfecto tiene varios usos importantes en la comunicación en hindi:

\subsubsection{3.1 Para Describir Hábitos Pasados}


\begin{itemize}

  \item {\hindifont जब मैं छोटा था, मैं हर रोज़ पार्क में खेलता था।} - Cuando yo era joven, jugaba en el parque todos los días.

  \item {\hindifont वह हर सुबह स्कूल जाता था।} - Él iba a la escuela todas las mañanas.

  \item {\hindifont हम दोपहर में घर पर खाना खाते थे।} - Comíamos en casa por la tarde.

  \item {\hindifont उसकी माँ हर दिन रात में कहानियाँ सुनाती थी।} - Su madre contaba historias todas las noches.

\end{itemize}

\subsubsection{3.2 Para Establecer el Contexto de una Historia}


El pasado imperfecto es ideal para describir el escenario en una narración:


\begin{itemize}

  \item {\hindifont एक बार एक राजा था। वह एक सुंदर महल में रहता था।} - Érase una vez un rey. Él vivía en un palacio hermoso.

  \item {\hindifont मौसम ठंढा था और बर्फ़ गिर रही थी।} - El tiempo estaba frío y estaba cayendo nieve.

  \item {\hindifont लोग सड़कों पर चलते थे और दुकानें खुली रहती थीं।} - La gente caminaba en las calles y las tiendas permanecían abiertas.

  \item {\hindifont बच्चे खेल रहे थे जब अध्यापक आया।} - Los niños estaban jugando cuando llegó el profesor. (Contraste con pasado simple para la acción principal)

\end{itemize}

\subsubsection{3.3 Para Describir Estados o Situaciones Persistentes en el Pasado}


\begin{itemize}

  \item {\hindifont वह बीमार रहता था।} - Él solía estar enfermo. / Él estaba enfermo con frecuencia.

  \item {\hindifont मैं उस समय दिल्ली में रहता था।} - Vivía en Delhi en ese tiempo.

  \item {\hindifont हम उस देश में खुश रहते थे।} - Éramos felices en ese país.

  \item {\hindifont मेरे पास पैसा नहीं रहता था।} - No tenía dinero.

\end{itemize}


\begin{tcolorbox}[colback=blue!5!white,colframe=blue!75!black,title=Nota/Clarificación]

\paragraph{Consejo del Profesor: Pasado Imperfecto vs. Pasado Simple} 

Una buena manera de distinguir entre ambos tiempos:


\begin{itemize}

  \item \textbf{Pasado Imperfecto:} Describe \textit{qué solía hacerse}, \textit{cómo era algo} o \textit{el entorno} en el pasado. "Solía...", "cada día...", "era costumbre..."

  \item \textbf{Pasado Simple:} Describe \textit{acciones específicas} que \textit{se completaron} en el pasado. "Fui...", "Hice...", "Vinieron..."

\end{itemize}


En una historia: El imperfecto describe el escenario y hábitos; el simple describe eventos específicos.


\end{tcolorbox}

\subsection{4. Expresiones Temporales Comunes con el Pasado Imperfecto}


Hay varias expresiones que indican claramente el uso del pasado imperfecto:


\begin{center}
\begin{tabularx}{\textwidth}{| X | X | X | X | X |}
\hline
\textbf{Español} & \textbf{Expresión en Hindi} & \textbf{Literalmente} & \textbf{Uso} & \textbf{Ejemplo} \ \hline
Cuando era niño/a & {\hindifont जब मैं छोटा था/थी} & Jab main chhota thaa/thii & Para eventos de la infancia & {\hindifont जब मैं छोटा था, मैं बहुत खेलता था।} (Cuando era niño, jugaba mucho.) \ \hline
Cada día & {\hindifont हर रोज़} & Har roz & Para hábitos diarios & {\hindifont हर रोज़ मैं स्कूल जाता था।} (Cada día iba a la escuela.) \ \hline
Siempre & {\hindifont हमेशा} & Hameshaa & Para estados persistentes & {\hindifont हमेशा मौसम गर्म रहता था।} (Siempre el clima era caliente.) \ \hline
A menudo & {\hindifont अक्सर} & Aksar & Para acciones repetidas & {\hindifont अक्सर वह मेरे घर आता था।} (A menudo él venía a mi casa.) \ \hline
Cada semana & {\hindifont हर हफ़्ते} & Har hafte & Para hábitos semanales & {\hindifont हर हफ़्ते हम सिनेमा जाते थे।} (Cada semana íbamos al cine.) \ \hline
En ese tiempo & {\hindifont उस समय} & Us samay & Para períodos pasados & {\hindifont उस समय मैं दिल्ली में रहता था।} (En ese tiempo vivía en Delhi.) \ \hline
Anteriormente & {\hindifont पहले} & Pahle & Para estados anteriores & {\hindifont पहले वह यहाँ रहता था।} (Anteriormente él vivía aquí.) \ \hline
\end{tabularx}
\end{center}

\subsection{5. Comunicación: Describir Hábitos de la Infancia y Situaciones Pasadas}


Ahora puedes comunicar experiencias pasadas de manera más rica y descriptiva:

\subsubsection{5.1 Frases para Hablar de Hábitos Pasados}


\begin{itemize}

  \item {\hindifont मैं बचपन में बहुत किताबें पढ़ता था।} - Leía muchos libros en la infancia.

  \item {\hindifont हम रोज़ पार्क में जाकर खेलते थे।} - Iíamos al parque todos los días y jugábamos.

  \item {\hindifont तुम जब छात्र थे तब क्या करते थे?} - ¿Qué hacías cuando eras estudiante?

  \item {\hindifont वह हर सुबह दौड़ता था।} - Él corría todas las mañanas.

  \item {\hindifont क्या आप बचपन में गाने गाते थे?} - ¿Cantabas canciones en la infancia?

  \item {\hindifont मैं अक्सर अपनी माँ की मदद करता था।} - A menudo ayudaba a mi madre.

\end{itemize}

\subsubsection{5.2 Descripciones de Situaciones Pasadas}


\begin{itemize}

  \item {\hindifont माहौल बहुत शांत रहता था।} - El ambiente solía ser muy tranquilo.

  \item {\hindifont हवा ठंढी और आरामदायक रहती थी।} - El aire solía ser fresco y cómodo.

  \item {\hindifont बच्चे स्कूल में खुश रहते थे।} - Los niños solían ser felices en la escuela.

  \item {\hindifont माता-पिता बच्चों के अध्ययन पर ध्यान देते थे।} - Los padres solían prestar atención al estudio de los niños.

  \item {\hindifont लोग एक-दूसरे से बातचीत करते थे।} - La gente solía conversar entre sí.

  \item {\hindifont उस समय तकनीक अत्यधिक सीमित रहती थी।} - La tecnología solía ser muy limitada en ese tiempo.

\end{itemize}

\subsubsection{5.3 Diálogos con Pasado Imperfecto}


\textbf{Diálogo 1: Hablando sobre hábitos pasados}


\begin{itemize}

  \item A: {\hindifont जब आप बच्चे थे तो आप क्या करते थे?} (Cuando Ud. era niño, ¿qué hacía Ud.?)

  \item B: {\hindifont हर रोज़ मैं पार्क में खेलता था। मैं लंबा समय तक खेलता रहता था।} (Todos los días jugaba en el parque. Solía jugar durante mucho tiempo.)

  \item A: {\hindifont क्या आप पढ़ाई करते थे?} (¿Estudiaba Ud.?)

  \item B: {\hindifont हाँ, मैं रोज़ पढ़ाई करता था। मैं अच्छा छात्र रहता था।} (Sí, estudiaba todos los días. Solía ser un buen estudiante.)

  \item A: {\hindifont क्या आप अपने मित्रों के साथ खेलते थे?} (¿Jugaba Ud. con sus amigos?)

  \item B: {\hindifont हाँ, आप ठीक कह रहे हैं। हम अक्सर एक-दूसरे के घर जाते थे और खेलते थे।} (Sí, Ud. tiene razón. Solíamos ir a las casas de los demás y jugar.)

\end{itemize}


\textbf{Diálogo 2: Narrando sobre una ciudad en el pasado}


\begin{itemize}

  \item A: {\hindifont क्या आप जानते हैं कि यह शहर पहले कैसा था?} (¿Sabe Ud. cómo era esta ciudad antes?)

  \item B: {\hindifont हाँ, मैं जानता हूँ। यह शहर पहले बहुत शांत रहता था।} (Sí, lo sé. Esta ciudad solía ser muy tranquila.)

  \item A: {\hindifont क्या यहाँ कम लोग रहते थे?} (¿Solía haber menos gente aquí?)

  \item B: {\hindifont हाँ, बहुत कम लोग रहते थे। लोग धीमी गति से जीवन जीते थे।} (Sí, solía haber muy poca gente. La gente vivía la vida con calma.)

  \item A: {\hindifont क्या तब कोई तकनीक नहीं थी?} (¿No había tecnología en ese momento?)

  \item B: {\hindifont थोड़ी थी। लेकिन लोग अधिक समय परिवार के साथ बिताते थे।} (Había un poco. Pero la gente solía pasar más tiempo con la familia.)

  \item A: {\hindifont यह दिलचस्प है। अब स्थिति बहुत अलग है।} (Eso es interesante. Ahora la situación es muy diferente.)

  \item B: {\hindifont हाँ, अब लोग अधिक व्यस्त रहते हैं।} (Sí, ahora la gente solía estar más ocupada.)

\end{itemize}

\subsection{6. Práctica Combinada: Pasado Simple vs. Imperfecto}


Es importante poder distinguir entre los dos usos:

\subsubsection{6.1 Situaciones de la Vida Real}


\begin{center}
\begin{tabularx}{\textwidth}{| X | X | X | X |}
\hline
\textbf{Situación} & \textbf{Con Pasado Imperfecto} & \textbf{Con Pasado Simple} & \textbf{Explicación} \ \hline
Hábito general & {\hindifont वह हर दिन खाना बनाती थी।} (Ella solía cocinar todos los días.) & {\hindifont उसने आज खाना बनाया।} (Ella cocinó hoy.) & Imperfecto para hábitos, simple para evento específico \ \hline
Estado general & {\hindifont मौसम ठंढा रहता था।} (El clima solía estar frío.) & {\hindifont मौसम ठंढा था।} (El clima estaba frío.) & Imperfecto para estado persistente, simple para estado en un momento \ \hline
Acción en progreso como contexto & {\hindifont वह पुस्तक पढ़ रहा था जब उसका दोस्त आया।} (Él estaba leyendo un libro cuando llegó su amigo.) & {\hindifont वह कल दिल्ली गया।} (Él fue a Delhi ayer.) & Imperfecto para acción en progreso, simple para acción concluida \ \hline
\end{tabularx}
\end{center}


\hrulefill

\section{Unidad 17: Verbos Modales (Obligación)}


\textbf{Objetivo:} Expresar diferentes grados de obligación y necesidad.

\subsection{1. Introducción a los Verbos Modales de Obligación}


Los verbos modales de obligación expresan que algo debe hacerse, es necesario hacerlo o es importante realizar una acción. En hindi, hay diferentes formas de expresar estos matices de obligación o necesidad.


\begin{tcolorbox}[colback=yellow!10!white,colframe=orange!75!black,title=Regla de Oro]

\paragraph{Concepto Fundamental} 

Los verbos modales de obligación en hindi no cambian según la persona del sujeto, pero se combinan con otros verbos (en forma oblicua o con partículas) para expresar distintos niveles de necesidad o deber.


\end{tcolorbox}

\subsection{2. Expresión de Obligación con el Infinitivo Oblicuo + है (hai)}


La primera forma de expresar obligación es con la estructura: \textbf{Infinitivo Oblicuo + है}

\subsubsection{2.1 Formación del Infinitivo Oblicuo}


El infinitivo oblicuo se forma quitando la terminación {\hindifont -ना} de los verbos en infinitivo y añadiendo {\hindifont -ने} o dejando solo la raíz dependiendo del verbo:


\begin{itemize}

  \item Verbos terminados en {\hindifont -ना} ({\hindifont -naa}): Quita {\hindifont -ना} y añade {\hindifont -ने}

\begin{itemize}

  \item {\hindifont करना} (karnaa - hacer) → {\hindifont करने} (karne)

  \item {\hindifont खाना} (khaanaa - comer) → {\hindifont खाने} (khaane)

  \item {\hindifont पीना} (piinaa - beber) → {\hindifont पीने} (piine)

\end{itemize}



  \item Verbos terminados en {\hindifont -ना} ({\hindifont -naa}) con raíces que terminan en vocal: A menudo solo se quita {\hindifont -ना}

\begin{itemize}

  \item {\hindifont लेना} (lena - tomar) → {\hindifont ले} (le)

  \item {\hindifont देना} (denaa - dar) → {\hindifont दे} (de)

\end{itemize}



\end{itemize}

\subsubsection{2.2 Estructura: Infinitivo Oblicuo + है (hai)}


Esta construcción expresa \textbf{una necesidad planeada o intencionada}. Se puede traducir como "tener que" (en un plan futuro).


\begin{center}
\begin{tabularx}{\textwidth}{| X | X | X | X | X |}
\hline
\textbf{Verbo Infinitivo} & \textbf{Forma Oblicua} & \textbf{Estructura} & \textbf{Ejemplo} & \textbf{Traducción} \ \hline
{\hindifont करना} (karnaa) & {\hindifont करने} (karne) & [Sujeto] + {\hindifont करने का/की/के} + {\hindifont है} & {\hindifont मुझे काम करने की ज़रूरत है।} & Necesito trabajar. (Tengo que trabajar.) \ \hline
{\hindifont जाना} (jaanaa) & {\hindifont जाने} (jaane) & {\hindifont मुझे जाने की ज़रूरत है।} & {\hindifont मुझे घर जाने की ज़रूरत है।} & Necesito ir a casa. \ \hline
{\hindifont पढ़ना} (padhnaa) & {\hindifont पढ़ने} (padhne) & {\hindifont तुम्हें पढ़ने की आवश्यकता है।} & {\hindifont तुम्हें परीक्षा पढ़ने की आवश्यकता है।} & Tienes que estudiar para el examen. \ \hline
{\hindifont खाना} (khaanaa) & {\hindifont खाने} (khaane) & {\hindifont उसे खाने की ज़रूरत है।} & {\hindifont उसे स्वास्थ्य के लिए अच्छा खाने की ज़रूरत है।} & Él necesita comer bien por su salud. \ \hline
{\hindifont लिखना} (likhnaa) & {\hindifont लिखने} (likhne) & {\hindifont हमें लिखने की आवश्यकता है।} & {\hindifont हमें रिपोर्ट लिखने की आवश्यकता है।} & Necesitamos escribir el informe. \ \hline
\end{tabularx}
\end{center}

\subsection{3. Expresión de Obligación con पड़ना (paRnaa)}


Otra forma de expresar la obligación, especialmente una obligación externa o indeseada, es con la estructura: \textbf{Verbo en gerundio + पड़ना}

\subsubsection{3.1 Formación del Gerundio}


El gerundio se forma de la siguiente manera:


\begin{itemize}

  \item La raíz del verbo + {\hindifont कर} ({\hindifont kar}) o {\hindifont ते/ती/ते} para formar una construcción gerundiva.

  \item Por ejemplo: {\hindifont खाना} (khaanaa - comer) → {\hindifont खाते-खाते} (khaate-khaate - comiendo-comiendo)

  \item Otro método común: {\hindifont ले} (le) para verbos como {\hindifont लेना}, {\hindifont दे} (de) para {\hindifont देना}, etc.

\end{itemize}


\begin{tcolorbox}[colback=blue!5!white,colframe=blue!75!black,title=Nota/Clarificación]

\paragraph{Nota del Profesor: El Verbo पड़ना} 

{\hindifont पड़ना} (paRnaa) literalmente significa "caer", pero en esta construcción modal significa que la acción "cae sobre" el sujeto como una obligación impuesta.


La estructura es: \textbf{[Sujeto] + [Objeto] + [Verbo en forma oblicua] + {\hindifont ना पड़ना}}


\end{tcolorbox}

\subsubsection{3.2 Estructura: Infinitivo + पड़ना}


Esto expresa una \textbf{obligación externa}, a menudo \textbf{indeseada} o \textbf{imprevista}. Se puede traducir como "tener que" (por circunstancias externas).


\begin{center}
\begin{tabularx}{\textwidth}{| X | X | X | X | X |}
\hline
\textbf{Verbo Infinitivo} & \textbf{Forma Oblicua + पड़ना} & \textbf{Ejemplo} & \textbf{Traducción} & \textbf{Comentario} \ \hline
{\hindifont करना} (karnaa - hacer) & {\hindifont करना पड़ा} (karanaa paRaa) & {\hindifont मुझे काम करना पड़ा।} & Tuve que trabajar. (Fue necesario por circunstancias.) & Obligación externa/imprevista \ \hline
{\hindifont जाना} (jaanaa - ir) & {\hindifont जाना पड़ा} (jaanaa paRaa) & {\hindifont हमें अस्पताल जाना पड़ा।} & Tuvimos que ir al hospital. & Por necesidad/emergencia \ \hline
{\hindifont पढ़ना} (padhnaa - leer/estudiar) & {\hindifont पढ़ना पड़ा} (padhnaa paRaa) & {\hindifont रात में भी पढ़ना पड़ा।} & Tuvimos que estudiar incluso por la noche. & Indica un esfuerzo adicional \ \hline
{\hindifont बोलना} (bolnaa - hablar) & {\hindifont बोलना पड़ा} (bolnaa paRaa) & {\hindifont उसे सच बोलना पड़ा।} & Él tuvo que decir la verdad. & En contra de su deseo \ \hline
{\hindifont उठना} (uthnaa - levantarse) & {\hindifont उठना पड़ा} (uthnaa paRaa) & {\hindifont मुझे सुबह 4 बजे उठना पड़ा।} & Tuve que levantarme a las 4 de la mañana. & Contrario a lo habitual/deseado \ \hline
\end{tabularx}
\end{center}

\subsubsection{3.3 Conjugación de पड़ना en Pasado}


El verbo {\hindifont पड़ना} se conjuga en el tiempo pasado y concuerda con el sujeto:


\begin{itemize}

  \item {\hindifont मैं} (main) masc. → {\hindifont पड़ा} (paRaa)

  \item {\hindifont मैं} (main) fem. → {\hindifont पड़ी} (paRii)

  \item {\hindifont वह} masc. → {\hindifont पड़ा} (paRaa)

  \item {\hindifont वह} fem. → {\hindifont पड़ी} (paRii)

  \item {\hindifont हम/ये/वे} masc. pl. → {\hindifont पड़े} (paRe)

  \item {\hindifont हम/ये/वे} fem. pl. → {\hindifont पड़ीं} (paRiin)

\end{itemize}

\subsection{4. Contraste entre las Diferentes Formas de Obligación}


\begin{tcolorbox}[colback=blue!5!white,colframe=blue!75!black,title=Nota/Clarificación]

\paragraph{Distinción Importante: Tipos de Obligación} 

Hay una diferencia marcada entre los diferentes modos de expresar obligación:


\begin{itemize}

  \item \textbf{Infinitivo oblicuo + है}: Obligación planeada o intencionada. "Tengo que hacer algo porque lo he decidido o es parte de mi plan."

  \item \textbf{Infinitivo + पड़ना}: Obligación impositiva, externa, a menudo indeseada. "Tengo que hacer algo porque las circunstancias me obligan o fue inevitable."

  \item \textbf{चाहिए} (chaahie): Necesidad o consejo. "Se debería hacer algo por razones de bienestar, sabiduría o necesidad."

\end{itemize}


\end{tcolorbox}


\begin{center}
\begin{tabularx}{\textwidth}{| X | X | X | X | X |}
\hline
\textbf{Tipo de Obligación} & \textbf{Construcción} & \textbf{Ejemplo} & \textbf{Traducción Española} & \textbf{Características} \ \hline
Necesidad Planeada & Infinitivo oblicuo + {\hindifont है} & {\hindifont मुझे काम करने की आवश्यकता है।} & Necesito trabajar. & Voluntaria, planificada \ \hline
Obligación Impuesta & Infinitivo + {\hindifont पड़ना} & {\hindifont मुझे काम करना पड़ा।} & Tuve que trabajar. & Por circunstancias, a menudo indeseada \ \hline
Consejo o Necesidad & {\hindifont को} + {\hindifont चाहिए} & {\hindifont तुम्हें आराम करना चाहिए।} & Deberías descansar. & Recomendación, necesidad para bienestar \ \hline
\end{tabularx}
\end{center}

\subsection{5. Repaso del Verbo Modal सकना (saknaa)}


Ya estudiamos anteriormente el verbo {\hindifont सकना} que expresa habilidad o posibilidad. Ahora lo contrastaremos con otros modales:


\begin{center}
\begin{tabularx}{\textwidth}{| X | X | X | X | X |}
\hline
\textbf{Modal} & \textbf{Significado} & \textbf{Estructura} & \textbf{Ejemplo} & \textbf{Traducción} \ \hline
{\hindifont सकना} & Capacidad/Poder & Raíz + {\hindifont सकना} + {\hindifont होना} (conjugado) & {\hindifont मैं हिंदी बोल सकता हूँ।} & Puedo hablar hindi. \ \hline
{\hindifont पाना} & Lograr/Conseguir hacer & Raíz + {\hindifont पाना} + {\hindifont होना} (conjugado) & {\hindifont मैं समस्या को हल पा गया।} & Logré resolver el problema. \ \hline
\end{tabularx}
\end{center}

\subsubsection{5.1 Contraste entre सकना y पड़ना}


Aunque ambos expresan posibilidad o necesidad, tienen matices diferentes:


\begin{itemize}

  \item {\hindifont सकना}: Posibilidad física o habilidad. "Puedo hacerlo si quiero."

  \item {\hindifont पड़ना}: Necesidad impositiva. "Tuve que hacerlo, no tuve elección."

\end{itemize}


Ejemplos de contraste:


\begin{itemize}

  \item {\hindifont मैं यह काम कर सकता हूँ।} - Puedo hacer este trabajo. (Capacidad)

  \item {\hindifont मुझे यह काम करना पड़ा।} - Tuve que hacer este trabajo. (Obligación externa)

  \item {\hindifont मैं देर से घर जा सकता हूँ।} - Puedo ir a casa tarde. (Capacidad/opción)

  \item {\hindifont मुझे देर से घर जाना पड़ा।} - Tuve que ir a casa tarde. (Por circunstancia externa)

\end{itemize}

\subsection{6. Comunicación: Expresar Obligación, Necesidad y Habilidad}


Con los nuevos conocimientos de verbos modales, puedes comunicar con mayor precisión:

\subsubsection{6.1 Frases para Expresar Obligación}


\begin{itemize}

  \item {\hindifont मुझे अभी जाना है।} - Tengo que irme ahora. (obligación planeada)

  \item {\hindifont हमें परीक्षा में अच्छा करना पड़ेगा।} - Tendremos que hacerlo bien en el examen. (obligación futura)

  \item {\hindifont मुझे नौकरी छोड़नी पड़ी।} - Tuve que dejar el trabajo. (circunstancia externa)

  \item {\hindifont तुम्हें आराम करना चाहिए।} - Deberías descansar. (recomendación)

  \item {\hindifont उसे अपनी गलती के लिए क्षमा मांगनी पड़ी।} - Él tuvo que pedir perdón por su error.

  \item {\hindifont मैंने अधिकारी से मिलना पड़ा।} - Tuve que reunirme con el oficial.

  \item {\hindifont वह रोज़ अपना काम करने की आवश्यकता महसूस करता है।} - Él siente la necesidad de hacer su trabajo todos los días.

  \item {\hindifont इस काम को पूरा करने के लिए तुम्हें अतिरिक्त समय लेना पड़ सकता है।} - Puede que tengas que tomar tiempo adicional para completar este trabajo.

\end{itemize}

\subsubsection{6.2 Frases para Expresar Habilidad y Posibilidad}


\begin{itemize}

  \item {\hindifont मैं तैरना सकता हूँ।} - Puedo nadar. (capacidad)

  \item {\hindifont क्या तुम गाना गा सकते हो?} - ¿Puedes cantar una canción?

  \item {\hindifont हम यात्रा के लिए पर्याप्त पैसा बचा सकते हैं।} - Podemos ahorrar suficiente dinero para viajar.

  \item {\hindifont अच्छी तरह से अध्ययन करके वह परीक्षा पास कर पाएगा।} - Logrará aprobar el examen estudiando bien. (combinación de habilidad y logro)

  \item {\hindifont थोड़ा अभ्यास करके तुम यह काम आसानी से कर पाओगे।} - Con un poco de práctica lograrás hacer este trabajo fácilmente.

\end{itemize}

\subsubsection{6.3 Diálogos con Verbos Modales}


\textbf{Diálogo 1: En la oficina}


\begin{itemize}

  \item Jefe: {\hindifont क्या आप आज रात तक यह काम पूरा कर सकेंगे?} - ¿Podrá Ud. completar este trabajo para esta noche?

  \item Empleado: {\hindifont हाँ, मैं इसे पूरा करने की कोशिश करूँगा। लेकिन मुझे बहुत काम करना पड़ेगा।} - Sí, intentaré completarlo. Pero tendré que trabajar mucho.

  \item Jefe: {\hindifont यह जरूरी है। ग्राहक को सुबह तक यह चाहिए।} - Es necesario. El cliente necesita esto para la mañana.

  \item Empleado: {\hindifont ठीक है। मुझे यह करना होगा। इसलिए आज मैं काम में रहूँगा।} - Bien. Tendré que hacerlo. Por eso me quedaré en el trabajo hoy.

  \item Jefe: {\hindifont मैं आपकी मेहनत की सराहना करता हूँ।} - Aprecio su esfuerzo.

  \item Empleado: {\hindifont धन्यवाद। मुझे अपने कर्तव्य को पूरा करना है।} - Gracias. Tengo que cumplir con mi deber.

\end{itemize}


\textbf{Diálogo 2: Conversación sobre hábitos y necesidades}


\begin{itemize}

  \item A: {\hindifont तुम अक्सर किताबें कहाँ खरीदते हो?} - ¿Dónde compras libros normalmente?

  \item B: {\hindifont मैं बाज़ार में खरीदता हूँ। लेकिन अब मुझे ऑनलाइन खरीदने की आवश्यकता है।} - Compro en el mercado. Pero ahora necesito comprar en línea.

  \item A: {\hindifont क्यों? क्या बाज़ार में नहीं मिल रहा?} - ¿Por qué? ¿No se encuentra en el mercado?

  \item B: {\hindifont मेरे पास बहुत कम समय है। इसलिए मुझे ऑनलाइन ऑर्डर करना पड़ा।} - No tengo mucho tiempo. Por eso tuve que ordenar en línea.

  \item A: {\hindifont मैं ऑनलाइन खरीद से बच सकता हूँ।} - Puedo evitar la compra en línea.

  \item B: {\hindifont हाँ, लेकिन मेरे पास विकल्प नहीं था। यह आवश्यक था।} - Sí, pero no tenía opción. Era necesario.

  \item A: {\hindifont क्या तुम तेज़ी से काम कर पाते हो?} - ¿Puedes trabajar rápidamente?

  \item B: {\hindifont ज़्यादातर हाँ, लेकिन कभी-कभी मुझे अधिक समय लेना पड़ता है।} - Mayormente sí, pero a veces tengo que tomar más tiempo.

\end{itemize}

\subsection{7. Práctica: Reconocer y Usar Diferentes Tipos de Obligación}


Identifica qué tipo de obligación se expresa en cada oración:


\begin{enumerate}

  \item {\hindifont उसे अस्पताल जाना पड़ा।} (Tuvo que ir al hospital.) - \textit{Obligación impositiva por circunstancia externa}

  \item {\hindifont तुम्हें अभी आराम करने की ज़रूरत है।} (Necesitas descansar ahora.) - \textit{Necesidad planeada}

  \item {\hindifont मुझे कल यात्रा करनी चाहिए।} (Debería viajar mañana.) - \textit{Consejo/necesidad}

  \item {\hindifont हमें घर जल्दी पहुँचना सकता है।} (Podemos llegar a casa pronto.) - \textit{Capacidad}

  \item {\hindifont मैंने इस प्रश्न का उत्तर देना पड़ा।} (Tuve que responder a esta pregunta.) - \textit{Obligación impositiva}

\end{enumerate}


\hrulefill

\section{Unidad 18: Adjetivos Comparativos y Superlativos}


\textbf{Objetivo:} Comparar personas, objetos y lugares usando adjetivos comparativos y superlativos.

\subsection{1. Introducción a las Comparaciones en Hindi}


En hindi, las comparaciones se forman de manera similar al español, pero con algunas peculiaridades específicas. Existen dos construcciones principales para hacer comparaciones:


\begin{itemize}

  \item El \textbf{Comparativo}: Para comparar dos elementos (más/menos que).

  \item El \textbf{Superlativo}: Para expresar el grado máximo o mínimo de una cualidad (el/la más/menos).

\end{itemize}


\begin{tcolorbox}[colback=yellow!10!white,colframe=orange!75!black,title=Regla de Oro]

\paragraph{Regla Fundamental de las Comparaciones} 

La estructura básica para comparaciones en hindi es:


\textbf{[Elemento A] + [Adjetivo/Adverbio] + [Partícula de Comparación] + [Elemento B]}


La partícula más común para comparaciones es {\hindifont से} (se), equivalente a la palabra española "que" en las comparaciones.


\end{tcolorbox}

\subsection{2. Formación del Comparativo}


Para formar comparaciones de superioridad, inferioridad o igualdad, usamos diferentes partículas con el adjetivo:

\subsubsection{2.1 Comparativo de Superioridad}


Para expresar "más ... que", usamos la partícula {\hindifont से} (se) después del elemento B y colocamos la palabra {\hindifont ज़्यादा} (zyaadaa) - "más" - o el adjetivo en su forma básica antes del verbo.


\begin{center}
\begin{tabularx}{\textwidth}{| X | X | X | X | X |}
\hline
\textbf{Tipos de Comparativo} & \textbf{Construcción} & \textbf{Ejemplo en Hindi} & \textbf{Traducción} & \textbf{Explicación} \ \hline
Comparativo Simple de Superioridad & A + {\hindifont B से अधिक/ज़्यादा} + Adj. + है & {\hindifont राम श्याम से लंबा है।} & Ram es más alto que Shayam. & {\hindifont राम} (A) + {\hindifont श्याम से} (de B) + {\hindifont लंबा} (más alto) + {\hindifont है} (es) \ \hline
Comparativo con ज़्यादा & A + {\hindifont B से ज़्यादा} + Adj. + है & {\hindifont डिल्ली मुंबई से ज्यादा बड़ा है।} & Delhi es más grande que Bombay. & {\hindifont डिल्ली} (A) + {\hindifont मुंबई से} (de B) + {\hindifont ज्यादा बड़ा} (más grande) + {\hindifont है} (es) \ \hline
Comparativo de Inteligencia & Sujeto A + {\hindifont से बुद्धिमान} + है & {\hindifont सीता राम से बुद्धिमान है।} & Sita es más inteligente que Ram. & {\hindifont सीता} (A) + {\hindifont राम से} (de B) + {\hindifont बुद्धिमान} (inteligente) + {\hindifont है} \ \hline
Comparativo de Velocidad & Sujeto A + {\hindifont विदा से ज़्यादा तेज़} + है & {\hindifont राम विदा से ज़्यादा तेज़ भागता है।} & Ram corre más rápido que Vida. & {\hindifont राम} + {\hindifont विदा से} + {\hindifont ज़्यादा तेज़} + {\hindifont भागता है} \ \hline
\end{tabularx}
\end{center}

\subsubsection{2.2 Comparative de Inferioridad}


Para expresar "menos ... que", usamos la partícula {\hindifont कम} (kam) - "menos" - con {\hindifont से}:


\begin{center}
\begin{tabularx}{\textwidth}{| X | X | X | X | X | X |}
\hline
\textbf{Elemento A} & \textbf{Partícula Comparativa} & \textbf{Elemento B} & \textbf{Adjetivo} & \textbf{Ejemplo} & \textbf{Traducción} \ \hline
{\hindifont वह} (vah - él) & {\hindifont ... से कम} (se kam - menos de) & {\hindifont मैं} (main - yo) & {\hindifont अच्छा} (acchhaa - bueno) & {\hindifont वह मुझ से कम अच्छा है।} & Él es menos bueno que yo. \ \hline
{\hindifont इस दुकान} (is dukaan - esta tienda) & {\hindifont ... से कम} & {\hindifont उस दुकान} (us dukaan - esa tienda) & {\hindifont महंगी} (mahaNgii - cara) & {\hindifont इस दुकान कम महंगी है।} & Esta tienda es menos cara que esa tienda. \ \hline
{\hindifont राम} (Raam) & {\hindifont ... से कम} & {\hindifont सीता} (Seetaa) & {\hindifont तेज़} (tez - rápido) & {\hindifont राम सीता से कम तेज़ है।} & Ram es menos rápido que Sita. \ \hline
{\hindifont यह घर} (yah ghar - esta casa) & {\hindifont ... से कम} & {\hindifont वह घर} (vah ghar - esa casa) & {\hindifont बड़ा} (badaa - grande) & {\hindifont यह घर वह घर से कम बड़ा है।} & Esta casa es menos grande que esa casa. \ \hline
\end{tabularx}
\end{center}

\subsubsection{2.3 Comparativo de Igualdad}


Para expresar "tan ... como", usamos {\hindifont इतना/इतनी/इतने} (itnaa/itnii/itne) - "tanto/tanta/tantos" - con la partícula {\hindifont जितना/जितनी/जितने} (jitnaa/jitnii/jitne) - "cuanto/cuanta/cuantos":


\begin{center}
\begin{tabularx}{\textwidth}{| X | X | X | X |}
\hline
\textbf{Construcción} & \textbf{Ejemplo} & \textbf{Traducción} & \textbf{Nota} \ \hline
{\hindifont A इतना/इतनी/इतने + Adj. + है जितना/जितनी/जितने + B} & {\hindifont राम इतना अच्छा है जितना श्याम।} & Ram es tan bueno como Shayam. & Concordancia con el sujeto de A \ \hline
Mismo patrón & {\hindifont यह घर उतना ही बड़ा है जितना वह घर।} & Esta casa es tan grande como esa casa. & {\hindifont उतना ही} (también tan...como) con énfasis \ \hline
Con adverbios & {\hindifont वह मैंने जितना तेज़ दौड़ता है।} & Él corre tan rápido como yo. & Comparación de acción/adverbio \ \hline
Con sustantivos & {\hindifont मुझे उतना ही पानी चाहिए जितना तुम्हें।} & Necesito tanta agua como tú. & Comparación de cantidad \ \hline
\end{tabularx}
\end{center}

\subsubsection{2.4 Adverbios Comparativos}


Las comparaciones con adverbios usan la misma estructura:


\begin{itemize}

  \item {\hindifont राम श्याम से ज़्यादा तेज़ चलता है।} - Ram camina más rápido que Shayam.

  \item {\hindifont वे हम से कम धीरे बोलते हैं।} - Ellos hablan más despacio que nosotros.

  \item {\hindifont मैं तुम्हारे जैसा तेज़ नहीं चल सकता।} - No puedo caminar tan rápido como tú.

  \item {\hindifont शहर में गाँव से ज़्यादा शोर होता है।} - En la ciudad hay más ruido que en el pueblo.

\end{itemize}

\subsection{3. Formación del Superlativo}


El superlativo en hindi se forma generalmente con la partícula {\hindifont सबसे} (sabse) - "más" o "el/la más" - colocada antes del adjetivo.

\subsubsection{3.1 Superlativo Positivo (el/la más)}


La estructura es: {\hindifont सबसे} + [Adjetivo] + [Sustantivo] + {\hindifont है}


\begin{center}
\begin{tabularx}{\textwidth}{| X | X | X | X | X | X |}
\hline
\textbf{Elemento} & \textbf{Partícula Superlativa} & \textbf{Adjetivo} & \textbf{Sujeto} & \textbf{Ejemplo} & \textbf{Traducción} \ \hline
Elemento & {\hindifont सबसे} & Adjetivo & Sujeto (en contexto) & {\hindifont राम सबसे अच्छा है।} & Ram es el mejor. \ \hline
 & {\hindifont सबसे} & {\hindifont सुंदर} (sundar - hermoso) & {\hindifont यह लड़की} (yah laRkii - esta chica) & {\hindifont यह लड़की सबसे सुंदर है।} & Esta chica es la más hermosa. \ \hline
 & {\hindifont सबसे} & {\hindifont बुद्धिमान} (buddhimaan - inteligente) & {\hindifont मेरा भाई} (meraa bhaai - mi hermano) & {\hindifont मेरा भाई सबसे बुद्धिमान है।} & Mi hermano es el más inteligente. \ \hline
 & {\hindifont सबसे} & {\hindifont लंबा} (lambaa - alto) & {\hindifont स्कूल के लड़कों में} (skul ke laRkon men - entre los chicos de la escuela) & {\hindifont स्कूल के लड़कों में राम सबसे लंबा है।} & Entre los chicos de la escuela, Ram es el más alto. \ \hline
\end{tabularx}
\end{center}

\subsubsection{3.2 Superlativo Negativo (el/la menos)}


Para expresar "el/la menos", usamos {\hindifont सबसे कम} (sabse kam) - "el menos":


\begin{itemize}

  \item {\hindifont राम सबसे कम आलसी है।} - Ram es el menos perezoso.

  \item {\hindifont यह सबसे कम महंगी किताब है।} - Este es el libro menos caro.

  \item {\hindifont हमारी कक्षा में वह सबसे कम तेज़ है।} - Él es el menos rápido en nuestra clase.

  \item {\hindifont यह घर सबसे कम बड़ा है।} - Esta casa es la menos grande.

\end{itemize}

\subsubsection{3.3 Superlativo Alternativo con सब (sab - todos)}


Otra forma de expresar superlativos es con {\hindifont सब} (sab) + {\hindifont से} (se) + [Adjetivo]:


\begin{itemize}

  \item {\hindifont राम सब से अच्छा है।} - Ram es el mejor de todos.

  \item {\hindifont डिल्ली सब से बड़ा शहर है।} - Delhi es la ciudad más grande de todas.

  \item {\hindifont ये सब से महंगी कार है।} - Este es el coche más caro de todos.

\end{itemize}

\subsection{4. Adjetivos Irregulares en Comparaciones}


Algunos adjetivos tienen formas irregulares en comparaciones:


\begin{center}
\begin{tabularx}{\textwidth}{| X | X | X | X | X | X |}
\hline
\textbf{Adjetivo Base} & \textbf{Comparativo de Superioridad} & \textbf{Comparativo de Inferioridad} & \textbf{Superlativo} & \textbf{Ejemplo Comparativo} & \textbf{Ejemplo Superlativo} \ \hline
{\hindifont अच्छा} (acchhaa - bueno) & {\hindifont बेहतर} (behtar - mejor) & {\hindifont कम अच्छा} (kam acchhaa - menos bueno) & {\hindifont सबसे अच्छा} (sabse acchhaa - el mejor) & {\hindifont तुम मुझ से बेहतर हो।} (Tú eres mejor que yo.) & {\hindifont राम सबसे अच्छा छात्र है।} (Ram es el mejor estudiante.) \ \hline
{\hindifont बुरा} (buraa - malo) & {\hindifont बदतर} (badtar - peor) & {\hindifont कम बुरा} (kam buraa - menos malo) & {\hindifont सबसे बुरा} (sabse buraa - el peor) & {\hindifont यह स्थिति उम्मीद से बदतर है।} (Esta situación es peor que la esperanza.) & {\hindifont यह सबसे बुरा अनुभव था।} (Este fue el peor experiencia.) \ \hline
{\hindifont बड़ा} (badaa - grande) & {\hindifont सबसे बड़ा} o {\hindifont अधिक बड़ा} & {\hindifont कम बड़ा} (kam badaa - menos grande) & {\hindifont सबसे बड़ा} (sabse badaa - el más grande) & {\hindifont वह घर इस घर से अधिक बड़ा है।} (Esa casa es más grande que esta casa.) & {\hindifont यह सबसे बड़ा शहर है।} (Ésta es la ciudad más grande.) \ \hline
{\hindifont छोटा} (chhoTaa - pequeño) & {\hindifont सबसे छोटा} o {\hindifont अधिक छोटा} & {\hindifont कम छोटा} (kam chhoTaa - menos pequeño) & {\hindifont सबसे छोटा} (sabse chhoTaa - el más pequeño) & {\hindifont यह घर वह घर से कम बड़ा है।} (Esta casa es más pequeña que esa casa.) & {\hindifont यह सबसे छोटा बच्चा है।} (Éste es el niño más pequeño.) \ \hline
\end{tabularx}
\end{center}

\subsection{5. Números Ordinales}


Los números ordinales se forman agregando {\hindifont वां} a los números cardinales (para masculino) o {\hindifont वीं} para femeninos:

\subsubsection{5.1 Formación de Números Ordinales}


\begin{center}
\begin{tabularx}{\textwidth}{| X | X | X | X | X |}
\hline
\textbf{Número Cardinal} & \textbf{Ordinal Masc.} & \textbf{Ordinal Fem.} & \textbf{Transliteración} & \textbf{Traducción} \ \hline
{\hindifont एक} (ek) & {\hindifont पहला} (pehlaa) & {\hindifont पहली} (pehlii) & pehlā/pehlī & primero/primera \ \hline
{\hindifont दो} (do) & {\hindifont दूसरा} (doosraa) & {\hindifont दूसरी} (doosrii) & doosrā/doosrī & segundo/segunda \ \hline
{\hindifont तीन} (tiin) & {\hindifont तीसरा} (teesraa) & {\hindifont तीसरी} (teesrii) & teesrā/teesrī & tercero/tercera \ \hline
{\hindifont चार} (char) & {\hindifont चौथा} (chaauthaa) & {\hindifont चौथी} (chaauthii) & chaūthā/chaūthī & cuarto/cuarta \ \hline
{\hindifont पाँच} (paanch) & {\hindifont पांचवा} (paanchvaa) & {\hindifont पांचवीं} (paanchviin) & pañcvā/pañcvī & quinto/quinta \ \hline
{\hindifont छह} (chhah) & {\hindifont छठा} (chhathaa) & {\hindifont छठी} (chhathii) & chhaṭhā/chhaṭhī & sexto/sexta \ \hline
{\hindifont सात} (saat) & {\hindifont सातवाँ} (saatvaan) & {\hindifont सातवीं} (saatviin) & saptam/saptamī & séptimo/séptima \ \hline
{\hindifont आठ} (aath) & {\hindifont आठवाँ} (aathvaan) & {\hindifont आठवीं} (aathviin) & ashtam/ashtamī & octavo/octava \ \hline
{\hindifont नौ} (nau) & {\hindifont नौवाँ} (navaan) & {\hindifont नौवीं} (naviin) & navam/navamī & noveno/novena \ \hline
{\hindifont दस} (das) & {\hindifont दसवाँ} (dasvaan) & {\hindifont दसवीं} (daviin) & dasam/dasamī & décimo/décima \ \hline
\end{tabularx}
\end{center}

\subsubsection{5.2 Uso de Números Ordinales}


Los números ordinales se comportan como adjetivos y concuerdan en género y número con el sustantivo al que modifican:


\begin{itemize}

  \item {\hindifont पहला लड़का} - El primer chico

  \item {\hindifont पहली लड़की} - La primera chica

  \item {\hindifont पहले लड़के} - Los primeros chicos

  \item {\hindifont पहली किताबें} - Las primeras libros

  \item {\hindifont अमेरिका में दूसरा नंबर रहा।} - Estuvo en segundo lugar en América.

  \item {\hindifont तीसरा सत्र शुरू हो गया।} - Ha comenzado la tercera sesión.

\end{itemize}

\subsection{6. Adverbios Comunes: धीरे, जल्दी, ज़ोर से, अच्छा}


Los adverbios son importantes para describir cómo se realiza una acción y son fundamentales para las comparaciones:


\begin{center}
\begin{tabularx}{\textwidth}{| X | X | X | X | X | X |}
\hline
\textbf{Español} & \textbf{Hindi} & \textbf{Transliteración} & \textbf{Uso} & \textbf{Ejemplo} & \textbf{Traducción} \ \hline
Despacio & {\hindifont धीरे} & dheere & Adverbio de modo & {\hindifont वह धीरे बोलता है।} & Él habla despacio. \ \hline
Pronto/Rápido & {\hindifont जल्दी} & jaldi & Adverbio de tiempo/modo & {\hindifont तुम जल्दी आ गए।} & Tú viniste rápidamente. \ \hline
Fuerte & {\hindifont ज़ोर से} & zor se & Adverbio de intensidad & {\hindifont बोलना ज़ोर से बोलो।} & Habla más fuerte. \ \hline
Bien & {\hindifont अच्छा} & achchhaa & Adverbio de modo/cualidad & {\hindifont वह हिंदी अच्छा बोलता है।} & Él habla hindi bien. \ \hline
Mal & {\hindifont खराब} & kharaab & Adverbio de modo/cualidad & {\hindifont तुम यह काम खराब करते हो।} & Tú haces este trabajo mal. \ \hline
Claro & {\hindifont स्पष्ट} & spasht & Adverbio de modo & {\hindifont स्पष्ट बोलो।} & Habla claramente. \ \hline
Lentamente & {\hindifont धीमा} & dheemaa & Adverbio de modo & {\hindifont वह धीमा चलता है।} & Él camina lentamente. \ \hline
Perfectamente & {\hindifont सही} & sahii & Adverbio de modo & {\hindifont तुमने यह काम सही किया।} & Tú hiciste este trabajo perfectamente. \ \hline
\end{tabularx}
\end{center}

\subsection{7. Comunicación: Comparar Personas, Objetos y Lugares}


Con los nuevos elementos gramaticales, podemos formular comparaciones más complejas:

\subsubsection{7.1 Frases para Comparar Personas}


\begin{itemize}

  \item {\hindifont राम सीता से बुद्धिमान है।} - Ram es más inteligente que Sita.

  \item {\hindifont सीता राम से अधिक लंबी है।} - Sita es más alta que Ram.

  \item {\hindifont लड़की लड़के से अधिक पढ़ती है।} - La chica estudia más que el chico.

  \item {\hindifont मेरे भाई सबसे अच्छा है।} - Mi hermano es el mejor.

  \item {\hindifont वह अपने दोस्तों से कम बुरा नहीं है।} - Él no es peor que sus amigos.

  \item {\hindifont मेरा भाई सबसे तेज़ है।} - Mi hermano es el más rápido.

\end{itemize}

\subsubsection{7.2 Frases para Comparar Objetos}


\begin{itemize}

  \item {\hindifont यह किताब वह किताब से ज़्यादा महंगी है।} - Este libro es más caro que ese libro.

  \item {\hindifont वह घर यह घर से कम बड़ा है।} - Esa casa es menos grande que esta casa.

  \item {\hindifont यह सबसे छोटा कमरा है।} - Esta es la habitación más pequeña.

  \item {\hindifont यह कार उस कार से ज़्यादा तेज़ है।} - Este coche es más rápido que ese coche.

  \item {\hindifont तुम्हारी किताब मेरी किताब से सबसे अधिक सुंदर है।} - Tu libro es más hermoso que mi libro.

  \item {\hindifont इस स्कूल की किताबें सबसे अच्छी हैं।} - Los libros de esta escuela son los mejores.

\end{itemize}

\subsubsection{7.3 Frases para Comparar Lugares}


\begin{itemize}

  \item {\hindifont डिल्ली मुंबई से बड़ा शहर है।} - Delhi es una ciudad más grande que Mumbai.

  \item {\hindifont गाँव शहर से शांत है।} - El pueblo es más tranquilo que la ciudad.

  \item {\hindifont मेरा कमरा तुम्हारे कमरे से छोटा है।} - Mi habitación es más pequeña que tu habitación.

  \item {\hindifont यह दुकान सबसे खूबसूरत है।} - Esta tienda es la más bonita.

  \item {\hindifont हिमालय सबसे ऊँचा पहाड़ है।} - El Himalaya es la montaña más alta.

  \item {\hindifont यह जगह उस जगह से बहुत अलग है।} - Este lugar es muy diferente de ese lugar.

  \item {\hindifont हमारा शहर सबसे साफ़ है।} - Nuestra ciudad es la más limpia.

\end{itemize}

\subsubsection{7.4 Diálogos con Comparativos}


\textbf{Diálogo 1: Comparando trabajos}


\begin{itemize}

  \item A: {\hindifont क्या तुम्हें अपना काम पसंद है?} - ¿Te gusta tu trabajo?

  \item B: {\hindifont हाँ, मुझे अपना काम बहुत पसंद है। यह अन्य कामों से ज़्यादा संतुष्टि देता है।} - Sí, me gusta mucho mi trabajo. Da más satisfacción que otros trabajos.

  \item A: {\hindifont क्या यह काम अन्य कामों से ज़्यादा कठिन है?} - ¿Es este trabajo más difícil que otros trabajos?

  \item B: {\hindifont नहीं, यह काम सबसे कम कठिन है।} - No, este trabajo es el menos difícil de todos.

  \item A: {\hindifont तो यह सबसे अच्छा काम है?} - ¿Entonces este es el mejor trabajo?

  \item B: {\hindifont हाँ, यह सबसे अच्छा काम है।} - Sí, este es el mejor trabajo.

\end{itemize}


\textbf{Diálogo 2: Comparando ciudades}


\begin{itemize}

  \item Turista: {\hindifont डिल्ली और मुंबई में से कौन सा शहर सबसे अच्छा है?} - ¿Cuál ciudad es la mejor entre Delhi y Mumbai?

  \item Guía: {\hindifont यह आपके अनुसार बदलता है। डिल्ली राजधानी है और इतिहास से भरा है।} - Depende de usted. Delhi es la capital y está llena de historia.

  \item Turista: {\hindifont मुंबई कैसा है?} - ¿Cómo es Mumbai?

  \item Guía: {\hindifont मुंबई सबसे बड़ा शहर है और फिल्म उद्योग का केंद्र है।} - Mumbai es la ciudad más grande y el centro de la industria cinematográfica.

  \item Turista: {\hindifont क्या मुंबई डिल्ली से ज़्यादा गरम है?} - ¿Es Mumbai más caliente que Delhi?

  \item Guía: {\hindifont हाँ, मुंबई डिल्ली से अधिक गरम और आर्द्र है।} - Sí, Mumbai es más caliente y húmeda que Delhi.

  \item Turista: {\hindifont यहाँ की जनता डिल्ली की जनता से अधिक मेहनती है?} - ¿La gente aquí es más trabajadora que en Delhi?

  \item Guía: {\hindifont यह भी आपके अनुसार है। लेकिन दोनों शहरों में लोग सबसे अच्छा काम करते हैं।} - Eso también depende de usted. Pero la gente en ambos ciudades hace su mejor trabajo.

\end{itemize}


\hrulefill

\section{Unidad 19: Conectores y Frases Subordinadas}


\textbf{Objetivo:} Unir ideas y construir frases más complejas con conectores y cláusulas subordinadas.

\subsection{1. Conectores Fundamentales}


Los conectores permiten unir palabras, frases u oraciones. Existen diferentes tipos de conectores que expresan relaciones lógicas distintas.

\subsubsection{1.1 Conectores de Adición y Secuencia}


\begin{center}
\begin{tabularx}{\textwidth}{| X | X | X | X | X | X |}
\hline
\textbf{Español} & \textbf{Hindi} & \textbf{Transliteración} & \textbf{Uso} & \textbf{Ejemplo} & \textbf{Traducción} \ \hline
y & {\hindifont और} & aur & Adición de elementos & {\hindifont राम और श्याम घूम रहे हैं।} & Ram y Shyam están paseando. \ \hline
entonces & {\hindifont फिर} & phir & Secuencia temporal & {\hindifont पहले वह पढ़ता है, फिर खेलता है।} & Primero él estudia, luego juega. \ \hline
luego & {\hindifont बाद में} & baad men & Secuencia temporal & {\hindifont हम खाना खाएंगे और बाद में पार्क जाएंगे।} & Comeremos y luego iremos al parque. \ \hline
también & {\hindifont भी} & bhi & Adición inclusiva & {\hindifont मैं भी आऊँगा।} & Yo también vendré. \ \hline
\end{tabularx}
\end{center}

\subsubsection{1.2 Conectores de Contraste}


\begin{center}
\begin{tabularx}{\textwidth}{| X | X | X | X | X | X |}
\hline
\textbf{Español} & \textbf{Hindi} & \textbf{Transliteración} & \textbf{Uso} & \textbf{Ejemplo} & \textbf{Traducción} \ \hline
pero & {\hindifont लेकिन} & lekin & Contraste o contradicción & {\hindifont मौसम अच्छा है, लेकिन मैं घर जाना चाहता हूँ।} & El tiempo es bueno, pero quiero ir a casa. \ \hline
aunque & {\hindifont हालांकि} & haalaanki & Contraste con concesión & {\hindifont हालांकि बीमार था, वह काम पर गया।} & Aunque estaba enfermo, él fue al trabajo. \ \hline
sin embargo & {\hindifont हालांकि} & haalaanki & Contraste formal & {\hindifont मौसम ठंढा था, हालांकि वह बाहर गया।} & Hacía frío, sin embargo él salió. \ \hline
\end{tabularx}
\end{center}

\subsubsection{1.3 Conectores de Causa/Consecuencia}


\begin{center}
\begin{tabularx}{\textwidth}{| X | X | X | X | X | X |}
\hline
\textbf{Español} & \textbf{Hindi} & \textbf{Transliteración} & \textbf{Uso} & \textbf{Ejemplo} & \textbf{Traducción} \ \hline
porque & {\hindifont क्योंकि} & kyonki & Indica causa o razón & {\hindifont मैं जल्दी जा रहा हूँ क्योंकि मुझे काम पर जल्दी पहुँचना है।} & Estoy yendo rápidamente porque necesito llegar temprano al trabajo. \ \hline
ya que & {\hindifont चूंकि} & chuunki & Indica causa o razón & {\hindifont चूंकि बारिश हो रही है, हम घर पर रहेंगे।} & Ya que está lloviendo, nos quedaremos en casa. \ \hline
por eso & {\hindifont इसलिए} & isliye & Indica consecuencia & {\hindifont बारिश हो रही है, इसलिए मैं घर पर हूँ।} & Está lloviendo, por eso estoy en casa. \ \hline
así que & {\hindifont इसलिए} & isliye & Indica consecuencia & {\hindifont तुम बीमार हो, इसलिए आराम करो।} & Estás enfermo, así que descansa. \ \hline
\end{tabularx}
\end{center}


\begin{tcolorbox}[colback=yellow!10!white,colframe=orange!75!black,title=Regla de Oro]

\paragraph{Regla de Oro: Posición de los Conectores} 

Los conectores generalmente van al \textbf{inicio de la cláusula} que introducen. En hindi, la cláusula principal puede ir antes o después de la cláusula subordinada, pero la posición del conector indica qué relación está expresando:


\begin{itemize}

  \item Cláusula principal + {\hindifont लेकिन} + Cláusula subordinada

  \item {\hindifont क्योंकि} + Cláusula subordinada + Cláusula principal

\end{itemize}


\end{tcolorbox}

\subsection{2. Tipos de Oraciones Compuestas}


Las oraciones compuestas consisten en dos o más cláusulas unidas por conectores. Existen diferentes tipos según la relación entre las cláusulas:

\subsubsection{2.1 Oraciones Coordinales}


Las oraciones coordinales contienen cláusulas de igual importancia, unidas por conectores de coordinación:


\begin{itemize}

  \item {\hindifont राम खाता है और श्याम पढ़ता है।} - Ram come y Shyam estudia.

  \item {\hindifont यह अच्छी किताब है, लेकिन कीमत ज़्यादा है।} - Éste es un buen libro, pero el precio es alto.

  \item {\hindifont हम पढ़ते हैं और हम खेलते हैं।} - Estudiamos y jugamos.

  \item {\hindifont तुम जाओ या मैं जाऊँ?} - ¿Vas tú o voy yo? ({\hindifont या} - o)

  \item {\hindifont वह तेज़ है और बुद्धिमान भी।} - Él es rápido y también inteligente.

\end{itemize}

\subsubsection{2.2 Oraciones Subordinadas}


Las oraciones subordinadas dependen de la oración principal y expresan relaciones como causa, tiempo, condición, etc.:

\paragraph{A. Cláusulas Adverbiales de Causa} 

\begin{itemize}

  \item {\hindifont क्योंकि यह त्योहार है, सब लोग खुश हैं।} - Porque es fiesta, todos están felices.

  \item {\hindifont उसने तेज़ पढ़ाई की क्योंकि उसे अच्छे अंक चाहिए थे।} - Él estudió intensamente porque quería buenas calificaciones.

  \item {\hindifont चूंकि समय देर हो रहा था, वे जल्दी से चले गए।} - Ya que se estaba haciendo tarde, se fueron rápidamente.

\end{itemize}

\paragraph{B. Cláusulas Adverbiales de Tiempo} 

\begin{itemize}

  \item {\hindifont जब मैं छोटा था, मैं बहुत खेलता था।} - Cuando yo era niño, jugaba mucho.

  \item {\hindifont जैसे ही बस आई, हम सवार हुए।} - Tan pronto como llegó el autobús, nos subimos. ({\hindifont जैसे ही} - tan pronto como)

  \item {\hindifont जब तक तुम पढ़ रहे हो, मैं तैयारी करता हूँ।} - Mientras tú estés estudiando, yo prepararé. ({\hindifont जब तक} - mientras)

  \item {\hindifont तुम खाना खा चुके थे जब मैं आया।} - Ya habías terminado de comer cuando yo llegué.

\end{itemize}

\paragraph{C. Cláusulas Adverbiales de Condición} 

\begin{itemize}

  \item {\hindifont यदि तुम जल्दी करोगे, तो तुम पहुँचोगे।} - Si te apuras, llegarás. ({\hindifont यदि... तो} - si...entonces)

  \item {\hindifont अगर बारिश होती, तो हम बाहर नहीं जाएँगे।} - Si llueve, no saldremos. ({\hindifont अगर... तो} - si...entonces)

  \item {\hindifont यदि तुम बीमार हो, तो डॉक्टर दिखाओ।} - Si estás enfermo, ve al médico.

\end{itemize}

\paragraph{D. Cláusulas Subordinadas de Propósito} 

\begin{itemize}

  \item {\hindifont मैंने तुम्हें याद दिलाने के लिए कहा।} - Te lo dije para recordártelo.

  \item {\hindifont वह काम जल्दी से करता है ताकि उसे आराम मिले।} - Él hace el trabajo rápidamente para que él pueda descansar. ({\hindifont ताकि} - para que)

  \item {\hindifont हम अधिक पढ़ रहे हैं ताकि हम परीक्षा में अच्छा कर सकें।} - Estamos estudiando más para que podamos hacerlo bien en el examen.

\end{itemize}

\subsection{3. Frases Subordinadas de Relativo}


Para conectar frases que describen un sustantivo, usamos cláusulas subordinadas de relativo:

\subsubsection{3.1 Uso de कि (ki) como Conector de Relativo}


El conector {\hindifont कि} (ki) puede funcionar como relativo en ciertos contextos, especialmente en frases subordinadas sustantivas:


\begin{itemize}

  \item {\hindifont मैंने सुना कि तुम घर गए।} - Oí que tú fuiste a casa.

  \item {\hindifont मैं उस बात को जानता हूँ कि तुम बहुत मेहनत करते हो।} - Sé el hecho de que tú trabajas mucho.

  \item {\hindifont वह एक अच्छा डॉक्टर है जिसे सब पसंद करते हैं।} - Él es un buen doctor al que todos aprecian.

  \item {\hindifont पुस्तक लाओ जो कमरे में है।} - Trae el libro que está en la habitación.

\end{itemize}

\subsubsection{3.2 Otros Conectores de Subordinación}


Hay más conectores que introducen cláusulas subordinadas:


\begin{center}
\begin{tabularx}{\textwidth}{| X | X | X | X | X |}
\hline
\textbf{Tipo de Subordinada} & \textbf{Conector} & \textbf{Transliteración} & \textbf{Ejemplo} & \textbf{Traducción} \ \hline
Condición & {\hindifont यदि} (yadi) & yadi & {\hindifont यदि तुम आते हो तो हम खुश होंगे।} & Si tú vienes, nosotros estaremos felices. \ \hline
Propósito & {\hindifont ताकि} (taaki) & taaki & {\hindifont हम जल्दी खाते हैं ताकि हम समय पर पहुँच सकें।} & Comemos rápidamente para que podamos llegar a tiempo. \ \hline
Concesión & {\hindifont चाहे} (chahe) & chahe & {\hindifont चाहे बारिश हो या धूप, हम जाएँगे।} & Sea lluvia o sol, iremos. \ \hline
Tiempo & {\hindifont जब} (jab) & jab & {\hindifont जब मैं छुट्टी पर था, मैं बहुत घूमता था।} & Cuando yo estaba de vacaciones, viajaba mucho. \ \hline
Lugar & {\hindifont जहाँ} (jahaan) & jahaan & {\hindifont जहाँ तक मैं जानता हूँ, वह अच्छा इंसान है।} & Por donde yo sé, él es una buena persona. \ \hline
Modo & {\hindifont जैसे} (jaise) & jaise & {\hindifont तुम जैसे कहो वैसे करेंगे।} & Haremos como tú digas. \ \hline
\end{tabularx}
\end{center}

\subsection{4. Comunicación: Formar Frases Completas con Conectores}


Veamos cómo usar conectores para formar frases más complejas y naturales:

\subsubsection{4.1 Expresar Causa y Efecto}


\begin{itemize}

  \item {\hindifont मैं थका हुआ हूँ क्योंकि मैंने आज सुबह से काम किया है।} - Estoy cansado porque he estado trabajando desde esta mañana.

  \item {\hindifont तुम्हें आराम करना चाहिए, क्योंकि तुम बहुत थके हुए हो।} - Deberías descansar, porque estás muy cansado.

  \item {\hindifont हम कल बाहर नहीं जा सकेंगे, क्योंकि बारिश होने वाली है।} - No podremos salir mañana, porque va a llover.

  \item {\hindifont बारिश हो रही है, इसलिए हम घर पर ही रहेंगे।} - Está lloviendo, por eso nos quedaremos en casa.

  \item {\hindifont तुम अपना डॉक्टर दिखाओ, इसलिए कि तुम बीमार हो।} - Ve al médico, porque estás enfermo.

  \item {\hindifont मैं जल्दी बोल रहा हूँ क्योंकि मैं देर से हूँ।} - Estoy hablando rápido porque estoy tarde.

\end{itemize}

\subsubsection{4.2 Expresar Contraste}


\begin{itemize}

  \item {\hindifont यह किताब अच्छी है, लेकिन कीमत बहुत अधिक है।} - Este libro es bueno, pero el precio es muy alto.

  \item {\hindifont मैं घर जाना चाहता हूँ, लेकिन काम पूरा करना है।} - Quiero ir a casa, pero necesito terminar el trabajo.

  \item {\hindifont वह बहुत पढ़ता है, लेकिन अच्छे अंक नहीं लाता।} - Él estudia mucho, pero no obtiene buenas calificaciones.

  \item {\hindifont हालांकि वह बहुत मेहनत करता है, उसे सफलता नहीं मिलती।} - Aunque él trabaja mucho, no obtiene éxito.

  \item {\hindifont तुम्हारे पास पैसा है, लेकिन तुम खर्चा नहीं कर रहे हो।} - Tienes dinero, pero no lo estás gastando.

  \item {\hindifont मौसम ठंढा है, हालांकि लोग घूमने बाहर जा रहे हैं।} - Hace frío, sin embargo la gente está saliendo a pasear.

\end{itemize}

\subsubsection{4.3 Expresar Propósitos y Condiciones}


\begin{itemize}

  \item {\hindifont मैं अधिक पढ़ रहा हूँ ताकि मैं परीक्षा में अच्छा कर सकूँ।} - Estoy estudiando más para poder hacerlo bien en el examen.

  \item {\hindifont तुम दवा लो ताकि तुम जल्दी ठीक हो जाओ।} - Toma medicina para que te recuperes rápido.

  \item {\hindifont यदि तुम मेरी सलाह मानोगे, तो तुम सफल होगे।} - Si aceptas mi consejo, tendrás éxito.

  \item {\hindifont अगर हम नियमित रूप से अभ्यास करेंगे, तो हम हिंदी सीख सकते हैं।} - Si practicamos regularmente, podremos aprender hindi.

  \item {\hindifont तुम जैसे कहोगे वैसे करूँगा।} - Haré como tú digas.

  \item {\hindifont हम समय पर पहुँचेंगे या फिर बहुत जल्दी चलेंगे।} - Llegaremos a tiempo o saldremos muy temprano.

\end{itemize}

\subsubsection{4.4 Diálogos con Conectores y Subordinadas}


\textbf{Diálogo 1: Conversación sobre planes y condiciones}


\begin{itemize}

  \item A: {\hindifont क्या तुम आज घूमने जाओगे?} (¿Irás a pasear hoy?)

  \item B: {\hindifont हाँ, अगर मौसम अच्छा हो तो घूमने जाऊँगा।} (Sí, si el tiempo es bueno, saldré a pasear.)

  \item A: {\hindifont लेकिन आज मौसम खराब होने की संभावना है।} (Pero hoy es probable que el tiempo esté malo.)

  \item B: {\hindifont तो फिर मैं गाना सुन रहा हूँ जब तक बारिश शुरू हो जाती है।} (Entonces escucharé música mientras comienza a llover.)

  \item A: {\hindifont मैं तुम्हें बाद में फ़ोन करूँगा जैसे ही घर पर पहुँच जाऊँ।} (Te llamaré más tarde tan pronto como llegue a casa.)

  \item B: {\hindifont ठीक है। यह अच्छा होगा कि हम सप्ताह के अंत में एक साथ घूमें।} (Bien. Será bueno que paseemos juntos al final de la semana.)

  \item A: {\hindifont मैं जैसे कहूँ तैसे तुम करोगे, तो हम अच्छी यात्रा कर सकते हैं।} (Si haces como digo, podremos hacer un buen viaje.)

  \item B: {\hindifont हाँ, मैं तुम्हारी बात मानने की पूरी कोशिश करूँगा।} (Sí, haré todo lo posible por escuchar lo que dices.)

\end{itemize}


\textbf{Diálogo 2: Conversación sobre motivos y consecuencias}


\begin{itemize}

  \item Paciente: {\hindifont डॉक्टर साहब, मैं बहुत थका हुआ हूँ।} (Doctor sahib, estoy muy cansado.)

  \item Doctor: {\hindifont क्या आप अच्छी तरह से सो रहे हैं?} (¿Está Ud. durmiendo bien?)

  \item Paciente: {\hindifont नहीं, मैं कई रातों से अच्छी नींद नहीं ले रहा हूँ।} (No, no he estado teniendo un buen sueño por varias noches.)

  \item Doctor: {\hindifont क्योंकि आप ज़्यादा काम कर रहे हैं, इसलिए आप थके हुए हैं।} (Porque Ud. está trabajando demasiado, por eso Ud. está cansado.)

  \item Paciente: {\hindifont हाँ, मेरे पास बहुत काम है। लेकिन मुझे अपने परिवार का भी ख्याल रखना है।} (Sí, tengo mucho trabajo. Pero también tengo que cuidar de mi familia.)

  \item Doctor: {\hindifont यह ठीक है कि आप अपने परिवार का ख्याल रखते हैं, लेकिन आपको अपने स्वास्थ्य का भी ध्यान देना चाहिए।} (Está bien que Ud. cuide de su familia, pero Ud. también debería cuidar de su salud.)

  \item Paciente: {\hindifont क्या आप बता सकते हैं कि मैं क्या करूँ?} (¿Puede Ud. decirme qué debo hacer?)

  \item Doctor: {\hindifont आप थोड़ा काम कम करें और अधिक आराम करें। यह महत्वपूर्ण है कि आप नियमित घूमते हैं।} (Reduzca Ud. un poco su trabajo y descanse más. Es importante que Ud. camine regularmente.)

\end{itemize}


\hrulefill

\section{Unidad 20: Participios y Verbos Compuestos}


\textbf{Objetivo:} Narrar secuencias de eventos de forma más fluida y natural.

\subsection{1. Participio Conjuntivo para Indicar Acción Previa}


El participio conjuntivo (también llamado "absoluto" o "gerundio" en hindi) se forma agregando {\hindifont कर} (kar) o {\hindifont करके} (karkar) a la raíz de los verbos. Esta forma permite indicar una acción que ocurre antes o simultáneamente a otra acción.


\begin{tcolorbox}[colback=yellow!10!white,colframe=orange!75!black,title=Regla de Oro]

\paragraph{Formación del Participio Conjuntivo} 

La estructura es: \textbf{Raíz del verbo + कर/करके}


\begin{itemize}

  \item La mayoría de los verbos usan {\hindifont कर} después de la raíz: {\hindifont खाना} (khaanaa - comer) → {\hindifont खा} (khaa) + {\hindifont कर} = {\hindifont खाकर} (khaakar - habiendo comido)

  \item Para mayor énfasis o claridad, se puede usar {\hindifont करके} (karkar) en lugar de {\hindifont कर} (kar)

  \item Esta forma permite unir dos acciones sin usar conectores explícitos

\end{itemize}


\end{tcolorbox}

\subsubsection{1.1 Ejemplos de Participios Conjuntivos}


\begin{center}
\begin{tabularx}{\textwidth}{| X | X | X | X | X | X |}
\hline
\textbf{Verbo Infinitivo} & \textbf{Raíz} & \textbf{Participio Conjuntivo} & \textbf{Transliteración} & \textbf{Ejemplo en Frase} & \textbf{Traducción} \ \hline
{\hindifont खाना} (comer) & {\hindifont खा} (khaa) & {\hindifont खाकर} (khaakar) & khaa-kar (habiendo comido) & {\hindifont मैं खाकर घर गया।} & Yo fui a casa habiendo comido. (Después de comer) \ \hline
{\hindifont पीना} (beber) & {\hindifont पी} (pii) & {\hindifont पीकर} (piikar) & pii-kar (habiendo bebido) & {\hindifont वह दूध पीकर सो गया।} & Él se durmió después de beber leche. \ \hline
{\hindifont जाना} (ir) & {\hindifont जा} (jaa) & {\hindifont जाकर} (jaakar) & jaa-kar (yendo y después) & {\hindifont वह दुकान जाकर पेन खरीदेगा।} & Él irá a la tienda y comprará un bolígrafo. \ \hline
{\hindifont आना} (venir) & {\hindifont आ} (aa) & {\hindifont आकर} (aakar) & aa-kar (viniendo y después) & {\hindifont तुम घर आकर आराम करो।} & Tú ven a casa y descansa. \ \hline
{\hindifont पढ़ना} (leer/estudiar) & {\hindifont पढ़} (padh) & {\hindifont पढ़कर} (padhkar) & padh-kar (habiendo leído) & {\hindifont वह पत्र पढ़कर जवाब देगा।} & Él dará respuesta después de leer la carta. \ \hline
{\hindifont लिखना} (escribir) & {\hindifont लिख} (likh) & {\hindifont लिखकर} (likhkar) & likh-kar (habiendo escrito) & {\hindifont मैं पत्र लिखकर भेजूँगा।} & Yo escribiré la carta y la enviaré. \ \hline
{\hindifont देखना} (ver/mirar) & {\hindifont देख} (dekh) & {\hindifont देखकर} (dekhkar) & dekh-kar (habiendo visto) & {\hindifont वह फिल्म देखकर मुझे बताएगा।} & Él verá la película y me lo contará. \ \hline
{\hindifont सुनना} (escuchar/oir) & {\hindifont सुन} (sun) & {\hindifont सुनकर} (sunkar) & sun-kar (habiendo oído) & {\hindifont मैं तुम्हारी सलाह सुनकर काम करूँगा।} & Haré el trabajo habiendo escuchado tu consejo. \ \hline
{\hindifont लेना} (tomar/obtener) & {\hindifont ले} (le) & {\hindifont लेकर} (lekar) & le-kar (tomando y después) & {\hindifont तुम किताब लेकर पढ़ो।} & Tú toma el libro y léelo. \ \hline
{\hindifont देना} (dar) & {\hindifont दे} (de) & {\hindifont देकर} (dekar) & de-kar (dando y después) & {\hindifont वह जवाब देकर खुश होगा।} & Él se alegrará después de dar la respuesta. \ \hline
\end{tabularx}
\end{center}

\subsubsection{1.2 Usos del Participio Conjuntivo}


\begin{itemize}

  \item \textbf{Secuencia de Acciones:} Indica una acción que precede a otra: {\hindifont घर जाकर खाना खाओ।} - Vete a casa y come comida.

  \item \textbf{Acción Simultánea:} Expresa dos acciones que ocurren al mismo tiempo: {\hindifont गाना गाकर नाचो।} - Cantando y bailando.

  \item \textbf{Modo o Manera:} Expresa cómo se realiza una acción: {\hindifont ध्यान से काम करके सफलता पाओ।} - Obten éxito trabajando con atención.

  \item \textbf{Condición:} Puede expresar condición implícita: {\hindifont समझकर बोलो।} - Habla habiendo comprendido (sabiendo lo que dices).

  \item \textbf{Causa/Razón:} Puede expresar la razón de una acción: {\hindifont गलती करके माफ़ी मांगो।} - Pide perdón por haber cometido errores.

  \item \textbf{Propósito:} A veces expresa propósito con la acción precedente: {\hindifont पैसा कमाकर घर बनाओ।} - Ganando dinero y construye la casa.

\end{itemize}

\subsubsection{1.3 Variantes del Participio Conjuntivo}


Algunos verbos tienen formas irregulares para formar el participio conjuntivo:


\begin{itemize}

  \item {\hindifont होना} (ser/estar) → {\hindifont होकर} (hokar) / {\hindifont होके} (hoke)

  \item {\hindifont लेना} (tomar) → {\hindifont लेकर} (lekar) / {\hindifont लेके} (leke)

  \item {\hindifont देना} (dar) → {\hindifont देकर} (dekar) / {\hindifont देके} (deke)

  \item {\hindifont रहना} (vivir/estar) → {\hindifont रहकर} (rehkar) / {\hindifont रहके} (rehke)

  \item {\hindifont बैठना} (sentarse) → {\hindifont बैठकर} (baithkar) / {\hindifont बैठके} (baithke)

  \item {\hindifont खड़ा होना} (pararse) → {\hindifont खड़ा कर} (khadaa kar) o {\hindifont खड़े होकर} (khade hokar)

\end{itemize}


\begin{tcolorbox}[colback=blue!5!white,colframe=blue!75!black,title=Nota/Clarificación]

\paragraph{Observación Importante: Género y Número} 

El participio conjuntivo \textbf{no concuerda} con el sujeto ni el objeto en género ni número. Es una forma invariable del verbo. La concordancia se mantiene en el verbo principal de la cláusula, no en el participio conjuntivo.


\begin{itemize}

  \item Correcto: {\hindifont लड़की खाकर सोई।} - La chica se durmió después de comer.

  \item Correcto: {\hindifont लड़के खाकर सोए।} - Los chicos se durmieron después de comer.

  \item En ambos casos, el participio es {\hindifont खाकर} (khaakar) - no cambia.

\end{itemize}


\end{tcolorbox}

\subsection{2. Infinitivo Oblicuo para Expresar Propósito}


El infinitivo oblicuo se forma quitando la terminación {\hindifont -ना} del verbo en infinitivo. Se usa frecuentemente con {\hindifont के लिए} (para) o {\hindifont करने के लिए} (para hacer) para expresar propósito.

\subsubsection{2.1 Formación del Infinitivo Oblicuo}


\begin{center}
\begin{tabularx}{\textwidth}{| X | X | X | X | X | X |}
\hline
\textbf{Verbo Infinitivo} & \textbf{Infinitivo Oblicuo} & \textbf{Transliteración} & \textbf{Uso en Propósito} & \textbf{Ejemplo} & \textbf{Traducción} \ \hline
{\hindifont खाना} (comer) & {\hindifont खाने} & khaane & con {\hindifont के लिए} & {\hindifont मैं खाने के लिए घर जा रहा हूँ।} & Estoy yendo a casa para comer. \ \hline
{\hindifont पीना} (beber) & {\hindifont पीने} & piine & con {\hindifont के लिए} & {\hindifont वह पानी पीने के लिए गया।} & Él fue para beber agua. \ \hline
{\hindifont जाना} (ir) & {\hindifont जाने} & jaane & con {\hindifont के लिए} & {\hindifont हम घूमने के लिए बाहर गए।} & Fuimos afuera para pasear. \ \hline
{\hindifont पढ़ना} (leer) & {\hindifont पढ़ने} & padhne & con {\hindifont के लिए} & {\hindifont मैं किताब पढ़ने के लिए पार्क गया।} & Fui al parque para leer libros. \ \hline
{\hindifont लिखना} (escribir) & {\hindifont लिखने} & likhne & con {\hindifont के लिए} & {\hindifont वह पत्र लिखने के लिए कमरा में बैठा।} & Él se sentó en la habitación para escribir la carta. \ \hline
{\hindifont सोना} (dormir) & {\hindifont सोने} & sonne & con {\hindifont के लिए} & {\hindifont हम जल्दी सोने के लिए कमरे में गए।} & Nosotros fuimos a la habitación para dormir temprano. \ \hline
{\hindifont देना} (dar) & {\hindifont देने} & denne & con {\hindifont के लिए} & {\hindifont मैं तुम्हें पुस्तक देने के लिए आया।} & Vine para darte el libro. \ \hline
{\hindifont लेना} (tomar) & {\hindifont लेने} & lenne & con {\hindifont के लिए} & {\hindifont तुम यहाँ से दवा लेने के लिए आए।} & Venísteis aquí para tomar medicina. \ \hline
{\hindifont खरीदना} (comprar) & {\hindifont खरीदने} & kharidne & con {\hindifont के लिए} & {\hindifont हम फल खरीदने के लिए बाजार जाते हैं।} & Vamos al mercado para comprar fruta. \ \hline
{\hindifont बनाना} (hacer/crear) & {\hindifont बनाने} & bananaa ne & con {\hindifont के लिए} & {\hindifont मैं खाना बनाने के लिए स्कूल नहीं गया।} & No fui a la escuela para preparar comida. \ \hline
\end{tabularx}
\end{center}


\begin{tcolorbox}[colback=yellow!10!white,colframe=orange!75!black,title=Regla de Oro]

\paragraph{Formas Alternativas para Expresar Propósito} 

Además de {\hindifont ...के लिए}, también se pueden usar:


\begin{itemize}

  \item {\hindifont ...करने के लिए} - Para hacer algo: {\hindifont पढ़ाई करने के लिए} (para estudiar)

  \item {\hindifont ...के लिए आया} - Vino para hacer algo: {\hindifont सलाह देने के लिए आया} (vino para dar consejo)

  \item {\hindifont ...के उद्देश्य से} - Con el propósito de: {\hindifont मदद करने के उद्देश्य से} (con el propósito de ayudar)

  \item {\hindifont ...के लिए जाना है} - Tengo que ir para hacer algo: {\hindifont डॉक्टर के पास जाना है} (tengo que ir al doctor)

\end{itemize}


\end{tcolorbox}

\subsection{3. Introducción a los Verbos Compuestos (V1 + V2)}


Los verbos compuestos en hindi son formaciones donde un verbo principal (V1) se combina con otro verbo (V2) para crear un nuevo verbo con un significado más específico o matizado. En el nivel A2, nos enfocaremos en el uso de algunos verbos auxiliares comunes como {\hindifont जाना}, {\hindifont लेना/देना} para añadir matices de completitud, dirección o aspecto.

\subsubsection{3.1 Verbos Compuestos Comunes con जाना (jaanaa)}


El verbo {\hindifont जाना} (ir) se combina con otros verbos para expresar completitud o dirección:


\begin{center}
\begin{tabularx}{\textwidth}{| X | X | X | X | X |}
\hline
\textbf{Verbo Compuesto} & \textbf{Composición} & \textbf{Significado} & \textbf{Ejemplo} & \textbf{Traducción} \ \hline
{\hindifont लगना} & {\hindifont लग + जाना} & comenzar a hacer algo, pegar, adherirse & {\hindifont मैं पढ़ाई लग गया।} & Empecé a estudiar. (Yo me puse a estudiar.) \ \hline
{\hindifont बैठना} & {\hindifont बैठ + जाना} & sentarse permanentemente & {\hindifont वह कमरे में बैठ गया।} & Él se sentó en la habitación. \ \hline
{\hindifont चुप होना} & {\hindifont चुप + हो + जाना} & hacerse callado & {\hindifont लड़के कमरे में चुप हो गए।} & Los chicos se hicieron callados en la habitación. \ \hline
{\hindifont मिलना} & {\hindifont मिल + जाना} & encontrarse, reunirse & {\hindifont हम दोपहर में मिल गए।} & Encontrémonos al mediodía. \ \hline
{\hindifont भूल जाना} & {\hindifont भूल + जाना} & olvidar completamente & {\hindifont तुम मेरा नाम भूल गए।} & Olvidaste mi nombre completamente. \ \hline
{\hindifont डर जाना} & {\hindifont डर + जाना} & asustarse, entrar en pánico & {\hindifont बच्चा अँधेरे से डर गया।} & El niño se asustó por la oscuridad. \ \hline
{\hindifont तैयार हो जाना} & {\hindifont तैयार + हो + जाना} & prepararse, estar listo & {\hindifont वह कल यात्रा के लिए तैयार हो गया।} & Él se preparó para el viaje de mañana. \ \hline
{\hindifont मान जाना} & {\hindifont मान + जाना} & rendirse, aceptar & {\hindifont उसने हार मान ली।} & Él aceptó la derrota. (Él se rindió.) \ \hline
{\hindifont लग जाना} & {\hindifont लग + जाना} & quedarse pegado, adherirse & {\hindifont गुन्हा उसके नाम पर लग गया।} & El crimen se le adjudicó a él. \ \hline
{\hindifont ले जाना} & {\hindifont ले + जाना} & llevar algo & {\hindifont वह किताब मेरे घर ले गया।} & Él llevó el libro a mi casa. \ \hline
\end{tabularx}
\end{center}

\subsubsection{3.2 Verbos Compuestos con लेना/देना (lena/dena)}


Los verbos {\hindifont लेना} (tomar) y {\hindifont देना} (dar) se combinan con otros verbos para expresar aspectos como reciprocidad, intensidad o duración:


\begin{itemize}

  \item {\hindifont लेना} + {\hindifont जाना} = {\hindifont ले जाना} - llevar algo

  \item {\hindifont लेना} + {\hindifont आना} = {\hindifont ले आना} - traer algo / tomar y venir con ello

  \item {\hindifont देना} + {\hindifont जाना} = {\hindifont दे जाना} - dar y irse

  \item {\hindifont देना} + {\hindifont आना} = {\hindifont दे आना} - dar y venir (regresar después de dar)

  \item {\hindifont मिलना} + {\hindifont लेना} = {\hindifont मिल लेना} - reunirse/recibirse mutuamente

  \item {\hindifont मिलना} + {\hindifont देना} = {\hindifont मिल देना} - reunirse/enviarse mutuamente

\end{itemize}


\begin{center}
\begin{tabularx}{\textwidth}{| X | X | X | X | X |}
\hline
\textbf{Verbo Compuesto} & \textbf{Composición} & \textbf{Significado} & \textbf{Ejemplo} & \textbf{Traducción} \ \hline
{\hindifont ले आना} & {\hindifont ले + आना} & traer algo & {\hindifont कल खाना ले आऊँगा।} & Mañana traeré comida. \ \hline
{\hindifont ले जाना} & {\hindifont ले + जाना} & llevar algo & {\hindifont किताब ले जाओ।} & Lleva el libro. \ \hline
{\hindifont दे देना} & {\hindifont दे + देना} & dar completamente (énfasis en completar la acción de dar) & {\hindifont यह पत्र मुझे दे दो।} & Entrégamelo completamente. / Dámelo. \ \hline
{\hindifont कर देना} & {\hindifont कर + देना} & hacer algo completamente, terminar una tarea & {\hindifont कल तक काम कर दो।} & Termina el trabajo mañana. / Hazlo completamente para mañana. \ \hline
{\hindifont आ लेना} & {\hindifont आ + लेना} & venir y tomar algo & {\hindifont तुम घर पर काम आ लो।} & Tú ven y haz el trabajo en casa. \ \hline
{\hindifont बैठ लेना} & {\hindifont बैठ + लेना} & sentarse, tomar asiento (voluntariamente o convenientemente) & {\hindifont तुम मुझे थोड़ा समय दो, मैं बैठ लेता हूँ।} & Dame un momento, me sentaré. (Yo tomaré asiento.) \ \hline
{\hindifont मिल लेना} & {\hindifont मिल + लेना} & reunirse/recibirse mutuamente & {\hindifont कल हम दोस्तों के साथ मिल लेंगे।} & Mañana nos reuniremos con amigos. / Tomaremos/recibiremos mutuamente con amigos. \ \hline
{\hindifont मिल देना} & {\hindifont मिल + देना} & ponerse de acuerdo/enviarse mutuamente & {\hindifont वे दोनों मिल दे रहे हैं।} & Ellos dos están negociando. \ \hline
{\hindifont जुड़ जाना} & {\hindifont जुड़ + जाना} & conectarse, unirse, adherirse & {\hindifont वह टीम में जुड़ गया।} & Él se unió al equipo. \ \hline
{\hindifont मान लेना} & {\hindifont मान + लेना} & aceptar, admitir, tomar como válido & {\hindifont हम तुम्हारी बात मान लेते हैं।} & Aceptamos lo que dices. / Tomamos tu palabra como válida. \ \hline
\end{tabularx}
\end{center}

\subsubsection{3.3 Foco en लेना, देना para Añadir Matices de Completitud o Dirección}


Estos verbos compuestos son especialmente útiles para expresar aspectos del verbo principal:


\begin{itemize}

  \item \textbf{Matices de completitud:} {\hindifont काम कर देना} - hacer/completar el trabajo

  \item \textbf{Matices de reciprocidad:} {\hindifont बातचीत कर लेना} - tener una charla mutua

  \item \textbf{Matices de intensidad:} {\hindifont मज़ा लेना} - disfrutar intensamente

  \item \textbf{Matices de duración:} {\hindifont थोड़ा समय लेना} - tomar un poco de tiempo (para hacer algo)

  \item \textbf{Matices de finalidad:} {\hindifont समझने के लिए पूछ लेना} - preguntar para entender (tomar la acción de preguntar con intención)

\end{itemize}


\begin{tcolorbox}[colback=blue!5!white,colframe=blue!75!black,title=Nota/Clarificación]

\paragraph{Observación sobre Verbos Compuestos} 

Los verbos compuestos en hindi no son simplemente la suma de dos verbos. Tienen significados más específicos o matizados que los verbos individuales. Por ejemplo:


\begin{itemize}

  \item {\hindifont जाना} = ir, pero {\hindifont लगना} (which is लग + जाना) = comenzar a hacer algo

  \item {\hindifont बैठना} = sentarse (acción) y {\hindifont बैठ जाना} = sentarse permanentemente / tomar asiento

  \item {\hindifont भूलना} = olvidar, pero {\hindifont भूल जाना} = olvidar por completo / totalmente olvidar

\end{itemize}


Estos matices hacen que el idioma hindi sea más rico y preciso en la expresión de matices de acción.


\end{tcolorbox}

\subsection{4. Comunicación: Narrar Secuencias de Eventos de Forma más Fluida}


Con los participios y verbos compuestos, podemos narrar eventos de forma más fluida:

\subsubsection{4.1 Frases con Secuencias de Eventos}


\begin{itemize}

  \item {\hindifont मैं सुबह उठकर दांत धोया और फिर नाश्ता किया।} - Me levanté por la mañana, me cepillé los dientes y luego desayuné.

  \item {\hindifont वह पानी पीकर दवा लेगा।} - Él beberá agua y tomará medicina.

  \item {\hindifont हम खाना खाकर पार्क में घूमा।} - Comimos y paseamos en el parque.

  \item {\hindifont तुम अपने काम के लिए समय लो।} - Toma tiempo para tu trabajo.

  \item {\hindifont राम ने किताब पढ़कर उत्तर लिखा।} - Ram leyó el libro y escribió la respuesta.

  \item {\hindifont वह जल्दी से दौड़कर बस पकड़ी।} - Corrió rápidamente y atrapó el autobús.

  \item {\hindifont मैंने डॉक्टर से मिलकर दवा ली।} - Me encontré con el doctor y tomé medicina.

  \item {\hindifont तुम अध्ययन करके परीक्षा में अच्छे अंक प्राप्त करोगे।} - Estudiando conseguiréis buenas calificaciones en el examen.

  \item {\hindifont समझकर बोलो, ताकि कोई गलती न हो।} - Hablad comprendiendo, para que no haya errores.

  \item {\hindifont मैंने इसकी जाँच करके ठीक किया।} - Comprobé esto y lo arreglé.

  \item {\hindifont तुम पहले आकर बैठो।} - Venid primero y sentaos.

  \item {\hindifont मुझे पता लगाने के लिए तुम्हें बुलाना पड़ा।} - Tuve que llamarte para averiguarlo.

  \item {\hindifont हमें यात्रा के लिए टिकट बुक करना है।} - Debemos reservar boletos para el viaje.

  \item {\hindifont वह खेलने के लिए बाहर गया।} - Él salió afuera para jugar.

  \item {\hindifont उन्हें समझाने के लिए अधिक समय देना पड़ा।} - Tuvieron que dedicar más tiempo para explicar.

\end{itemize}

\subsubsection{4.2 Diálogos con Participios y Verbos Compuestos}


\textbf{Diálogo 1 - Planificando un proyecto}


\begin{itemize}

  \item A: {\hindifont क्या तुम इस प्रोजेक्ट के लिए तैयार हो?} - ¿Estás listo para este proyecto?

  \item B: {\hindifont हाँ, मैं जल्दी से तैयार हो रहा हूँ। तुम्हें कब तक काम पूरा करना चाहिए?} - Sí, me estoy preparando rápidamente. ¿Cuándo debes completar el trabajo tú?

  \item A: {\hindifont कल तक, इसलिए मैंने जाकर जानकारी ली है।} - Para mañana, así que fui y recolecté información.

  \item B: {\hindifont यह अच्छा है। मैं भी अधिक जानकारी जुटा लूंगा।} - Eso es bueno. Yo también recopilaré más información.

  \item A: {\hindifont तो हम जानकारी जुटाकर एक साथ बैठकर काम करेंगे।} - Entonces recopilaremos información y trabajaremos juntos sentados.

  \item B: {\hindifont हाँ, और हम तेजी से काम करके समय पर पूरा करेंगे।} - Sí, y trabajaremos rápidamente para completarlo a tiempo.

  \item A: {\hindifont यह अच्छा है। क्या तुमने अपने अध्यापक से बात कर ली है?} - Eso es bueno. ¿Has hablado ya con tu profesor?

  \item B: {\hindifont हाँ, मैंने उनसे समझकर बात की है। वे हमें मदद करेंगे।} - Sí, he hablado con ellos comprendiendo. Ellos nos ayudarán.

  \item A: {\hindifont मैं उन्हें समय पर जाने के लिए कहूंगा।} - Les diré que vengan a tiempo.

  \item B: {\hindifont ठीक है। अगर हम जल्दी से काम ले जाएँगे तो हम अधिक समय बचा सकते हैं।} - Bien. Si llevamos el trabajo rápidamente, podremos ahorrar más tiempo.

\end{itemize}


\textbf{Diálogo 2 - Hablando sobre una experiencia pasada}


\begin{itemize}

  \item A: {\hindifont तुम अपनी छुट्टियों के बारे में क्या सोचते हो?} - ¿Qué piensas sobre tus vacaciones?

  \item B: {\hindifont मैं उन्हें बहुत अच्छा मानता हूँ। मैंने पिछले महीने कश्मीर जाकर घूमा था।} - Las considero muy buenas. Fui a Cachemira el mes pasado y viajé.

  \item A: {\hindifont कश्मीर सुंदर है। क्या तुम ने वहाँ के लोगों से मिलकर बात की थी?} - Cachemira es hermoso. ¿Hablaste con la gente local allí?

  \item B: {\hindifont हाँ, मैंने उनसे मिल लिया। वे बहुत मेहनती थे।} - Sí, me reuní con ellos. Ellos eran muy trabajadores.

  \item A: {\hindifont तो तुमने वहाँ के खाने का स्वाद लिया?} - Entonces probaste la comida local allí.

  \item B: {\hindifont हाँ, मैंने खाकर यह माना कि वहाँ का खाना स्वादिष्ट है।} - Sí, comiendo reconocí que la comida allí era sabrosa.

  \item A: {\hindifont यह अच्छा है। क्या तुम ने वहाँ के प्राकृतिक स्थानों को देखकर तस्वीरें लीं?} - Eso es bueno. ¿Tomaste fotos después de ver los lugares naturales allí?

  \item B: {\hindifont हाँ, मैंने बहुत सुंदर स्थानों को देखकर तस्वीरें लीं। उसके बाद मैंने तुम्हें भेज दिया।} - Sí, después de ver lugares muy hermosos tomé fotos. Luego te las envié.

  \item A: {\hindifont तुमने मुझे कब भेजा था?} - ¿Cuándo me las enviaste?

  \item B: {\hindifont तुम्हारे लिए मैंने तुरंत भेजा था। तुम ने उन्हें देखा?} - Para ti, te las envié inmediatamente. ¿Las viste tú?

  \item A: {\hindifont हाँ, मैंने उन्हें देखकर तुम्हें प्रशंसा की।} - Sí, las vi y te elogié.

  \item B: {\hindifont यह मेरे लिए बहुत खुशी की बात है। अगली बार तुम मेरे साथ आओ।} - Eso es muy feliz para mí. La próxima vez ven conmigo.

\end{itemize}


\hrulefill

\section{Unidad 21: Vocabulario Temático: Salud, Ocio y Tecnología}


\textbf{Objetivo:} Adquirir vocabulario específico en temas de salud, ocio y tecnología.

\subsection{1. Vocabulario: Salud y Medicina}


Este vocabulario es esencial para expresar estados de salud, enfermedades o visitar al médico:

\subsubsection{1.1 Partes del Cuerpo (Repaso y Ampliación)}


\begin{center}
\begin{tabularx}{\textwidth}{| X | X | X | X | X | X | X |}
\hline
\textbf{Español} & \textbf{Hindi} & \textbf{Transliteración} & \textbf{Género} & \textbf{Uso} & \textbf{Ejemplo en Frase} & \textbf{Traducción} \ \hline
Cabeza & {\hindifont सिर} & sir & Masculino & Parte del cuerpo & {\hindifont मेरा सिर दर्द कर रहा है।} & Me duele la cabeza. \ \hline
Ojo & {\hindifont आँख} & aankh & Femenino & Órgano de visión & {\hindifont उसकी आँखें लाल हैं।} & Sus ojos están rojos. \ \hline
Nariz & {\hindifont नाक} & naak & Femenino & Órgano respiratorio & {\hindifont उसे नाक में सर्दी है।} & Tiene resfriado en la nariz. \ \hline
Boca & {\hindifont मुँह} & muh & Masculino & Abertura facial & {\hindifont मुँह बंद करो।} & Cierra la boca. \ \hline
Mano & {\hindifont हाथ} & haath & Masculino & Extremidad superior & {\hindifont हाथ धो लो।} & lávate las manos. \ \hline
Pie & {\hindifont पैर} & pair & Masculino & Extremidad inferior & {\hindifont पैर में चोट है।} & El pie está lesionado. \ \hline
Corazón & {\hindifont दिल} & dil & Masculino & Órgano vital & {\hindifont दिल से बात करो।} & Habla desde el corazón. \ \hline
Estómago & {\hindifont पेट} & pet & Masculino & Parte corporal & {\hindifont पेट में दर्द है।} & El estómago duele. \ \hline
Garganta & {\hindifont गला} & gala & Masculino & Parte corporal & {\hindifont गला खराब है।} & La garganta está dañada. \ \hline
Hombro & {\hindifont कंधा} & kandhaa & Masculino & Parte corporal & {\hindifont कंधे पर बोझ है।} & Hay carga en el hombro. \ \hline
\end{tabularx}
\end{center}

\subsubsection{1.2 Vocabulario de Enfermedades y Síntomas}


\begin{center}
\begin{tabularx}{\textwidth}{| X | X | X | X | X | X | X |}
\hline
\textbf{Español} & \textbf{Hindi} & \textbf{Transliteración} & \textbf{Género} & \textbf{Uso} & \textbf{Ejemplo en Frase} & \textbf{Traducción} \ \hline
Enfermedad & {\hindifont बीमारी} & beemaarii & Femenino & Estado de malestar & {\hindifont मैं बीमारी में हूँ।} & Estoy enfermo. \ \hline
Dolor & {\hindifont दर्द} & dard & Masculino & Sensación física & {\hindifont सिर में दर्द है।} & Tengo dolor de cabeza. \ \hline
Fiebre & {\hindifont बुखार} & bukhaar & Masculino & Condición médica & {\hindifont मुझे बुखार है।} & Tengo fiebre. \ \hline
Resfriado & {\hindifont सर्दी} & sardi & Femenino & Condición médica & {\hindifont उसे सर्दी है।} & Él tiene resfriado. \ \hline
ToS & {\hindifont खांसी} & khaansii & Femenino & Síntoma respiratorio & {\hindifont तुम्हें खांसी हो रही है।} & Estás tosiendo. \ \hline
Medicina & {\hindifont दवा} & davaa & Femenino & Tratamiento & {\hindifont डॉक्टर ने दवा दी।} & El médico dio medicina. \ \hline
Doctor & {\hindifont डॉक्टर} & doctor & Masculino/Femenino & Profesional médico & {\hindifont मैं डॉक्टर के पास गया।} & Fui al doctor. \ \hline
Hospital & {\hindifont अस्पताल} & aspataal & Masculino & Institución médica & {\hindifont रोगी अस्पताल में है।} & El paciente está en el hospital. \ \hline
Paciente & {\hindifont रोगी} & rogii & Masculino/Femenino & Persona enferma & {\hindifont रोगी को दवा चाहिए।} & El paciente necesita medicina. \ \hline
Consultorio & {\hindifont क्लीनिक} & clinic & Masculino & Centro médico más pequeño & {\hindifont वह क्लीनिक में काम करता है।} & Él trabaja en el consultorio. \ \hline
\end{tabularx}
\end{center}


\begin{tcolorbox}[colback=blue!5!white,colframe=blue!75!black,title=Nota/Clarificación]

\paragraph{Frases Comunes en Contextos Médicos} 

\begin{itemize}

  \item {\hindifont मुझे डॉक्टर के पास जाना है।} - Tengo que ir al doctor.

  \item {\hindifont मुझे बहुत दर्द हो रहा है।} - Tengo mucho dolor.

  \item {\hindifont तुम्हारे बुखार के लिए यह दवा लो।} - Toma esta medicina para tu fiebre.

  \item {\hindifont क्या तुम स्वस्थ हो?} - ¿Estás sano?

  \item {\hindifont मैं ठीक नहीं हूँ।} - No estoy bien.

  \item {\hindifont आज मैं घर पर रहने का विचार कर रहा हूँ।} - Estoy considerando quedarme en casa hoy.

  \item {\hindifont मुझे सर्दी लग गई है।} - Me he resfriado.

  \item {\hindifont तुम्हें तुरंत अस्पताल जाना चाहिए।} - Deberías ir al hospital inmediatamente.

  \item {\hindifont मैं एक रोगी से मिलने के लिए तैयार हूँ।} - Estoy listo para ver a un paciente.

\end{itemize}


\end{tcolorbox}

\subsection{2. Vocabulario: Ocio y Actividades Recreativas}


Ahora vamos a explorar vocabulario relacionado con actividades de ocio y entretenimiento:

\subsubsection{2.1 Actividades de Ocio y Pasatiempos}


\begin{center}
\begin{tabularx}{\textwidth}{| X | X | X | X | X | X | X |}
\hline
\textbf{Español} & \textbf{Hindi} & \textbf{Transliteración} & \textbf{Género} & \textbf{Uso} & \textbf{Ejemplo en Frase} & \textbf{Traducción} \ \hline
Hobbie & {\hindifont शौक} & shaok & Masculino & Actividad recreativa & {\hindifont मेरा शौक पढ़ाई है।} & Mi hobby es estudiar. \ \hline
Música & {\hindifont संगीत} & sangeet & Masculino & Arte musical & {\hindifont मैं संगीत सुनना पसंद करता हूँ।} & Me gusta escuchar música. \ \hline
Canción & {\hindifont गाना} & gaanaa & Masculino & Composición musical & {\hindifont यह एक सुंदर गाना है।} & Ésta es una canción hermosa. \ \hline
Video & {\hindifont वीडियो} & video & Masculino & Contenido audiovisual & {\hindifont हम एक वीडियो देख रहे हैं।} & Estamos viendo un video. \ \hline
Televisión & {\hindifont टेलीविज़न} & televizan & Masculino & Dispositivo/equipo & {\hindifont बच्चे टेलीविज़न देख रहे हैं।} & Los niños están viendo la televisión. \ \hline
Película & {\hindifont फ़िल्म} & film & Femenino & Producción cinematográfica & {\hindifont इस फिल्म में राम अभिनेता है।} & Ram es actor en esta película. \ \hline
Deportes & {\hindifont खेल} & khel & Masculino (en plural) & Actividades físicas & {\hindifont तुम्हें कौन सा खेल पसंद है?} & ¿Qué deporte te gusta? \ \hline
Juego & {\hindifont खेल} & khel & Masculino (en sing.) & Actividad recreativa & {\hindifont यह खेल बहुत मज़ेदार है।} & Este juego es muy divertido. \ \hline
Lectura & {\hindifont पढ़ाई} & padhaaii & Femenino & Actividad de leer & {\hindifont मेरी पढ़ाई में पुस्तकों के लिए ज्यादा समय लगता है।} & En mi lectura se necesita mucho tiempo para libros. \ \hline
Excursión & {\hindifont यात्रा} & yaatraa & Femenino & Salida recreativa & {\hindifont हम लोग एक यात्रा करने की योजना बना रहे हैं।} & Estamos planeando una excursión. \ \hline
\end{tabularx}
\end{center}

\subsubsection{2.2 Sustantivos Relacionados con Ocio}


\begin{center}
\begin{tabularx}{\textwidth}{| X | X | X | X | X | X | X |}
\hline
\textbf{Español} & \textbf{Hindi} & \textbf{Transliteración} & \textbf{Género} & \textbf{Uso} & \textbf{Ejemplo en Frase} & \textbf{Traducción} \ \hline
Librería & {\hindifont पुस्तकालय} & pustakaalay & Masculino & Lugar para leer & {\hindifont हम पुस्तकालय में पढ़ते हैं।} & Leemos en la biblioteca. \ \hline
Parque & {\hindifont पार्क} & park & Masculino & Lugar recreativo & {\hindifont हम पार्क में घूमते हैं।} & Paseamos en el parque. \ \hline
Cine & {\hindifont सिनेमा} & sinemaa & Masculino & Lugar para ver películas & {\hindifont क्या तुम सिनेमा जाना चाहोगे?} & ¿Quieres ir al cine? \ \hline
Centro recreativo & {\hindifont मनोरंजन केन्द्र} & manoranjan kendr & Masculino & Instalación recreativa & {\hindifont बच्चे मनोरंजन केन्द्र पर खेलते हैं।} & Los niños juegan en el centro recreativo. \ \hline
Mercado & {\hindifont बाजार} & baazaar & Masculino & Lugar comercial & {\hindifont बाजार में घूमना मज़ेदार है।} & Pasear por el mercado es divertido. \ \hline
Teatro & {\hindifont नाट्यशाला} & naatyashaalaa & Femenino & Lugar para representaciones & {\hindifont हम नाट्यशाला में नाटक देखते हैं।} & Veremos una obra en el teatro. \ \hline
Club & {\hindifont क्लब} & klab & Masculino & Organización social & {\hindifont क्लब में सांस्कृतिक कार्यक्रम होते हैं।} & Hay programas culturales en el club. \ \hline
Estadio & {\hindifont स्टेडियम} & stadiyam & Masculino & Lugar deportivo & {\hindifont खेल स्टेडियम में आयोजित होते हैं।} & Los deportes se organizan en el estadio. \ \hline
Playa & {\hindifont समुद्र तट} & samudra tatt & Masculino & Lugar de recreo & {\hindifont हम समुद्र तट पर घूमने जाते हैं।} & Vamos a pasear a la playa. \ \hline
Montaña & {\hindifont पहाड़} & pahaad & Masculino & Lugar natural & {\hindifont हम पहाड़ पर यात्रा करते हैं।} & Vamos de excursión a la montaña. \ \hline
\end{tabularx}
\end{center}

\subsubsection{2.3 Verbos Relacionados con Ocio y Entretenimiento}


\begin{itemize}

  \item {\hindifont गाना गाना} (gaanaa gaanaa) - cantar canciones

  \item {\hindifont नृत्य करना} (nrity karnaa) - bailar

  \item {\hindifont संगीत सुनना} (sangeet sunnaa) - escuchar música

  \item {\hindifont फ़िल्म देखना} (film dekhnaa) - ver película

  \item {\hindifont पुस्तक पढ़ना} (pustak padhnaa) - leer libros

  \item {\hindifont खेलना} (khelnaa) - jugar

  \item {\hindifont पिकनिक मनाना} (picnic manaanaa) - celebrar picnic

  \item {\hindifont घूमना} (ghoomnaa) - pasear, viajar

  \item {\hindifont तैरना} (tairnaa) - nadar

  \item {\hindifont चित्र बनाना} (chitr banaanaa) - dibujar imágenes

  \item {\hindifont वीडियो देखना} (video dekhnaa) - ver videos

  \item {\hindifont टेलीविजन देखना} (television dekhnaa) - ver televisión

  \item {\hindifont यात्रा करना} (yaatraa karnaa) - viajar, hacer excursión

  \item {\hindifont हँसना} (hansnaa) - reír

  \item {\hindifont रोना} (ronaa) - llorar

  \item {\hindifont खिलौने खेलना} (khilone khelnaa) - jugar con juguetes

\end{itemize}

\subsection{3. Vocabulario: Tecnología}


El vocabulario relacionado con la tecnología es cada vez más importante en el mundo moderno:

\subsubsection{3.1 Dispositivos Tecnológicos}


\begin{center}
\begin{tabularx}{\textwidth}{| X | X | X | X | X | X | X |}
\hline
\textbf{Español} & \textbf{Hindi} & \textbf{Transliteración} & \textbf{Género} & \textbf{Uso} & \textbf{Ejemplo en Frase} & \textbf{Traducción} \ \hline
Móvil & {\hindifont मोबाइल} & mobile & Masculino & Dispositivo de comunicación & {\hindifont मुझे मोबाइल से कॉल करो।} & Llámanos con el móvil. \ \hline
Ordenador & {\hindifont कंप्यूटर} & kampyooter & Masculino & Dispositivo de trabajo & {\hindifont कंप्यूटर पर काम कर रहा हूँ।} & Estoy trabajando en el ordenador. \ \hline
Internet & {\hindifont इंटरनेट} & internet & Masculino & Red de comunicación & {\hindifont इंटरनेट से जानकारी मिलती है।} & Se obtiene información por internet. \ \hline
Correo electrónico & {\hindifont ई-मेल} & iimeel & Masculino & Forma de comunicación & {\hindifont तुमने मुझे ई-मेल भेजी।} & Me enviaste un correo electrónico. \ \hline
Chat & {\hindifont चैट} & chat & Masculino & Comunicación digital & {\hindifont हम चैट कर रहे हैं।} & Estamos chateando. \ \hline
Correo electrónico & {\hindifont ईमेल} & imeel & Masculino & Comunicación electrónica & {\hindifont मैंने नई ईमेल लिखी।} & Escribí un nuevo correo electrónico. \ \hline
Red social & {\hindifont सोशल मीडिया} & soshal miidiyaa & Masculino & Plataforma digital & {\hindifont लोग सोशल मीडिया पर घूमते हैं।} & La gente navega en redes sociales. \ \hline
Aplicación & {\hindifont एप्लिकेशन} & aplikeshan & Masculino & Software & {\hindifont यह एप्लिकेशन उपयोगी है।} & Esta aplicación es útil. \ \hline
Computadora portátil & {\hindifont लैपटॉप} & laptoo & Femenino & Dispositivo móvil & {\hindifont तुम्हारी लैपटॉप नई है।} & Tu ordenador portátil es nueva. \ \hline
Tablet & {\hindifont टैबलेट} & tablet & Femenino & Dispositivo digital & {\hindifont हम टैबलेट पर पढ़ रहे हैं।} & Estamos leyendo en la tablet. \ \hline
\end{tabularx}
\end{center}

\subsubsection{3.2 Vocabulario Relacionado con Internet y Comunicación Digital}


\begin{itemize}

  \item {\hindifont वेबसाइट} (websait) - sitio web

  \item {\hindifont सर्च} (search) - búsqueda

  \item {\hindifont डाउनलोड} (download) - descargar

  \item {\hindifont अपलोड} (upload) - subir

  \item {\hindifont लॉगिन} (login) - iniciar sesión

  \item {\hindifont लॉगआउट} (logout) - cerrar sesión

  \item {\hindifont पासवर्ड} (password) - contraseña

  \item {\hindifont यूज़र} (user) - usuario

  \item {\hindifont कमेंट} (comment) - comentario

  \item {\hindifont लाइक} (like) - me gusta

  \item {\hindifont शेयर} (share) - compartir

  \item {\hindifont वीडियो कॉल} (video call) - videollamada

  \item {\hindifont ऑनलाइन} (online) - en línea

  \item {\hindifont ऑफलाइन} (offline) - fuera de línea

  \item {\hindifont वाई-फाई} (wifi) - Wi-Fi

\end{itemize}


\begin{tcolorbox}[colback=yellow!10!white,colframe=orange!75!black,title=Regla de Oro]

\paragraph{Uso de 'vālā' con Tecnología y Ocio} 

El sufijo {\hindifont वाला} es especialmente útil para describir dispositivos con características específicas:


\begin{itemize}

  \item {\hindifont मोबाइल वाला} - el que tiene móvil / el tipo de móvil

  \item {\hindifont इंटरनेट वाला} - con internet / que tiene internet

  \item {\hindifont कंप्यूटर वाला} - el que tiene ordenador / el que trabaja con ordenador

  \item {\hindifont म्यूज़िक वाला} - con música / el tipo que ama música

  \item {\hindifont गाना गा वाला} - el que canta canciones

  \item {\hindifont खेल वाला} - el que juega / relacionado con juegos

\end{itemize}


\end{tcolorbox}

\subsection{4. Números Ordinales: पहला, दूसरा, तीसरा...}


Los números ordinales se utilizan para expresar posición o secuencia:


\begin{center}
\begin{tabularx}{\textwidth}{| X | X | X | X |}
\hline
\textbf{Ordinal} & \textbf{Hindi} & \textbf{Transliteración} & \textbf{Traducción} \ \hline
Primero & {\hindifont पहला} & pahlaa & Primero (masc.)/{\hindifont पहली} (pahlii) - fem./{\hindifont पहले} (pahle) - masc.pl. \ \hline
Segundo & {\hindifont दूसरा} & doosraa & Segundo (masc.)/{\hindifont दूसरी} (doosrii) - fem./{\hindifont दूसरे} (doosre) - masc.pl. \ \hline
Tercero & {\hindifont तीसरा} & teesraa & Tercero (masc.)/{\hindifont तीसरी} (teesrii) - fem./{\hindifont तीसरे} (teesre) - masc.pl. \ \hline
Cuarto & {\hindifont चौथा} & chaauthaa & Cuarto (masc.)/{\hindifont चौथी} (chaauthii) - fem./{\hindifont चौथे} (chaauthe) - masc.pl. \ \hline
Quinto & {\hindifont पाँचवाँ} & paanchvaa(n) & Quinto (masc.)/{\hindifont पाँचवीं} (paanchviin) - fem./{\hindifont पाँचवें} (paanchven) - masc.pl. \ \hline
Sexto & {\hindifont छठवाँ} & chhathvaa(n) & Sexto (masc.)/{\hindifont छठवीं} (chhathviin) - fem./{\hindifont छठवें} (chhathven) - masc.pl. \ \hline
Séptimo & {\hindifont सातवाँ} & saatvaa(n) & Séptimo (masc.)/{\hindifont सातवीं} (saatviin) - fem./{\hindifont सातवें} (saatven) - masc.pl. \ \hline
Octavo & {\hindifont आठवाँ} & aathvaa(n) & Octavo (masc.)/{\hindifont आठवीं} (aathviin) - fem./{\hindifont आठवें} (aathven) - masc.pl. \ \hline
Noveno & {\hindifont नौवाँ} & navvaa(n) & Noveno (masc.)/{\hindifont नौवीं} (navviin) - fem./{\hindifont नौवें} (navven) - masc.pl. \ \hline
Décimo & {\hindifont दसवाँ} & dasvaa(n) & Décimo (masc.)/{\hindifont दसवीं} (dasviin) - fem./{\hindifont दसवें} (dasven) - masc.pl. \ \hline
\end{tabularx}
\end{center}

\subsubsection{4.1 Uso de los Números Ordinales}


Los números ordinales concuerdan con el sustantivo en género y número como cualquier adjetivo:


\begin{itemize}

  \item {\hindifont पहला लड़का} - El primer chico (masculino singular)

  \item {\hindifont पहली लड़की} - La primera chica (femenino singular)

  \item {\hindifont पहले लड़के} - Los primeros chicos (masculino plural)

  \item {\hindifont पहली किताबें} - Las primeras libros (femenino plural)

  \item {\hindifont दूसरा पाला} - El segundo turno

  \item {\hindifont तीसरी मंज़िल} - El tercer piso

  \item {\hindifont चौथे दिन} - El cuarto día

  \item {\hindifont पाँचवीं कक्षा} - El quinto grado

\end{itemize}

\subsection{5. Adverbios: धीरे, जल्दी, ज़ोर से, अच्छा}


A continuación veremos una variedad de adverbios que podemos usar para modificar verbos, adjetivos u otros adverbios:

\subsubsection{5.1 Adverbios de Modo}


\begin{center}
\begin{tabularx}{\textwidth}{| X | X | X | X | X | X |}
\hline
\textbf{Español} & \textbf{Hindi} & \textbf{Transliteración} & \textbf{Uso} & \textbf{Ejemplo en Frase} & \textbf{Traducción} \ \hline
Despacio & {\hindifont धीरे} & dheere & Modo de acción & {\hindifont धीरे-धीरे बोलो।} & Habla despacio. \ \hline
Rápido/Pronto & {\hindifont जल्दी} & jaldee & Velocidad de acción & {\hindifont हम जल्दी घर जाएँगे।} & Volveremos pronto a casa. \ \hline
Fuerte & {\hindifont ज़ोर से} & zor se & Intensidad de acción & {\hindifont ज़ोर से बोलो।} & Habla fuerte. \ \hline
Bien & {\hindifont अच्छा} & achchhaa & Calidad de acción & {\hindifont वह हिंदी अच्छा बोलता है।} & Él habla bien hindi. \ \hline
Claramente & {\hindifont स्पष्ट} & spashtt & Calidad de comprensibilidad & {\hindifont कृपया स्पष्ट बताओ।} & Por favor, explica claramente. \ \hline
Correctamente & {\hindifont सही} & sahii & Calidad de precisión & {\hindifont सही से काम करो।} & Haced el trabajo correctamente. \ \hline
Silenciosamente & {\hindifont चुपचाप} & chupchaap & Modo de acción tranquila & {\hindifont कक्षा में चुपचाप बैठो।} & Sentaos silenciosamente en clase. \ \hline
Suavemente & {\hindifont नरम से} & nam se & Modo gentil & {\hindifont बच्चे को नरम से छूओ।} & Toca al niño suavemente. \ \hline
Atentamente & {\hindifont ध्यान से} & dhyaan se & Modo atento & {\hindifont ध्यान से पढ़ो।} & Leed atentamente. \ \hline
Incorrectamente & {\hindifont गलत} & galat & Calidad de error & {\hindifont गलत से काम करना ठीक नहीं है।} & No está bien hacer trabajo incorrectamente. \ \hline
\end{tabularx}
\end{center}

\subsubsection{5.2 Adverbios de Frecuencia}


\begin{center}
\begin{tabularx}{\textwidth}{| X | X | X | X | X | X |}
\hline
\textbf{Español} & \textbf{Hindi} & \textbf{Transliteración} & \textbf{Uso} & \textbf{Ejemplo en Frase} & \textbf{Traducción} \ \hline
Siempre & {\hindifont हमेशा} & hameshaa & Frecuencia constante & {\hindifont वह हमेशा सच बोलता है।} & Él siempre dice la verdad. \ \hline
A veces & {\hindifont कभी-कभी} & kabhi-kabhi & Frecuencia esporádica & {\hindifont हम कभी-कभी पार्क जाते हैं।} & A veces vamos al parque. \ \hline
Nunca & {\hindifont कभी नहीं} & kabhi nahin & ausencia total & {\hindifont मैं कभी झूठ नहीं बोलता।} & Nunca miento. \ \hline
A menudo & {\hindifont अक्सर} & aksar & Frecuencia alta & {\hindifont अक्सर वह खेलने जाता है।} & A menudo él va a jugar. \ \hline
Raramente & {\hindifont शायद} & shaayad & Frecuencia baja & {\hindifont वह शायद ही बाहर जाता है।} & Raramente él sale. \ \hline
Cada día & {\hindifont हर रोज़} & har roz & Frecuencia diaria & {\hindifont मैं हर रोज़ पढ़ता हूँ।} & Yo leo todos los días. \ \hline
Todos los días & {\hindifont रोज़} & roz & Frecuencia diaria & {\hindifont वह रोज़ काम पर जाता है।} & Él va al trabajo todos los días. \ \hline
Constantemente & {\hindifont लगातार} & lagataar & Frecuencia continua & {\hindifont लगातार बारिश हो रही है।} & Está lloviendo constantemente. \ \hline
Todas las semanas & {\hindifont हर हफ्ते} & har hafte & Frecuencia semanal & {\hindifont हर हफ्ते एक फिल्म देखते हैं।} & Vemos una película cada semana. \ \hline
Regularmente & {\hindifont नियमित} & niyamat & Frecuencia sistemática & {\hindifont नियमित रूप से व्यायाम करें।} & Haz ejercicio regularmente. \ \hline
\end{tabularx}
\end{center}

\subsubsection{5.3 Adverbios de Lugar}


\begin{center}
\begin{tabularx}{\textwidth}{| X | X | X | X | X | X |}
\hline
\textbf{Español} & \textbf{Hindi} & \textbf{Transliteración} & \textbf{Uso} & \textbf{Ejemplo en Frase} & \textbf{Traducción} \ \hline
Aquí & {\hindifont यहाँ} & yahaan & Ubicación próxima & {\hindifont तुम यहाँ रहो।} & Tú quédate aquí. \ \hline
Allí & {\hindifont वहाँ} & vahaan & Ubicación distante & {\hindifont वह वहाँ खड़ा है।} & Él está de pie allí. \ \hline
Cerca & {\hindifont पास} & paas & Proximidad & {\hindifont दुकान घर के पास है।} & La tienda está cerca de la casa. \ \hline
Lejos & {\hindifont दूर} & door & Distancia & {\hindifont शहर बहुत दूर है।} & La ciudad está muy lejos. \ \hline
Delante & {\hindifont आगे} & aage & Anterioridad espacial & {\hindifont कार घर के आगे है।} & El coche está delante de la casa. \ \hline
Detrás & {\hindifont पीछे} & piiche & Posterioridad espacial & {\hindifont लड़का घर के पीछे है।} & El chico está detrás de la casa. \ \hline
Encima & {\hindifont ऊपर} & upar & Superposición vertical & {\hindifont पक्षी आकाश के ऊपर उड़ रहा है।} & El pájaro está volando encima del cielo. \ \hline
Debajo & {\hindifont नीचे} & nichhe & Posición inferior & {\hindifont मछली पानी के नीचे है।} & El pez está debajo del agua. \ \hline
Entre & {\hindifont बीच} & beech & Medialidad & {\hindifont लड़के घरों के बीच खेल रहे हैं।} & Los chicos están jugando entre las casas. \ \hline
Alrededor & {\hindifont आसपास} & aaspaas & Ubicación periférica & {\hindifont दुकानों के आसपास बहुत भीड़ है।} & Hay mucha gente alrededor de las tiendas. \ \hline
\end{tabularx}
\end{center}

\subsection{6. El Sufijo 'वाला' (vālā)}


El sufijo {\hindifont वाला} es una de las herramientas gramaticales más versátiles en hindi. Se puede usar tanto como adjetivo como sustantivo y expresa posesión, característica, profesión o relación con algo.


\begin{tcolorbox}[colback=yellow!10!white,colframe=orange!75!black,title=Regla de Oro]

\paragraph{Función del Sufijo 'वाला'} 

El sufijo {\hindifont वाला} indica que algo o alguien \textbf{tiene}, \textbf{posee}, \textbf{contiene}, \textbf{está relacionado con} o \textbf{realiza la acción} de lo que precede al sufijo.


\begin{itemize}

  \item {\hindifont गाड़ी वाला आदमी} - El hombre que tiene coche

  \item {\hindifont लाल वाला कपड़ा} - La ropa que es roja

  \item {\hindifont चाय वाला} - El vendedor de té

  \item {\hindifont पढ़ाई वाला लड़का} - El chico que estudia

\end{itemize}


\end{tcolorbox}

\subsubsection{6.1 Formación del Sufijo 'वाला'}


El sufijo {\hindifont वाला} se comporta como un adjetivo y concuerda con el sustantivo que acompaña en género y número:


\begin{itemize}

  \item Masculino singular: {\hindifont -वाला} (-vaalaa)

  \item Femenino singular: {\hindifont -वाली} (-vaalii)

  \item Masculino plural: {\hindifont -वाले} (-vaale)

\end{itemize}


\begin{center}
\begin{tabularx}{\textwidth}{| X | X | X | X | X | X |}
\hline
\textbf{Palabra Base} & \textbf{Masculino Sing.} & \textbf{Femenino Sing.} & \textbf{Masculino Pl.} & \textbf{Ejemplo} & \textbf{Traducción} \ \hline
Rojo ({\hindifont लाल}) & {\hindifont लाल वाला} & {\hindifont लाल वाली} & {\hindifont लाल वाले} & {\hindifont लाल वाली कमीज़} & La camisa roja \ \hline
Verbo (लिखना, escribir) & {\hindifont लिखने वाला} & {\hindifont लिखने वाली} & {\hindifont लिखने वाले} & {\hindifont लिखने वाला आदमी} & El hombre que escribe \ \hline
Verbo (पढ़ना, leer) & {\hindifont पढ़ने वाला} & {\hindifont पढ़ने वाली} & {\hindifont पढ़ने वाले} & {\hindifont पढ़ने वाली लड़कियाँ} & Las chicas que estudian \ \hline
Objeto (किताब, libro) & {\hindifont किताब वाला} & {\hindifont किताब वाली} & {\hindifont किताब वाले} & {\hindifont किताब वाले दुकानदार} & El librero \ \hline
Nombre (राम) & {\hindifont राम वाला} & {\hindifont राम वाली} & {\hindifont राम वाले} & {\hindifont राम वाला लड़का} & El chico llamado Ram \ \hline
Característica (थोड़ा, poco) & {\hindifont थोड़ा वाला} & {\hindifont थोड़ी वाली} & {\hindifont थोड़े वाले} & {\hindifont थोड़ा वाला पैसा} & El dinero que es poco \ \hline
Profesión (चाय) & {\hindifont चाय वाला} & {\hindifont चाय वाली} & {\hindifont चाय वाले} & {\hindifont चाय वाले दुकानदार} & El vendedor de té \ \hline
Objeto (फल, fruta) & {\hindifont फल वाला} & {\hindifont फल वाली} & {\hindifont फल वाले} & {\hindifont फल वाली दुकान} & La tienda de frutas \ \hline
Verbo (खाना, comer) & {\hindifont खाने वाला} & {\hindifont खाने वाली} & {\hindifont खाने वाले} & {\hindifont खाने वाले लोग} & La gente que come \ \hline
Adjetivo (तेज़, rápido) & {\hindifont तेज़ वाला} & {\hindifont तेज़ वाली} & {\hindifont तेज़ वाले} & {\hindifont तेज़ वाली गाड़ी} & El coche que es rápido \ \hline
\end{tabularx}
\end{center}

\subsubsection{6.2 Usos del Sufijo 'वाला'}


\begin{itemize}

  \item \textbf{Para expresar posesión:} {\hindifont किताब वाला आदमी} - El hombre que tiene libro / El que tiene libros

  \item \textbf{Para expresar características:} {\hindifont लंबा वाला लड़का} - El chico que es alto

  \item \textbf{Para expresar profesiones:} {\hindifont चाय वाला} - Vendedor de té, {\hindifont दूध वाला} - Lechero

  \item \textbf{Para expresar relaciones:} {\hindifont राम वाला घर} - La casa de Ram / La casa donde vive Ram

  \item \textbf{Para expresar acciones pasadas:} {\hindifont पढ़ने वाला छात्र} - El estudiante que está estudiando / El estudiante que estudia

  \item \textbf{Para formar adjetivos compuestos:} {\hindifont लाल वाला} - Que es rojo, de color rojo

  \item \textbf{Para formar sustantivos:} {\hindifont लिखने वाला} - El que escribe (sustantivo derivado de verbo)

\end{itemize}

\subsubsection{6.3 Ejemplos de Frases con 'वाला'}


\begin{itemize}

  \item {\hindifont लाल वाली कमीज़ मुझे पसंद है।} - Me gusta la camisa roja.

  \item {\hindifont सफेद वाला कुर्ता तुम्हारे लिए अच्छा है।} - La kurta blanca es buena para ti.

  \item {\hindifont वह लड़की रोटी बनाने वाली है।} - Esa chica es la que cocina pan.

  \item {\hindifont तुम वहाँ के लोगों को तेज़ चलने वाले पाते हो।} - Notas que la gente allí camina rápido.

  \item {\hindifont चाय वाले ने हमें अच्छी चाय दी।} - El vendedor de té nos dio buen té.

  \item {\hindifont मुझे किताब पढ़ने वाले लोग पसंद हैं।} - Me gustan las personas que leen libros.

  \item {\hindifont गाड़ी वाले लोगों को ध्यान से चलाना चाहिए।} - Los que tienen coches deberían conducir con cuidado.

  \item {\hindifont तुमने तेज़ बोलने वाले वक्ता को सुना है?} - ¿Has escuchado al orador que habla rápido?

  \item {\hindifont यह स्कूल खेलने वाले बच्चों के लिए बहुत अच्छा है।} - Esta escuela es muy buena para los niños que juegan.

  \item {\hindifont हमारे घर वाले बहुत मेहमाननवाज़ हैं।} - La gente de nuestra casa es muy hospitalaria.

\end{itemize}

\subsection{7. Comunicación: Describir Rutinas con Mayor Precisión y Seleccionar Objetos}


Con los adverbios y el sufijo {\hindifont वाला}, podemos comunicarnos con mayor precisión y detalle:

\subsubsection{7.1 Descripciones de Rutinas Diarias con Adverbios}


\begin{itemize}

  \item {\hindifont हर रोज़ मैं सुबह जल्दी उठता हूँ।} - Todos los días me levanto temprano por la mañana.

  \item {\hindifont कभी-कभी वह धीरे पढ़ता है।} - A veces él estudia despacio.

  \item {\hindifont हम अक्सर देर रात तक जागते हैं।} - A menudo nos quedamos despiertos hasta tarde por la noche.

  \item {\hindifont तुम ज़ोर से बोल रहे थे।} - Tú estabas hablando fuerte.

  \item {\hindifont वह नरम से बोलती है।} - Ella habla suavemente.

  \item {\hindifont मैं लगातार पढ़ रहा हूँ।} - Estoy leyendo continuamente.

  \item {\hindifont हमेशा वह ठीक से काम करता है।} - Él siempre hace el trabajo correctamente.

  \item {\hindifont रोज़ वह तेज़ चलता है।} - Él camina rápido todos los días.

  \item {\hindifont कभी नहीं वह मेहनत से काम नहीं करता।} - Él nunca trabaja diligentemente.

  \item {\hindifont नियमित रूप से तुम खेलते हो।} - Tú juegas regularmente.

\end{itemize}

\subsubsection{7.2 Selección y Descripción de Objetos con 'वाला'}


\begin{itemize}

  \item {\hindifont यह लाल वाली किताब मेरी है।} - Este libro rojo es mío.

  \item {\hindifont कृपया उस चाय वाले के पास जाओ।} - Por favor, id donde el vendedor de té.

  \item {\hindifont मुझे लंबा वाला कोट चाहिए।} - Necesito un abrigo largo.

  \item {\hindifont वह गाड़ी वाला आदमी मेरा दोस्त है।} - El hombre que tiene coche es mi amigo.

  \item {\hindifont तुम उन लिखने वाले लोगों को जानते हो?} - ¿Conoces a esas personas que escriben?

  \item {\hindifont कौन सा फल खाने वाला तुम्हें पसंद है?} - ¿Qué tipo de fruta comer te gusta?

  \item {\hindifont ये सब किताबें लिखने वाले लेखक की हैं।} - Todos estos libros son del escritor que escribe.

  \item {\hindifont तुम उस तेज़ चलने वाली बस का इंतज़ार कर रहे थे।} - Estabas esperando el autobús que anda rápido.

  \item {\hindifont यह एक फिल्म देखने वाले लोगों के लिए अच्छी है।} - Ésta es buena para la gente que ve películas.

  \item {\hindifont इस बाजार में फल वाली दुकानें बहुत हैं।} - En este mercado hay muchas tiendas de frutas.

\end{itemize}

\subsubsection{7.3 Diálogo: En la Tienda de Ropa}


Este diálogo combina adverbios y el sufijo {\hindifont वाला}:


\begin{itemize}

  \item Cliente: {\hindifont क्या आपके पास लाल वाला कुर्ता है?} - ¿Tiene Ud. una kurta roja?

  \item Comerciante: {\hindifont हाँ, बहुत सारे लाल वाले कुर्ते हैं। क्या आप उन्हें धीरे से देखना चाहते हैं?} - Sí, hay muchas kurtas rojas. ¿Quiere Ud. verlas despacio?

  \item Cliente: {\hindifont हाँ, मैं अच्छा से देखना चाहता हूँ। फिर मैं खरीदने का निर्णय लूंगा।} - Sí, quiero verlas bien. Luego tomaré la decisión de comprar.

  \item Comerciante: {\hindifont ठीक है। यह लाल वाले कुर्ते हैं। यह नीले वाले कुर्ते हैं। और यह हरे वाले कुर्ते हैं।} - Bien. Éstas son kurtas rojas. Éstas son kurtas azules. Y éstas son kurtas verdes.

  \item Cliente: {\hindifont यह कुर्ता कैसा है?} - ¿Cómo es esta kurta?

  \item Comerciante: {\hindifont यह बहुत अच्छा है। इसका डिज़ाइन खूबसूरत है। यह लंबा वाला है।} - Ésta es muy buena. Su diseño es hermoso. Ésta es larga.

  \item Cliente: {\hindifont क्या यह ठीक से बैठता है?} - ¿Se ajusta correctamente?

  \item Comerciante: {\hindifont हाँ, यह तेज़ी से तैयार किया गया था। इसलिए यह ठीक से बैठेगा।} - Sí, fue rápidamente preparado. Por eso se ajustará correctamente.

  \item Cliente: {\hindifont यह कितने का है?} - ¿De cuánto es?

  \item Comerciante: {\hindifont यह दो हज़ार रुपये का है। लेकिन आप जैसे ग्राहक के लिए छूट दे सकता हूँ।} - Ésta es de dos mil rupias. Pero para clientes como Ud. puedo dar descuento.

  \item Cliente: {\hindifont क्या आपके पास कम वाला कुर्ता भी है?} - ¿Tiene Ud. también una kurta que sea más económica?

  \item Comerciante: {\hindifont हाँ, यहाँ पर थोड़ा सस्ते वाले कुर्ते भी हैं। आप उन्हें देख सकते हैं।} - Sí, aquí también hay kurtas un poco más baratas. Puede Ud. verlas.

  \item Cliente: {\hindifont ठीक है। मैं देखता हूँ। फिर निर्णय लेता हूँ।} - Bien. Veré. Luego tomaré una decisión.

  \item Comerciante: {\hindifont अच्छा, ध्यान से देखें। आप जल्दी से निर्णय लेंगे।} - Bien, vea atentamente. Tomarás rápido la decisión.

\end{itemize}


\hrulefill

\section{Unidad 22: Vocabulario Temático: Salud, Ocio y Tecnología}


\textbf{Objetivo:} Adquirir vocabulario específico en temas de salud, ocio y tecnología.

\subsection{1. Vocabulario Temático: Salud}


En esta unidad aprenderemos vocabulario específico relacionado con el cuerpo humano, la salud, enfermedades y el sistema médico:

\subsubsection{1.1 Partes del Cuerpo Humano}


\begin{center}
\begin{tabularx}{\textwidth}{| X | X | X | X | X | X |}
\hline
\textbf{Español} & \textbf{Hindi} & \textbf{Transliteración} & \textbf{Uso} & \textbf{Ejemplo} & \textbf{Traducción} \ \hline
Cabeza & {\hindifont सिर} & sir & Parte superior del cuerpo & {\hindifont मेरा सिर दर्द कर रहा है।} & Mi cabeza duele. \ \hline
Ojo & {\hindifont आँख} & aankh & Órgano de la visión & {\hindifont उसकी आँखें छोटी हैं।} & Sus ojos son pequeños. \ \hline
Nariz & {\hindifont नाक} & naak & Órgano respiratorio & {\hindifont उसकी नाक लाल है।} & Su nariz está roja. \ \hline
Boca & {\hindifont मुँह} & muh & Órgano de la alimentación y habla & {\hindifont उसका मुँह बहुत बड़ा है।} & Su boca es muy grande. \ \hline
Labios & {\hindifont होंठ} & honth & Parte exterior de la boca & {\hindifont उसके होंठ सूखे हैं।} & Sus labios están secos. \ \hline
Lengua & {\hindifont जीभ} & jiibh & Órgano para hablar y saborear & {\hindifont जीभ स्वाद लेती है।} & La lengua saborea. \ \hline
Dientes & {\hindifont दाँत} & daant & Órganos para masticar & {\hindifont उसके दाँत सफेद हैं।} & Sus dientes son blancos. \ \hline
Cuello & {\hindifont गला} & galaa & Parte que une cabeza con torso & {\hindifont उसका गला सूखा है।} & Su garganta está seca. \ \hline
Hombro & {\hindifont कंधा} & kandhaa & Parte superior del torso & {\hindifont कंधे पर बोझ है।} & Hay carga en el hombro. \ \hline
Brazo & {\hindifont बाँह} & baanh & Extremidad superior & {\hindifont उसकी बाँह मजबूत है।} & Su brazo es fuerte. \ \hline
Mano & {\hindifont हाथ} & haath & Extremidad superior & {\hindifont हाथ धो लो।} & Lavaos las manos. \ \hline
Palma & {\hindifont हथेली} & hatheli & Parte interior de la mano & {\hindifont हथेली पर लिखो।} & Escribid en la palma. \ \hline
Dedo & {\hindifont उँगली} & ungalii & Parte de la mano & {\hindifont उसकी उँगली काटी गई।} & Su dedo fue cortado. \ \hline
Corazón & {\hindifont दिल} & dil & Órgano cardiovascular & {\hindifont दिल धड़क रहा है।} & El corazón late. \ \hline
Pecho & {\hindifont छाती} & chaatii & Parte superior del torso & {\hindifont छाती दर्द हो रहा है।} & El pecho duele. \ \hline
Estómago & {\hindifont पेट} & pet & Órgano digestivo & {\hindifont पेट खाली है।} & El estómago está vacío. \ \hline
Cintura & {\hindifont कमर} & kamar & Zona entre pecho y abdomen & {\hindifont कमर दर्द हो रहा है।} & Dolor de espalda/cintura. \ \hline
Pierna & {\hindifont पैर} & pair & Extremidad inferior & {\hindifont पैर छुट्टी है।} & El pie está libre. \ \hline
Rodilla & {\hindifont घुटना} & ghutnaa & Articulación de la pierna & {\hindifont घुटना का दर्द हो रहा है।} & Dolor de rodilla. \ \hline
Tobillo & {\hindifont टखना} & tkhanaa & Parte inferior de la pierna & {\hindifont टखना में चोट है।} & Hay herida en el tobillo. \ \hline
Pies & {\hindifont पैर} & pair & Extremidad inferior (pl.) & {\hindifont पैर धो लो।} & Lavaos los pies. \ \hline
\end{tabularx}
\end{center}

\subsubsection{1.2 Vocabulario Médico y de Salud}


\begin{center}
\begin{tabularx}{\textwidth}{| X | X | X | X | X | X |}
\hline
\textbf{Español} & \textbf{Hindi} & \textbf{Transliteración} & \textbf{Uso} & \textbf{Ejemplo} & \textbf{Traducción} \ \hline
Doctor/Médico & {\hindifont डॉक्टर} & doctor & Profesional médico & {\hindifont डॉक्टर से मिलो।} & Id a ver al médico. \ \hline
Enfermera & {\hindifont नर्स} & nars & Profesional de enfermería & {\hindifont नर्स मरीज़ की देखभाल करती है।} & La enfermera cuida al paciente. \ \hline
Paciente & {\hindifont रोगी} & rogii & Persona que recibe tratamiento & {\hindifont रोगी को दवा चाहिए।} & El paciente necesita medicina. \ \hline
Hospital & {\hindifont अस्पताल} & aspataal & Institución médica & {\hindifont रोगी अस्पताल में है।} & El paciente está en el hospital. \ \hline
Clinica & {\hindifont क्लीनिक} & clinic & Centro médico pequeño & {\hindifont वह क्लीनिक में काम करता है।} & Él trabaja en la clínica. \ \hline
Consultorio & {\hindifont क्लीनिक} o {\hindifont डॉक्टर की दुकान} & clinic & Oficina médica & {\hindifont डॉक्टर के क्लीनिक पर जाओ।} & Id a la consulta médica del doctor. \ \hline
Medicina & {\hindifont दवा} & davaa & Remedio farmacológico & {\hindifont मुझे दवा लेनी है।} & Yo tengo que tomar medicina. \ \hline
Receta & {\hindifont नुस्खा} & nuskhaa & Documento médico & {\hindifont डॉक्टर का नुस्खा पढ़ो।} & Leed la receta médica. \ \hline
Consulta & {\hindifont जांच} & jaanch & Revisión médica & {\hindifont डॉक्टर ने जाँच की।} & El médico realizó el examen. \ \hline
Operación & {\hindifont सर्जरी} & sarjari & Procedimiento quirúrgico & {\hindifont उसकी सर्जरी हो चुकी है।} & Su operación ya se realizó. \ \hline
Enfermedad & {\hindifont बीमारी} & beemaarii & Condición médica & {\hindifont मुझे बीमारी है।} & Estoy enfermo. \ \hline
Dolor & {\hindifont दर्द} & dard & Sensación desagradable & {\hindifont सिर में दर्द है।} & Dolor de cabeza. \ \hline
Inyección & {\hindifont इंजेक्शन} & injekshan & Administración de medicamento & {\hindifont इंजेक्शन लगाना है।} & Necesita una inyección. \ \hline
Termómetro & {\hindifont थर्मामीटर} & thermaamiiter & Instrumento de medición & {\hindifont थर्मामीटर से तापमान मापो।} & Medid la temperatura con el termómetro. \ \hline
Fiebre & {\hindifont बुखार} & bukhaar & Síntoma de enfermedad & {\hindifont मुझे बुखार है।} & Tengo fiebre. \ \hline
Resfriado & {\hindifont सर्दी} & sardi & Enfermedad común & {\hindifont उसे सर्दी है।} & Él tiene resfriado. \ \hline
Tos & {\hindifont खांसी} & khaansii & Síntoma respiratorio & {\hindifont तुम्हें खांसी हो रही है।} & Tú estás tosiendo. \ \hline
Estornudo & {\hindifont छींक} & cheenk & Reacción refleja & {\hindifont वह छींक रहा है।} & Él está estornudando. \ \hline
Vacuna & {\hindifont टीका} & teeka & Prevención médica & {\hindifont टीका लगाओ।} & Tomad la vacuna. \ \hline
Sanitario & {\hindifont स्वास्थ्यकर्मी} & svaasthyakarmii & Trabajador sanitario & {\hindifont स्वास्थ्यकर्मी मदद करता है।} & El trabajador sanitario ayuda. \ \hline
Emergencia & {\hindifont आपातकाल} & aapatkaal & Situación crítica & {\hindifont यह आपातकाल की स्थिति है।} & Ésta es una situación de emergencia. \ \hline
\end{tabularx}
\end{center}


\begin{tcolorbox}[colback=yellow!10!white,colframe=orange!75!black,title=Regla de Oro]

\paragraph{Frases Comunes en Contexto Médico} 

\begin{itemize}

  \item {\hindifont मैं बीमार हूँ।} - Estoy enfermo.

  \item {\hindifont मुझे डॉक्टर के पास जाना है।} - Tengo que ir al médico.

  \item {\hindifont मुझे बुखार है।} - Tengo fiebre.

  \item {\hindifont मुझे सर्दी है।} - Tengo resfriado.

  \item {\hindifont मुझे खांसी हो रही है।} - Estoy tosiendo.

  \item {\hindifont मुझे दर्द हो रहा है।} - Me duele.

  \item {\hindifont कृपया मेरी जाँच करें।} - Por favor, examíname.

  \item {\hindifont क्या दवा लेनी है?} - ¿Qué medicina debo tomar?

  \item {\hindifont कितने समय तक?} - ¿Por cuánto tiempo?

  \item {\hindifont डॉक्टर का क्या कहना है?} - ¿Qué dice el médico?

  \item {\hindifont मुझे बेहतर महसूस हो रहा है।} - Me siento mejor.

  \item {\hindifont क्या यह गंभीर है?} - ¿Es grave?

  \item {\hindifont मुझे आराम करना है।} - Tengo que descansar.

  \item {\hindifont कृपया तबीयत सुधारें।} - Por favor, mejora tu salud.

  \item {\hindifont यह नुस्खा डॉक्टर ने लिखा है।} - El médico escribió esta receta.

\end{itemize}


\end{tcolorbox}

\subsection{2. Vocabulario Temático: Ocio}


El vocabulario de ocio incluye actividades recreativas, lugares de entretenimiento y expresiones relacionadas con el tiempo libre:

\subsubsection{2.1 Actividades de Ocio y Entretenimiento}


\begin{center}
\begin{tabularx}{\textwidth}{| X | X | X | X | X | X |}
\hline
\textbf{Español} & \textbf{Hindi} & \textbf{Transliteración} & \textbf{Uso} & \textbf{Ejemplo} & \textbf{Traducción} \ \hline
Ocio & {\hindifont छुट्टी} & chhutti & Tiempo libre & {\hindifont हम छुट्टियों में घूमते हैं।} & Viajamos en vacaciones. \ \hline
Entretenimiento & {\hindifont मनोरंजन} & manoranjan & Actividad recreativa & {\hindifont यह फिल्म मनोरंजन है।} & Esta película es entretenimiento. \ \hline
Pasatiempo & {\hindifont शौक} & shaok & Actividad recreativa & {\hindifont मेरा शौक पढ़ना है।} & Mi hobby es leer. \ \hline
Deporte & {\hindifont खेल} & khel & Actividad física & {\hindifont तुम खेलना पसंद करते हो।} & A vosotros os gusta jugar. \ \hline
Fútbol & {\hindifont फुटबॉल} & futbol & Deporte popular & {\hindifont बच्चे फुटबॉल खेल रहे हैं।} & Los niños están jugando fútbol. \ \hline
Cricket & {\hindifont क्रिकेट} & kriket & Deporte nacional & {\hindifont भारत में क्रिकेट लोकप्रिय है।} & El cricket es popular en la India. \ \hline
Música & {\hindifont संगीत} & sangeet & Arte sonoro & {\hindifont मैं संगीत सुनना पसंद करता हूँ।} & Me gusta escuchar música. \ \hline
Canción & {\hindifont गाना} & gaanaa & Composición musical & {\hindifont वह गाना गा रहा है।} & Él está cantando una canción. \ \hline
Cantar & {\hindifont गाना} & gaanaa & Producir melodía vocal & {\hindifont तुम अच्छा गाते हो।} & Vosotros cantáis bien. \ \hline
Baile & {\hindifont नृत्य} & nrity & Arte del movimiento & {\hindifont वह नृत्य सीख रहा है।} & Él está aprendiendo a bailar. \ \hline
Teatro & {\hindifont नाटक} & naatak & Arte escénico & {\hindifont हम नाटक देखने गए।} & Fuimos a ver teatro. \ \hline
Cine & {\hindifont सिनेमा} & cinema & Sala cinematográfica & {\hindifont हम सिनेमा में फिल्म देखते हैं।} & Vemos películas en el cine. \ \hline
Película & {\hindifont फिल्म} & film & Obra cinematográfica & {\hindifont यह फिल्म बहुत अच्छी है।} & Esta película es muy buena. \ \hline
Televisión & {\hindifont टेलीविज़न} & televizan & Dispositivo electrónico & {\hindifont हम टेलीविज़न देखते हैं।} & Vemos televisión. \ \hline
Programa & {\hindifont कार्यक्रम} & kaaryakram & Emisión televisiva & {\hindifont यह कार्यक्रम दिलचस्प है।} & Este programa es interesante. \ \hline
Leyendo & {\hindifont पढ़ना} & padhnaa & Actividad intelectual & {\hindifont तुम पुस्तक पढ़ रहे हो।} & Vosotros estáis leyendo libros. \ \hline
Libro & {\hindifont किताब} & kitab & Medio de lectura & {\hindifont यह किताब महंगी है।} & Este libro es caro. \ \hline
Jugando & {\hindifont खेलना} & khelnaa & Actividad recreativa & {\hindifont बच्चे बाहर खेल रहे हैं।} & Los niños están jugando afuera. \ \hline
Paseando & {\hindifont घूमना} & ghoomnaa & Movimiento recreativo & {\hindifont हम पार्क में घूमते हैं।} & Paseamos en el parque. \ \hline
Viajando & {\hindifont यात्रा करना} & yaatraa karnaa & Movimiento a lugares & {\hindifont हम गर्मियों में यात्रा करते हैं।} & Viajamos en verano. \ \hline
\end{tabularx}
\end{center}

\subsubsection{2.2 Lugares de Ocio}


\begin{center}
\begin{tabularx}{\textwidth}{| X | X | X | X | X | X |}
\hline
\textbf{Español} & \textbf{Hindi} & \textbf{Transliteración} & \textbf{Uso} & \textbf{Ejemplo} & \textbf{Traducción} \ \hline
Parque & {\hindifont पार्क} & park & Lugar recreativo & {\hindifont हम पार्क में घूमते हैं।} & Paseamos en el parque. \ \hline
Cine & {\hindifont सिनेमा} & cinema & Sala de películas & {\hindifont सिनेमा में फिल्म देखो।} & Ver la película en el cine. \ \hline
Teatro & {\hindifont नाट्यशाला} & naatyashaalaa & Local teatral & {\hindifont नाट्यशाला में नाटक है।} & Hay teatro en la sala de teatro. \ \hline
Biblioteca & {\hindifont पुस्तकालय} & paustakaalay & Depósito de libros & {\hindifont पुस्तकालय में चुपचाप पढ़ो।} & Leed silenciosamente en la biblioteca. \ \hline
Playa & {\hindifont समुद्र तट} & samudra tatt & Litoral marino & {\hindifont हम समुद्र तट पर घूमते हैं।} & Paseamos por la playa. \ \hline
Montaña & {\hindifont पहाड़} & pahar & Rugosidad terrestre & {\hindifont हम पहाड़ पर यात्रा करते हैं।} & Hacemos viajes a las montañas. \ \hline
Hotel & {\hindifont होटल} & hotel & Alojamiento & {\hindifont हम होटल में ठहरते हैं।} & Nos alojamos en el hotel. \ \hline
Restaurante & {\hindifont रेस्तरां} & restaurant & Lugar comunal & {\hindifont हम रेस्तरां में खाते हैं।} & Comemos en el restaurante. \ \hline
Centro comercial & {\hindifont शॉपिंग मॉल} & shoping maal & Lugar de compra & {\hindifont हम शॉपिंग मॉल में खरीदते हैं।} & Compramos en el centro comercial. \ \hline
Parque de diversiones & {\hindifont मनोरंजन पार्क} & manoranjan park & Área recreativa & {\hindifont बच्चे मनोरंजन पार्क पर जाते हैं।} & Los niños van al parque de diversiones. \ \hline
Estadio & {\hindifont स्टेडियम} & stadiyam & Área de eventos & {\hindifont क्रिकेट मैच स्टेडियम में होता है।} & El partido de cricket es en el estadio. \ \hline
Zoológico & {\hindifont चिड़ियाघर} & chidiyaghar & Área de animales & {\hindifont हम चिड़ियाघर में जानवर देखते हैं।} & Vemos animales en el zoológico. \ \hline
Museo & {\hindifont संग्रहालय} & sangrahaalay & Área cultural & {\hindifont हम संग्रहालय में वस्तुएँ देखते हैं।} & Vemos objetos en el museo. \ \hline
Galería de arte & {\hindifont कला गृह} & kalaa griha & Área artística & {\hindifont हम कला गृह में चित्र देखते हैं।} & Vemos pinturas en la galería de arte. \ \hline
Templo & {\hindifont मंदिर} & mandir & Lugar religioso & {\hindifont लोग मंदिर में पूजा करते हैं।} & La gente reza en el templo. \ \hline
Cafetería & {\hindifont कॉफी शॉप} & coffee shop & Lugar de bebida & {\hindifont हम कॉफी शॉप में बैठक करते हैं।} & Hacemos reuniones en la cafetería. \ \hline
Banco & {\hindifont बैंक} & bank & Institución financiera & {\hindifont हम बैंक में पैसे जमा करते हैं।} & Depositamos dinero en el banco. \ \hline
Parque temático & {\hindifont थीम पार्क} & theme park & Área recreativa temática & {\hindifont हम थीम पार्क में मज़ा करते हैं।} & Divertimos en el parque temático. \ \hline
Casa de té & {\hindifont चाय बिक्री स्थान} & chaa bikrii sthaan & Lugar de té & {\hindifont हम चाय बिक्री स्थान पर चाय पीते हैं।} & Tomamos té en la casa de té. \ \hline
Piscina & {\hindifont तालाब} & taalaab & Área acuática & {\hindifont हम तालाब में तैरते हैं।} & Nadamos en la piscina. \ \hline
\end{tabularx}
\end{center}


\begin{tcolorbox}[colback=blue!5!white,colframe=blue!75!black,title=Nota/Clarificación]

\paragraph{Expresiones Comunes para Hablar de Ocio} 

\begin{itemize}

  \item {\hindifont मैं फिल्म देखना पसंद करता हूँ।} - Me gusta ver películas.

  \item {\hindifont हम बाहर घूमते हैं।} - Salimos a pasear.

  \item {\hindifont तुम पुस्तकें पढ़ना पसंद करते हो।} - A vosotros os gusta leer libros.

  \item {\hindifont वह संगीत सुनना पसंद करता है।} - A él le gusta escuchar música.

  \item {\hindifont क्या तुम खेलना पसंद करते हो?} - ¿Os gusta jugar?

  \item {\hindifont हम पार्क में जल्दी जाते हैं।} - Vamos temprano al parque.

  \item {\hindifont तुम समुद्र तट पर जाना पसंद करते हो।} - A vosotros os gusta ir a la playa.

  \item {\hindifont मैं घूमना पसंद करता हूँ।} - Me gusta viajar.

  \item {\hindifont हम रोज़ पढ़ते हैं।} - Leemos todos los días.

  \item {\hindifont वह नाटक देखना पसंद करता है।} - A él le gusta ver teatro.

  \item {\hindifont तुम संगीत सुनना पसंद करते हो।} - A vosotros os gusta escuchar música.

  \item {\hindifont मैं टेलीविज़न देखना पसंद करता हूँ।} - Me gusta ver televisión.

  \item {\hindifont हम खाना बनाना पसंद करते हैं।} - Nos gusta cocinar.

  \item {\hindifont तुम बाहर जाना पसंद करते हो।} - A vosotros os gusta salir.

  \item {\hindifont वह तस्वीर खींचना पसंद करता है।} - A él le gusta tomar fotos.

\end{itemize}


\end{tcolorbox}

\subsection{3. Vocabulario Temático: Tecnología}


El vocabulario técnico y digital es cada vez más importante en el mundo actual:

\subsubsection{3.1 Dispositivos Electrónicos}


\begin{center}
\begin{tabularx}{\textwidth}{| X | X | X | X | X | X |}
\hline
\textbf{Español} & \textbf{Hindi} & \textbf{Transliteración} & \textbf{Uso} & \textbf{Ejemplo} & \textbf{Traducción} \ \hline
Teléfono móvil & {\hindifont मोबाइल} & mobile & Comunicación portátil & {\hindifont मेरा मोबाइल नया है।} & Mi móvil es nuevo. \ \hline
Ordenador & {\hindifont कंप्यूटर} & kampyooter & Herramienta de trabajo & {\hindifont कंप्यूटर पर काम करता हूँ।} & Trabajo en el ordenador. \ \hline
Portátil & {\hindifont लैपटॉप} & laptop & Ordenador móvil & {\hindifont लैपटॉप लाओ।} & Traed el portátil. \ \hline
Tableta & {\hindifont टैबलेट} & tablet & Dispositivo táctil & {\hindifont बच्चे टैबलेट पर पढ़ते हैं।} & Los niños leen en la tableta. \ \hline
Smartphone & {\hindifont स्मार्ट फ़ोन} & smart phoon & Teléfono avanzado & {\hindifont स्मार्ट फ़ोन से सब कुछ हो सकता है।} & Con un smartphone puede hacerse todo. \ \hline
Cámara & {\hindifont कैमरा} & kameraa & Dispositivo de imágenes & {\hindifont कैमरे से तस्वीर लो।} & Tomad la foto con la cámara. \ \hline
Impresora & {\hindifont प्रिंटर} & printer & Dispositivo de impresión & {\hindifont प्रिंटर से दस्तावेज़ छापो।} & Imprimid documentos con la impresora. \ \hline
Altavoz & {\hindifont स्पीकर} & speaker & Dispositivo de sonido & {\hindifont स्पीकर से संगीत चलाओ।} & Reproducid música con el altavoz. \ \hline
Auriculares & {\hindifont हेडफोन} & headphon & Dispositivo auditivo & {\hindifont हेडफोन लगाओ।} & Poneos los auriculares. \ \hline
Pantalla & {\hindifont स्क्रीन} & screen & Superficie de visualización & {\hindifont स्क्रीन साफ़ करो।} & Limpiad la pantalla. \ \hline
Teclado & {\hindifont कीबोर्ड} & keyboard & Dispositivo de entrada & {\hindifont कीबोर्ड पर टाइप करो।} & Tipead en el teclado. \ \hline
Ratón & {\hindifont माउस} & mouse & Dispositivo de puntero & {\hindifont माउस क्लिक करो।} & Haced clic con el ratón. \ \hline
Monitor & {\hindifont मॉनिटर} & monitor & Pantalla de computadora & {\hindifont मॉनिटर पर ध्यान दो।} & Prestad atención al monitor. \ \hline
Router & {\hindifont राउटर} & router & Dispositivo de red & {\hindifont राउटर से इंटरनेट चलता है।} & El internet funciona con el router. \ \hline
Disco duro & {\hindifont हार्ड डिस्क} & hard disk & Almacenamiento de datos & {\hindifont हार्ड डिस्क में फाइलें सुरक्षित रहती हैं।} & Los archivos se mantienen seguros en el disco duro. \ \hline
USB & {\hindifont यूएसबी} & USB & Dispositivo de almacenamiento & {\hindifont डेटा यूएसबी में सुरक्षित करो।} & Guardad los datos en el USB. \ \hline
Memoria & {\hindifont मेमोरी कार्ड} & memory card & Almacenamiento extraíble & {\hindifont तस्वीरें मेमोरी कार्ड में हैं।} & Las fotos están en la tarjeta de memoria. \ \hline
Reloj inteligente & {\hindifont स्मार्ट घड़ी} & smart ghadii & Dispositivo tecnológico & {\hindifont स्मार्ट घड़ी से संगीत सुनो।} & Escuchad música con el reloj inteligente. \ \hline
Altavoz inteligente & {\hindifont स्मार्ट स्पीकर} & smart speaker & Dispositivo de voz & {\hindifont स्मार्ट स्पीकर से गाना बजाओ।} & Reproducid canciones con el altavoz inteligente. \ \hline
Altavoz Bluetooth & {\hindifont ब्लूटूथ स्पीकर} & Bluetooth speaker & Dispositivo inalámbrico & {\hindifont ब्लूटूथ स्पीकर से संगीत सुनो।} & Escuchad música con el altavoz Bluetooth. \ \hline
\end{tabularx}
\end{center}

\subsubsection{3.2 Internet y Comunicación Digital}


\begin{center}
\begin{tabularx}{\textwidth}{| X | X | X | X | X | X |}
\hline
\textbf{Español} & \textbf{Hindi} & \textbf{Transliteración} & \textbf{Uso} & \textbf{Ejemplo} & \textbf{Traducción} \ \hline
Internet & {\hindifont इंटरनेट} & internet & Red global & {\hindifont इंटरनेट पर जानकारी मिलती है।} & La información se encuentra en internet. \ \hline
Buscar & {\hindifont खोजना} & khonjnaa & Acción de encontrar & {\hindifont इंटरनेट पर कुछ खोजो।} & Buscad algo en internet. \ \hline
Correo electrónico & {\hindifont ईमेल} & email & Comunicación digital & {\hindifont ईमेल भेजो।} & Enviad correo electrónico. \ \hline
Red social & {\hindifont सोशल मीडिया} & soshal miidiyaa & Plataforma de interacción & {\hindifont सोशल मीडिया पर पोस्ट करो।} & Publicad en redes sociales. \ \hline
Aplicación & {\hindifont एप्लिकेशन} & application & Programa digital & {\hindifont यह एप्लिकेशन उपयोगी है।} & Esta aplicación es útil. \ \hline
Descargar & {\hindifont डाउनलोड करना} & download karnaa & Obtener archivos & {\hindifont फ़ाइल डाउनलोड करो।} & Descargad archivos. \ \hline
Subir & {\hindifont अपलोड करना} & upload karnaa & Enviar archivos & {\hindifont तस्वीर अपलोड करो।} & Subid fotos. \ \hline
Actualizar & {\hindifont अपडेट करना} & update karnaa & Mejorar sistema & {\hindifont सॉफ्टवेयर अपडेट करो।} & Actualizad el software. \ \hline
Conectarse & {\hindifont कनेक्ट होना} & connect honaa & Establecer conexión & {\hindifont वाई-फाई से कनेक्ट हो जाओ।} & Conectaos al wifi. \ \hline
Conexión & {\hindifont कनेक्शन} & connection & Estado de conexión & {\hindifont कनेक्शन अच्छा है।} & La conexión es buena. \ \hline
Wi-Fi & {\hindifont वाई-फाई} & wifi & Conexión inalámbrica & {\hindifont वाई-फाई उपलब्ध है।} & Wi-Fi está disponible. \ \hline
Contraseña & {\hindifont पासवर्ड} & password & Clave de acceso & {\hindifont पासवर्ड बदलो।} & Cambiad la contraseña. \ \hline
Usuario & {\hindifont यूज़र} & user & Identificador & {\hindifont यूज़र नाम दर्ज करो।} & Introducid el nombre de usuario. \ \hline
Inicio de sesión & {\hindifont लॉगिन} & login & Acceder al sistema & {\hindifont लॉगिन करो।} & Iniciad sesión. \ \hline
Cerrar sesión & {\hindifont लॉगआउट} & logout & Salir del sistema & {\hindifont लॉगआउट करो।} & Cerrad sesión. \ \hline
Comentario & {\hindifont टिप्पणी} & tippaNi & Opinión en línea & {\hindifont वीडियो पर टिप्पणी करो।} & Comentad en el video. \ \hline
Me gusta & {\hindifont लाइक करना} & like karnaa & Mostrar agrado & {\hindifont पोस्ट को लाइक करो।} & Dadle me gusta a la publicación. \ \hline
Compartir & {\hindifont शेयर करना} & share karnaa & Distribuir contenido & {\hindifont लिंक शेयर करो।} & Compartid el enlace. \ \hline
Grabar & {\hindifont रिकॉर्ड करना} & record karnaa & Registrar audio/video & {\hindifont वीडियो रिकॉर्ड करो।} & Grabad el video. \ \hline
Transmitir & {\hindifont लाइव करना} & live karnaa & Transmisión en directo & {\hindifont लाइव टीवी देखो।} & Ved televisión en directo. \ \hline
Chat & {\hindifont चैट करना} & chat karnaa & Conversación escrita & {\hindifont मित्र के साथ चैट करो।} & Chatead con el amigo. \ \hline
\end{tabularx}
\end{center}

\subsubsection{3.3 Frases Comunes sobre Tecnología}


\begin{itemize}

  \item {\hindifont मेरा मोबाइल फ़ोन बंद हो गया।} - Mi teléfono móvil se apagó.

  \item {\hindifont कंप्यूटर बहुत धीमा है।} - El ordenador es muy lento.

  \item {\hindifont इंटरनेट काम नहीं कर रहा है।} - Internet no está funcionando.

  \item {\hindifont वाई-फाई पासवर्ड क्या है?} - ¿Cuál es la contraseña del wifi?

  \item {\hindifont ईमेल भेजना आसान है।} - Enviar correos electrónicos es fácil.

  \item {\hindifont मैंने एप्लिकेशन डाउनलोड की।} - Yo descargué la aplicación.

  \item {\hindifont सोशल मीडिया पर बहुत समय बिताता हूँ।} - Paso mucho tiempo en redes sociales.

  \item {\hindifont हम वीडियो कॉल करते हैं।} - Hacemos videollamadas.

  \item {\hindifont लैपटॉप की बैटरी समाप्त हो रही है।} - La batería del portátil se está agotando.

  \item {\hindifont कैमरा से तस्वीरें लेना अच्छा है।} - Tomar fotos con la cámara es bueno.

  \item {\hindifont क्या तुम मेरा नंबर सेव कर सकते हो?} - ¿Podéis guardar mi número?

  \item {\hindifont मैं ऑफ़लाइन हूँ।} - Estoy desconectado.

  \item {\hindifont वह ऑनलाइन है।} - Él está conectado.

  \item {\hindifont मैं अपना पासवर्ड भूल गया।} - Olvidé mi contraseña.

  \item {\hindifont लेकिन इंटरनेट स्पीड बहुत कम है।} - Pero la velocidad de internet es muy baja.

  \item {\hindifont हमें नया सॉफ़्टवेयर अपडेट करना चाहिए।} - Debemos actualizar el software nuevo.

  \item {\hindifont गाना यूट्यूब पर खोजो।} - Buscad la canción en YouTube.

  \item {\hindifont वीडियो डाउनलोड करना बंद करो।} - Dejad de descargar vídeos.

  \item {\hindifont इस एप्लिकेशन में बहुत अच्छी सुविधाएँ हैं।} - Esta aplicación tiene muchas características buenas.

  \item {\hindifont तुम स्मार्ट फ़ोन के बिना नहीं रह सकते।} - No podéis vivir sin teléfono inteligente.

\end{itemize}

\subsection{4. Números Mayores de 100}


Los números mayores de 100 son esenciales para expresar cantidades, precios, años, etc.:

\subsubsection{4.1 Números del 100 al 1000}


\begin{center}
\begin{tabularx}{\textwidth}{| X | X | X | X | X | X |}
\hline
\textbf{Número} & \textbf{Hindi} & \textbf{Transliteración} & \textbf{Formación} & \textbf{Ejemplo} & \textbf{Traducción} \ \hline
100 & {\hindifont एक सौ} & ek sau & एक + सौ & {\hindifont एक सौ रुपये।} & Cien rupias. \ \hline
200 & {\hindifont दो सौ} & do sau & दो + सौ & {\hindifont दो सौ विद्यार्थी।} & Doscientos estudiantes. \ \hline
300 & {\hindifont तीन सौ} & teen sau & तीन + सौ & {\hindifont तीन सौ किताबें।} & Trescientas libros. \ \hline
400 & {\hindifont चार सौ} & chaar sau & चार + सौ & {\hindifont चार सौ गाने।} & Cuatrocientas canciones. \ \hline
500 & {\hindifont पाँच सौ} & paanch sau & पाँच + सौ & {\hindifont पाँच सौ किलोमीटर।} & Quinientas kilómetros. \ \hline
600 & {\hindifont छः सौ} & chhah sau & छः + सौ & {\hindifont छः सौ लोग।} & Seiscientas personas. \ \hline
700 & {\hindifont सात सौ} & saat sau & सात + सौ & {\hindifont सात सौ घर।} & Setecientas casas. \ \hline
800 & {\hindifont आठ सौ} & aath sau & आठ + सौ & {\hindifont आठ सौ पेड़।} & Ochocientas árboles. \ \hline
900 & {\hindifont नौ सौ} & naum sau & नौ + सौ & {\hindifont नौ सौ सवाल।} & Novecientas preguntas. \ \hline
1000 & {\hindifont एक हज़ार} & ek hazaar & एक + हज़ार & {\hindifont एक हज़ार रुपये।} & Mil rupias. \ \hline
\end{tabularx}
\end{center}

\subsubsection{4.2 Números Compuestos (101-999)}


Para formar números entre 101-999, combinamos las centenas con decenas y unidades:


\begin{itemize}

  \item 101: {\hindifont एक सौ एक} (ek sau ek) - Ciento uno

  \item 125: {\hindifont एक सौ पच्चीस} (ek sau pachhiis) - Ciento veinticinco

  \item 250: {\hindifont दो सौ पचास} (do sau pachaas) - Doscientos cincuenta

  \item 367: {\hindifont तीन सौ सरसठ} (teen sau sarsath) - Trescientos sesenta y siete

  \item 489: {\hindifont चार सौ नवासी} (chaar sau navaasii) - Cuatrocientos ochenta y nueve

  \item 543: {\hindifont पाँच सौ तैंतालीस} (paanch sau taintaalis) - Quinientas cuarenta y tres

  \item 692: {\hindifont छः सौ बानवे} (chhah sau baannave) - Seiscientas noventa y dos

  \item 734: {\hindifont सात सौ चौंतीस} (saat sau chauntaalis) - Setecientas treinta y cuatro

  \item 876: {\hindifont आठ सौ छिहत्तर} (aath sau chihattar) - Ochocientas setenta y seis

  \item 999: {\hindifont नौ सौ निन्यानवे} (naum sau ninyaannave) - Novecientas noventa y nueve

\end{itemize}

\subsubsection{4.3 Números Mayores (1000-10,00,000)}


\begin{center}
\begin{tabularx}{\textwidth}{| X | X | X | X | X |}
\hline
\textbf{Número} & \textbf{Hindi} & \textbf{Transliteración} & \textbf{Traducción} & \textbf{Uso} \ \hline
1,000 & {\hindifont एक हज़ार} & ek hazaar & Mil & Unidades monetarias, población \ \hline
2,500 & {\hindifont ढाई हज़ार} & dhaaee hazaar & 2,500 & 2.5 mil \ \hline
5,000 & {\hindifont पाँच हज़ार} & paanch hazaar & Cinco mil & Unidades monetarias \ \hline
10,000 & {\hindifont दस हज़ार} & das hazaar & Diez mil & Gran cantidad \ \hline
50,000 & {\hindifont पचास हज़ार} & pachaas hazaar & Cincuenta mil & Muchas unidades \ \hline
1,00,000 (1 lakh) & {\hindifont एक लाख} & ek laakh & Cien mil & Sistema numérico indio \ \hline
2,50,000 (2.5 lakh) & {\hindifont ढाई लाख} & dhaaee laakh & 250,000 & 2.5 lakh \ \hline
10,00,000 (10 lakh) & {\hindifont दस लाख} & das laakh & Un millón & Millones \ \hline
1,00,00,000 (1 crore) & {\hindifont एक करोड़} & ek karod & Diez millones & Grandes cantidades \ \hline
5,00,00,000 & {\hindifont पाँच करोड़} & paanch karod & Cincuenta millones & Muchas unidades \ \hline
\end{tabularx}
\end{center}


\begin{tcolorbox}[colback=yellow!10!white,colframe=orange!75!black,title=Regla de Oro]

\paragraph{Sistema Numérico Indio} 

El sistema numérico en hindi utiliza un sistema diferente al occidental:


\begin{itemize}

  \item 1 = {\hindifont एक} (ek)

  \item 10 = {\hindifont दस} (das)

  \item 100 = {\hindifont सौ} (sau)

  \item 1,000 = {\hindifont हज़ार} (hazaar)

  \item 1,00,000 = {\hindifont लाख} (laakh) = Cien mil

  \item 1,00,00,000 = {\hindifont करोड़} (karod) = Diez millones

\end{itemize}


Las separaciones numéricas también son diferentes en hindi: 10,00,000 (diez millones) en lugar de 10,000,000


\end{tcolorbox}

\subsection{5. Comunicación: Describir Hábitos con Mayor Precisión y Seleccionar Objetos}


Con todos los elementos gramaticales aprendidos, podemos expresar rutinas, preferencias y selecciones con mayor precisión:

\subsubsection{5.1 Frases para Describir Hábitos con Precisión}


\begin{itemize}

  \item {\hindifont हर रोज़ सुबह मैं ध्यान से जागता हूँ।} - Todos los días por la mañana me despierto con atención.

  \item {\hindifont कभी-कभी शाम में मैं संगीत सुनता हूँ।} - A veces por la tarde escucho música.

  \item {\hindifont रोज़ तुम खेलने के बाद पढ़ाई करते हो।} - Todos los días después de jugar estudiais.

  \item {\hindifont मैं अक्सर अपने दोस्तों के साथ बाहर जाता हूँ।} - A menudo salgo con mis amigos.

  \item {\hindifont हम लोग छुट्टियों में घूमने जाते हैं।} - Nosotros vamos de viaje en vacaciones.

  \item {\hindifont जब मैं छुट्टियों पर था, मैंने बहुत घूमा।} - Cuando yo estaba de vacaciones, viajé mucho.

  \item {\hindifont तुम हमेशा बज़ार में सब्ज़ियाँ खरीदते हो।} - Vosotros siempre compráis verduras en el mercado.

  \item {\hindifont वह लड़की कभी नहीं देर से आती है।} - Esa chica nunca llega tarde.

  \item {\hindifont रात को वह तेज़ी से सो जाता है।} - Por la noche él se duerme rápidamente.

  \item {\hindifont हम नियमित रूप से खेलते हैं।} - Jugamos regularmente.

\end{itemize}

\subsubsection{5.2 Expresiones para Seleccionar Objetos}


\begin{itemize}

  \item {\hindifont मुझे लाल वाली किताब चाहिए।} - Necesito el libro rojo. (Literally: Book with red I want)

  \item {\hindifont तुम उस पेड़ वाले स्थान पर जाओ।} - Id al lugar con ese árbol.

  \item {\hindifont वह लड़की जो खेल रही है, यह मेरी बहन है।} - La chica que está jugando es mi hermana.

  \item {\hindifont तुम उस डॉक्टर के पास जाओ जो अच्छा है।} - Id donde el doctor que es bueno.

  \item {\hindifont मैंने वह किताब खरीदी जो महंगी नहीं थी।} - Compré ese libro que no era caro.

  \item {\hindifont यह वह गाड़ी है जिसके पास बहुत शक्ति है।} - Éste es el coche que tiene mucha potencia.

  \item {\hindifont तुम वहाँ से कौन सा खाना चुन रहे हो?} - ¿Cuál comida estáis eligiendo de allí?

  \item {\hindifont मैं वह फिल्म देखना पसंद करता हूँ जो दिलचस्प है।} - Me gusta ver la película que es interesante.

  \item {\hindifont इन दोस्तों में से जो अधिक पढ़ाई करता है, वह सफल होगा।} - De entre estos amigos, quien estudia más tendrá éxito.

  \item {\hindifont तुम उस तरह का काम कर रहे हो जो तुम्हारा पसंदीदा है।} - Estás haciendo el tipo de trabajo que es tu favorito.

\end{itemize}

\subsubsection{5.3 Diálogos Integradores}


\textbf{Diálogo 1 - Compra de un móvil}


\begin{itemize}

  \item Cliente: {\hindifont क्या आपके पास नए मोबाइल हैं?} - ¿Tiene Ud. móviles nuevos?

  \item Comerciante: {\hindifont हाँ, हमारे पास बहुत नए मोबाइल हैं। आप कौन सा चाहते हैं?} - Sí, tenemos muchos móviles nuevos. ¿Cuál quiere Ud.?

  \item Cliente: {\hindifont मुझे वह मोबाइल चाहिए जो कैमरा और इंटरनेट वाला हो।} - Necesito el móvil que tenga cámara e internet.

  \item Comerciante: {\hindifont ठीक है, यह मॉडल कैमरा और इंटरनेट दोनों वाला है।} - Bien, este modelo tiene cámara e internet ambos.

  \item Cliente: {\hindifont क्या इसकी बैटरी ज्यादा टिकती है?} - ¿Esta batería dura mucho?

  \item Comerciante: {\hindifont हाँ, इस मोबाइल की बैटरी दिन भर टिकती है।} - Sí, la batería de este móvil dura todo el día.

  \item Cliente: {\hindifont क्या इस पर आप ऑनलाइन खरीद सकते हैं?} - ¿Puedes comprar en línea en éste?

  \item Comerciante: {\hindifont हाँ, आप इसके साथ ऑनलाइन खरीद सकते हैं। और आप सोशल मीडिया भी उपयोग कर सकते हैं।} - Sí, puedes comprar en línea con éste. Y puedes usar también redes sociales.

  \item Cliente: {\hindifont यह रंग आपके पसंद के अनुसार है। लेकिन क्या यह महंगा है?} - Este color es según tu gusto. Pero ¿es caro?

  \item Comerciante: {\hindifont यह इस कीमत का सामान है जो अच्छा है। और हम आपके लिए कुछ छूट दे सकते हैं।} - Éste es un producto de este precio que es bueno. Y podemos darte un poco de descuento para ti.

  \item Cliente: {\hindifont ठीक है, मैं यही लेना चाहता हूँ। क्या आप इसकी गारंटी भी दे सकते हैं?} - Bien, quiero tomar éste. ¿Puedes también darme garantía de éste?

  \item Comerciante: {\hindifont हाँ, हम एक साल की गारंटी दे रहे हैं। और हम इसका सॉफ्टवेयर अपडेट भी करते हैं।} - Sí, estamos dando garantía de un año. Y también actualizamos el software de éste.

\end{itemize}


\textbf{Diálogo 2 - Visita médica}


\begin{itemize}

  \item Paciente: {\hindifont नमस्कार डॉक्टर जी, मुझे बुखार है।} - Namaste doctor ji, tengo fiebre.

  \item Doctor: {\hindifont नमस्कार! कितने दिन से बुखार है?} - ¡Namaste! ¿Desde cuántos días tienes fiebre?

  \item Paciente: {\hindifont तीन दिन से बुखार है। और मुझे सरदी भी हो रही है।} - Desde tres días tengo fiebre. Y también me está dando resfriado.

  \item Doctor: {\hindifont क्या आपको गला खराब भी है?} - ¿También tienes dolor de garganta?

  \item Paciente: {\hindifont हाँ, मेरा गला खराब है और मैं खांस भी रहा हूँ।} - Sí, mi garganta está mal y también estoy tosiendo.

  \item Doctor: {\hindifont ठीक है, मैं आपकी जाँच करता हूँ। आज आप घर पर रहें।} - Bien, voy a examinarte. Hoy quédate en casa.

  \item Paciente: {\hindifont क्या मैं काम पर जा सकता हूँ?} - ¿Puedo ir a trabajar?

  \item Doctor: {\hindifont नहीं, आपको आराम करना चाहिए। और आपको यह दवा लेनी चाहिए।} - No, debes descansar. Y debes tomar esta medicina.

  \item Paciente: {\hindifont कितने समय तक दवा लेनी है?} - ¿Por cuánto tiempo debo tomar la medicina?

  \item Doctor: {\hindifont तीन दिन तक सुबह और शाम को। और आपको ज्यादा पानी पीना चाहिए।} - Tres días, por la mañana y por la noche. Y debes beber más agua.

  \item Paciente: {\hindifont क्या मैं कल फिर आ सकता हूँ अगर मुझे अच्छा नहीं लगा?} - ¿Puedo venir de nuevo mañana si no me siento mejor?

  \item Doctor: {\hindifont हाँ, अगर स्थिति में सुधार नहीं होता तो आप फिर से आ सकते हैं।} - Sí, si no hay mejora en la situación, puedes venir de nuevo.

\end{itemize}

\subsection{6. Componente Cultural: Festividades (Diwali, Holi)}


Para completar esta unidad, exploremos dos de las festividades más importantes de la India:

\subsubsection{6.1 Diwali (Festival de las Luces)}


Diwali, también conocido como Deepavali, es uno de los festivales más importantes del hinduismo. Se celebra durante cinco días y representa la victoria del bien sobre el mal, de la luz sobre la oscuridad y del conocimiento sobre la ignorancia.


\begin{itemize}

  \item {\hindifont दीपावली भारत में सबसे प्रसिद्ध त्योहार में से एक है।} - Diwali es uno de los festivales más famosos en la India.

  \item {\hindifont लोग अपने घरों को दीयों और रंगोली से सजाते हैं।} - La gente decora sus casas con lámparas de aceite y rangolis.

  \item {\hindifont व्यापारी लोग अपने नए लेखापोत्र खोलते हैं।} - La gente de negocio abre sus nuevos libros de contabilidad.

  \item {\hindifont लोग एक दूसरे को मिठाइयाँ और उपहार देते हैं।} - La gente se da dulces y regalos mutuamente.

  \item {\hindifont लक्ष्मी जी की पूजा की जाती है।} - Se realiza la adoración a la diosa Lakshmi.

  \item {\hindifont बच्चे पटाखे जलाते हैं।} - Los niños queman fuegos artificiales.

  \item {\hindifont परिवार एक साथ खाना खाता है और खुशी मनाता है।} - Las familias comen juntas y celebran la alegría.

  \item {\hindifont यह अंधकार को दूर करने और ज्ञान के प्रकाश को अपनाने का त्योहार है।} - Éste es un festival para alejar la oscuridad y aceptar la luz del conocimiento.

  \item {\hindifont लोग पूजा करते हैं ताकि वे समृद्धि प्राप्त कर सकें।} - La gente reza para que puedan obtener prosperidad.

  \item {\hindifont दीपावली एक साथ मिलकर जश्न मनाने का समय है।} - Diwali es un tiempo para celebrar juntos.

\end{itemize}

\subsubsection{6.2 Holi (Festival de los Colores)}


Holi es una celebración primaveral que simboliza la victoria del bien sobre el mal y celebra la llegada de la primavera. Es conocido como el "festival de los colores".


\begin{itemize}

  \item {\hindifont होली रंगों का त्योहार है।} - Holi es el festival de los colores.

  \item {\hindifont लोग एक दूसरे पर रंग और गुलाल फेंकते हैं।} - La gente se lanza colores y gulal mutuamente.

  \item {\hindifont होली के पहले दिन होलिका दहन होता है।} - El primer día de Holi es Holika Dahan (hoguera).

  \item {\hindifont लोग खुशी से नाचते और गाते हैं।} - La gente baila y canta con alegría.

  \item {\hindifont हम ठाकुरजी को भोग लगाते हैं।} - Ofrecemos alimentos sagrados a Thakurji.

  \item {\hindifont होली समाज में भाईचारा बनाए रखने में मदद करती है।} - Holi ayuda a mantener la camaradería en la sociedad.

  \item {\hindifont लोग थंडाई पीते हैं।} - La gente bebe thandai (bebida tradicional).

  \item {\hindifont यह त्योहार लोगों के बीच भेदभाव को दूर करता है।} - Este festival elimina la discriminación entre la gente.

  \item {\hindifont होली का मतलब है कि हम सभी खुश रहें।} - El significado de Holi es que todos nosotros seamos felices.

  \item {\hindifont लोग अलग-अलग जाति और समुदाय के होते हुए भी एक साथ मिलकर खेलते हैं।} - La gente de diferentes castas y comunidades juega junta.

\end{itemize}


\begin{tcolorbox}[colback=blue!5!white,colframe=blue!75!black,title=Nota/Clarificación]

\paragraph{Importancia Cultural de las Festividades} 

Las festividades como Diwali y Holi desempeñan un papel fundamental en la cultura india:


\begin{itemize}

  \item Fortalecen los vínculos familiares y comunitarios

  \item Preservan las tradiciones y valores culturales

  \item Proporcionan un descanso de las responsabilidades diarias

  \item Expresan alegría y unidad entre las personas

  \item Destacan principios espirituales y morales

  \item Ofrecen oportunidades de compartir con los demás

  \item Representan diferentes aspectos del calendario agrícola y natural

\end{itemize}


\end{tcolorbox}


\hrulefill

\section{Unidad 29: Consolidación A2}


\textbf{Objetivo:} Repasar todos los contenidos del nivel A2 y evaluar competencias.

\subsection{1. Introducción a la Consolidación A2}


Esta unidad sintetiza todos los contenidos aprendidos durante el nivel A2. Revisaremos las estructuras gramaticales, vocabulario y funciones comunicativas que has aprendido para fortalecer tu dominio del idioma hindi.


\begin{tcolorbox}[colback=yellow!10!white,colframe=orange!75!black,title=Regla de Oro]

\paragraph{Competencias del Nivel A2} 

Al finalizar el nivel A2, debes ser capaz de:


\begin{itemize}

  \item Comprender frases y expresiones de uso frecuente relacionadas con áreas de experiencia que tratan de necesidades inmediatas.

  \item Comunicarte en tareas simples y cotidianas que requieran un intercambio sencillo de información.

  \item Describir con palabras sencillas aspectos de tu pasado y tu entorno así como cosas relacionadas con tus necesidades inmediatas.

\end{itemize}


\end{tcolorbox}

\subsection{2. Repaso General de Contenidos}

\subsubsection{2.1 Tiempos Verbales}


\begin{center}
\begin{tabularx}{\textwidth}{| X | X | X | X | X |}
\hline
\textbf{Tiempo} & \textbf{Estructura} & \textbf{Uso} & \textbf{Ejemplo} & \textbf{Traducción} \ \hline
Presente Habitual & Sujeto + Raíz + ता/ती/ते + होना (concordando con sujeto) & Hábitos, rutinas, hechos generales & {\hindifont वह प्रतिदिन पढ़ाई करता है।} & Él estudia todos los días. \ \hline
Presente Continuo & Sujeto + रहा/रही/रहे + होना & Acción en progreso en el momento actual & {\hindifont वह पढ़ाई कर रहा है।} & Él está estudiando. \ \hline
Futuro Simple & Sujeto + Raíz + एगा/एगी/एंगे/एंगी & Acciones futuras planificadas & {\hindifont वह कल पढ़ाई करेगा।} & Él estudiará mañana. \ \hline
Pasado Simple (Intransitivo) & Verbo en pasado (concordando con sujeto) & Acciones completadas en el pasado & {\hindifont वह घर पर था।} & Él estaba en casa. \ \hline
Pasado Simple (Transitivo con ने) & Sujeto + ने + Objeto + Verbo (concordando con objeto) & Acciones sobre objeto en pasado & {\hindifont राम ने किताब पढ़ी।} & Ram leyó el libro. \ \hline
Pasado Imperfecto & Sujeto + Raíz + ता/ती/ते + था/थी/थे & Hábitos pasados o acciones en progreso & {\hindifont राम किताब पढ़ता था।} & Ram solía leer libros. (Hábito pasado) \ \hline
Imperativo & Raíz + o (formal/informal), ईए (formal), इए (formal) & Órdenes o instrucciones & {\hindifont तुम पढ़ो। / आप पढ़िए।} & Leed (vosotros). / Leed (ustedes). \ \hline
\end{tabularx}
\end{center}

\subsubsection{2.2 Sistema Ergativo}


Revisamos la construcción ergativa que se usa con verbos transitivos en pasado:


\begin{itemize}

  \item \textbf{Estructura:} Sujeto + ने + Objeto + Verbo (concordando con el objeto)

  \item \textbf{Ejemplo:} {\hindifont राम ने किताब पढ़ी।} (Ram leyó el libro.)

  \item \textbf{Concordancia:} El verbo concuerda con el objeto directo, no con el sujeto

  \item \textbf{Raíz:} El verbo no se conjuga directamente, sino que forma participios

\end{itemize}

\paragraph{Tabla de Concordancia en el Sistema Ergativo (Pasado Simple Transitivo)} 

\begin{center}
\begin{tabularx}{\textwidth}{| X | X | X | X | X | X |}
\hline
\textbf{Objeto Directo} & \textbf{Masculino Singular} & \textbf{Femenino Singular} & \textbf{Masculino Plural} & \textbf{Femenino Plural} & \textbf{Ejemplo} \ \hline
Singular Masc. & {\hindifont -आ} (aa) & - & - & - & {\hindifont राम ने सेब खाया।} (comió un manzano) \ \hline
Singular Fem. & {\hindifont -ई} (ii) & {\hindifont -ई} (ii) & - & - & {\hindifont राम ने रोटी खाई।} (comió pan - fem.) \ \hline
Plural Masc. & {\hindifont -ए} (e) & {\hindifont -ई} (ii) & {\hindifont -ए} (e) & {\hindifont -ए} (e) & {\hindifont राम ने सेब खाए।} (comió manzanas - masc. pl.) \ \hline
Plural Fem. & {\hindifont -ईं} (iin) & {\hindifont -ईं} (iin) & {\hindifont -ए} (e) & {\hindifont -ईं} (iin) & {\hindifont राम ने रोटियाँ खाईं।} (comió panes - fem. pl.) \ \hline
\end{tabularx}
\end{center}

\subsubsection{2.3 Caso Oblicuo}


Revisamos el caso oblicuo singular y plural:

\paragraph{Caso Oblicuo Singular (de Unidad 7 de A1):} 

\begin{center}
\begin{tabularx}{\textwidth}{| X | X | X | X |}
\hline
\textbf{Singular Nominativo} & \textbf{Caso Oblicuo Singular} & \textbf{Ejemplo con {\hindifont को}} & \textbf{Traducción} \ \hline
{\hindifont लड़का} (chico) & {\hindifont लड़के} & {\hindifont लड़के को} & Al chico \ \hline
{\hindifont किताब} (libro) & {\hindifont किताब} & {\hindifont किताब को} & Al libro \ \hline
{\hindifont लड़की} (chica) & {\hindifont लड़की} & {\hindifont लड़की को} & A la chica \ \hline
{\hindifont कमरा} (habitación) & {\hindifont कमरे} & {\hindifont कमरे में} & En la habitación \ \hline
\end{tabularx}
\end{center}

\paragraph{Caso Oblicuo Plural (de Unidad 22):} 

\begin{center}
\begin{tabularx}{\textwidth}{| X | X | X | X |}
\hline
\textbf{Plural Nominativo} & \textbf{Caso Oblicuo Plural} & \textbf{Ejemplo con {\hindifont को}} & \textbf{Traducción} \ \hline
{\hindifont लड़के} (chicos) & {\hindifont लड़कों} & {\hindifont लड़कों को} & A los chicos \ \hline
{\hindifont लड़कियाँ} (chicas) & {\hindifont लड़कियों} & {\hindifont लड़कियों को} & A las chicas \ \hline
{\hindifont किताबें} (libros) & {\hindifont किताबों} & {\hindifont किताबों को} & A los libros \ \hline
{\hindifont घर} (casas) & {\hindifont घरों} & {\hindifont घरों में} & En las casas \ \hline
\end{tabularx}
\end{center}

\subsubsection{2.4 Verbos Modales}


Revisamos diferentes construcciones modales:

\paragraph{Verbos Modales de Obligación:} 

\begin{center}
\begin{tabularx}{\textwidth}{| X | X | X | X |}
\hline
\textbf{Construcción} & \textbf{Significado} & \textbf{Ejemplo} & \textbf{Traducción} \ \hline
Infinitivo oblicuo + {\hindifont का है/हैं} & Obligación planificada & {\hindifont मुझे जाना का है।} & Yo tengo que ir. (Planificado) \ \hline
Infinitivo + {\hindifont पड़ना} & Obligación externa & {\hindifont मुझे जाना पड़ा।} & Yo tuve que ir. (Por circunstancia) \ \hline
{\hindifont चाहिए} & Necesidad/Consejo & {\hindifont तुम्हें आराम करना चाहिए।} & Tú deberías descansar. \ \hline
{\hindifont सकना} & Habilidad/Potencialidad & {\hindifont मैं यह कर सकता हूँ।} & Yo puedo hacer esto. \ \hline
{\hindifont पाना} & Logro/Consecución & {\hindifont मैं इसे कर पाया।} & Logré hacer esto. \ \hline
\end{tabularx}
\end{center}

\subsubsection{2.5 Adjetivos Comparativos y Superlativos}


\begin{center}
\begin{tabularx}{\textwidth}{| X | X | X | X |}
\hline
\textbf{Tipo} & \textbf{Construcción} & \textbf{Ejemplo} & \textbf{Traducción} \ \hline
Comparativo de superioridad & {\hindifont से ज़्यादा}, {\hindifont से बेहतर} & {\hindifont यह वह से ज़्यादा अच्छा है।} & Éste es mejor que aquél. \ \hline
Comparativo de inferioridad & {\hindifont से कम}, {\hindifont से ख़राब} & {\hindifont यह वह से कम अच्छा है।} & Éste es menos bueno que aquél. \ \hline
Superlativo & {\hindifont सबसे} + adjetivo & {\hindifont यह सबसे अच्छा है।} & Éste es el mejor. \ \hline
\end{tabularx}
\end{center}

\subsubsection{2.6 Cláusulas Subordinadas}


Revisamos el uso de {\hindifont कि} para introducir cláusulas subordinadas:


\begin{itemize}

  \item {\hindifont वह कहता है कि वह आएगा।} - Él dice que vendrá.

  \item {\hindifont मैं समझता हूँ कि यह ठीक है।} - Entiendo que esto está bien.

  \item {\hindifont यह लगता है कि मौसम ठीक हो रहा है।} - Parece que el tiempo se está poniendo bien.

\end{itemize}

\subsubsection{2.7 Participios y Formas Compuestas}


Revisamos formas como el participio conjuntivo:


\begin{itemize}

  \item {\hindifont खाकर} - Habiendo comido

  \item {\hindifont जाकर} - Yendo y después...

  \item {\hindifont करके} - Habiendo hecho

  \item Ejemplo: {\hindifont घर जाकर आराम करो।} - Id a casa y descansad.

\end{itemize}

\subsection{3. Práctica Integradora: Ejercicios de Consolidación}

\subsubsection{3.1 Ejercicio de Completar Frases}


Completa las siguientes frases con la forma correcta del verbo:


\begin{enumerate}

  \item राम \_\_\_\_\_\_\_ करता था। (पढ़ना) - Ram solía \_\_\_\_\_\_\_. (leer)

  \item मैं \_\_\_\_\_\_\_ जाना है। (स्कूल) - Yo tengo que ir a \_\_\_\_\_\_\_. (escuela)

  \item तुम \_\_\_\_\_\_\_ खा लो। (भात) - Come \_\_\_\_\_\_\_. (arroz)

  \item वह \_\_\_\_\_\_\_ कर पाया। (समझना) - Él logró \_\_\_\_\_\_\_. (entender)

  \item यह \_\_\_\_\_\_\_ सबसे अच्छा है। (तरीका) - Éste \_\_\_\_\_\_\_ es el mejor. (método)

  \item राम ने किताब \_\_\_\_\_\_\_. (पढ़ना, म. सि.) - Ram \_\_\_\_\_\_\_ el libro. (leyó)

  \item मैं सोचता हूँ कि यह \_\_\_\_\_\_\_ है। (ठीक) - Pienso que esto \_\_\_\_\_\_\_. (está bien)

  \item तुम घर जाकर \_\_\_\_\_\_\_। (आराम करना) - Id a casa y \_\_\_\_\_\_\_, vosotros. (descansad)

  \item हमें \_\_\_\_\_\_\_ समय पर आना चाहिए। (कार्यालय) - Deberíamos llegar a \_\_\_\_\_\_\_ a tiempo.

  \item यह स्थान वह स्थान से \_\_\_\_\_\_\_ है। (खूबसूरत) - Este lugar es \_\_\_\_\_\_\_ que aquél. (más hermoso)

  \item मैं \_\_\_\_\_\_\_ पढ़ रहा था। (किताब म.सि.) - Yo estaba \_\_\_\_\_\_\_ un libro.

  \item तुम्हारे पास \_\_\_\_\_\_\_ करने का समय था। (काम) - Tú tenías tiempo para \_\_\_\_\_\_\_. (trabajar)

  \item उसने कहा कि वह \_\_\_\_\_\_\_। (आना) - Él dijo que \_\_\_\_\_\_\_. (vendría)

  \item तुम \_\_\_\_\_\_\_ बहुत तेज़ चल रहे थे। (सड़क पर) - Vosotros \_\_\_\_\_\_\_ muy rápido por la calle.

  \item यह खाना वह खाना से \_\_\_\_\_\_\_ है। (महंगा) - Esta comida es \_\_\_\_\_\_\_ que aquélla.

\end{enumerate}


\begin{tcolorbox}[colback=blue!5!white,colframe=blue!75!black,title=Nota/Clarificación]

\paragraph{Solución del Ejercicio} 

\begin{enumerate}

  \item राम {\hindifont पढ़ाई} करता था।

  \item मैं {\hindifont स्कूल} जाना है।

  \item तुम {\hindifont भात} खा लो।

  \item वह {\hindifont समझ} कर पाया।

  \item यह {\hindifont तरीका} सबसे अच्छा है।

  \item राम ने किताब {\hindifont पढ़ी}.

  \item मैं सोचता हूँ कि यह {\hindifont ठीक} है।

  \item तुम घर जाकर {\hindifont आराम करो}.

  \item हमें {\hindifont कार्यालय} समय पर आना चाहिए।

  \item यह स्थान वह स्थान से {\hindifont खूबसूरत} है।

  \item मैं {\hindifont किताब} पढ़ रहा था।

  \item तुम्हारे पास {\hindifont काम} करने का समय था।

  \item उसने कहा कि वह {\hindifont आएगा}.

  \item तुम {\hindifont सड़क पर} बहुत तेज़ चल रहे थे।

  \item यह खाना वह खाना से {\hindifont महंगा} है।

\end{enumerate}


\end{tcolorbox}

\subsubsection{3.2 Traducción del Español al Hindi}


Traduce las siguientes frases al hindi:


\begin{enumerate}

  \item Yo solía estudiar mucho cuando era joven.

  \item Él tiene que salir ahora.

  \item Ellos tuvieron que esperar mucho tiempo.

  \item Este libro es mejor que ese libro.

  \item Yo puedo resolver este problema.

  \item Él logró entender la lección.

  \item Yo pienso que esto es correcto.

  \item Después de comer, descansamos.

  \item Él dice que vendrá mañana.

  \item Esta es la chica más inteligente de la clase.

  \item El chico que está leyendo allí es mi hermano.

  \item Si llueve, nos quedaremos en casa.

  \item Cuando tú vengas, nosotros hablaremos.

  \item Él compró el libro que tú leíste.

  \item La razón por la que no vino es que estaba enfermo.

\end{enumerate}


\begin{tcolorbox}[colback=blue!5!white,colframe=blue!75!black,title=Nota/Clarificación]

\paragraph{Respuestas} 

\begin{enumerate}

  \item {\hindifont जब मैं छोटा था, मैं बहुत पढ़ाई करता था।}

  \item {\hindifont उसे अभी निकलना है।}

  \item {\hindifont उन्हें बहुत समय तक प्रतीक्षा करनी पड़ी।}

  \item {\hindifont यह किताब वह किताब से बेहतर है।}

  \item {\hindifont मैं इस समस्या को हल कर सकता हूँ।}

  \item {\hindifont उसने पाठ को समझने का प्रयास किया।} o {\hindifont वह पाठ को समझ पाया।}

  \item {\hindifont मैं सोचता हूँ कि यह सही है।}

  \item {\hindifont खाने के बाद, हम आराम करते हैं।}

  \item {\hindifont वह कहता है कि वह कल आएगा।}

  \item {\hindifont यह कक्षा की सबसे समझदार लड़की है।}

  \item {\hindifont वह लड़का जो वहाँ पढ़ रहा है, मेरा भाई है।}

  \item {\hindifont अगर बारिश होती है, तो हम घर पर रहेंगे।}

  \item {\hindifont जब तुम आओगे, तब हम बात करेंगे।}

  \item {\hindifont उसने वह किताब खरीदी जो तुमने पढ़ी थी।}

  \item {\hindifont उसके न आने का कारण यह था कि वह बीमार था।}

\end{enumerate}


\end{tcolorbox}

\subsubsection{3.3 Traducción del Hindi al Español}


Traduce las siguientes frases al español:


\begin{enumerate}

  \item {\hindifont तुम पिछले वर्ष परीक्षा में उत्तीर्ण हुए थे।}

  \item {\hindifont मैं उस दिन काम पर नहीं गया क्योंकि बुखार था।}

  \item {\hindifont यह वह घर है जिसमें राम रहता है।}

  \item {\hindifont हम सभी को नियमों का पालन करना चाहिए।}

  \item {\hindifont राम के घर में गाय है।}

  \item {\hindifont तुम जल्दी करो ताकि हम समय पर पहुँच सकें।}

  \item {\hindifont मैं नई किताबें पढ़ने का शौक रखता हूँ।}

  \item {\hindifont वह पहले घर गया, फिर काम पर गया।}

  \item {\hindifont मैंने समझाने की कोशिश की, लेकिन वह नहीं माना।}

  \item {\hindifont यह फल उस फल से अधिक मीठा है।}

\end{enumerate}


\begin{tcolorbox}[colback=blue!5!white,colframe=blue!75!black,title=Nota/Clarificación]

\paragraph{Respuestas} 

\begin{enumerate}

  \item Tú aprobaste el examen el año pasado.

  \item Yo no fui al trabajo ese día porque tenía fiebre.

  \item Ésta es la casa en la que vive Ram.

  \item Todos nosotros deberíamos seguir las reglas.

  \item Hay una vaca en la casa de Ram.

  \item Haced prisa para que podamos llegar a tiempo.

  \item Yo tengo el gusto de leer libros nuevos.

  \item Él fue primero a casa, luego al trabajo.

  \item Yo intenté explicar, pero él no aceptó.

  \item Esta fruta es más dulce que esa fruta.

\end{enumerate}


\end{tcolorbox}

\subsection{4. Conversaciones Completas}

\subsubsection{4.1 Diálogo en el Hospital}


\begin{itemize}

  \item Doctor: {\hindifont नमस्कार! क्या हाल है?} - ¡Namaste! ¿Cómo estás?

  \item Paciente: {\hindifont नमस्कार डॉक्टर साहब। मुझे बुखार है।} - ¡Namaste Dr. Sahib! Tengo fiebre.

  \item Doctor: {\hindifont कितने दिन से?} - ¿Desde cuántos días?

  \item Paciente: {\hindifont तीन दिन से। मैं बहुत थका हुआ महसूस कर रहा हूँ।} - Desde tres días. Me siento muy cansado.

  \item Doctor: {\hindifont क्या आपको दर्द भी है?} - ¿También tienes dolor?

  \item Paciente: {\hindifont हाँ, मुझे सिर और शरीर में दर्द है।} - Sí, tengo dolor de cabeza y cuerpo.

  \item Doctor: {\hindifont ठीक है। मैं आपका तापमान चेक करता हूँ।} - Bien. Yo comprobaré vuestra temperatura.

  \item Paciente: {\hindifont क्या यह गंभीर है?} - ¿Es grave?

  \item Doctor: {\hindifont जैसा मैं देखता हूँ, आपके पास सामान्य बुखार है। यह गंभीर नहीं है।} - Como veo, vosotros tenéis fiebre común. No es grave.

  \item Paciente: {\hindifont फिर आप क्या सलाह देंगे?} - Entonces, ¿qué aconsejaréis?

  \item Doctor: {\hindifont आप यह दवा लें। आपको अधिक आराम करना चाहिए। और अधिक पानी पिएं। तीन दिन में आप ठीक हो जाएंगे।} - Tomad esta medicina. Deberíais descansar más. Y bebed más agua. En tres días vosotros os pondréis bien.

  \item Paciente: {\hindifont धन्यवाद, डॉक्टर साहब।} - Gracias, Dr. Sahib!

  \item Doctor: {\hindifont कोई बात नहीं। यदि कोई समस्या हो, तो कृपया मुझे बुलाएं।} - No hay problema. Si hay algún problema, por favor llamadme.

\end{itemize}

\subsubsection{4.2 Diálogo en un Restaurante}


\begin{itemize}

  \item Garzón: {\hindifont नमस्कार! क्या मैं आपकी कुछ सहायता कर सकता हूँ?} - ¡Namaste! ¿Puedo ayudaros en algo?

  \item Cliente: {\hindifont हाँ, क्या आप मुझे मेनू दिखा सकते हैं?} - Sí, ¿podéis mostrarme el menú?

  \item Garzón: {\hindifont जी हाँ, यहाँ है मेनू। हमारे यहाँ बहुत सारे विकल्प हैं।} - Sí, aquí está el menú. Aquí tenemos muchas opciones.

  \item Cliente: {\hindifont मैं सब्ज़ी खाना पसंद करता हूँ। क्या आपके पास कोई सब्ज़ी विकल्प है?} - Me gusta comer verduras. ¿Tenéis alguna opción de verduras?

  \item Garzón: {\hindifont हाँ, हमारे पास बहुत सारी सब्ज़ियाँ हैं। आलू, पालक, बैंगन, गोभी आदि।} - Sí, tenemos muchas verduras. Patatas, espinacas, berenjenas, col, etc.

  \item Cliente: {\hindifont ठीक है, मैं पालक का ऑर्डर देना चाहता हूँ। और दो रोटी।} - Bien, yo quiero pedir espinacas. Y dos panes (roti).

  \item Garzón: {\hindifont ठीक है। क्या आपको कुछ पीना भी चाहिए?} - Bien. ¿También queréis beber algo?

  \item Cliente: {\hindifont हाँ, मुझे चाय चाहिए।} - Sí, necesito té.

  \item Garzón: {\hindifont आपका ऑर्डर हमारे पास पहुँच गया है। थोड़ी देर में आपका खाना तैयार हो जाएगा।} - Vuestro pedido ha llegado a nosotros. En un momento vuestro comida estará lista.

  \item Cliente: {\hindifont बहुत अच्छा। धन्यवाद।} - Muy bien. Gracias.

  \item Garzón: {\hindifont कोई बात नहीं। यदि आपको कुछ और चाहिए, तो मुझे बुलाएं।} - No hay problema. Si necesitáis algo más, llamadme.

  \item Cliente: {\hindifont निश्चित रूप से, अगर मुझे कुछ और चाहिए होगा तो मैं आपको बुलाऊंगा।} - Ciertamente, si necesito algo más, yo te llamaré.

  \item Garzón: {\hindifont यह सही है। मैं आपके खाने की तैयारी के बारे में पता करता हूँ।} - Eso es correcto. Yo me entero sobre la preparación de vuestra comida.

\end{itemize}

\subsubsection{4.3 Diálogo en un Viaje}


\begin{itemize}

  \item Turista: {\hindifont क्या आप इस स्थान के बारे में मुझे कुछ बता सकते हैं?} - ¿Podéis contarme algo sobre este lugar?

  \item Guía: {\hindifont हाँ, यह एक प्राचीन स्थान है। यहाँ के इतिहास में बहुत कुछ है।} - Sí, éste es un lugar antiguo. En la historia de aquí hay mucho.

  \item Turista: {\hindifont यहाँ के लोग कैसे रहते हैं?} - ¿Cómo viven la gente aquí?

  \item Guía: {\hindifont यहाँ के लोग सरल जीवन जीते हैं। यहाँ का माहौल बहुत शांतिपूर्ण है।} - La gente aquí vive una vida simple. El ambiente aquí es muy pacífico.

  \item Turista: {\hindifont क्या यहाँ कोई विशेष त्योहार है?} - ¿Hay algún festival especial aquí?

  \item Guía: {\hindifont हाँ, यहाँ हर साल एक बड़ा त्योहार होता है। इसमें सभी लोग भाग लेते हैं।} - Sí, aquí cada año hay un gran festival. En éste participan todas las personas.

  \item Turista: {\hindifont यह तो बहुत अच्छा है। क्या मैं यहाँ कुछ खरीद सकता हूँ?} - Eso es muy bueno. ¿Puedo comprar algo aquí?

  \item Guía: {\hindifont हाँ, यहाँ बहुत सारी हस्तशिल्प दुकानें हैं। आप वहाँ से खरीद सकते हैं।} - Sí, aquí hay muchas tiendas de artesanía. Vosotros podéis comprar allí.

  \item Turista: {\hindifont मैं यहाँ कुछ दिन रहना चाहता हूँ। क्या यहाँ कोई होटल है?} - Yo quiero quedarme aquí unos días. ¿Hay algún hotel aquí?

  \item Guía: {\hindifont हाँ, हमारे पास एक छोटा सा होटल है। आप वहाँ ठहर सकते हैं।} - Sí, tenemos un hotel pequeño. Vosotros podéis alojaros allí.

  \item Turista: {\hindifont यह बहुत सहायक रहा। आपका धन्यवाद।} - Ésta fue muy útil. Gracias.

  \item Guía: {\hindifont यह मेरा कर्तव्य है। आप यहाँ सुखी हों, यह मुझे खुश करता है।} - Éste es mi deber. Que vosotros seáis felices aquí me alegra.

  \item Turista: {\hindifont मैं इस जगह को अपने दोस्तों को बताऊँगा।} - Yo contaré sobre este lugar a mis amigos.

  \item Guía: {\hindifont यह बहुत अच्छा होगा। हम इसे और लोगों के साथ साझा करना चाहते हैं।} - Ésto será muy bueno. Nosotros queremos compartirlo con más gente.

\end{itemize}

\subsection{5. Ejercicios de Comprensión Auditiva y Oral (Simulación)}


Aunque en este formato no podemos proporcionar audios reales, aquí presentamos ejercicios basados en conversaciones que típicamente se escucharían en una evaluación auditiva:

\subsubsection{5.1 Escuchar y Responder}


Imagina que escuchas esta conversación y responde las siguientes preguntas:


\textbf{Transcripción:} {\hindifont राम: तुमने कल क्या किया? सीता: मैंने कल स्कूल गया। फिर मैंने घर पर पढ़ाई की। राम: क्या तुम आज कहीं घूमने जा रहे हो? सीता: नहीं, मैं घर पर रहूंगा। मैं अपने काम को पूरा करना चाहता हूँ।}

\paragraph{Preguntas:} 

\begin{enumerate}

  \item ¿Qué hizo Sita ayer?

  \item ¿Va Sita a salir hoy?

  \item ¿Por qué Sita va a permanecer en casa?

  \item ¿Cuál es el plan de Sita para hoy?

\end{enumerate}


\begin{tcolorbox}[colback=blue!5!white,colframe=blue!75!black,title=Nota/Clarificación]

\paragraph{Respuestas:} 

\begin{enumerate}

  \item Sita fue a la escuela ayer y luego estudió en casa.

  \item No, Sita no va a salir hoy.

  \item Sita va a permanecer en casa porque quiere completar su trabajo.

  \item El plan de Sita para hoy es quedarse en casa y completar su trabajo.

\end{enumerate}


\end{tcolorbox}

\subsubsection{5.2 Escuchar y Completar}


Imagina que escuchas esta descripción y completa los espacios:


\textbf{Transcripción:} {\hindifont हम यहाँ से एक छोटा सा \_\_\_\_\_\_\_ देख सकते हैं। वहाँ पर कई \_\_\_\_\_\_\_ हैं। लोग उन पेड़ों के नीचे \_\_\_\_\_\_\_ करते हैं। यह स्थान बहुत \_\_\_\_\_\_\_ है। हम यहाँ थोड़ी देर \_\_\_\_\_\_\_ कर सकते हैं।}


\begin{tcolorbox}[colback=blue!5!white,colframe=blue!75!black,title=Nota/Clarificación]

\paragraph{Respuestas posibles:} 

Podría ser: {\hindifont हम यहाँ से एक छोटा सा झील देख सकते हैं। वहाँ पर कई पेड़ हैं। लोग उन पेड़ों के नीचे आराम करते हैं। यह स्थान बहुत शांत है। हम यहाँ थोड़ी देर बैठ सकते हैं।}


Traducción: Podemos ver un pequeño lago desde aquí. Allí hay muchos árboles. La gente descansa debajo de esos árboles. Este lugar es muy tranquilo. Podemos sentarnos aquí un rato.


\end{tcolorbox}

\subsection{6. Evaluación de Nivel A2}


Para completar esta unidad de consolidación, incluimos algunos ejercicios tipo examen que evaluarían las competencias de nivel A2:

\subsubsection{6.1 Ejercicio de Gramática}


Elige la opción correcta:


\begin{enumerate}

  \item राम ने तुम्हें \_\_\_\_\_\_\_। (a) देखा (b) देखी (c) देखे

  \item मैं \_\_\_\_\_\_\_ करता हूँ कि यह सच है। (a) सोचता (b) सोचती (c) सोचते

  \item यह किताब वह किताब से \_\_\_\_\_\_\_ है। (a) ज़्यादा अच्छा (b) सबसे अच्छा (c) अच्छे

  \item तुम \_\_\_\_\_\_\_ पार्क जा सकते हो। (a) यह (b) इस (c) इसके

  \item मैं जल्दी से घर \_\_\_\_\_\_\_। (a) आया था (b) आया रहा था (c) आया है

\end{enumerate}


\begin{tcolorbox}[colback=blue!5!white,colframe=blue!75!black,title=Nota/Clarificación]

\paragraph{Respuestas:} 

\begin{enumerate}

  \item (a) देखा - Ram te vio. (Objeto femenino: "तुम्हें" con को se vuelve masc. sing. en verbo)

  \item (a) सोचता - Yo pienso que esto es cierto. (sujeto masculino)

  \item (a) ज़्यादा अच्छा - Este libro es mejor que ese libro.

  \item (b) इस - Vosotros podéis ir a este parque.

  \item (a) आया था - Yo vine rápidamente a casa. (Pasado simple intransitivo)

\end{enumerate}


\end{tcolorbox}

\subsubsection{6.2 Ejercicio de Vocabulario}


Empareja las palabras con sus significados:


\begin{center}
\begin{tabularx}{\textwidth}{| X | X |}
\hline
\textbf{Palabra (Hindi)} & \textbf{Significado} \ \hline
1. {\hindifont इतिहास} & A. Alegría \ \hline
2. {\hindifont खुशी} & B. Historia \ \hline
3. {\hindifont मौसम} & C. Medicina \ \hline
4. {\hindifont दवा} & D. Tiempo (climático) \ \hline
5. {\hindifont गाना} & E. Canción \ \hline
\end{tabularx}
\end{center}


\begin{tcolorbox}[colback=blue!5!white,colframe=blue!75!black,title=Nota/Clarificación]

\paragraph{Respuestas:} 

1-B, 2-A, 3-D, 4-C, 5-E


\end{tcolorbox}

\subsubsection{6.3 Ejercicio de Comprensión Escrita}


Lee el siguiente texto y responde las preguntas:


\begin{tcolorbox}[colback=blue!5!white,colframe=blue!75!black,title=Nota/Clarificación]


{\hindifont हम छात्रों को पढ़ाते हैं। हम उन्हें ज्ञान देते हैं। छात्र हमारे साथ सीखते हैं। वे पुस्तकें पढ़ते हैं। वे लिखते हैं। और वे अभ्यास करते हैं। हमें लगता है कि सीखना बहुत महत्वपूर्ण है। हम यह समझते हैं कि ज्ञान जीवन का आधार है।}


Traducción: Nosotros enseñamos a los estudiantes. Les damos conocimiento. Los estudiantes aprenden con nosotros. Ellos leen libros. Ellos escriben. Y ellos practican. A nosotros nos parece que aprender es muy importante. Nosotros entendemos que el conocimiento es la base de la vida.


\end{tcolorbox}

\paragraph{Preguntas:} 

\begin{enumerate}

  \item ¿Qué hacen los profesores con los estudiantes?

  \item ¿Qué hacen los estudiantes?

  \item ¿Por qué es importante aprender según el texto?

  \item ¿Qué consideran los profesores que es la base de la vida?

\end{enumerate}


\begin{tcolorbox}[colback=blue!5!white,colframe=blue!75!black,title=Nota/Clarificación]

\paragraph{Respuestas:} 

\begin{enumerate}

  \item Los profesores enseñan a los estudiantes y les dan conocimiento.

  \item Los estudiantes leen libros, escriben y practican.

  \item Porque aprender es muy importante.

  \item Consideran que el conocimiento es la base de la vida.

\end{enumerate}


\end{tcolorbox}

\subsubsection{6.4 Producción Escrita}


Escribe un párrafo de 5-7 líneas sobre uno de estos temas:


\begin{itemize}

  \item Mi ciudad ideal

  \item Un viaje memorable

  \item Mi rutina diaria

  \item La importancia de la educación

\end{itemize}

\paragraph{Ejemplo de respuesta (sobre "Mi rutina diaria"):} 

\begin{tcolorbox}[colback=blue!5!white,colframe=blue!75!black,title=Nota/Clarificación]


{\hindifont हर दिन मैं सुबह छह बजे उठता हूँ। फिर मैं अपना नाश्ता बनाता हूँ और खाता हूँ। उसके बाद मैं अपने काम पर जाता हूँ। शाम को मैं घर लौटता हूँ। फिर मैं रात का खाना बनाता हूँ। रात में मैं पुस्तक पढ़ता हूँ। यह मेरी दैनिक दिनचर्या है।}


Traducción: Todos los días me levanto a las seis de la mañana. Luego preparo y como mi desayuno. Después voy a mi trabajo. Por la tarde vuelvo a casa. Luego preparo la cena. Por la noche leo libros. Ésta es mi rutina diaria.


\end{tcolorbox}

\subsection{7. Componente Cultural: Festividades (Diwali, Holi)}


Para completar nuestra consolidación A2, exploremos dos de las festividades más importantes de la India:

\subsubsection{7.1 Diwali (Festival de las Luces)}


Diwali es una de las festividades más importantes del hinduismo. Se celebra durante cinco días y simboliza la victoria del bien sobre el mal, de la luz sobre la oscuridad y del conocimiento sobre la ignorancia.


\begin{itemize}

  \item {\hindifont दीपावली भारत में सबसे अधिक उत्साह के साथ मनाई जाने वाली छुट्टियों में से एक है।} - Diwali es una de las fiestas más celebradas con entusiasmo en India.

  \item {\hindifont लोग अपने घरों को दीयों और रंगोली से सजाते हैं।} - La gente decora sus casas con lámparas de aceite y rangoli.

  \item {\hindifont वह लोग एक दूसरे को मिठाइयाँ और उपहार देते हैं।} - La gente se da dulces y regalos mutuamente.

  \item {\hindifont कई स्थानों पर लोग पटाखे भी जलाते हैं।} - En muchos lugares, la gente también quema petardos.

  \item {\hindifont यह लक्ष्मी जी के सम्मान में मनाया जाता है।} - Éste se celebra en honor a Lakshmi Ji.

\end{itemize}

\subsubsection{7.2 Holi (Festival de los Colores)}


Holi es una festividad primaveral que celebra el triunfo del bien sobre el mal y la llegada de la primavera. También es conocida como el "festival de los colores".


\begin{itemize}

  \item {\hindifont होली रंगों का त्योहार है।} - Holi es el festival de los colores.

  \item {\hindifont लोग एक दूसरे पर रंग और गुलाल फेंकते हैं।} - La gente se lanza colores y gulal mutuamente.

  \item {\hindifont वे खुशी से नाचते और गाते हैं।} - Ellos bailan y cantan con alegría.

  \item {\hindifont हम ठाकुरजी को भोग लगाते हैं।} - Ofrecemos alimentos sagrados a Thakurji.

  \item {\hindifont यह लोगों के बीच भाईचारा बढ़ाता है।} - Éste aumenta la camaradería entre la gente.

  \item {\hindifont हम ठंढाई पीते हैं और मिठाइयाँ खाते हैं।} - Bebemos thandai y comemos dulces.

  \item {\hindifont होली का मतलब है कि सभी को खुश रहना चाहिए।} - El significado de Holi es que todos deberían ser felices.

  \item {\hindifont इस त्योहार में लोग वर्ण, धर्म और समाज के अंतर को भूल जाते हैं।} - En este festival, la gente olvida las diferencias de casta, religión y sociedad.

  \item {\hindifont होली का आनंद लेने के लिए लोग घूमते हैं।} - La gente pasea para disfrutar del placer de Holi.

  \item {\hindifont हम सभी को इस उत्सव को साथ में मनाना चाहिए।} - Todos nosotros deberíamos celebrar este festival juntos.

\end{itemize}

\subsection{8. Conclusión}


Con esta unidad completamos el nivel A2 del curso de hindi. Has aprendido:


\begin{itemize}

  \item A formar y usar todos los tiempos verbales fundamentales

  \item A usar el sistema ergativo con verbos transitivos en pasado

  \item A construir cláusulas subordinadas con {\hindifont कि}

  \item A usar conectores condicionales y temporales

  \item A expresar comparativos y superlativos

  \item A formular expresiones modales de obligación, posibilidad y habilidad

  \item A usar participios y formas compuestas

  \item A comprender y producir textos escritos de nivel A2

  \item A comunicarte en situaciones cotidianas y en contextos específicos

  \item Acerca de aspectos culturales importantes del mundo hindi

\end{itemize}


Estás preparado/a para avanzar al nivel A2 del curso de hindi, donde desarrollarás aún más tu competencia comunicativa en contextos más complejos.

\end{document}